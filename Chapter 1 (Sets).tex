\documentclass{book}
\usepackage[english]{babel}
\usepackage{geometry,amsmath,amssymb,enumerate,bbm,latexsym,theorem,makeidx}
\makeindex
\geometry{a4paper, paperheight= auto, paperwidth= auto}

%%%%%%%%%% Start TeXmacs macros
\catcode`\<=\active \def<{
\fontencoding{T1}\selectfont\symbol{60}\fontencoding{\encodingdefault}}
\catcode`\>=\active \def>{
\fontencoding{T1}\selectfont\symbol{62}\fontencoding{\encodingdefault}}
\newcommand{\Leftrightarrowlim}{\mathop{\leftrightarrow}\limits}
\newcommand{\Nu}{\mathrm{N}}
\newcommand{\Rightarrowlim}{\mathop{\rightarrow}\limits}
\newcommand{\comma}{{,}}
\newcommand{\equallim}{\mathop{=}\limits}
\newcommand{\matheuler}{\gamma}
\newcommand{\nin}{\not\in}
\newcommand{\nobracket}{}
\newcommand{\nocomma}{}
\newcommand{\noplus}{}
\newcommand{\nosymbol}{}
\newcommand{\tmmathbf}[1]{\ensuremath{\boldsymbol{#1}}}
\newcommand{\tmop}[1]{\ensuremath{\operatorname{#1}}}
\newcommand{\tmstrong}[1]{\textbf{#1}}
\newcommand{\tmtextbf}[1]{{\bfseries{#1}}}
\newcommand{\tmverbatim}[1]{{\ttfamily{#1}}}
\newcommand{\um}{-}
\newcommand{\upl}{+}
\newenvironment{enumeratealpha}{\begin{enumerate}[a{\textup{)}}] }{\end{enumerate}}
\newenvironment{itemizedot}{\begin{itemize} \renewcommand{\labelitemi}{$\bullet$}\renewcommand{\labelitemii}{$\bullet$}\renewcommand{\labelitemiii}{$\bullet$}\renewcommand{\labelitemiv}{$\bullet$}}{\end{itemize}}
\newenvironment{proof}{\noindent\textbf{Proof\ }}{\hspace*{\fill}$\Box$\medskip}
\newtheorem{axiom}{Axiom}
\newtheorem{corollary}{Corollary}
\newtheorem{definition}{Definition}
{\theorembodyfont{\rmfamily}\newtheorem{example}{Example}}
\newtheorem{lemma}{Lemma}
\newtheorem{notation}{Notation}
{\theorembodyfont{\rmfamily}\newtheorem{note}{Note}}
\newtheorem{theorem}{Theorem}
\providecommand{\xRightarrow}[2][]{\mathop{\Longrightarrow}\limits_{#1}^{#2}}
%%%%%%%%%% End TeXmacs macros

\begin{document}

{\locus{{\id{/home/marc/Documents/Measures/Chapter 1
(Sets).tm}}}{{\observer{+Nvp8DlRdqED2VX}{mirror-notify}}}{\chapter{Classes}

\section{Basic concepts about classes}

We use Von Neuman's set of axioms to define a set and class theory. We start
with two undefined notions: \tmtextbf{class} and the \tmtextbf{membership
relation $\in$}. The objects of our discourse will be classes. Later we will
introduce the notion of \tmtextbf{set} that is a special kind of class. Every
\tmtextbf{set} will be a class but not every \tmtextbf{class} will be a
\tmtextbf{set}. Intuitive we understand a class to be a kind of collection and
$x \in C$ to mean that $x$ is in the collection $C$. The elements of a class
are themselves considered classes. As a notation shorthand we always use
uppercase for classes (if we think of them as a collection) and lowercase for
elements of a class (although they are also classes).

\begin{notation}
  We introduce the following notational terms
  \begin{eqnarray*}
    \wedge & \tmop{meaning} & \tmop{and}\\
    \vee & \tmop{meaning} & \tmop{or}\\
    \neg & \tmop{meaning} & \tmop{not}\\
    \Rightarrow & \tmop{meaning} & \tmop{implies}\\
    \Leftrightarrow & \tmop{meaning} & \tmop{is} \tmop{equivalent}
    \tmop{with}\\
    \in & \tmop{meaning} & \tmop{is} \tmop{element}\\
    \vdash & \tmop{meaning} & \tmop{with}\\
    \vDash & \tmop{meaning} & \tmop{where}\\
    \forall & \tmop{meaning} & \tmop{forall}\\
    \exists & \tmop{meaning} & \tmop{there} \tmop{exists}\\
    \exists ! & \tmop{meaning} & \tmop{there} \tmop{exists} a \tmop{unique}
  \end{eqnarray*}
\end{notation}

\begin{definition}
  {\index{$\in$}}We say that $x$ is a element if there exists a class $A$ such
  that $x \in A$
\end{definition}

\begin{definition}
  {\index{$\nin$}}$x \nin A$ is the same as $\neg (x \in A)$
\end{definition}

\begin{definition}
  \label{equality of classes}{\index{$A = B$}}Let $A, B$ be classes then we
  say that $A = B$ if and only if $x \in A \Rightarrow x \in B \wedge x \in B
  \Rightarrow x \in A$
\end{definition}

So two classes are considered equal if they have the same elements. Once we
have equality defined for classes we can \ establish our first axiom
concerning classes.

\begin{axiom}[Axiom of extent]
  \label{axiom of extent}{\index{axiom of extent}}If $x = y$ and $x \in A
  \Rightarrow y \in A$. 
\end{axiom}

\begin{definition}
  \label{subclasses}{\index{$\subseteq$}}Let $A$ and $B$ be classes then $A
  \subseteq B$ iff $x \in A \Rightarrow x \in B$, we call $A$ a subclass of
  $B$
\end{definition}

\begin{definition}
  \label{proper subclass}{\index{$\subset$}}Let $A$ and $B$ be classes then $A
  \subset B$ iff $A \subseteq B$ and $A \neq B$, we call $A$ a proper subclass
  of $B$
\end{definition}

\begin{theorem}
  \label{properties of classes}For all classes $A, B$ and $C$ the following
  hold:
  \begin{enumerate}
    \item $A = A$
    
    \item $A = B \Rightarrow B = A$
    
    \item $A = B$ and $B = C \Rightarrow A = C$
    
    \item $A \subseteq B$ and $B \subseteq A \Rightarrow A = B$
    
    \item $A \subseteq B$ and $B \subseteq C \Rightarrow A \subseteq C$
  \end{enumerate}
\end{theorem}

\begin{proof}
  
  \begin{enumerate}
    \item $x \in A \Rightarrow x \in A$ and $x \in A \Rightarrow x \in A$ is
    obviously true thus by definition \ref{equality of classes} we have $A =
    A$.
    
    \item If $A = B$ then $x \in A \Rightarrow x \in B \wedge x \in B
    \Rightarrow x \in A$ and thus $x \in B \Rightarrow x \in A \wedge x \in A
    \Rightarrow x \in B$ and thus $B = A$
    
    \item If $A = B$ and $B = C$ then we have $x \in A \Rightarrow x \in B$
    and $x \in B \Rightarrow x \in C$ so that we have $x \in A \Rightarrow x
    \in C$ (a). Also we have $x \in C \Rightarrow x \in B$ and $x \in B
    \Rightarrow x \in A$ giving thus $x \in C \Rightarrow x \in A$ (b) and
    thus by (a) and (b) we have $A = C$
    
    \item From $A \subseteq B$ we have $x \in A \Rightarrow x \in B$ and from
    $B \subseteq A$ we have $x \in B \Rightarrow x \in A$ and thus by
    definition $A = B$
    
    \item From $A \subseteq B$ it follows that $x \in A \Rightarrow x \in B$
    (1) and from $B \subseteq C$ it follows that $x \in B \Rightarrow x \in C$
    (2) and from (1) and (2) we have then $x \in A \Rightarrow x \in C$ and
    thus $A \subseteq B$
  \end{enumerate}
\end{proof}

Lets now consider different ways of making classes. Let $P (x)$ denotes a
statement about $x$ which can be expressed in terms of the symbols $\in
\nocomma$, $\wedge$, $\vee$, $\neg$, $\Rightarrow$, $\exists$, $\forall,
\vdash, \vDash$ brackets and variables $x, y, z \ldots, A, B, C, \ldots$ then
we have.

\begin{axiom}[Axiom of class construction]
  \label{axiom of construction}{\index{axiom of class construction}}There
  exists a class $C$ such that $x \in C$ iff \ x is a element and $P (x)$ is
  true, we note this as $C = \{ x | P (x) \nobracket \}$
\end{axiom}

\begin{note}
  This definition avoids the classical Russel's Paradox as this allows us to
  form a class consisting of all elements $x$ which satisfies $P (x)$ but not
  the class of all classes $x$ such that $P (x)$. So if we take $P (x)$ to be
  $x \nin x$ then $C = \{ x | x \nin x \nobracket \}$ and we have the
  following two possibilities
  \begin{enumerate}
    \item $C \in C$ is impossible as this would imply that $C$ is a element
    and $C \in C \Rightarrow C \nin C$
    
    \item $C \nin C$ then we have the following possibilities
    \begin{enumerate}
      \item $C$ is a element but then $C \in C$ a contradiction
      
      \item $C$ is not a element so that indeed $C \nin C$
    \end{enumerate}
  \end{enumerate}
  So we have reduced Russel's paradox to the harmless statement that $C \nin
  C$ and $C$ is not a element. The use of lowercase in $C = \{ x | \nobracket
  P (x) \}$ is essential as we have agreed to designate elements by lower
  cases so $C = \{ x | x \tmop{is} a \tmop{element} \tmop{and} P (x)
  \nobracket \}$. Another notation we use is $\{ x \in A | P (x) \nobracket
  \}$ which is the same as $\{ x | x \in A \wedge P (x) \nobracket \}$ (note
  that $x \in A$ already implements that $x$ is a element).
\end{note}

Using the axiom of class construction we can now define the union of two
classes

\begin{definition}
  {\index{$\bigcup$}}Let $A$ and $B$ be classes then $A \bigcup B = \{ x | x
  \in A \vee x \in B \nobracket \}$
\end{definition}

\begin{definition}
  {\index{$\bigcap$}}Let $A$ and $B$ be classes then $A \bigcap B = \{ x | x
  \in A \wedge x \in B \nobracket \}$
\end{definition}

\begin{definition}
  {\index{$\mathcal{U}$}}We define the universal class \ $\mathcal{U}$ to be
  $\mathcal{U}= \{ x | x = x \nobracket \}$ as the class of all elements ($x =
  x$ is always true) [note that we use the short notation so $\mathcal{U}= \{
  x|x \tmop{is} a \tmop{element} \tmop{and} x = x \}$]
\end{definition}

\begin{definition}
  {\index{$\emptyset$}}We define the empty class $\emptyset$ by $\emptyset =
  \{ x | x \neq x \nobracket \}$ which is the class with no elements for if $x
  \in \emptyset$ then $x \neq x$ which is a contradiction.
\end{definition}

\begin{theorem}
  For every class $A$ we have $\emptyset \subseteq A$ and $A \subseteq
  \mathcal{U}$
\end{theorem}

\begin{proof}
  
  \begin{enumerate}
    \item We prove $\emptyset \subseteq A$ by contra positivity so $x \in
    \emptyset \Rightarrow x \in A$ is equivalent with $x \nin A \Rightarrow x
    \nin \emptyset$. Now if $x \nin A \Rightarrow x \nin \emptyset$ as
    $\emptyset$ has no elements.
    
    \item If $x \in A$ then $x$ is a element and $x = x$ so $x \in
    \mathcal{U}$
  \end{enumerate}
\end{proof}

\begin{definition}
  Two classes $A, B$ are disjoint if $A \bigcap B = \emptyset$ (they have no
  common elements)
\end{definition}

\begin{definition}
  {\index{$A^c$}}If $A$ is a class then we define $A^c = \{ x : x \nin A \}$
\end{definition}

\begin{definition}
  If $A$ and $B$ are classes then the class $A \backslash B = A \bigcap B^c$.
\end{definition}

\begin{theorem}
  If $A$ and $B$ are classes then $A \backslash B = \{ x \in A | x \nin B
  \nobracket \}$
\end{theorem}

\begin{proof}
  
  \begin{eqnarray*}
    x \in A \backslash B & \Leftrightarrow & x \in A \bigcap B^c\\
    & \Leftrightarrow & x \in A \wedge x \in B^c\\
    & \Leftrightarrow & x \in A \wedge x \nin B\\
    & \Leftrightarrow & x \in \{ x \in A | x \nin B \nobracket \}
  \end{eqnarray*}
  
\end{proof}

\

\section{Class Operations}

\

\begin{theorem}
  \label{relation classes and intersection and union}If $A, B$ are classes
  then we have
  \begin{enumerate}
    \item $A \subseteq A \bigcup B$ and $B \subseteq A \bigcup B$
    
    \item $A \bigcap B \subseteq A$ and $A \bigcap B \subseteq B$
  \end{enumerate}
\end{theorem}

\begin{proof}
  
  \begin{enumerate}
    \item $x \in A \Rightarrow x \in A \vee x \in B \Rightarrow x \in A
    \bigcup B$, $x \in B \Rightarrow x \in A \vee x \in B \Rightarrow x \in A
    \bigcup B$
    
    \item $x \in A \bigcap B \Rightarrow x \in A \wedge x \in B \Rightarrow x
    \in A$ and thus $A \bigcap B \subseteq A$. $x \in A \bigcap B \Rightarrow
    x \in B$ and thus $A \bigcap B \subseteq B$
  \end{enumerate}
\end{proof}

\begin{theorem}
  \label{relation subclass and union or intersection}If $A, B$ are classes
  then we have
  \begin{enumerate}
    \item $A \subseteq B$ if and only if $A \bigcup B = B$
    
    \item $A \subseteq B$ if and only if $A \bigcap B = A$
  \end{enumerate}
\end{theorem}

\begin{proof}
  
  \begin{enumerate}
    \item
    \begin{enumerate}
      \item If $A \subseteq B$ then $x \in A \Rightarrow x \in B$ so if $x \in
      A \bigcup B \Rightarrow x \in A \vee x \in B \Rightarrow x \in B \vee x
      \in B \Rightarrow x \in B$ and thus $A \bigcup B \subseteq B$. From the
      previous theorem we have $B \subseteq A \bigcup B$ so by \ref{properties
      of classes} we have $A \bigcup B = B$
      
      \item If $A \bigcup B = B$ then $(x \in A \vee x \in B) \Rightarrow x
      \in B$ so $x \in A \Rightarrow x \in B$ and thus $A \subseteq B$
    \end{enumerate}
    \item
    \begin{enumerate}
      \item If $A \subseteq B$ then $x \in A \Rightarrow x \in B$ so if $x \in
      A \Rightarrow x \in A \wedge x \in B$ and thus $A \subseteq A \bigcap B$
      and from the previous theorem we have $A \bigcap B \subseteq A$ so by
      \ref{properties of classes} we have $A \bigcap B = A$
      
      \item If $A \bigcap B = A$ we have $x \in A \Rightarrow (x \in A \wedge
      x \in B)$ so $x \in A \Rightarrow x \in B$ and thus $A \subseteq B$
    \end{enumerate}
  \end{enumerate}
\end{proof}

\begin{theorem}[Absorption Laws]
  \label{absorption laws}{\index{absorption laws}}If $A, B$ are classes then
  \begin{enumerate}
    \item $A \bigcup \left( A \bigcap B \right) = A$
    
    \item $A \bigcap \left( A \bigcup B \right) = A$
  \end{enumerate}
  \begin{proof}
    
    \begin{enumerate}
      \item By \ref{relation classes and intersection and union} we have $A
      \bigcap B \subseteq A$ and thus by the previous theorem we have $A
      \bigcup \left( A \bigcap B \right) = A$
      
      \item By \ref{relation classes and intersection and union} we have $A
      \subseteq A \bigcup B$ and thus by the previous theorem we have $A
      \bigcap \left( A \bigcup B \right) = A$
    \end{enumerate}
  \end{proof}
\end{theorem}

\begin{theorem}
  For every class $A$ we have $(A^c)^c = A$
\end{theorem}

\begin{proof}
  If $x \in A \Rightarrow x \nin A^c \Rightarrow x \in (A^c)^c \Rightarrow A
  \subseteq (A^c)^c$. If $x \in (A^c)^c \Rightarrow x \nin A^c \Rightarrow x
  \in A \Rightarrow (A^c)^c \subseteq A$ (if $x \nin A \Rightarrow x \in A^c$
  contradicting $x \nin A^c$)
\end{proof}

\begin{theorem}[DeMorgan's Law]
  \label{DeMorgan's Laws}{\index{DeMorgan's Laws}}For all classes $A, B$ we
  have
  \begin{enumerate}
    \item $\left( A \bigcap B \right)^c = A^c \bigcup B^c$
    
    \item $\left( A \bigcup B \right)^c = A^c \bigcap B^c$
  \end{enumerate}
\end{theorem}

\begin{proof}
  
  \begin{enumerate}
    \item First
    \begin{eqnarray*}
      x \in \left( A \bigcap B \right)^c & \Rightarrow & x \nin \left( A
      \bigcap B \right)\\
      & \Rightarrow & \neg (x \in A \wedge x \in B)\\
      & \Rightarrow & x \nin A \vee x \nin B\\
      & \Rightarrow & x \in A^c \vee x \in B^c\\
      & \Rightarrow & x \in A^c \bigcup B^c
    \end{eqnarray*}
    Second
    \begin{eqnarray*}
      x \in A^c \bigcup B^c & \Rightarrow & x \in A^c \vee x \in B^c\\
      & \Rightarrow & x \nin A \vee x \nin B\\
      & \Rightarrow & \neg (x \in A \wedge x \in B)\\
      & \Rightarrow & \neg \left( x \in A \bigcap B \right)\\
      & \Rightarrow & x \in \left( A \bigcap B \right)^c
    \end{eqnarray*}
    \item First
    \begin{eqnarray*}
      x \in \left( A \bigcup B \right)^c & \Rightarrow & x \nin \left( A
      \bigcup B \right)\\
      & \Rightarrow & \neg \left( x \in A \bigcup B \right)\\
      & \Rightarrow & \neg (x \in A \vee x \in B)\\
      & \Rightarrow & x \nin A \wedge x \nin B\\
      & \Rightarrow & x \in A^c \wedge x \in B^c\\
      & \Rightarrow & x \in A^c \bigcap B^c
    \end{eqnarray*}
    Second
    \begin{eqnarray*}
      x \in A^c \bigcap B^c & \Rightarrow & x \in A^c \wedge x \in B^c\\
      & \Rightarrow & x \nin A \wedge x \nin B\\
      & \Rightarrow & \neg (x \in A \vee x \in B)\\
      & \Rightarrow & \neg \left( x \in A \bigcup B \right)\\
      & \Rightarrow & x \nin \left( A \bigcup B \right)\\
      & \Rightarrow & x \in \left( A \bigcup B \right)^c
    \end{eqnarray*}
  \end{enumerate}
\end{proof}

\begin{theorem}
  \label{commutative laws of union and intersection}If $A, B$ are classes then
  \begin{enumerate}
    \item $A \bigcup B = B \bigcup A$
    
    \item $A \bigcap B = B \bigcap A$
  \end{enumerate}
  \begin{proof}
    
    \begin{enumerate}
      \item We have
      \begin{eqnarray*}
        x \in A \bigcup B & \Leftrightarrow & x \in A \vee x \in B\\
        & \Leftrightarrow & x \in B \vee x \in A\\
        & \Leftrightarrow & x \in B \bigcap A
      \end{eqnarray*}
      \item We have
      \begin{eqnarray*}
        x \in A \bigcap B & \Leftrightarrow & x \in A \wedge x \in B\\
        & \Leftrightarrow & x \in B \wedge x \in A\\
        & \Leftrightarrow & x \in B \bigcap A
      \end{eqnarray*}
    \end{enumerate}
  \end{proof}
  
  \begin{theorem}
    \label{idempotent laws}For any class $A$ we have
    \begin{enumerate}
      \item $A \bigcup A = A$
      
      \item $A \bigcap A = A$
    \end{enumerate}
  \end{theorem}
  
  \begin{proof}
    
    \begin{enumerate}
      \item We have
      \begin{eqnarray*}
        x \in A \bigcup A & \Leftrightarrow & x \in A \vee x \in A\\
        & \Leftrightarrow & x \in A
      \end{eqnarray*}
      \item We have
      \begin{eqnarray*}
        x \in A \bigcap A & \Leftrightarrow & x \in A \wedge x \in A\\
        & \Leftrightarrow & x \in A
      \end{eqnarray*}
    \end{enumerate}
  \end{proof}
  
  \begin{theorem}
    \label{associative laws of union and intersection}For any class $A, B, C$
    we have
    \begin{enumerate}
      \item $A \bigcup \left( B \bigcup C \right) = \left( A \bigcup B \right)
      \bigcup C$
      
      \item $A \bigcap \left( B \bigcap C \right) = \left( A \bigcap B \right)
      \bigcap C$
    \end{enumerate}
  \end{theorem}
  
  \begin{proof}
    
    \begin{enumerate}
      \item We have
      \begin{eqnarray*}
        x \in A \bigcup \left( B \bigcup C \right) & \Leftrightarrow & x \in A
        \vee x \in \left( B \bigcup C \right)\\
        & \Leftrightarrow & x \in A \vee (x \in B \vee x \in C)\\
        & \Leftrightarrow & (x \in A \vee x \in B) \vee x \in C\\
        & \Leftrightarrow & x \in A \bigcup B \vee x \in C\\
        & \Leftrightarrow & x \in \left( A \bigcup B \right) \bigcup C
      \end{eqnarray*}
      \item We have
      \begin{eqnarray*}
        x \in A \bigcap \left( B \bigcap C \right) & \Leftrightarrow & x \in A
        \wedge x \in B \bigcap C\\
        & \Leftrightarrow & x \in A \wedge (x \in B \wedge x \in C)\\
        & \Leftrightarrow & (x \in A \wedge x \in B) \wedge x \in C\\
        & \Leftrightarrow & x \in A \bigcap B \wedge x \in C\\
        & \Leftrightarrow & x \in \left( A \bigcap B \right) \bigcap C
      \end{eqnarray*}
    \end{enumerate}
  \end{proof}
\end{theorem}

\begin{theorem}
  \label{distributivity of union or intersection}If $A, B, C$ are classes then
  we have
  \begin{enumerate}
    \item $A \bigcap \left( B \bigcup C \right) = \left( A \bigcap B \right)
    \bigcup \left( A \bigcap C \right)$
    
    \item $A \bigcup \left( B \bigcap C \right) = \left( A \bigcup B \right)
    \bigcap \left( A \bigcup C \right)$
    
    \item $A\backslash \left( B \bigcup C \right) = (A\backslash B) \bigcap
    (A\backslash C)$
    
    \item $A\backslash \left( B \bigcap C \right) = (A\backslash B) \bigcup
    (A\backslash C)$
    
    \item If $B \subseteq A$ then $A\backslash (A\backslash B) = B$
  \end{enumerate}
  \begin{proof}
    
    \begin{enumerate}
      \item We have
      \begin{eqnarray*}
        x \in A \bigcap \left( B \bigcup C \right) & \Leftrightarrow & x \in A
        \wedge x \in B \bigcup C\\
        & \Leftrightarrow & x \in A \wedge (x \in B \vee x \in C)\\
        & \Leftrightarrow & (x \in A \wedge x \in B) \vee (x \in A \wedge x
        \in C)\\
        & \Leftrightarrow & x \in A \bigcap B \vee x \in A \bigcap C\\
        & \Leftrightarrow & x \in \left( A \bigcap B \right) \bigcup \left( A
        \bigcap C \right)
      \end{eqnarray*}
      \item We have
      \begin{eqnarray*}
        x \in A \bigcup \left( B \bigcap C \right) & \Leftrightarrow & x \in A
        \vee x \in \left( B \bigcap C \right)\\
        & \Leftrightarrow & x \in A \vee (x \in B \wedge x \in C)\\
        & \Leftrightarrow & (x \in A \vee x \in B) \wedge (x \in A \vee x \in
        C)\\
        & \Leftrightarrow & x \in A \bigcup B \wedge x \in A \bigcup C\\
        & \Leftrightarrow & x \in \left( A \bigcup B \right) \bigcap \left( A
        \bigcup C \right)
      \end{eqnarray*}
      \item We have
      \begin{eqnarray*}
        A\backslash \left( B \bigcup C \right) & \equallim_{\tmop{definition}
        \tmop{of} \backslash} & A \bigcap \left( B \bigcup C \right)^c\\
        & \equallim_{\text{\ref{DeMorgan's Laws}}} & A \bigcap \left( B^c
        \bigcap C^c \right)\\
        & = & \left( A \bigcap A \right) \bigcap \left( B^c \bigcap C^c
        \right)\\
        & \equallim_{\text{\ref{associative laws of union and intersection}}}
        & \left( A \bigcap \left( B^c \bigcap C^c \right) \right) \bigcap A\\
        & \equallim_{\text{\ref{associative laws of union and intersection}}}
        & \left( \left( A \bigcap B^c \right) \bigcap C^c \right) \bigcap A\\
        & \equallim_{\text{\ref{associative laws of union and intersection}}}
        & \left( A \bigcap B^c \right) \bigcap \left( C^c \bigcap A \right)\\
        & \equallim_{\text{\ref{commutative laws of union and intersection}}}
        & \left( A \bigcap B^c \right) \bigcap \left( A \bigcap C^c \right)\\
        & = & (A\backslash B) \bigcap (A\backslash C)
      \end{eqnarray*}
      \item We have
      \begin{eqnarray*}
        A\backslash \left( B \bigcap C \right) & = & A \bigcap \left( B
        \bigcap C \right)^c\\
        & \equallim_{\text{\ref{DeMorgan's Laws}}} & A \bigcap \left( B^c
        \bigcup C^c \right)\\
        & \equallim_{(1)} & \left( A \bigcap B^c \right) \bigcup \left( A
        \bigcap C^c \right)\\
        & = & (A\backslash B) \bigcup (A\backslash C)
      \end{eqnarray*}
      \item We have if $B \subseteq A$
      \begin{eqnarray*}
        A\backslash (A\backslash B) & = & A \bigcap (A\backslash B)^c\\
        & = & A \bigcap \left( A \bigcap B^c \right)^c\\
        & = & A \bigcap \left( A^c \bigcup (B^c)^c \right)\\
        & = & A \bigcap \left( A^c \bigcup B \right)\\
        & = & \left( A \bigcap A^c \right) \bigcup \left( A \bigcap B
        \right)\\
        & = & \emptyset \bigcup \left( A \bigcap B \right)\\
        & = & A \bigcap B\\
        & \equallim_{B \subseteq A} & B
      \end{eqnarray*}
    \end{enumerate}
  \end{proof}
\end{theorem}

\begin{theorem}
  \label{properties of empty set and universal class}For every class we have
  \begin{enumerate}
    \item $A \bigcup \emptyset = A$
    
    \item $A \bigcap \emptyset = \emptyset$
    
    \item $A \bigcup \mathcal{U}=\mathcal{U}$
    
    \item $A \bigcap \mathcal{U}= A$
    
    \item $\mathcal{U}^c = \emptyset$
    
    \item $\emptyset^c =\mathcal{U}$
    
    \item $A \bigcup A^c =\mathcal{U}$
    
    \item $A \bigcap A^c = \emptyset$
  \end{enumerate}
\end{theorem}

\begin{proof}
  
  \begin{enumerate}
    \item We have
    \begin{eqnarray*}
      x \in A \bigcup \emptyset & \Leftrightarrow & x \in A \vee x \in
      \emptyset\\
      & \Leftrightarrowlim_{x \in \emptyset \tmop{is} \tmop{false}} & x \in A
    \end{eqnarray*}
    \item We have
    \begin{eqnarray*}
      x \in A \bigcap \emptyset & \Rightarrow & x \in A \wedge x \in
      \emptyset\\
      & \Rightarrow & x \in \emptyset\\
      x \in \emptyset & \Rightarrowlim_{x \in \emptyset \tmop{is}
      \tmop{false}} & x \in A \wedge x \in \emptyset\\
      & \Rightarrow & x \in A \bigcap \emptyset
    \end{eqnarray*}
    \item We have
    \begin{eqnarray*}
      x \in A \bigcup \mathcal{U} & \Leftrightarrow & x \in A \vee x \in
      \mathcal{U}\\
      & \Leftrightarrow & x \in A \vee x = x\\
      & \Leftrightarrow & x = x
    \end{eqnarray*}
    \item We have
    \begin{eqnarray*}
      x \in A \bigcap \emptyset & \Rightarrow & x \in A \wedge x \neq x\\
      & \Rightarrow & x \neq x\\
      & \Rightarrow & x \in \emptyset\\
      x \in \emptyset & \Rightarrow & x \in A (\tmop{vacuously})\\
      & \Rightarrow & x \in A \bigcap \emptyset
    \end{eqnarray*}
    \item We have
    \begin{eqnarray*}
      \emptyset & \subseteq & \mathcal{U}^c\\
      x \in \mathcal{U}^c & \Rightarrow & x \in \mathcal{U}^c\\
      & \Rightarrow & x \nin \mathcal{U}\\
      & \Rightarrow & x \neq x\\
      & \Rightarrow & x \in \emptyset
    \end{eqnarray*}
    so $\emptyset =\mathcal{U}^c$
    
    \item We have
    \begin{eqnarray*}
      \emptyset^c & \subseteq & \mathcal{U}\\
      x \in \mathcal{U} & \Rightarrow & x = x\\
      & \Rightarrow & \neg (x \neq x)\\
      & \Rightarrow & x \nin \emptyset\\
      & \Rightarrow & x \in \emptyset^c
    \end{eqnarray*}
    \item We have
    \begin{eqnarray*}
      x \in A \bigcup A^c & \Leftrightarrow & x \in A \vee x \in A^c\\
      & \Leftrightarrow & x \in A \vee x \nin A (\tmop{always} \tmop{true})\\
      & \Leftrightarrow & x = x (\tmop{always} \tmop{true})\\
      & \Leftrightarrow & x \in \mathcal{U}
    \end{eqnarray*}
    \item We have
    \begin{eqnarray*}
      x \in A \bigcap A^c & \Leftrightarrow & x \in A \wedge x \in A^c\\
      & \Leftrightarrow & x \in A \wedge x \nin A (\tmop{always}
      \tmop{false})\\
      & \Leftrightarrow & x \neq x (\tmop{always} \tmop{false})\\
      & \Leftrightarrow & x \in \emptyset
    \end{eqnarray*}
  \end{enumerate}
\end{proof}

\begin{theorem}
  \label{union and intersection of disjoint classes}Let $A, B, C$ be classes
  such that $A \bigcap B = \emptyset$, $C = A \bigcup B$ then $A = C\backslash
  B$ and $B = C\backslash A$
\end{theorem}

\begin{proof}
  
  \begin{enumerate}
    \item If $x \in A \subseteq A \bigcup B = C \Rightarrow x \in C$ and if $x
    \in B$ we would have $x \in A \bigcap B = \emptyset$ a contradiction so $x
    \nin B \Rightarrow x \in C\backslash B \Rightarrow A \subseteq C\backslash
    B$
    
    \item If $x \in C\backslash B$ then $x \in C = A \bigcup B$ so $x \in A
    \vee x \in B$ as $x \nin B$ we must have $x \in A \Rightarrow C\backslash
    B \subseteq A$
    
    \item If $x \in B \subseteq A \bigcup B = C \Rightarrow x \in C$ and if $x
    \in A$ we would have $x \in A \bigcap B = \emptyset$ a contradiction so $x
    \nin A \Rightarrow x \in C\backslash A \Rightarrow B \subseteq C\backslash
    A$
    
    \item If $x \in C\backslash A$ then $x \in C = A \bigcup B$ so $x \in A
    \vee x \in B$ as $x \nin A$ we must have $x \in B \Rightarrow C\backslash
    A \subseteq B$
  \end{enumerate}
\end{proof}

\section{Cartesian Products}

If $a$ is a element we can use the axiom of class construction \ref{axiom of
construction} to define the class $\{ x | x = a \nobracket \}$, this leads to
the following definition of a singleton.

\begin{definition}
  If $a$ is a element then $\{ a \} = \{ x | x = a \nobracket \}$ is a class
  containing only one element. The class $\{ a \}$ is called a singleton.
\end{definition}

If $a, b$ are elements then we can define the class $\{ x | x = a \vee x = b
\nobracket \}$ consisting of two elements. This leads to the following
definition.

\begin{definition}
  If $a, b$ are elements then $\{ a, b \} = \{ x | x = a \vee x = b \nobracket
  \}$ is called a unordered pair.
\end{definition}

We want to form classes in which $\{ a, b \}$ are members, to be able to this,
we assume the following axiom.

\begin{axiom}[Axiom of Pairing]
  \label{unordered pair is a element}{\index{axiom of pairing}}If $a, b$ are
  elements then $\{ a, b \}$ is a element
\end{axiom}

\begin{lemma}
  If $a$ is a element then $\{ a, a \} = \{ a \}$ 
\end{lemma}

\begin{proof}
  
  \begin{eqnarray*}
    x \in \{ a, a \} & \Leftrightarrow & x = a \vee x = a\\
    & \Leftrightarrow & x = a\\
    & \Leftrightarrow & x \in \{ a \}
  \end{eqnarray*}
\end{proof}

\begin{theorem}
  If $a$ is a element then $\{ a \}$ is a element
\end{theorem}

\begin{proof}
  As $a$ is a element we have that $a, a$ are elements and by the previous
  axiom we have that $\{ a, a \} = \{ a \}$ is a element.
\end{proof}

\begin{theorem}
  \label{equality of unordered pairs}If $x, y, x', y'$ are elements then
  from$\{ x, y \} = \{ x', y' \}$ we have $(x = x' \wedge y = y') \vee (x = y'
  \wedge y = x')$
\end{theorem}

\begin{proof}
  Lets's consider the following possible cases
  \begin{enumerate}
    \item $x = y$ then using the previous lemma we have $\{ x, y \} = \{ x \}
    = \{ x', y' \}$ then from $x' \in \{ x, y \}$ we have $x' \in \{ x \}
    \Rightarrow x = x'$ and from $y' \in \{ x, y \}$ we have $y' = x = y$ so
    we have that $(x = x' \wedge y = y') \Rightarrow (x = x' \wedge y = y')
    \vee (x = y' \wedge y = x')$ \
    
    \item $x \neq y$ then we have from $\{ x, y \} = \{ x', y' \}$ the
    following cases for $x$
    \begin{enumerate}
      \item $x = x'$ for $y$ we have then the following cases
      \begin{enumerate}
        \item $y = x'$ but then $x = y$ which we have excluded
        
        \item $y = y'$
      \end{enumerate}
      So we conclude that $x = x' \wedge y = y' \Rightarrow (x = x' \wedge y =
      y') \vee (x = y' \wedge y = x')$
      
      \item $x = y'$ for $y$ we have then following cases
      \begin{enumerate}
        \item $y = y'$ but then $x = y$ which we have excluded
        
        \item $y = x'$ so we have then $x = y' \wedge y = x' \Rightarrow (x =
        x' \wedge y = y') \vee (x = y' \wedge y = x')$
      \end{enumerate}
    \end{enumerate}
    So we have always the following possible cases $(x = x' \wedge y = y')
    \vee (x = y' \wedge y = x')$
  \end{enumerate}
\end{proof}

We can now define a ordered pair of elements

\begin{definition}
  \label{pair}If $a, b$ are elements then $(a, b) = \{ \{ a \}, \{ a, b \} \}$
  (which is a element as $\{ a, b \}$ and $\{ a \}$ are elements)
\end{definition}

\begin{theorem}
  If $(a, b) = (c, d)$ then $a = c$ and $b = d$
\end{theorem}

\begin{proof}
  If $(a, b) = (c, d)$ then we have by definition $\{ \{ a \}, \{ a, b \} \} =
  \{ \{ c \}, \{ c, d \} \}$. Using \ref{equality of unordered pairs} we have
  then the following possible cases
  \begin{enumerate}
    \item $\{ a \} = \{ c \} \wedge \{ a, b \} = \{ c, d \}$ from this we
    conclude that $a = c$ and using \ref{equality of unordered pairs} again we
    have either
    \begin{enumerate}
      \item $a = c \wedge b = d$ giving the outcome of our theorem
      
      \item $a = d \wedge b = c \Rightarrowlim_{a = c} a = c = d \wedge b = d
      \Rightarrow a = c \wedge b = d$
    \end{enumerate}
    \item $\{ a \} = \{ c, d \} \wedge \{ a, b \} = \{ c \}$ then $a = c = d$
    and $a = b = c$ and thus $a = c \wedge b = d$
  \end{enumerate}
\end{proof}

\begin{definition}
  \label{cartesian product}{\index{$A \times B$}}If $A, B$ are classes then
  the class $A \times B = \{ z | z = (a, b) \tmop{where} a \in A \wedge b \in
  B \} \nobracket$ is noted as $A \times B = \{ (x, y) | x \in A \wedge y \in
  B \nobracket \}$ is the Cartesian product of $A$ and $B$ (note that $A
  \times B$ is a class as $x \in A \wedge y \in B$ means $x, y$ are elements
  so that $(x, y)$ is a element)
\end{definition}

The following is a consequence of this definition

\begin{lemma}
  \label{product of subclasses}If $A, B, C, D$ are classes with $A \subseteq
  B$ and $C \subseteq D$ then $A \times C \subseteq B \times D$
\end{lemma}

\begin{proof}
  
  \begin{eqnarray*}
    z \in A \times C & \Rightarrow & z = (x, y) \wedge x \in A \wedge x \in
    C\\
    & \Rightarrow & z = (x, y) \wedge x \in B \wedge x \in D\\
    & \Rightarrow & z \in B \times D
  \end{eqnarray*}
  Proving that $A \times C \subseteq B \times D$
\end{proof}

We have also

\begin{lemma}
  \label{product of sets properties}For all classes $A, B$ and $C$ we have
  \begin{enumerate}
    \item $A \times \left( B \bigcap C \right) = (A \times B) \bigcap (A
    \times C)$
    
    \item $A \times \left( B \bigcup C \right) = (A \times B) \bigcup (A
    \times C)$
    
    \item $(A \times B) \bigcap (C \times D) = \left( A \bigcap C \right)
    \times \left( B \bigcap C \right)$
    
    \item $\left( B \bigcap C \right) \times A = (B \times A) \bigcap (C
    \times A)$
    
    \item $\left( B \bigcup C \right) \times A = (B \times A) \bigcup (C
    \times A)$
    
    \item $\left( A \bigcap C \right) \times \left( B \bigcap D \right) = (A
    \times B) \bigcap (C \times D)$
    
    \item $(A \times B) \backslash (C \times D) = ((A\backslash C) \times B)
    \bigcup (A \times (B\backslash D))$
  \end{enumerate}
  \begin{proof}
    
    \begin{enumerate}
      \item We have
      \begin{eqnarray*}
        z \in A \times \left( B \bigcap C \right) & \Leftrightarrow & z = (x,
        y) \wedge x \in A \wedge y \in \left( B \bigcap C \right)\\
        & \Leftrightarrow & z = (x, y) \wedge x \in A \wedge (y \in B \wedge
        y \in C)\\
        & \Leftrightarrow & (z = (x, y) \wedge x \in A \wedge y \in B) \wedge
        (z = (x, y) \wedge x \in A \wedge y \in C)\\
        & \Leftrightarrow & z \in A \times B \wedge z \in A \times C\\
        & \Leftrightarrow & z \in (A \times B) \bigcap (A \times C)
      \end{eqnarray*}
      \item We have
      \begin{eqnarray*}
        z \in A \times \left( B \bigcup C \right) & \Leftrightarrow & z = (x,
        y) \wedge x \in A \wedge y \in \left( B \bigcup C \right)\\
        & \Leftrightarrow & z = (x, y) \wedge x \in A \wedge (y \in B \vee y
        \in C)\\
        & \Leftrightarrow & (z = (x, y) \wedge x \in A \wedge y \in B) \vee
        (z = (x, y) \wedge x \in A \wedge y \in C)\\
        & \Leftrightarrow & z \in A \times B \vee z \in A \times C\\
        & \Leftrightarrow & z \in (A \times B) \bigcup (A \times C)
      \end{eqnarray*}
      \item We have
      \begin{eqnarray*}
        z \in (A \times B) \bigcap (C \times D) & \Leftrightarrow & z \in A
        \times B \wedge z \in C \times D\\
        & \Leftrightarrow & (z = (x, y) \wedge x \in A \wedge y \in B) \wedge
        (z = (x', y') \wedge x' \in C \wedge y' \in D)\\
        & \Leftrightarrowlim_{(x, y) = z = (x', y') \Rightarrow x = x', y =
        y'} & z = (x, y) \wedge x \in A \wedge y \in B \wedge x \in C \wedge y
        \in D\\
        & \Leftrightarrow & z = (x, y) \wedge (x \in A \wedge x \in C) \wedge
        (y \in B \wedge y \in D)\\
        & \Leftrightarrow & z = (x, y) \wedge \left( x \in A \bigcap C
        \right) \wedge \left( y \in B \bigcap D \right)\\
        & \Leftrightarrow & z \in \left( A \bigcap C \right) \times \left( B
        \bigcap D \right)
      \end{eqnarray*}
      \item We have
      \begin{eqnarray*}
        z \in \left( B \bigcap C \right) \times A & \Leftrightarrow & z = (x,
        y) \wedge x \in B \bigcap C \wedge y \in A\\
        & \Leftrightarrow & z = (x, y) \wedge x \in B \wedge x \in C \wedge y
        \in A\\
        & \Leftrightarrow & (z = (x, y) \wedge x \in B \wedge y \in A) \wedge
        (z = (x, y) \wedge x \in C \wedge y \in A)\\
        & \Leftrightarrow & z \in B \times A \wedge z \in C \times A\\
        & \Leftrightarrow & z \in (B \times A) \bigcap (C \times A)
      \end{eqnarray*}
      \item We have
      \begin{eqnarray*}
        z \in \left( B \bigcup C \right) \times A & \Leftrightarrow & z = (x,
        y) \wedge x \in B \bigcup C \wedge y \in A\\
        & \Leftrightarrow & z = (x, y) \wedge (x \in B \vee x \in C) \wedge y
        \in A\\
        & \Leftrightarrow & (z = (x, y) \wedge x \in B \wedge y \in A) \vee
        (z = (x, y) \wedge x \in C \wedge y \in A)\\
        & \Leftrightarrow & (z \in B \times A) \vee (z \in C \times A)\\
        & \Leftrightarrow & z \in (B \times A) \bigcup (C \times A)
      \end{eqnarray*}
      \item We have
      \begin{eqnarray*}
        z \in \left( A \bigcap C \right) \times \left( B \bigcap D \right) &
        \Leftrightarrow & z = (x, y) \wedge x \in A \bigcap C \wedge y \in B
        \bigcap D\\
        & \Leftrightarrow & z = (x, y) \wedge x \in A \wedge x \in C \wedge y
        \in B \wedge y \in D\\
        & \Leftrightarrow & z = (x, y) \wedge x \in A \wedge y \in B \wedge z
        = (x, y) \wedge x \in C \wedge y \in D\\
        & \Leftrightarrow & z \in A \times B \wedge z \in C \times D
      \end{eqnarray*}
      \item We have
      \begin{eqnarray*}
        z \in (A \times B) \backslash (C \times D) & \Leftrightarrow & z \in A
        \times B \wedge z \nin C \times D\\
        & \Leftrightarrow & z = (x, y) \wedge (x, y) \in A \times B \wedge
        (x, y) \nin C \times D\\
        & \Leftrightarrow & z = (x, y) \wedge x \in A \wedge y \in B \wedge
        \neg (x \in C \wedge y \in D)\\
        & \Leftrightarrow & z = (x, y) \wedge x \in A \wedge y \in B \wedge
        (x \nin C \vee y \nin D)\\
        & \Leftrightarrow & (z = (x, y) \wedge x \in A \wedge y \in B \wedge
        x \nin C) \vee (z = (x, y) \wedge x \in A \wedge y \in B \wedge y \nin
        D)\\
        & \Leftrightarrow & (z \in (A\backslash C) \times B) \vee (z \in A
        \times (B\backslash D))\\
        & \Leftrightarrow & z \in ((A\backslash C) \times B) \bigcup (A
        \times (B\backslash D))
      \end{eqnarray*}
    \end{enumerate}
    
  \end{proof}
\end{lemma}

\begin{notation}
  From now on we note the class $\{ z | z \nobracket = (x, y) \wedge P (x, y)
  \}$ as $\{ (x, y) | P (x, y) \nobracket \}$. This makes sense for if we want
  to prove that a class $A$ of tupple's is equal to a class $B$ of tupple's
  then we must prove that $z \in A = \{ (x, y) | P (x, y) \nobracket \}
  \Leftrightarrow z \in B = \{ (x, y) | Q (x, y) \nobracket \}$ but this is
  equivalent to $z = (x, y) \wedge P (x, y) \Leftrightarrow z = (x', y')
  \wedge Q (x', y')$ which because of the fact that $(x, y) = z = (x', y')$
  implies that $x = x' \wedge y = y'$ is equivalent with $P (x, y) = Q (x,
  y)$.
\end{notation}

\section{Graphs}

\begin{definition}
  A graph is a subclass of $\mathcal{U} \times \mathcal{U}$ or in other words
  a graph is a class of ordered pairs
\end{definition}

\begin{example}
  \label{product of sets is a graph}If $A, B$ are classes then $A \times B$ is
  a class and by \ref{product of subclasses} and $A, B \subseteq \mathcal{U}$
  we have $A \times B \subseteq \mathcal{U} \times \mathcal{U}$ and thus $A
  \times B$ is a graph.
\end{example}

\begin{lemma}
  If $G, H$ are graphs then $G \bigcup H$ is a graph
\end{lemma}

\begin{proof}
  As $G, H$ are graphs we have $G, H$ are classes and $G \subseteq \mathcal{U}
  \times \mathcal{U} \wedge H \subseteq \mathcal{U} \times \mathcal{U}
  \Rightarrow G \bigcup H \subseteq \mathcal{U} \times \mathcal{U}$ proving
  that the class $G \bigcup H$ is a graph
\end{proof}

\begin{definition}
  If $G$ is a graph then $G^{- 1} = \{ (x, y) | (y, x) \in G \nobracket \}$
\end{definition}

\begin{definition}
  \label{composition of graphs}If $G, H$ are graphs then $G \circ H = \{ (x,
  y) | \exists z \vdash (x, z) \in H \wedge (z, x) \in G \} \subseteq
  \mathcal{U} \times \mathcal{U}$ is a graph, the composition of two graphs.
\end{definition}

\begin{theorem}
  \label{properties of composition of mappings}If $G, H, J$ are graphs then we
  have
  \begin{enumerate}
    \item $(G \circ H) \circ J = G \circ (H \circ J)$
    
    \item $(G^{- 1})^{- 1} = G$
    
    \item $(G \circ H)^{- 1} = H^{- 1} \circ G^{- 1}$
  \end{enumerate}
\end{theorem}

\begin{proof}
  
  \begin{enumerate}
    \item We have
    \begin{eqnarray*}
      (x, y) \in (G \circ H) \circ J & \Leftrightarrow & \exists z \vdash (x,
      z) \in J \wedge (z, y) \in (G \circ H)\\
      & \Leftrightarrow & \exists z, \exists z' \vdash (x, z) \in J \wedge
      (z, z') \in H \wedge (z', y) \in G\\
      & \Leftrightarrow & \exists z' \vdash (x, z') \in H \circ J \wedge (x'
      y) \in G\\
      & \Leftrightarrow & (x, y) \in (H \circ J) \circ G
    \end{eqnarray*}
    \item We have
    \begin{eqnarray*}
      (x, y) \in (G^{- 1})^{- 1} & \Leftrightarrow & (y, x) \in G^{- 1}\\
      & \Leftrightarrow & (x, y) \in G
    \end{eqnarray*}
    \item we have
    \begin{eqnarray*}
      (x, y) \in (G \circ H)^{- 1} & \Leftrightarrow & (y, x) \in G \circ H\\
      & \Leftrightarrow & \exists z \vdash (y, z) \in H \wedge (z, x) \in G\\
      & \Leftrightarrow & \exists z \vdash (z, y) \in H^{- 1} \wedge (x, z)
      \in G^{- 1}\\
      & \Leftrightarrow & (x, y) \in H^{- 1} \circ G^{- 1}
    \end{eqnarray*}
  \end{enumerate}
\end{proof}

\begin{definition}
  {\index{domain}}If $G$ is a graph then $\tmop{dom} (G)$ is the class defined
  by $\tmop{dom} (G) = \{ x | \exists y \nobracket \vdash (x, y) \in G \}$
\end{definition}

\begin{definition}
  {\index{range}}If $G$ is a graph then $\tmop{range} (G)$ is the class
  defined by $\tmop{range} (G) = \{ y | \exists x \nobracket \vdash (x, y) \in
  G \}$
\end{definition}

\begin{theorem}
  \label{properties of domains and ranges of graphs}If $G$ and $H$ are graphs
  then
  \begin{enumerate}
    \item $\tmop{dom} (G) = \tmop{range} (G^{- 1})$
    
    \item $\tmop{dom} (G \circ H) \subseteq \tmop{dom} (H)$
    
    \item $\tmop{range} (G) = \tmop{dom} (G^{- 1})$
    
    \item $\tmop{range} (G \circ H) \subseteq \tmop{range} (G)$
  \end{enumerate}
\end{theorem}

\begin{proof}
  
  \begin{enumerate}
    \item This is proved by
    \begin{eqnarray*}
      x \in \tmop{dom} (G) & \Leftrightarrow & \exists y \vdash (x, y) \in G\\
      & \Leftrightarrow & \exists y \vdash (y, x) \in G^{- 1}\\
      & \Leftrightarrow & x \in \tmop{range} (G^{- 1})
    \end{eqnarray*}
    \item This is proved by
    \begin{eqnarray*}
      x \in \tmop{dom} (G \circ H) & \Rightarrow & \exists y \vdash (x, y) \in
      G \circ H\\
      & \Rightarrow & \exists z \vdash (x, z) \in H \wedge (z, y) \in G\\
      & \Rightarrow & x \in \tmop{dom} (H)
    \end{eqnarray*}
    \item This is proved by
    \begin{eqnarray*}
      x \in \tmop{range} (G) & \Leftrightarrow & \exists y \vdash (y, x) \in
      G\\
      & \Leftrightarrow & \exists y \vdash (x, y) \in G^{- 1}\\
      & \Leftrightarrow & x \in \tmop{dom} (G^{- 1})
    \end{eqnarray*}
    \item This is proved by
    \begin{eqnarray*}
      x \in \tmop{range} (G \circ H) & \Rightarrow & \exists y \vdash (y, x)
      \in G \circ H\\
      & \Rightarrow & \exists z \vdash (y, z) \in H \wedge (z, x) \in G\\
      & \Rightarrow & x \in \tmop{range} (G)
    \end{eqnarray*}
  \end{enumerate}
\end{proof}

\begin{theorem}
  \label{domain property of composition}If $G, H$ are graphs with
  $\tmop{range} (H) \subseteq \tmop{dom} (G)$ then $\tmop{dom} (G \circ H) =
  \tmop{dom} (H)$
\end{theorem}

\begin{proof}
  This is proved by
  \begin{eqnarray*}
    \tmop{dom} (G \circ H) & \subseteq & \tmop{dom} (H)  (\tmop{previous}
    \tmop{theorem})\\
    x \in \tmop{dom} (H) & \Rightarrow & \exists y \vdash (x, y) \in H\\
    & \Rightarrow & y \in \tmop{range} (H)\\
    & \Rightarrow & y \in \tmop{dom} (G)\\
    & \Rightarrow & \exists z \vdash (y, z) \in G\\
    & \Rightarrow & (x, z) \in G \circ H\\
    & \Rightarrow & x \in \tmop{dom} (G \circ H)\\
    & \Rightarrow & \tmop{dom} (H) \subseteq \tmop{dom} (G \circ H)
  \end{eqnarray*}
  So we have proved that $\tmop{dom} (G \circ H) = \tmop{dom} (H)$
\end{proof}

\begin{note}
  Instead in the following we use the shorthand notation
  \begin{eqnarray*}
    \forall x \in A \vDash P (i) & \tmop{to} \tmop{mean} & \forall x \vDash x
    \in A \Rightarrow P (i)\\
    \exists x \in A \vdash P (i) & \tmop{to} \tmop{mean} & \exists x \vdash x
    \in A \wedge P (i)
  \end{eqnarray*}
\end{note}

\begin{theorem}
  \label{union of two graphs}If $G$ and $H$ are graphs then
  \begin{enumerate}
    \item $\tmop{dom} \left( G \bigcup H \right) = \tmop{dom} (G) \bigcup
    \tmop{dom} (H)$
    
    \item $\tmop{range} \left( G \bigcup H \right) = \tmop{range} (G) \bigcup
    \tmop{range} (H)$
  \end{enumerate}
\end{theorem}

\begin{proof}
  
  \begin{enumerate}
    \item 
    \begin{eqnarray*}
      x \in \tmop{dom} \left( G \bigcup H \right) & \Leftrightarrow & \exists
      y \vdash (x, y) \in G \bigcup H\\
      & \Leftrightarrow & \exists y \vdash (x, y) \in G \vee (x, y) \in H\\
      & \Leftrightarrow & (\exists y \vdash (x, y) \in G) \vee (\exists z
      \vdash (x, z) \in H)\\
      & \Leftrightarrow & x \in \tmop{dom} (G) \bigcup \tmop{dom} (H)
    \end{eqnarray*}
    \item 
    \begin{eqnarray*}
      y \in \tmop{range} \left( G \bigcup H \right) & \Leftrightarrow &
      \exists x \vdash (x, y) \in G \bigcup H\\
      & \Leftrightarrow & \exists x \vdash (x, y) \in G \vee (x, y) \in H\\
      & \Leftrightarrow & (\exists x \vdash (x, y) \in G) \vee (\exists z
      \vdash (z, y) \in H)\\
      & \Leftrightarrow & x \in \tmop{range} (G) \bigcup \tmop{range} (H)
    \end{eqnarray*}
  \end{enumerate}
  
\end{proof}

\begin{definition}[function graph]
  \label{function graph}{\index{function graph}}A graph $G$ is a function
  graph if $\forall (x, y), (x, y') \in G$ we have $y = y'$
\end{definition}

\section{Families}

\begin{definition}
  \label{family of classes}{\index{family of classes}}{\index{$\{ G_i \}_{i
  \in I}$}}A family of classes is a pair $\langle G, I \rangle$ where $G$ is a
  graph $G$ with $\tmop{dom} (G) \subseteq I$. If $\langle G, I \rangle$ is a
  family then we define $\forall i \in I$ that $G_i = \{ x| (i, x) \in G \}$.
  As the $G_i$ are the most important objects of a family we note a family as
  $\{ G_i \}_{i \in I}$ instead of $\langle G, I \rangle$. The graph of a
  family $\{ G_i \}_{i \in I}$ is noted as $\tmop{graph} (\{ G_i \}_{i \in
  I})$ and $I$ is the index of the family, so that we must have $\tmop{dom}
  (\tmop{graph} (\{ G_i \}_{i \in I})) \subseteq I$.
  
  Notice that in this notation a index variable is introduced which is not
  really needed (which is a disadvantage of this kind of notations) so $\{ G_i
  \}_{i \in I} = \{ G_j \}_{j \in I}$.
\end{definition}

\begin{example}
  \label{empty family}If $\{ G_i \}_{i \in I}$ is a family with $I =
  \emptyset$ then $\{ G_i \}_{i \in I} = \langle G, I \rangle$ where $G$ is a
  graph and $\tmop{dom} (G) \subseteq \emptyset \Rightarrow \tmop{dom} (G) =
  \emptyset$. If now $(x, y) \in G$ then $x \in \tmop{dom} (G) = \emptyset$ a
  contradiction so that we must have then $G = \emptyset$ or we have $\{ G_i
  \}_{i \in I} = \langle \emptyset, \emptyset \rangle$.
\end{example}

\begin{example}
  If $G = \{ (1, a), (1, b), (1, c), (2, c), (2, d), (3, a) \}$ then we write
  $G$ as $\{ G_i \}_{i \in I}$ where $I = \{ 1, 2, 3 \}$ and $G_1 = \{ a, b, c
  \}, G_2 = \{ c, d \}$ and $G_3 = \{ a \}$
\end{example}

In the above example we have that $\tmop{dom} (\{ G_i \}_{i \in I}) = I$ which
for general families is not needed. The price of this generalization is that
we can have empty $G_i$'s as the following theorem illustrates.

\begin{theorem}
  If $\{ G_i \}_{i \in I}$ is a family of classes then we have $\tmop{dom}
  (\tmop{graph} (\{ G_i \}_{i \in I})) = I \Leftrightarrow \forall i \in I
  \vDash G_i \neq \emptyset$
\end{theorem}

\begin{proof}[$\Rightarrow$][$\Leftarrow$]
  
  
  If $\tmop{dom} (\{ G_i \}_{i \in I}) = I$ then if $i \in I$ there exists a
  $x$ such that $(i, x) \in G$ and thus $x \in G_i \Rightarrow G_i \neq
  \emptyset$.
  
  If $\forall i \in I \vDash G_i \neq \emptyset$ then if $i \in I$ there
  exists a $x \in G_i \Rightarrow (i, x) \in G \Rightarrow I \subseteq
  \tmop{dom} (\tmop{graph} (\{ G_i \}_{i \in I}))$ and as by definition we
  have $\tmop{dom} (\tmop{graph} (\{ G_i \}_{i \in I})) \subseteq I$ we have
  $\tmop{dom} (\tmop{graph} (\{ G_i \}_{i \in I})) = I$.
\end{proof}

\begin{definition}
  If $\{ G_i \}_{i \in I}$ is a indexed family of classes then $\bigcup_{i \in
  I} G_i = \{ x | \exists i \in I \vdash x \in G_i \nobracket \}$ or written
  using the graph and index we have $\bigcup_{i \in I} G_i = \{ x | \exists i
  \nobracket \vdash i \in I \wedge (i, x) \in \tmop{graph} (\{ G_i \}_{i \in
  I}) \}$
\end{definition}

\begin{definition}
  If $\{ G_i \}_{i \in I}$ is a indexed family of classes then $\bigcap_{i \in
  I} G_i = \{ x | \forall i \in I \vDash x \in G_i \nobracket \}$ or written
  using the graph and index we have $\bigcap_{i \in I} G_i = \{ x | \forall i
  \in I \nobracket \vDash (i, x) \in \tmop{graph} (\{ G_i \}_{i \in I}) \}$
\end{definition}

The following example proves that the normal definition of union and
intersection corresponds with the definition of a union.

\begin{example}
  If $A$ and $B$ are classes define then for $I = \{ 0, 1 \}$ where $0, 1$ are
  different elements (we will see later that they exists) and $G = (\{ 0 \}
  \times A) \bigcup (\{ 1 \} \times B)$ we have then that
  \begin{enumerate}
    \item $G = \{ G_i \}_{i \in \{ 0, 1 \}}$ is a indexed family of classes
    with $G_0 = A$ and $G_1 = B$
    
    \item $\bigcup_{i \in \{ 0, 1 \}} G_i = A \bigcup B_{}$
    
    \item $\bigcap_{i \in \{ 0, 1 \}} G_i = A \bigcap B$
  \end{enumerate}
\end{example}

\begin{proof}
  First we must prove that $\tmop{dom} (G) \subseteq \{ 0, 1 \}$. If $x \in
  \tmop{dom} (G) \Rightarrow \exists y \vdash (x, y) \in G \Rightarrow (x, y)
  \in \{ 0 \} \times A \vee (x, y) \in \{ 1 \} \times B \Rightarrow x \in \{ 0
  \} \vee x \in \{ 1 \} \Rightarrow x = 0 \vee x = 1 \Rightarrow x \in \{ 0, 1
  \} \Rightarrow \tmop{dom} (G) \subseteq \{ 0, 1 \}$. So we have that $G$ is
  a family indexed by $I = \{ 0, 1 \}$. Now we have
  \begin{eqnarray*}
    x \in G_0 & \Leftrightarrow & (0, x) \in (\{ 0 \} \times A) \bigcup (\{ 1
    \} \times B)\\
    & \Leftrightarrow & (0, x) \in \{ 0 \} \times A \vee (0, x) \in \{ 1 \}
    \times B\\
    & \Leftrightarrow & (0, x) \in \{ 0 \} \times A \vee (0 = 1 \wedge x \in
    B)\\
    & \Leftrightarrowlim_{0 \neq 1} & (0, x) \in \{ 0 \} \times A\\
    & \Leftrightarrow & x \in A
  \end{eqnarray*}
  proving that $G_0 = A$. Likewise we have
  \begin{eqnarray*}
    x \in G_1 & \Leftrightarrow & (1, x) \in (\{ 0 \} \times A) \bigcup (\{ 1
    \} \times B)\\
    & \Leftrightarrow & (1, x) \in \{ 0 \} \times A \vee (1, x) \in \{ 1 \}
    \times B\\
    & \Leftrightarrow & (1 = 0 \wedge x \in A) \vee (1, x) \in \{ 1 \} \times
    B\\
    & \Leftrightarrowlim_{0 \neq 1} & (1, x) \in \{ 1 \} \times B\\
    & \Leftrightarrow & x \in B
  \end{eqnarray*}
  Proving that $G_1 = B$. Now
  \begin{eqnarray*}
    x \in \bigcap_{i \in \{ 0, 1 \}} G_i & \Leftrightarrow & \forall i \in I
    \vDash (i, x) \in G\\
    & \Leftrightarrow & (0, x) \in G_0 \wedge (1, x) \in G_1\\
    & \Leftrightarrow & x \in A \wedge x \in B\\
    & \Leftrightarrow & x \in A \bigcap B
  \end{eqnarray*}
  proving that $\bigcap_{i \in \{ 0, 1 \}} G_i = A \bigcap B$. Also
  \begin{eqnarray*}
    x \in \bigcup_{i \in \{ 0, 1 \}} G_i & \Leftrightarrow & \exists i \in I
    \vdash (i, x) \in G\\
    & \Leftrightarrow & (0, x) \in G \vee (1, x) \in G\\
    & \Leftrightarrow & x \in A \vee x \in B\\
    & \Leftrightarrow & x \in A \bigcup B
  \end{eqnarray*}
\end{proof}

\begin{example}
  \label{union and intersection of emptyset}If $\{ G_i \}_{i \in \emptyset} =
  \langle \emptyset, \emptyset \rangle$ (see \ref{empty family}) then we have
  \begin{enumerate}
    \item $\bigcup_{i \in \emptyset} G_i = \emptyset$
    
    \item $\bigcap_{i \in \emptyset} G_i =\mathcal{U}$
  \end{enumerate}
\end{example}

\begin{proof}
  
  \begin{enumerate}
    \item If $x \in \bigcup_{i \in \emptyset} G_i$ then $\exists i \in I$ such
    that $x \in G_i$ which as $I = \emptyset$ is impossible and thus
    $\bigcup_{i \in \emptyset} G_i = \emptyset$
    
    \item $\bigcap_{i \in \emptyset} G_i = \{ x| \forall i \in I \vDash x \in
    G_i \} \equallim_{\tmop{defined}} \{ x|x \tmop{is} a \tmop{element} \wedge
    \forall i \in I \vDash x \in G_i \}$. As $\bigcap_{i \in \emptyset} G_i
    \subseteq \mathcal{U}$ we must prove that $\mathcal{U} \subseteq
    \bigcap_{i \in \emptyset} G_i$ so if $x \in \mathcal{U}$ then $x$ is a
    element $\Rightarrow$ $x$ is a element $\wedge$ ($\forall i \in \emptyset
    \vDash x \in G_i$) [is vacuously true] so that $x \in \bigcap_{i \in
    \emptyset} G_i$.
  \end{enumerate}
\end{proof}



As the only defining elements for classes are classes and the relation is a
element of, we can consider a class itself as a collection of classes (these
classes are then elements (and as we later will see by definition sets)). This
gives rise to another type of general union and intersection.

\begin{definition}
  If $\mathcal{A}$ is a class then $\bigcap_{A \in \mathcal{A}} A = \{ x |
  \forall A \in \mathcal{A} \nobracket \vDash x \in A \}$ and $\bigcup_{A \in
  \mathcal{A}} = \{ x | \exists A \in \mathcal{A} \nobracket \vDash x \in A
  \}$
\end{definition}

Now if $\mathcal{A}$ is a class of classes then we can consider
$\mathcal{I}^{\mathcal{A}} = \{ (A, x) | A \in \mathcal{A} \wedge x \in A
\nobracket \}$ then \ $\mathcal{I}^{\mathcal{A}} \subseteq \mathcal{A} \times
\bigcup_{A \in \mathcal{A}} A \subseteq \mathcal{U} \times \mathcal{U}$ so
that $\mathcal{I}^{\mathcal{A}}$ is a graph with $\tmop{dom}
(\mathcal{I}^{\mathcal{A}}) \subseteq \mathcal{A}$. Also if $A \in
\mathcal{A}$ then $(\mathcal{I}^{\mathcal{A}})_A = \{ x | (A, x) \in
\mathcal{I}^{\mathcal{A}} \nobracket \} = \{ x | A \in \mathcal{A} \wedge x
\in A \nobracket \} \equallim_{A \in \mathcal{A} \tmop{is} \tmop{true}} \{ x|x
\in A \} = A$ so that we have the family $\{ (\mathcal{I}^{\mathcal{A}})_A
\}_{A \in \mathcal{A}}$ can be be written as $\{ A \}_{A \in \mathcal{A}}$.
Essentially we can consider a class $\mathcal{A}$ as a family $\{
(\mathcal{I}^{\mathcal{A}})_A \}_{A \in \mathcal{A}}$ with $A \in \mathcal{A}
\Leftrightarrow (\mathcal{I}^{\mathcal{A}})_A = A$, that is we index the class
by itself. This leads to the following definition:

\begin{definition}
  \label{family of classes indexed by itself}If $\mathcal{A}$ is a class of
  classes then $\{ A \}_{A \in \mathcal{A}}$ is equal to $\{ A_i \}_{i \in
  \mathcal{A}}$ where $\tmop{graph} (\{ A_i \}_{i \in \mathcal{A}})
  =\mathcal{I}^{\mathcal{A}} = \{ (A, x) |A \in \mathcal{A} \wedge x \in A \}$
\end{definition}

We have then \ following theorem that says that the two types of unions and
intersections introduced are actually the same.

\begin{theorem}
  \label{union of family of classes}If $\mathcal{A}$ is a class then
  $\bigcup_{A \in \mathcal{A}} (\mathcal{I}^{\mathcal{A}})_A = \bigcup_{A \in
  \mathcal{A}} A$ and $\bigcap_{A \in \mathcal{A}} (\mathcal{I}^A)_A =
  \bigcap_{A \in \mathcal{A}} A$ 
\end{theorem}

\begin{proof}
  
  \begin{eqnarray*}
    x \in \bigcup_{A \in \mathcal{A}} (\mathcal{I}^{\mathcal{A}})_A &
    \Leftrightarrow & \exists A \in \mathcal{A} \vdash x \in
    (\mathcal{I}^{\mathcal{A}})_A\\
    & \Leftrightarrowlim_{(\mathcal{I}^{\mathcal{A}})_A = A} & \exists A \in
    \mathcal{A} \vdash x \in A\\
    & \Leftrightarrow & x \in \bigcup_{A \in A} A\\
    x \in \bigcap_{A \in \mathcal{A}} (\mathcal{I}^{\mathcal{A}})_A &
    \Leftrightarrow & \forall A \in \mathcal{A} \vdash x \in
    (\mathcal{I}^{\mathcal{A}})_A\\
    & \Leftrightarrowlim_{(\mathcal{I}^{\mathcal{A}})_A = A} & \forall A \in
    \mathcal{A} \vdash x \in A\\
    & \Leftrightarrow & 
  \end{eqnarray*}
  
\end{proof}

\begin{theorem}
  If $\{ G_i \}_{i \in I}$ is a family of classes and $B$ is a class then we
  have
  \begin{enumerate}
    \item $\forall i \in I \vdash G_i \subseteq \bigcup_{i \in I} G_i$
    
    \item $\forall i \in I \vdash \bigcap_{i \in I} G_i \subseteq G_i$
    
    \item If $\forall i \in I$ we have $A_i \subseteq B$ then $\bigcup_{i \in
    I} A_i \subseteq B$
    
    \item If $\forall i \in I$ we have $B \subseteq A_i$ then $B \subseteq
    \bigcap A_{i \in I}$
  \end{enumerate}
  \begin{proof}
    
    \begin{enumerate}
      \item Take $i \in I$ then if $x \in G_i \Rightarrow \exists j \vdash x
      \in G_j  (\tmop{take} i = j) \Rightarrow x \in \bigcup_{i \in I} G_i$
      
      \item Take $i \in I$ then if $x \in \bigcap_{j \in I} G_j \Rightarrow
      \forall j \in I \vdash x \in G_j \Rightarrow x \in G_i$
      
      \item We have
      \begin{eqnarray*}
        x \in \bigcup_{i \in I} A_i & \Rightarrow & \exists i \in I \vdash x
        \in A_i\\
        & \Rightarrowlim_{A_i \subseteq B} & x \in B\\
        & \Rightarrow & \bigcup_{i \in I} A_i \subseteq B
      \end{eqnarray*}
      \item We have
      \begin{eqnarray*}
        x \in B & \Rightarrow & \forall i \in I \vDash x \in A_i\\
        & \Rightarrow & x \in \bigcap_{i \in I} A_i\\
        & \Rightarrow & B \subseteq \bigcap_{i \in I} A_i
      \end{eqnarray*}
    \end{enumerate}
    
  \end{proof}
\end{theorem}

\begin{theorem}[Generalized deMorgan's Laws]
  \label{generalized deMorgan's Laws}{\index{generalized deMorgan's Law}}Let
  $\{ G_i \}_{i \in I}$ be an indexed family of classes. Then
  \begin{enumerate}
    \item $\left( \bigcup_{i \in I} A_i \right)^c = \bigcap_{i \in I} A_i^c$
    
    \item $\left( \bigcap_{i \in I} A_i \right)^c = \bigcup_{i \in I} A_i^c$
  \end{enumerate}
\end{theorem}

\begin{proof}
  
  \begin{enumerate}
    \item This is proved by
    \begin{eqnarray*}
      x \in \left( \bigcup_{i \in I} A_i \right)^c & \Leftrightarrow & x \nin
      \bigcup_{i \in I} A_i\\
      & \Leftrightarrow & \neg (\exists i \in I \vdash x \in A_i)\\
      & \Leftrightarrow & \forall i \in I \vDash x \nin A_i\\
      & \Leftrightarrow & \forall i \in I \vDash x \in A_i^c\\
      & \Leftrightarrow & x \in \bigcap_{i \in I} A_i^c
    \end{eqnarray*}
    \item This is proved by
    \begin{eqnarray*}
      x \in \left( \bigcap_{i \in I} A_i \right)^c & \Leftrightarrow & x \nin
      \bigcap_{i \in I} A_i\\
      & \Leftrightarrow & \neg (\forall i \in I \vDash x \in A_i)\\
      & \Leftrightarrow & \exists i \in I \vDash x \nin A_i\\
      & \Leftrightarrow & \exists i \in I \vDash x \in A_i^c\\
      & \Leftrightarrow & x \in \bigcup_{i \in I} A_i^c
    \end{eqnarray*}
  \end{enumerate}
\end{proof}

\begin{theorem}
  \label{generalized distributive laws}If $\{ A_i \}_{i \in I}$ and $\{ B_j
  \}_{j \in J}$ are indexed families of classes then
  \begin{enumerate}
    \item $\left( \bigcup_{i \in I} A_i \right) \bigcap \left( \bigcup_{j \in
    J} B_j \right) = \bigcup_{(i, j) \in I \times J} \left( A_i \bigcap B_j
    \right)$
    
    \item $\left( \bigcap_{i \in I} A_i \right) \bigcup \left( \bigcap_{j \in
    J} B_j \right) = \bigcap_{(i, j) \in I \times J} \left( A_i \bigcup B_j
    \right)$
    
    \item $\bigcup_{i \in I} A_i = \left( \bigcup_{i \in I \backslash \{ j \}}
    A_i \right) \bigcup A_j$ if $j \in I$
    
    \item $\bigcap_{i \in I} A_i = \left( \bigcap_{i \in I \backslash \{ j \}}
    A_j \right) \bigcap A_j$ if $j \in I$
  \end{enumerate}
\end{theorem}

\begin{proof}
  
  \begin{enumerate}
    \item This is proved by
    \begin{eqnarray*}
      x \in \left( \bigcup_{i \in I} A_i \right) \bigcap \left( \bigcup_{j \in
      J} B_j \right) & \Leftrightarrow & x \in \bigcup_{i \in I} A_i \wedge x
      \in \bigcup_{j \in J} B_j\\
      & \Leftrightarrow & (\exists i \in I \vdash x \in A_i) \wedge (\exists
      j \in J \vdash x \in B_j)\\
      & \Leftrightarrow & \exists (i, j) \in I \times J \vdash x \in A_i
      \wedge x \in B_j\\
      & \Leftrightarrow & \exists (i, j) \in I \times J \vdash x \in A_i
      \bigcap B_j\\
      & \Leftrightarrow & x \in \bigcup_{(i.j) \in I \times J} \left( A_i
      \bigcap B_j \right)
    \end{eqnarray*}
    \item This is proved by
    \begin{eqnarray*}
      x \in \left( \bigcap_{i \in I} A_i \right) \bigcup \left( \bigcap_{j \in
      J} B_j \right) & \Leftrightarrow & x \in \bigcap_{i \in I} A_i \vee x
      \in \bigcap_{j \in J} B_j\\
      & \Leftrightarrow & (\forall i \in I \vDash x \in A_i) \vee (\forall j
      \in J \vDash x \in B_j)\\
      & \Leftrightarrow & \forall (i, j) \in I \times J \vDash (x \in A_i
      \vee x \in B_j)\\
      & \Leftrightarrow & \forall (i, j) \in I \times J \vDash x \in A_i
      \bigcup B_j\\
      & \Leftrightarrow & x \in \bigcap_{(i, j) \in I \times J} \left( A_i
      \bigcup B_j \right)
    \end{eqnarray*}
    \item If $j \in I \Rightarrow I = (I \backslash \{ j \}) \bigcup \{ j \}$
    \begin{eqnarray*}
      x \in \bigcup_{i \in I} A_i & \Leftrightarrow & \exists i \in I \vdash x
      \in A_i\\
      & \Leftrightarrow & \exists i \in \left( I \backslash \{ j \} \bigcup
      \{ j \} \right) \vdash x \in A_i\\
      & \Leftrightarrow & (\exists i \in I \backslash \{ j \} \vdash x \in
      A_i) \vee (\exists i \in \{ j \} \vdash x \in A_i)\\
      & \Leftrightarrow & (\exists i \in I \backslash \{ j \} \vdash x \in
      A_i) \vee \nobracket x \in A_j)\\
      & \Leftrightarrow & x \in \bigcup_{i \in I \backslash \{ j \}} A_i
      \bigcup A_j
    \end{eqnarray*}
    \item If $j \in I \Rightarrow I = (I \backslash \{ j \}) \bigcup \{ j \}$
    \begin{eqnarray*}
      x \in \bigcup_{i \in I} A_i & \Leftrightarrow & \forall i \in I \vDash x
      \in A_i\\
      & \Leftrightarrow & \forall i \in (I \backslash \{ j \}) \bigcup \{ j
      \} \vDash x \in A_i\\
      & \Leftrightarrow & (\forall i \in (I \backslash \{ j \}) \vDash x \in
      A_i) \wedge x \in A_j\\
      & \Leftrightarrow & \left( x \in \bigcap_{i \in I \backslash \{ j \}}
      A_j \right) \wedge x \in A_j\\
      & \Leftrightarrow & x \in \left( \left( \bigcap_{i \in I \backslash \{
      j \}} A_i \right) \bigcap A_j \right)
    \end{eqnarray*}
  \end{enumerate}
  
\end{proof}

\begin{theorem}
  \label{generalized difference}If $A$ is a class and $\{ A_i)_{i \in I}$ a
  family of classes then
  \begin{enumerate}
    \item $A\backslash \left( \bigcap_{i \in I} A_i \right) = \bigcup_{i \in
    I} (A\backslash A_i)$
    
    \item $A\backslash \left( \bigcup_{i \in I} A_i \right) = \bigcap_{i \in
    I} (A\backslash A_i)$
    
    \item $\left( \bigcup_{i \in I} A_i \right) \backslash A = \bigcup_{i \in
    I} (A_i \backslash A)$
    
    \item $\left( \bigcap_{i \in I} A_i \right) \backslash A = \bigcap_{i \in
    I} (A_i \backslash A)$
    
    \item $A \times \left( \bigcup_{i \in I} A_i \right) = \bigcup_{i \in I}
    (A \times A_i)$
    
    \item $\left( \bigcup_{i \in I} A_i \right) \times A = \bigcup_{i \in I}
    (A_i \times A)$
  \end{enumerate}
\end{theorem}

\begin{proof}
  
  \begin{enumerate}
    \item We have
    \begin{eqnarray*}
      A\backslash \left( \bigcap_{i \in I} A_i \right) & = & A \bigcap \left(
      \bigcap_{i \in I} A_i \right)^c\\
      & \equallim_{\text{\ref{generalized deMorgan's Laws}}} & A \bigcap
      \left( \bigcup_{i \in I} A_i^c \right)\\
      & \equallim_{\text{\ref{generalized distributive laws}}} & \bigcup_{i
      \in I} \left( A \bigcap A^c_i \right)\\
      & = & \bigcup_{i \in I} (A\backslash A_i)
    \end{eqnarray*}
    \item We have
    \begin{eqnarray*}
      A\backslash \left( \bigcup_{i \in I} A_i \right) & = & A \bigcap \left(
      \bigcup_{i \in I} A_i \right)^c\\
      & = & A \bigcap \left( \bigcap_{i \in I} A_i^c \right)
    \end{eqnarray*}
    Now if $x \in A \bigcap \left( \bigcap_{i \in I} A_i^c \right)$ then $x
    \in A$ and $\forall i \in I$ we have $x \nin A_i \Rightarrow x \in
    \bigcap_{i \in I} (A\backslash A_i)$. If $x \in \bigcap_{i \in I}
    (A\backslash A_i)$ then $\forall i \in I$ we have $x \in A \wedge x \nin
    A_i \Rightarrow x \in A \wedge \forall i \in I \vDash x \in A_i^c
    \Rightarrow x \in A \bigcap \left( \bigcap_{i \in I} A_i^c \right)$
    proving that $A \bigcap \left( \bigcap_{i \in I} A_i^c \right) =
    \bigcap_{i \in I} (A\backslash A_i)$ and thus that $A\backslash \left(
    \bigcup_{i \in I} A_i \right) = \bigcap_{i \in I} (A\backslash A_i)$.
    
    \item We have $x \in \left( \bigcup_{i \in I} A_i \right) \backslash A
    \Leftrightarrow x \nin A \wedge \exists i \in I \vdash x \in A_i
    \Leftrightarrow \exists i \in I \vdash (x \nin A \wedge x \in A_i)
    \Leftrightarrow x \in \bigcup_{i \in I} (A_i \backslash A)$
    
    \item We have $x \in \left( \bigcap_{i \in I} A_i \right) \backslash A
    \Leftrightarrow x \in A \wedge \forall i \in I \vdash x \in A_i
    \Leftrightarrow \forall i \in I \vDash (x \in A_i \wedge x \nin A)
    \Leftrightarrow x \in \bigcap_{i \in I} (A_i \backslash A)$
    
    \item We have
    \begin{eqnarray*}
      (x, y) \in A \times \left( \bigcup_{i \in I} A_i \right) &
      \Leftrightarrow & x \in A \wedge y \in \bigcup_{i \in I} A_i\\
      & \Leftrightarrow & x \in A \wedge \exists i \in I \vDash y \in A_i\\
      & \Leftrightarrow & \exists i \in I \vDash x \in A \wedge y \in A_i\\
      & \Leftrightarrow & (x, y) \in \bigcup_{i \in I} (A \times A_i)
    \end{eqnarray*}
    \item We have
    \begin{eqnarray*}
      (x, y) \in \left( \bigcup_{i \in I} A_i \right) \times A &
      \Leftrightarrow & x \in \bigcup_{i \in I} A_i \wedge y \in A\\
      & \Leftrightarrow & y \in A \wedge \exists i \in I \vDash x \in A_i\\
      & \Leftrightarrow & \exists i \in I \vDash x \in A_i \wedge y \in A\\
      & \Leftrightarrow & (x, y) \in \bigcup_{i \in I} (A_i \times A)
    \end{eqnarray*}
  \end{enumerate}
  
\end{proof}

\begin{theorem}
  \label{domain an range of graph}Let $\{ G_i \}_{i \in I}$ be a family of
  graphs then
  \begin{enumerate}
    \item $\tmop{dom} \left( \bigcup_{i \in I} G_i \right) = \bigcup_{i \in I}
    (\tmop{dom} (G_i))$
    
    \item $\tmop{range} \left( \bigcup_{i \in I} G_i \right) = \bigcup_{i \in
    I} (\tmop{range} (G_i))$
  \end{enumerate}
\end{theorem}

\begin{proof}
  
  \begin{enumerate}
    \item We have
    \begin{eqnarray*}
      x \in \tmop{dom} \left( \bigcup_{i \in I} G_i \right) & \Leftrightarrow
      & \exists y \vdash (x, y) \in \bigcup_{i \in I} G_i\\
      & \Leftrightarrow & \exists y \vdash (\exists i \in I \vdash (x, y) \in
      G_i)\\
      & \Leftrightarrow & \exists i \in I \vdash (\exists y \vdash (x, y) \in
      G_i)\\
      & \Leftrightarrow & \exists i \in I \vdash x \in \tmop{dom} (G_i)\\
      & \Leftrightarrow & x \in \bigcup_{i \in I} \tmop{dom} (G_i)
    \end{eqnarray*}
    \item This is proved by
    \begin{eqnarray*}
      x \in \tmop{range} \left( \bigcup_{i \in I} G_i \right) &
      \Leftrightarrow & \exists y \vdash (y, x) \in \bigcup_{i \in I} G_i\\
      & \Leftrightarrow & \exists y \vdash (\exists i \in I \vdash (y, x) \in
      G_i)\\
      & \Leftrightarrow & \exists i \in I \vdash (\exists y \vdash (y, x) \in
      G_i)\\
      & \Leftrightarrow & \exists i \in I \vdash (x \in \tmop{range} (G_i))\\
      & \Leftrightarrow & \bigcup_{i \in I} \tmop{range} (G_i)
    \end{eqnarray*}
  \end{enumerate}
  
\end{proof}

\section{Sets}

\begin{definition}
  A set is a class that is an element of a class. In formula form $x$ is a set
  $\Leftrightarrow$ $\exists \mathcal{A}$ with $x \in \mathcal{A}$
\end{definition}

Rephrasing the axiom of pairing we have then

\begin{theorem}
  \label{axiom of pairing}If $x \tmop{and} y$ are sets then $\{ x, y \}$ is a
  set
\end{theorem}

\begin{axiom}[Axiom of Subsets]
  \label{axiom of subsets}{\index{axiom of subsets}}Every subclass of a set is
  a set
\end{axiom}

\begin{theorem}
  If $A$ is a set and $B$ a class then $A \bigcap B$ is a set. In particular
  we have that the intersection of two sets is a set
\end{theorem}

\begin{proof}
  By theorem \ref{relation classes and intersection and union} we have $A
  \bigcap B \subseteq A$ and thus by the axiom of subsets (see \ref{axiom of
  subsets}) we have $A \bigcap B$ is a subset
\end{proof}

\begin{axiom}[Axiom of Unions]
  \label{axiom of unions}{\index{axiom of unions}}If $\mathcal{A}$ is a set of
  sets then $\bigcup_{A \in \mathcal{A}} A$ is a set
\end{axiom}

\begin{theorem}
  \label{union of two sets is a set}If $A, B$ are sets then $A \bigcup B$ is a
  set
\end{theorem}

\begin{proof}
  Take $\mathcal{A}= \{ A, B \}$ then $\mathcal{A}$ is a set by \ref{axiom of
  pairing} and thus a set of sets. Using the axiom of unions we have then that
  $\bigcup_{C \in \mathcal{A}} C$ is a set. Next we prove that $A \bigcup B =
  \bigcup_{C \in \mathcal{A}} C$
  \begin{eqnarray*}
    x \in A \bigcup B & \Leftrightarrow & x \in A \vee x \in B\\
    & \Leftrightarrow & \exists C \in \{ A, B \} \vdash x \in C\\
    & \Leftrightarrow & x \in \bigcup_{C \in \mathcal{A}} C
  \end{eqnarray*}
\end{proof}

Now if $A$ is a set then if $B \subseteq A$ we have that $B$ is a set so $B
\subseteq A$ is equivalent with $B$ is a element and $B \subseteq A$ (and thus
a element) by the axiom of subsets. Using \ref{axiom of construction} we have
then that $\{ B | B \subseteq A \nobracket \}$ is a class (as $B \subseteq A$
is equivalent with $B$ is a set and $B \subseteq A$). This leads to the
following definition.

\begin{definition}
  If $A$ is a set then $\mathcal{P} (A)$ is the class defined by $\mathcal{P}
  (A) = \{ B | B \subseteq A \nobracket \}$
\end{definition}

We state now that $\mathcal{P} (A)$ is a set

\begin{axiom}[Axiom of Power Sets]
  \label{axiom of power sets}{\index{axiom of powersets}}{\index{$\mathcal{P}
  (A)$}}If $A$ is a set then $\mathcal{P} (A)$ is a set
\end{axiom}

\begin{theorem}
  If $A$ is a set and $P$ a predicate about $X$ then $\{ X | X \subseteq A
  \wedge P (X) \nobracket \}$ is a set
\end{theorem}

\begin{proof}
  First as $X \subseteq A \Rightarrow X$ is a set and thus a element (see
  \ref{axiom of subsets}) so we have that $X \subseteq A \wedge P (X)$ is
  equivalent with $(X \tmop{is} a \tmop{element} \wedge X \subseteq A \wedge P
  (X))$ and thus $B = \{ X | X \subseteq A \wedge P (X) \nobracket \}$ is a
  class. Now if $X \in B \Rightarrow X \subseteq A \Rightarrow X \in
  \mathcal{P} (A) \Rightarrow B \subseteq \mathcal{P} (X)$ and thus by the
  axiom of subsets \ref{axiom of subsets} we have that $B$ is a set.
\end{proof}

Not every class is a set as the following theorem proves:

\begin{theorem}
  The universal class $\mathcal{U}= \{ x|x \tmop{is} a \tmop{element}
  \tmop{and} x = x \}$ is not a set
\end{theorem}

\begin{proof}
  If $\mathcal{U}$ is a set (thus a element) then $\mathcal{U} \in
  \mathcal{U}$. Define now the class $\mathcal{B}= \{ x|x \tmop{is} a
  \tmop{element} \tmop{and} x \nin x \}$ then if $x \in \mathcal{B}$ we have
  $x$ is a element and thus $x \in \mathcal{U}$ so we have $\mathcal{B}
  \subseteq \mathcal{U}$ and thus by the axiom of subsets (see \ref{axiom of
  subsets}) we must have that $\mathcal{B}$ is a set. If now $\mathcal{B} \in
  \mathcal{B}$ then we have as $\mathcal{B}$ is a set that $\mathcal{B} \nin
  \mathcal{B}$ contradicting $\mathcal{B} \in \mathcal{B}$ so we must have
  that $\mathcal{B} \in \mathcal{B}$ but this leads then to $\mathcal{B} \nin
  \mathcal{B}$ again a contradiction, so in all cases we have a contradiction
  and thus our initial assumption that \tmverbatim{$\mathcal{U}$} is a set
  must be false.
\end{proof}

\begin{definition}
  \label{P'(X)}{\index{$\mathcal{P}' (A)$}}Let $A$ be a set then $\mathcal{P}'
  (A) =\mathcal{P} (A) \backslash \{ \emptyset \}$ (this is a set as
  $\mathcal{P} (A)$ is a set (by\ref{axiom of power sets} and \ref{axiom of
  subsets}). It is essentially the set of all nonempty subsets of $A$
\end{definition}

If $A$ is a set then $\{ A \} = \{ A, A \}$ is a set by \ref{axiom of pairing}
and thus by \ref{union of two sets is a set} $A \bigcup \{ A \}$ is a set.
This leads to the following definition

\begin{definition}
  \label{successor of a set}{\index{successor}}If $A$ is a set then the set $s
  (A)$ is defined by $s (A) = A \bigcup \{ A \}$
\end{definition}

\begin{definition}[Successor Set]
  \label{successor set}{\index{successor set}}We say that a set $A$ is a
  successor set iff
  \begin{enumerate}
    \item $\emptyset \in A$
    
    \item If $X \in A \Rightarrow s (X) \in A$
  \end{enumerate}
\end{definition}

We have now the axiom of infinity, which will be used later to define the set
of natural numbers, from which the whole numbers and finally the reals will
follow.

\begin{axiom}[Axiom of Infinity]
  \label{axiom of infinity}{\index{axiom of infinity}}There exists a successor
  set
\end{axiom}

One consequence of this theorem is that $\emptyset$ is a set

\begin{theorem}
  \label{empty set is a set}$\emptyset$ is a set
\end{theorem}

\begin{proof}
  By the axiom of infinity there exists a successor set $A$. But then
  $\emptyset \in A$ and thus the class $\emptyset$ is a set.
\end{proof}

\begin{example}
  The set $0 = \emptyset$ it has no elements, the set 1 is the set $\{
  \emptyset \} = \{ 0 \}$ it has one element, and $2$ is the set $\{
  \emptyset, \{ \emptyset \} \} = \{ 0, 1 \}$ which has two elements (for if
  $\emptyset = \{ \emptyset \}$ we would have a contradiction from $\emptyset
  \in \{ \emptyset \} \Rightarrow \emptyset \in \emptyset
  \Rightarrowlim_{\emptyset = \{ x | x \neq x | \}} \emptyset \neq \emptyset$)
\end{example}

Let's now proceed to prove that the product of sets is a set, first we need
the following lemma.

\begin{lemma}
  \label{product of sets as element}If $A, B$ are sets then $A \times B
  \subseteq \mathcal{P} \left( \mathcal{P} \left( A \bigcup B \right) \right)$
\end{lemma}

\begin{proof}
  We have
  \begin{eqnarray*}
    (x, y) \in A \times B & \Rightarrow & (x, y) = \{ \{ x \}, \{ x, y \} \}
    \wedge x \in A \wedge x \in B\\
    & \Rightarrow & \{ x \} \subseteq A \subseteq A \bigcup B \wedge \{ x, y
    \} \subseteq A \bigcup B\\
    & \Rightarrow & \{ x \} \subseteq A \bigcup B \wedge \{ x, y \} \subseteq
    A \bigcup B\\
    & \Rightarrow & \{ x \} \in \mathcal{P} \left( A \bigcup B \right) \wedge
    \{ x, y \} \in \mathcal{P} \left( A \bigcup B \right)\\
    & \Rightarrow & \{ \{ x \}, \{ x, y \} \} \subseteq \mathcal{P} \left( A
    \bigcup B \right)\\
    & \Rightarrow & \{ \{ x \}, \{ x, y \} \} \in \mathcal{P} \left(
    \mathcal{P} \left( A \bigcup B \right) \right)\\
    & \Rightarrow & (x, y) \in \mathcal{P} \left( \mathcal{P} \left( A
    \bigcup B \right) \right)
  \end{eqnarray*}
\end{proof}

\begin{theorem}
  \label{product of sets is a set}If $A, B$ are sets then $A \times B$ is a
  set
\end{theorem}

\begin{proof}
  By \ref{union of two sets is a set} we have that $A \bigcup B$ is a set, so
  that by \ref{axiom of power sets} $\mathcal{P} \left( A \bigcup B \right)$
  is set and again by \ref{axiom of power sets} we have that $\mathcal{P}
  \left( \mathcal{P} \left( A \bigcup B \right) \right)$ is a set. Now by the
  previous lemma we have that $A \times B \subseteq \mathcal{P} \left(
  \mathcal{P} \left( A \bigcup B \right) \right)$ and using \ref{axiom of
  subsets} we have then that $A \times B$ is a set.
\end{proof}

\chapter{Partial functions, functions and relations}

\section{Functions}

\subsection{Partial functions}

First we define how we can form pairs and triples (not to be confused with
ordered pairs). We need a new definition for if $A, B$ are pure classes then
we can not just form $(A, B) = \{ A, \{ B \} \}$ because this would mean that
$A, B$ are elements and not pure classes. So we need another way of forming
pairs, triples and so on.

\begin{definition}
  \label{pair of classes}{\index{pair of classes}}{\index{$\langle x, y
  \rangle$}}A pair $\langle x, y \rangle$ of classes is the class defined by
  $(x \times \{ \emptyset \}) \bigcup \{ y \times \{ \{ \emptyset \} \} \}$
  [which is a class because $\emptyset$ is a set (see \ref{empty set is a
  set}) so $\{ \emptyset \} \tmop{and} \{ \{ \emptyset \} \}$ are sets (see
  \ref{axiom of pairing}) and thus $\emptyset, \{ \emptyset \}, \{ \{
  \emptyset \} \}$ are classes which means that $x \times \{ \emptyset \}, \{
  y \times \{ \{ \emptyset \} \} \}$ are defined and thus their union is a
  class].
\end{definition}

\begin{theorem}
  If $\langle x, y \rangle$ and $\langle x', y' \rangle$ are pairs then
  $\langle x, y \rangle = \langle x', y' \rangle$ iff $x = x' \wedge y = y'$
\end{theorem}

\begin{proof}
  
  
  $\Rightarrow$
  
  Assume $\langle x, y \rangle = \langle x', y' \rangle$ then we have
  \begin{eqnarray*}
    e \in x & \Rightarrow & (e, \emptyset) \in x \times \{ \emptyset \}\\
    & \Rightarrow & (e, \emptyset) \in \langle x, y \rangle\\
    & \Rightarrow & (e, \emptyset) \in \langle x', y' \rangle\\
    & \Rightarrow & (e, \emptyset) \in x' \times \{ \emptyset \}  (\tmop{as}
    (e, \emptyset) \nin y' \times \{ \{ \emptyset \} \})\\
    & \Rightarrow & e \in x'\\
    e \in x' & \Rightarrow & (e, \emptyset) \in x' \times \{ \emptyset \}\\
    & \Rightarrow & (e, \emptyset) \in \langle x', y' \rangle\\
    & \Rightarrow & (e, \emptyset) \in \langle x, y \rangle\\
    & \Rightarrow & (e, \emptyset) \in x \times \{ \emptyset \} \tmop{as} (e,
    \emptyset) \nin y \times \{ \{ \emptyset \} \}\\
    & \Rightarrow & e \in x
  \end{eqnarray*}
  So we have $x = x'$
  \begin{eqnarray*}
    e \in y & \Rightarrow & (e, \{ \emptyset \}) \in y \times \{ \{ \emptyset
    \} \}\\
    & \Rightarrow & (e, \{ \emptyset \}) \in \langle x, y \rangle\\
    & \Rightarrow & (e, \{ \emptyset \}) \in \langle x', y' \rangle\\
    & \Rightarrow & (e, \{ \emptyset \}) \in y' \times \{ \{ \emptyset \} \}
    \tmop{as} (e, \{ \emptyset \}) \nin x' \times \{ \emptyset \}\\
    & \Rightarrow & e \in y'\\
    e \in y' & \Rightarrow & (e, \{ \emptyset \}) \in y' \times \{ \{
    \emptyset \} \}\\
    & \Rightarrow & (e, \{ \emptyset \}) \in \langle x', y' \rangle\\
    & \Rightarrow & (e, \{ \emptyset \}) \in \langle x, y \rangle\\
    & \Rightarrow & (e, \{ \emptyset \}) \in y \times \{ \{ \emptyset \} \}
    \tmop{as} (e, \{ \emptyset \}) \nin x \times \{ \emptyset \}\\
    & \Rightarrow & e \in y
  \end{eqnarray*}
  So we have $y = y'$
  
  $\Leftarrow$
  
  Assume now that $x = x' \wedge y = y'$ then
  \begin{eqnarray*}
    e \in \langle x, y \rangle & \Leftrightarrow & e \in (x \times \{
    \emptyset \}) \bigcup (y \times \{ \{ \emptyset \} \})\\
    & \Leftrightarrow & (e = (f, \emptyset) \wedge f \in x) \vee (e = (g, \{
    \emptyset \}) \wedge g \in y)\\
    & \Leftrightarrow & (e = (f, \emptyset) \wedge f \in x') \vee (e = (g, \{
    \emptyset \}) \wedge g \in y')\\
    & \Leftrightarrow & e \in (x' \times \{ \emptyset \}) \bigcup (y' \times
    \{ \{ \emptyset \} \})\\
    & \Leftrightarrow & e \in \langle x', y' \rangle
  \end{eqnarray*}
\end{proof}

\begin{definition}
  \label{tripple of classes}{\index{tripple of classes}}{\index{$\langle x, y,
  z \rangle$}}A triple $\langle x, y, z \rangle$ where $x, y, z$ are classes
  is the class defined by $(x \times \{ \emptyset \}) \bigcup (y \times \{ \{
  \emptyset \} \}) \bigcup (z \times \{ \{ \{ \emptyset \} \} \})$ (which is a
  class because $\emptyset$ is a set (see \ref{empty set is a set}) so $\{
  \emptyset \}, \{ \{ \emptyset \} \}$ and $\{ \{ \{ \emptyset \} \} \}$ are
  sets (see \ref{axiom of pairing}) thus $\langle x, y, z \rangle$ is a
  class). 
\end{definition}

\begin{theorem}
  If $(x, y, z)$ and $(x', y', z')$ are triples then $\langle x, y, z \rangle
  = \langle x', y', z' \rangle$ iff $x = x' \wedge y = y' \wedge z = z'$
\end{theorem}

\begin{proof}
  
  
  $\Rightarrow$
  
  Assume $\langle x, y, z \rangle = \langle x', y', z' \rangle$ then we have
  \begin{eqnarray*}
    e \in x & \Rightarrow & (e, \emptyset) \in x \times \{ \emptyset \}\\
    & \Rightarrow & (e, \emptyset) \in \langle x, y, z \rangle\\
    & \Rightarrow & (e, \emptyset) \in \langle x', y', z' \rangle\\
    & \Rightarrow & (e, \emptyset) \in x' \times \{ \emptyset \} \tmop{as}
    (e, \emptyset) \nin y' \times \{ \{ \emptyset \} \}, z' \times \{ \{ \{
    \emptyset \} \} \}\\
    & \Rightarrow & e \in x'\\
    e \in x' & \Rightarrow & (e, \emptyset) \in x' \times \{ \emptyset \}\\
    & \Rightarrow & (e, \emptyset) \in \langle x', y', z' \rangle\\
    & \Rightarrow & (e, \emptyset) \in \langle x, y, z \rangle\\
    & \Rightarrow & (e, \emptyset) \in x \times \{ \emptyset \} \tmop{as} (e,
    \emptyset) \nin y \times \{ \{ \emptyset \} \}, z \times \{ \{ \{
    \emptyset \} \} \}\\
    & \Rightarrow & e \in x
  \end{eqnarray*}
  So we have $x = x'$
  \begin{eqnarray*}
    e \in y & \Rightarrow & (e, \{ \emptyset \}) \in y \times \{ \{ \emptyset
    \} \}\\
    & \Rightarrow & (e, \{ \emptyset \}) \in \langle x, y, z \rangle\\
    & \Rightarrow & (e, \{ \emptyset \}) \in \langle x', y', z' \rangle\\
    & \Rightarrow & (e, \{ \emptyset \}) \in y' \times \{ \{ \emptyset \} \}
    \tmop{as} (e, \{ \emptyset \}) \nin x' \times \{ \emptyset \}, z' \times
    \{ \{ \{ \emptyset \} \} \}\\
    & \Rightarrow & e \in y'\\
    e \in y' & \Rightarrow & (e, \{ \emptyset \}) \in y' \times \{ \{
    \emptyset \} \}\\
    & \Rightarrow & (e, \{ \emptyset \}) \in \langle x', y', z' \rangle\\
    & \Rightarrow & (e, \{ \emptyset \}) \in \langle x, y, z \rangle\\
    & \Rightarrow & (e, \{ \emptyset \}) \in y \times \{ \{ \emptyset \} \}
    \tmop{as} (e, \{ \emptyset \}) \nin x \times \{ \emptyset \}, z \times \{
    \{ \{ \emptyset \} \} \}\\
    & \Rightarrow & e \in y
  \end{eqnarray*}
  So we have $y = y'$
  \begin{eqnarray*}
    e \in z & \Rightarrow & (e, \{ \{ \emptyset \} \}) \in z \times \{ \{ \{
    \emptyset \} \} \}\\
    & \Rightarrow & (e, \{ \{ \emptyset \} \}) \in \langle x, y, z \rangle\\
    & \Rightarrow & (e, \{ \{ \emptyset \} \}) \in \langle x', y', z'
    \rangle\\
    & \Rightarrow & (e, \{ \{ \emptyset \} \}) \in z' \times \{ \{ \{
    \emptyset \} \} \} \tmop{as} (e, \{ \{ \emptyset \} \}) \nin x \times \{
    \emptyset \}, y \times \{ \{ \emptyset \} \}\\
    & \Rightarrow & e \in z'\\
    e \in z' & \Rightarrow & (e, \{ \{ \emptyset \} \}) \in z' \times \{ \{ \{
    \emptyset \} \} \}\\
    & \Rightarrow & (e, \{ \{ \emptyset \} \}) \in \langle x', y', z'
    \rangle\\
    & \Rightarrow & (e, \{ \{ \emptyset \} \}) \in \langle x, y, z \rangle\\
    & \Rightarrow & (e, \{ \{ \emptyset \} \}) \in z \times \{ \{ \{
    \emptyset \} \} \}\\
    & \Rightarrow & e \in z
  \end{eqnarray*}
  So we have $z = z'$
  
  $\Leftarrow$
  
  Assume now that $x = x' \wedge y = y \wedge z = z'$ then
  \begin{eqnarray*}
    e \in \langle x, y, z \rangle & \Leftrightarrow & e \in (x \times \{
    \emptyset \}) \bigcup (y \times \{ \{ \emptyset \} \}) \bigcup (z \times
    \{ \{ \{ \emptyset \} \} \})\\
    & \Leftrightarrow & (e = (f, \emptyset) \wedge f \in x) \vee (e = (g, \{
    \emptyset \}) \wedge g \in y) \vee (e = (f, \{ \{ \emptyset \} \}) \wedge
    f \in z)\\
    & \Leftrightarrow & (e = (f, \emptyset) \wedge f \in x') \vee (e = (g, \{
    \emptyset \}) \wedge g \in y') \vee (e = (f, \{ \{ \emptyset \} \}) \wedge
    f \in z')\\
    & \Leftrightarrow & e \in (x' \times \{ \emptyset \}) \bigcup (y' \times
    \{ \{ \emptyset \} \}) \bigcup (z' \times \{ \{ \{ \emptyset \} \} \})\\
    & \Leftrightarrow & e \in \langle x', y', z' \rangle
  \end{eqnarray*}
\end{proof}

\begin{definition}
  {\index{partial function}}{\index{partial mapping}}A partial function from
  $A$ to $B$ is a triple of objects $\langle f, A, B \rangle$ [where $A, B$
  are classes and $f \subseteq A \times B$ is a class (the graph of the
  partial function) (see \ref{axiom of subsets} (so $\langle f, A, B \rangle$
  is a class] such that if $(x, y) \in f \wedge (x, y') \in f$ then $y = y'$.
  In other words the graph of a partial function is a function graph (see
  \ref{function graph}).
\end{definition}

\begin{notation}
  Instead of writing $\langle f, A, B \rangle$ for a partial function we write
  $f : A \rightarrow B$
\end{notation}

\begin{theorem}
  \label{alternative definition of partial function}If $f$ is a graph and $A,
  B$ are classes then $\langle f, A, B \rangle$ is a partial function iff
  \begin{enumerate}
    \item $\tmop{range} (f) \subseteq B$
    
    \item $\tmop{dom} (f) \subseteq A$
    
    \item If $(x, y) \in f \wedge (x, y') \in f \Rightarrow y = y'$
  \end{enumerate}
\end{theorem}

\begin{proof}
  
  
  $\Rightarrow$
  
  If $\langle f, A, B \rangle$ is a partial function then $f \subseteq A
  \times B$ so $f$ is clearly a graph and we have
  \begin{enumerate}
    \item $y \in \tmop{range} (f) \Rightarrow \exists x \vdash (x, y) \in f
    \subseteq A \times B \Rightarrow y \in B \Rightarrow \tmop{range} (f)
    \subseteq B$
    
    \item $x \in \tmop{dom} (f) \Rightarrow \exists y \vdash (x, y) \in f
    \subseteq A \times B \Rightarrow x \in A \Rightarrow \tmop{dom} (f)
    \subseteq A$
    
    \item If $(x, y) \in f \wedge (x, y') \in f \Rightarrow y = y'$
  \end{enumerate}
  $\Leftarrow$
  
  If $\langle f, A, B \rangle$ fulfills (1),(2) and (3) then $(x, y) \in f
  \Rightarrow x \in \tmop{dom} (f) \wedge y \in \tmop{range} (f) \Rightarrow x
  \in A \wedge y \in B \Rightarrow (x, y) \in A \times B \Rightarrow f
  \subseteq A \times B$ and because of (3) we have that $f$ is a partial
  function.
\end{proof}

\begin{definition}
  {\index{$f (C)$}}{\index{$f^{- 1} (D)$}}Let $f : A \rightarrow B$ be a
  partial function then if $C \subseteq A$ and $D \subseteq B$ we define $f
  (C) = \{ y \in B| \exists x \in C \vdash (x, y) \in f \}$ (\tmtextbf{the
  image of C by f}) and $f^{- 1} (D) = \{ x \in A| \exists y \in D \vdash (x,
  y) \in F \}$ (\tmtextbf{the reverse image of D by f})
\end{definition}

\begin{theorem}
  \label{composition of partial functions}If $f : A \rightarrow B$ is a
  partial function and $g : C \rightarrow D$ is a partial function then $g
  \circ f : A \rightarrow D$ is a partial function.
\end{theorem}

\begin{proof}
  We use \ref{alternative definition of partial function} to prove this.
  \begin{enumerate}
    \item $g \circ f$ is a graph (see \ref{composition of graphs})
    
    \item Using \ref{properties of domains and ranges of graphs} we have that
    $\tmop{range} (g \circ f) \subseteq \tmop{range} (g) \subseteq D$
    
    \item Again using \ref{properties of domains and ranges of graphs} we have
    that $\tmop{dom} (g \circ f) \subseteq \tmop{dom} (f) \subseteq A$
    
    \item Finally if $(x, z) \in g \circ f \wedge (x, z') \in g \circ f$ then
    $\exists y, y'$ such that $(x, y) \in f \wedge (y, z) \in g \wedge (x, y')
    \in f \wedge (y', z') \Rightarrowlim_{f \tmop{is} a \tmop{partial}
    \tmop{function}} y = y' \Rightarrow (y, z) \in g \wedge (y, z') \in g
    \Rightarrowlim_{g \tmop{is} a \tmop{partial} \tmop{function}} z = z'$
    proving that $g \circ f$ is a partial function.
  \end{enumerate}
\end{proof}

\begin{theorem}
  \label{dom and range of a composition of partial functions}If $f : A
  \rightarrow B$ and $g : C \rightarrow D$ are partial functions then
  $\tmop{dom} (g \circ f) = \tmop{dom} (f) \bigcap f^{- 1} (\tmop{dom} (g))$
  and $\tmop{range} (g \circ f) = g \left( \tmop{range} (f) \bigcap \tmop{dom}
  (g) \right)$ for $g \circ f : A \rightarrow D$
\end{theorem}

\begin{proof}
  If $x \in \tmop{dom} (g \circ f) \Rightarrow \exists y \vdash (x, y) \in g
  \circ f \Rightarrow \exists z \vdash (x, z) \in f \wedge (z, y) \in g
  \Rightarrow x \in \tmop{dom} (f) \wedge z \in \tmop{dom} (g) \subseteq C
  \wedge (z, y) \in g \wedge (x, z) \in f \Rightarrow x \in \tmop{dom} (f)
  \wedge x \in f^{- 1} (\tmop{dom} (g)) \Rightarrow x \in \tmop{dom} (f)
  \bigcap f^{- 1} (\tmop{dom} (g))$ proving that
  \begin{equation}
    \label{eq 0.2.1} \tmop{dom} (g \circ f) \subseteq \tmop{dom} (f) \bigcap
    f^{- 1} (\tmop{dom} (g))
  \end{equation}
  If $x \in \tmop{dom} (f) \bigcap f^{- 1} (\tmop{dom} (g)) \Rightarrow x \in
  \tmop{dom} (f) \wedge x \in f^{- 1} (\tmop{dom} (g)) \Rightarrow \exists y
  \vdash (x, y) \in f \wedge \exists z \in \tmop{dom} (g) \vdash (x, z) \in f
  \Rightarrowlim_{f \tmop{is} a \tmop{partial} \tmop{function} \Rightarrow y =
  z} \exists y \vdash (x, y) \in f \wedge y \in \tmop{dom} (g) \Rightarrow
  \exists w, y \vdash (x, y) \in f \wedge (y, w) \in g \Rightarrow \exists w
  \vdash (x, w) \in g \circ f \Rightarrow x \in \tmop{dom} (g \circ f)$
  proving that
  \begin{equation}
    \label{eq 0.2.2} \tmop{dom} (f) \bigcap f^{- 1} (\tmop{dom} (g)) \subseteq
    \tmop{dom} (g \circ f)
  \end{equation}
  If $z \in \tmop{range} (g \circ f) \Rightarrow \exists x \vdash (x, z) \in g
  \circ f \Rightarrow \exists x, y \vdash (x, y) \in f \wedge (y, z) \in g
  \Rightarrow \exists y \vdash y \in \tmop{range} (f) \wedge y \in \tmop{dom}
  (g) \wedge (y, z) \in g \Rightarrow \exists y \vdash y \in \tmop{range} (f)
  \bigcap \tmop{dom} (g) \wedge (y, z) \in g \Rightarrow z \in g \left(
  \tmop{range} (f) \bigcap \tmop{dom} (g) \right)$ proving that
  \begin{equation}
    \label{eq 0.2.3} \tmop{range} (g \circ f) \subseteq g \left( \tmop{range}
    (f) \bigcap \tmop{dom} (g) \right)
  \end{equation}
  If $z \in g \left( \tmop{range} (f) \bigcap \tmop{dom} (g) \right)
  \Rightarrow \exists y \vdash y \in \tmop{range} (f) \wedge y \in \tmop{dom}
  (g) \wedge (y, z) \in g \Rightarrow \exists x, y \vdash (x, y) \in f \wedge
  (y, z) \in g \Rightarrow \exists x \vdash (x, z) \in g \circ f \Rightarrow z
  \in \tmop{range} (g \circ f)$ proving that
  \begin{equation}
    \label{eq 0.2.4} g \left( \tmop{range} (f) \bigcap \tmop{dom} (g) \right)
    \subseteq \tmop{range} (g \circ f)
  \end{equation}
  Using \ref{eq 0.2.1}, \ref{eq 0.2.2} we have that $\tmop{dom} (g \circ f) =
  \tmop{dom} (f) \bigcap f^{- 1} (\tmop{dom} (g))$ and using \ref{eq 0.2.3},
  \ref{eq 0.2.4} we have then $g \left( \tmop{range} (f) \bigcap \tmop{dom}
  (g) \right) = \tmop{range} (g \circ f)$ proving the theorem.$\text{}$
\end{proof}

\subsection{Functions}

\begin{definition}
  {\index{function}}{\index{mapping}}A partial function $f : A \rightarrow B$
  is a function if $\forall x \in A \vDash \exists y \in B \vdash (x, y) \in
  f$
\end{definition}

\begin{theorem}
  \label{alternative definition of a function (1)}A partial function $f : A
  \rightarrow B$ is a function iff $\tmop{dom} (f) = A$
\end{theorem}

\begin{proof}
  \
  
  $\Rightarrow$
  
  Assume that $f$ is a function. If $x \in \tmop{dom} (f) \Rightarrow \exists
  y \vdash (x, y) \in f \subseteq A \times B \Rightarrow x \in A \Rightarrow
  \tmop{dom} (f) \subseteq A$. If $x \in A \Rightarrowlim_{f \tmop{is} a
  \tmop{function}} \exists y \in B \vdash (x, y) \in f \Rightarrow x \in
  \tmop{dom} (f) \Rightarrow A \subseteq \tmop{dom} (f)$. From this we
  conclude that $\tmop{dom} (f) = A$
  
  $\Leftarrow$
  
  Assume that $f$ is a partial function and that $\tmop{dom} (f) = A$. Then
  \begin{eqnarray*}
    x \in A & \Leftrightarrow & x \in \tmop{dom} (f)\\
    & \Leftrightarrow & \exists y \in B \vdash (x, y) \in f\\
    & \Rightarrow & \forall x \in A \vDash \exists y \in B \vdash (x, y) \in
    f
  \end{eqnarray*}
\end{proof}

\begin{example}
  $\emptyset : \emptyset \rightarrow \emptyset$ is a function (the null
  function).
\end{example}

\begin{notation}
  If $f : A \rightarrow B$ is a partial function then for every $x \in
  \tmop{dom} (f)$ we have a unique $y \in B$ such that $(x, y) \in f$ we call
  this $y$ $f (x)$. So $y = f (x) \Leftrightarrow (x, y) \in f$
\end{notation}

Using this notation we can rewrite the definition of a function as follows

\begin{theorem}
  \label{alternative definition of a function}$\langle f, A, B \rangle$ is a
  function iff
  \begin{enumerate}
    \item $\forall x \in A \vDash \exists y \in B \vdash y = f (x)$
    
    \item If $y_1 = f (x) \wedge y_2 = f (x) \Rightarrow y_1 = y_2$
  \end{enumerate}
\end{theorem}

\begin{theorem}
  \label{condition for equality of functions}If $f : A \rightarrow B$ and $g :
  A \rightarrow B$ are two functions then $f = g$ iff $\forall x \in A \vdash
  f (x) = g (x)$
\end{theorem}

\begin{proof}
  
  \begin{description}
    \item[$\Rightarrow$] Assume that $f = g$ then if $x \in A = \tmop{dom} (f)
    = \tmop{dom} (g)$ there exists a $y, y' \vdash (x, y) \in f \wedge (x, y')
    \in g \Rightarrowlim_{f = g} (x, y), (x, y') \in g \Rightarrowlim_{g
    \tmop{is} a \tmop{function}} y = y' \Rightarrow f (x) = y = y' = g (x)
    \Rightarrow f (x) = g (x)$.
    
    \item[$\Leftarrow$] Assume that $\forall x \in A$ we have $f (x) = g (x)$
    then
    \begin{eqnarray*}
      (x, y) \in f & \Leftrightarrow & x \in A \wedge y = f (x)\\
      & \Leftrightarrow & x \in A \wedge y = g (x)\\
      & \Leftrightarrow & (x, y) \in g
    \end{eqnarray*}
  \end{description}
\end{proof}

\begin{definition}
  A partial function $f : A \rightarrow B$ is injective iff $(x, y) \in f
  \wedge (x', y) \in f \Rightarrow y = y'$. Using the $f (x)$ notation this is
  equivalent with $f (x) = f (x') \Rightarrow x = x'$.
\end{definition}

\begin{theorem}
  \label{composition of injective partial functions}If $f : A \rightarrow B$
  and $g : C \rightarrow D$ are injective partial functions then $g \circ f :
  A \rightarrow D$ is a injective partial function.
\end{theorem}

\begin{proof}
  By \ref{composition of partial functions} $g \circ f$ is a partial function.
  Now if $(x, z), (x', z) \in g \circ f$ then $\exists y, y' \vdash (x, y),
  (x', y') \in f \wedge (y, z), (y', z) \in g \Rightarrowlim_{g \tmop{is}
  \tmop{injective}} y = y' \Rightarrow (x, y), (x', y) \in f \Rightarrowlim_{f
  \tmop{is} \tmop{injective}} x = x'$ proving that $g \circ f$ is injective.
\end{proof}

\begin{definition}
  A partial function $f : A \rightarrow B$ is invertible if $f^{\um 1} : B
  \rightarrow A$ is a partial function
\end{definition}

\begin{lemma}
  \label{condition for invertible partial functions}A partial function $f : A
  \rightarrow B$ is invertible iff $f$ is injective
\end{lemma}

\begin{proof}
  
  
  $\Rightarrow$
  
  So assume that $f^{\um 1} : B \rightarrow A$ is a partial function then if
  $(x, y), (x', y) \in f \Rightarrow (y, x), (y, x') \in f^{- 1}
  \Rightarrowlim_{f^{- 1} \tmop{is} a \tmop{partial} \tmop{function}} x = x'$
  so we have that $f : A \rightarrow B$ is injective.
  
  $\Leftarrow$
  
  Assume now that $f$ is injective then we use \ref{alternative definition of
  partial function} to prove that $f^{- 1}$ is a partial function.
  \begin{enumerate}
    \item $x \in \tmop{range} (f^{- 1}) \Rightarrow \exists y \vdash (y, x)
    \in f^{- 1} \Rightarrow (x, y) \in f \Rightarrow x \in \tmop{dom} (f)
    \Rightarrowlim_{f \tmop{is} a \tmop{partial} \tmop{function}} x \in A$
    
    \item $x \in \tmop{dom} (f^{- 1}) \Rightarrow \exists y \vdash (x, y) \in
    f^{- 1} \Rightarrow (y, x) \in f \Rightarrow x \in \tmop{range} (f)
    \Rightarrowlim_{f \tmop{is} a \tmop{partial} \tmop{function}} x \in B$
    
    \item $(x, y), (x, y') \in f^{- 1} \Rightarrow (y, x), (y', x) \in f
    \Rightarrowlim_{f \tmop{is} \tmop{injective}} y = y'$
  \end{enumerate}
\end{proof}

\begin{corollary}
  \label{condition for invertable function}If $f : A \rightarrow B$ is a
  injective function then $f^{- 1} : f (B) \rightarrow A$ is a function
\end{corollary}

\begin{proof}
  By the above lemma we have that $f^{- 1} : B \rightarrow A$ is a partial
  function. To prove that it is a function, let $x \in f (B)$ then there exist
  a $y \in B$ such that $y = f (x)$ or $x = f^{- 1} (y)$, using
  \ref{alternative definition of a function} we have then that $f^{- 1} : f
  (B) \rightarrow A$ is a function.
\end{proof}

\begin{definition}
  A partial function $f : A \rightarrow B$ is surjective iff $\forall y \in B
  \vDash \exists x \in A \vdash (x, y) \in f$ (or $\forall y \in B \vDash
  \exists x \in A \vdash y = f (x)$)
\end{definition}

\begin{note}
  A partial function $f : A \rightarrow B$ is surjective iff $\tmop{range} (f)
  = B$
\end{note}

\begin{proof}
  \
  
  $\Rightarrow$
  
  Assume that $f$ is surjective. If $y \in \tmop{range} (f) \Rightarrow
  \exists x \in A \vdash (x, y) \in f \subseteq A \times B \Rightarrow y \in B
  \Rightarrow \tmop{range} (f) \subseteq B$. If $y \in B \Rightarrowlim_{f
  \tmop{is} \tmop{surjective}} \exists x \in A \vdash (x, y) \in f \Rightarrow
  y \in \tmop{range} (f) \Rightarrow B \subseteq \tmop{range} (f)$. So we
  conclude that $\tmop{range} (f) = B$
  
  $\Leftarrow$
  
  Assume that $\tmop{range} (f) = B$. Then if $y \in B \Rightarrow y \in
  \tmop{range} (f) \Rightarrow \exists x \in A \vdash (x, y) \in f \Rightarrow
  \forall y \in B \vDash \exists x \in A \vdash (x, y) \in f \Rightarrow f$ is
  surjective.
\end{proof}

\subsection{Bijections}

\begin{definition}
  A function $f : A \rightarrow B$ is bijective iff $f$ is surjective and $f$
  is injective.
\end{definition}

\begin{definition}
  \label{one to one correspondence}{\index{one to one
  correspondence}}{\index{$\approx$}}If $A$ and $B$ are classes such that
  there exists a bijective function $f : A \rightarrow B$ between $A$ and $B$
  then we say that $A$ an $B$ are \tmtextbf{bijective} or A and B are in
  \tmtextbf{a one to one correspondence} and we note this as $A \approx B$
\end{definition}

\begin{example}[Identity Function]
  \label{identity function}{\index{$1_A$}}If $A$ is a class then $1_A : A
  \rightarrow A$ defined by $1_A = \{ (x, x) | x \in A \nobracket \}$ is a
  bijection.
\end{example}

\begin{proof}
  We have trivially that $1_A \subseteq A \times A$ and $(x, y) \in 1_A \wedge
  (x, y') \in 1_A \Rightarrow y = x \wedge y' = x \Rightarrow y = y'$ so $1_A$
  is a partial function. Also $\forall x \in A \vDash (x, x) \in 1_A
  \Rightarrow \forall x \in A \vDash \exists y \in A \vdash (x, y) \in 1_A
  \Rightarrow f$ is a function. If $\forall x \in A \vDash (x, x) \in 1_A
  \Rightarrow \forall x \in A \vDash \exists y \in A \vdash (y, x) \in 1_A
  \Rightarrow 1_A$ is surjective. Finally if $(x, y) \in 1_A \wedge (x', y)
  \in 1_A \Rightarrow x = y \wedge x' = y \Rightarrow x = x'$ so $1_A$ is
  injective and thus bijective.
\end{proof}

\begin{example}[Inclusion Function]
  \label{inclusion function}{\index{inclusion function}}{\index{$e_A$}}If $A$
  is a class and $B \subseteq A$ is a subclass then $e_B : B \rightarrow A$
  defined by $e_B = \{ (x, x) | x \in B \nobracket \}$ is a injective
  function.
\end{example}

\begin{proof}
  We have trivially that $e_B \subseteq B \times B \subseteq B \times A$ and
  $(x, y) \in e_B \wedge (x, y') \in e_B \Rightarrow y = x = y' \Rightarrow y
  = y'$ so $e_B$ is a partial function. Next if $x \in B$ then $(x, x) \in e_B
  \Rightarrow \forall x \in B \vDash \exists y \in A \vdash (x, y) \in e_B
  \Rightarrow e_B$ is a function. Finally if $(x, y) \in e_B \wedge (x', y)
  \in e_B \Rightarrow x = y = x' \Rightarrow x = x'$ so $e_B$ is injective.
\end{proof}

\begin{example}
  \label{characteristics function}{\index{characteristics
  function}}{\index{$C_B$}}If $A$ is a class and $B \subseteq A$ then the
  characteristics function $C_B : A \rightarrow 2 = \{ 0, 1 \}$ is defined as
  follows $C_B = (B \times \{ 1 \}) \bigcup ((A \backslash B) \times \{ 0
  \})$. Then $C_B : A \rightarrow \{ 0, 1 \} = 2$ is a function
\end{example}

\begin{proof}
  We use \ref{alternative definition of partial function} to prove that $C_B$
  is a function
  \begin{enumerate}
    \item If $y \in \tmop{range} (C_B) \Rightarrow \exists x \vdash (x, y) \in
    C_B \Rightarrow x \in A = B \bigcup (A \backslash B)$ and we have two
    possible cases
    \begin{enumerate}
      \item $x \in B \Rightarrow y = 1$
      
      \item $x \nin B \Rightarrow x \in A \backslash B \Rightarrow y = 0$
    \end{enumerate}
    so $\tmop{range} (C_B) \subseteq \{ 0, 1 \} = 2$
    
    \item If $x \in A$ we have the following cases
    \begin{enumerate}
      \item $x \in B \Rightarrow (x, 1) \in C_B$
      
      \item $x \nin B \Rightarrow x \in (A \backslash B) \Rightarrow (x, 0)
      \in C_B$
    \end{enumerate}
    \item If $(x, y), (x, y') \in C_B$ then we have two excluding cases
    \begin{enumerate}
      \item If $x \in B \Rightarrow x \nin (A \backslash B) \Rightarrow (x,
      y), (x, y') \in B \times \{ 1 \} \Rightarrow y = 1 = y'$
      
      \item If $x \in A \backslash B \Rightarrow (x, y), (x, y') \in (A
      \backslash B) \times \{ 0 \} \Rightarrow y = 0 = y'$
    \end{enumerate}
  \end{enumerate}
\end{proof}

\begin{example}
  $\emptyset : \emptyset \rightarrow \emptyset$ is a a bijection
\end{example}

\begin{proof}[injective][surjective]
  
  \begin{enumerate}
    \item $\forall x, x' \in \emptyset$ we have that $\emptyset (x) =
    \emptyset (x')$ is satisfied vacuously.
    
    \item $\forall y \in \emptyset$ we have that there exists a $x \in
    \emptyset$ is satisfied vacuously.
  \end{enumerate}
\end{proof}

\subsection{Operations on functions}

\begin{theorem}
  \label{restricted partial function}{\index{restricted partial function}}If
  $f : A \rightarrow B$ is a partial function and $C \subseteq A$ is a
  subclass of $A$ then $f_{| C \nobracket} : C \rightarrow B$ with $f_{| C
  \nobracket} = \{ (x, y) \in f|x \in C \}$ is a partial function. Furthermore
  if $f$ is a function then $f_{| C \nobracket}$ is a function
\end{theorem}

\begin{proof}
  First as $f_{| C \nobracket} \subseteq f$ we have that $(x, y) \in f_{| C
  \nobracket} \wedge (x, y') \in f_{| C \nobracket} \Rightarrow (x, y) \in f
  \wedge (x, y') \in f \Rightarrowlim_{f \tmop{is} a \tmop{partial}
  \tmop{function}} y = y'$, also if $(x, y) \in f_{|C} \Rightarrow x \in C
  \wedge (x, y) \in f \Rightarrow (x, y) \in C \times B \Rightarrow f_{|C}
  \subseteq C \times B$ so that $f_{| C \nobracket}$ is a partial function.
  From \ref{alternative definition of partial function} we are left to prove
  that $C \subseteq \tmop{dom} (f_{| C \nobracket})$ if $f$ is a function. Now
  if $x \in C \subseteq A \Rightarrowlim_{f \tmop{is} a \tmop{function}} x \in
  A = \tmop{dom} (f) \Rightarrow \exists y \in B \vdash (x, y) \in f
  \Rightarrowlim_{x \in C} (x, y) \in f_{| C \nobracket} \Rightarrow x \in
  \tmop{dom} (f_{| C \nobracket}) \Rightarrow C \subseteq \tmop{dom} (f_{| C
  \nobracket})$
\end{proof}

\begin{note}
  If $f : A \rightarrow B$ is a (partial) function and $C \subseteq A$ then
  $f_{|C} : C \rightarrow B$ is sometimes written as $f : C \rightarrow B$
  meaning that its graph is $f_{|C}$ and $\tmop{dom} (f_{|C}) = C$.
\end{note}

\

\begin{theorem}
  If $f : A \bigcup B \rightarrow C$ is a partial function then $f = f_{| A
  \nobracket} \bigcup f_{| B \nobracket}$
\end{theorem}

\begin{proof}
  If $(x, y) \in f \subseteq \left( A \bigcup B \right) \times C \Rightarrow
  (x \in A \vee x \in B) \wedge (x, y) \in f \Rightarrow (x \in A \wedge (x,
  y) \in f) \vee (x \in B \wedge (x, y) \in f) \Rightarrow (x, y) \in f_{|A}
  \vee (x, y) \in f_{|B} \Rightarrow (x, y) \in f_{|A} \bigcup f_{|B}$ proving
  that $f \subseteq f_{|B} \bigcup f_{|C}$. As $f_{|A} \bigcup f_{|C}
  \subseteq f \bigcup f = f$ we have then finally $f = f_{|A} \bigcup f_{|B}$.
\end{proof}

\begin{theorem}
  \label{union of functions}If $f_1 : B \rightarrow A$ and $f_2 : C
  \rightarrow A$ are functions with $B \bigcap C = \emptyset$ then if $f = f_1
  \bigcup f_2$ the following hold
  \begin{enumerate}
    \item $f : B \bigcup C \rightarrow A$ is a function
    
    \item $f_1 = f_{| B \nobracket}$ and $f_2 = f_{| C \nobracket}$
    
    \item If $x \in B \Rightarrow f (x) = f_1 (x)$ and if $x \in C \Rightarrow
    f (x) = f_2 (x)$
  \end{enumerate}
\end{theorem}

\begin{proof}
  We begin the proof by proving
  \begin{enumeratealpha}
    \item $(x, y) \in f \wedge x \in B \Leftrightarrow (x, y) \in f_1$
    
    \begin{proof}
      If $(x, y) \in f \wedge x \in B$ then $(x, y) \in f_1 \vee (x, y) \in
      f_2$. Now if $(x, y) \in f_2 \Rightarrow (x, y) \in C \times A
      \Rightarrow x \in C \Rightarrow x \in B \bigcap C$ a contradiction. So
      we conclude that $(x, y) \in f_1$. On the other hand if $(x, y) \in f_1
      \Rightarrowlim_{f_1 \subseteq f \wedge f_1 \subseteq B \times A} (x, y)
      \in f \wedge x \in B$
    \end{proof}
    
    \item $(x, y) \in f \wedge x \in C \Leftrightarrow (x, y) \in f_2$
    
    \begin{proof}
      If $(x, y) \in f \wedge x \in C$ then $(x, y) \in f_1 \vee (x, y) \in
      f_2$. Now if $(x, y) \in f_1 \Rightarrow (x, y) \in B \times A
      \Rightarrow x \in B \Rightarrow x \in B \bigcap C$ a contradiction. So
      we conclude that $(x, y) \in f_2$. On the other hand if $(x, y) \in f_2
      \Rightarrowlim_{f_2 \subseteq f \wedge f_2 \subseteq C \times A} (x, y)
      \in f \wedge x \in C$
    \end{proof}
  \end{enumeratealpha}
  We proceed now as follows
  \begin{enumerate}
    \item If $(x, y) \in f \wedge (x, y') \in f$ then we have the following
    cases
    \begin{enumerate}
      \item $x \in B$ but then $(x, y) \in f_1 \wedge (x, y') \in f_1
      \Rightarrowlim_{f_{1 \tmop{is} a \tmop{function}}} y = y'$
      
      \item $x \in C$ but then $(x, y) \in f_2 \wedge (x, y') \in f_2
      \Rightarrowlim_{f_2 \tmop{is} a \tmop{function}} y = y'$.
    \end{enumerate}
    As we have also $f = f_1 \bigcup f_2 \subseteq (B \times A) \bigcup (C
    \times A) = \left( B \bigcup C \right) \times A$ we have that $f$ is
    indeed a partial function. To prove that it is a function consider
    \begin{eqnarray*}
      x \in B \bigcup C & \Rightarrow & x \in B \vee x \in C\\
      & \Rightarrowlim_{f_1, f_2 \tmop{are} \tmop{functions}} & (\exists y
      \in A \vdash (x, y) \in f_1) \vee (\exists y' \in A \vdash (x, y') \in
      f_2)\\
      & \Rightarrowlim_{f_1, f_2 \subseteq f_1 \bigcup f_2} & \left( \exists
      y \in A \vdash (x, y) \in f_1 \bigcup f_2 \right) \vee \left( \exists y'
      \in A \vdash (x, y') \in f_1 \bigcup f_2 \right)\\
      & \Rightarrow & x \in \tmop{dom} \left( f_1 \bigcup f_2 \right) \vee x
      \in \tmop{dom} \left( f_1 \bigcup f_2 \right)\\
      & \Rightarrow & x \in \tmop{dom} \left( f_1 \bigcup f_2 \right)
    \end{eqnarray*}
    and thus $\left\langle f_1 \bigcup f_2, B \bigcup C, A \right\rangle$ is a
    function.
    
    \item We have
    \begin{eqnarray*}
      (x, y) \in f_{| B \nobracket} & \Leftrightarrow & (x, y) \in f \bigcap
      (B \times A)\\
      & \Leftrightarrow & (x, y) \in f \wedge x \in B \wedge y \in A\\
      & \Leftrightarrowlim_{f \subseteq \left( B \bigcup C \right) \times A
      \Rightarrow y \in A} & (x, y) \in f \wedge x \in B\\
      & \Leftrightarrowlim_{f_1 \subseteq B \times A \Rightarrow x \in B} &
      (x, y) \in f_1\\
      (x, y) \in f_{| C \nobracket} & \Leftrightarrow & (x, y) \in f \bigcap
      (C \times A)\\
      & \Leftrightarrow & (x, y) \in f \wedge x \in C \wedge y \in A\\
      & \Leftrightarrowlim_{f \subseteq \left( B \bigcup C \right) \times A
      \Rightarrow y \in A} & (x, y) \in f \wedge x \in C\\
      & \Leftrightarrowlim_{f_2 \subseteq C \times A \Rightarrow x \in C} &
      (x, y) \in f_2
    \end{eqnarray*}
    \item Finally we have
    \begin{enumerate}
      \item $x \in B \Rightarrowlim_{\tmop{dom} (f) = B \bigcup C} \exists y
      \vdash (x, y) \in f \Rightarrow x \in B \wedge (x, y) \in f \Rightarrow
      (x, y) \in f_1 \Rightarrow f (x) = y = f_1 (x)$
      
      \item $x \in C \Rightarrowlim_{\tmop{dom} (f) = B \bigcup C} \exists y
      \vdash (x, y) \in f \Rightarrow x \in C \wedge (x, y) \in f \Rightarrow
      (x, y) \in f_2 \Rightarrow f (x) = y = f_2 (x)$
    \end{enumerate}
  \end{enumerate}
  
\end{proof}

\begin{notation}
  \label{union definition of functions}If $f_1 : B \rightarrow A$ and $f_2 : C
  \rightarrow A$ are functions, $B \bigcap C = \emptyset$ then $f_1 \bigcup
  f_2 : B \bigcup C \rightarrow A$ is noted by $f : B \bigcup C \rightarrow A$
  defined by $x \rightarrow f (x)$ where $f (x) = \left\{ \begin{array}{l}
    f_1 (x) \tmop{if} x \in B\\
    f_2 (x) \tmop{if} x \in C
  \end{array} \right.$, this is easily extended to more then two functions.
\end{notation}

\begin{theorem}
  \label{composition of functions (3)}If $f : A \rightarrow B$ and $g : C
  \rightarrow D$ are functions then $g \circ f : A \bigcap f^{- 1} (C)
  \rightarrow D$ is a function.
\end{theorem}

\begin{proof}
  As $f : A \rightarrow B$ and $g : C \rightarrow D$ are functions we have by
  \ref{composition of partial functions} that $g \circ f$ is a partial
  function. Using \ref{dom and range of a composition of partial functions} we
  have that $\tmop{dom} (g \circ f) = \tmop{dom} (f) \bigcap f^{- 1}
  (\tmop{dom} (g)) \equallim_{\tmop{dom} (f) = A \wedge \tmop{dom} (g) = C
  \tmop{by} \text{\ref{alternative definition of a function}}} A \bigcap f^{-
  1} (C)$ so that by \ref{alternative definition of a function (1)} again we
  have that $f : A \bigcap f^{- 1} (C) \rightarrow D$ is a function.
\end{proof}

\begin{theorem}
  \label{composition of functions}If $f : A \rightarrow B$ and $g : C
  \rightarrow D$ are functions with $\tmop{range} (f) \subseteq C$ then $g
  \circ f : A \rightarrow D$ is a function.
\end{theorem}

\begin{proof}
  If $\tmop{range} (f) \subseteq C$ then if $x \in A$ we have $f (x) \in C
  \Rightarrow f (A) \subseteq C \Rightarrow A \subseteq f^{- 1} (C)
  \Rightarrow A \bigcap f^{- 1} (C) = A$. So using the previous theorem we
  have that $f : A = A \bigcap f^{- 1} (C) \rightarrow D$ is a functction.
\end{proof}

\begin{theorem}
  \label{properties of compositions}We have the following properties of the
  composition of functions
  \begin{enumerate}
    \item If $f : A \rightarrow B$, $g : B - C$ and $h : C \rightarrow D$ are
    functions then $h \circ (g \circ f) : A \rightarrow D$ is equal to $(h
    \circ g) \circ f : A \rightarrow E$
    
    \item If $f : A \rightarrow B$ is a function then $f \circ 1_A : A
    \rightarrow B$ is equal to $f : A \rightarrow B$
    
    \item If $f : A \rightarrow B$ is a function then $1_B \circ f : A
    \rightarrow B$ is equal to $f : A \rightarrow B$
  \end{enumerate}
\end{theorem}

\begin{proof}
  
  \begin{enumerate}
    \item If $x \in A$ then $(h \circ (g \circ f)) (x) = h ((g \circ f) (x)) =
    h (g (f (x))) = (h \circ g) (f (x)) = ((h \circ g) \circ f) (x)$ so that
    $h \circ (g \circ f) = (h \circ g) \circ f$
    
    \item If $x \in A$ then $(f \circ 1_A) (x) = f (1_A (x)) = f (x)
    \Rightarrow f \circ 1_A = f$
    
    \item If $x \in A$ then $(1_B \circ f) (x) = 1_B (f (x)) = f (x)
    \Rightarrow 1_B \circ f = f$
  \end{enumerate}
\end{proof}

\begin{theorem}
  \label{composition of functions and function application}If $f : A
  \rightarrow B$ and $g : C \rightarrow D$ are functions with  then $\forall x
  \in A \bigcap f^{- 1} (C) \vDash (g \circ f) (x) = g (f (x))$
\end{theorem}

\begin{proof}
  We use \ref{condition for equality of functions} so if $x \in A \bigcap f^{-
  1} (C)$ then
  \begin{eqnarray*}
    z = g (f (x)) & \Leftrightarrow & (f (x), z) \in g\\
    & \Leftrightarrow & (y, z) \in g \wedge y = f (x)\\
    & \Leftrightarrow & (y, z) \in g \wedge (x, y) \in f\\
    & \Leftrightarrow & (x, z) \in g \circ f\\
    & \Leftrightarrow & z = (g \circ f) (x)
  \end{eqnarray*}
  
\end{proof}

\begin{definition}
  A function $f : A \rightarrow B$ is a invertible if $f^{- 1} : B \rightarrow
  A$ is a function.
\end{definition}

\begin{theorem}
  If $f : A \rightarrow B$ is a function then $f$ is invertible iff $f$ is
  bijective
\end{theorem}

\begin{proof}
  
  
  $\Rightarrow$
  
  Assume that $f$ is invertible then it is invertible as a partial function
  and thus by \ref{condition for invertible partial functions} we have that
  $f$ is injective. Now if $x \in B \Rightarrowlim_{f^{- 1} \tmop{is} a
  \tmop{function}} \exists y \vdash (x, y) \in f^{- 1} \Rightarrow (y, x) \in
  f \Rightarrow x \in \tmop{range} (f) \Rightarrow B \subseteq \tmop{range}
  (f) \subseteq B$ and $f$ is surjective. So we have that $f$ is bijective.
  
  $\Leftarrow$
  
  Assume that $f$ is bijective then it is injective so that $f^{- 1}$ is a
  partial function by \ref{condition for invertible partial functions}. Now if
  $x \in B \Rightarrowlim_{f \tmop{is} \tmop{surjective}} \exists y \vdash (y,
  x) \in f \Rightarrow (x, y) \in f^{- 1} \Rightarrow x \in \tmop{dom} (f^{-
  1}) \Rightarrow B \subseteq \tmop{dom} (f^{- 1}) \subseteq B$ so $\tmop{dom}
  (f^{- 1}) = B$ proving that $f^{- 1}$ is a function.
\end{proof}

\begin{theorem}
  \label{bijection and its inverse}If $f : A \rightarrow B$ is a bijection
  then $f^{- 1} : B \rightarrow A$ is a bijection.
\end{theorem}

\begin{proof}
  By the previous theorem we have that $f^{- 1}$ is invertible and thus a
  function. Now if $(x, y), (x', y) \in f^{- 1} \Rightarrow (y, x), (y, x')
  \in f \Rightarrowlim_{f \tmop{is} a \tmop{partial} \tmop{function}} x = x'
  \Rightarrow f^{- 1}$ is injective. If $x \in A \Rightarrowlim_{f \tmop{is} a
  \tmop{function}} x \in \tmop{dom} (f) \Rightarrow \exists y \vdash (x, y)
  \in f \Rightarrow (y, x) \in f^{- 1} \Rightarrow x \in \tmop{range} (f^{-
  1}) \Rightarrow A \subseteq \tmop{range} (f^{- 1}) \subseteq A \Rightarrow
  \tmop{range} (f^{- 1}) = A$ and thus $f^{- 1}$ is surjective.
\end{proof}

\begin{theorem}
  If $f : A \rightarrow B$ is bijective then
  \begin{enumerate}
    \item $f \circ f^{- 1} = 1_B$
    
    \item $f^{- 1} \circ f = 1_A$
  \end{enumerate}
\end{theorem}

\begin{proof}
  
  \begin{enumerate}
    \item $f \circ f^{- 1} : B \rightarrow B$ is a function as a composition
    of two functions. Further we have
    \begin{eqnarray*}
      (x, y) \in f \circ f^{- 1} & \Rightarrow & \exists z \vdash (x, z) \in
      f^{- 1} \wedge (z, y) \in f\\
      & \Rightarrow & \exists z \vdash (z, x) \in f \wedge (z, y) \in f\\
      & \Rightarrowlim_{f \tmop{is} a \tmop{function}} & x = y\\
      & \Rightarrowlim_{\tmop{dom} (f \circ f^{- 1}) = B} & (x, y) \in 1_B\\
      (x, y) \in 1_B & \Rightarrow & x \in B \wedge x = y\\
      & \Rightarrowlim_{f^{- 1} \tmop{is} \tmop{afunction}} & \exists z
      \vdash (x, z) \in f^{- 1}\\
      & \Rightarrow & \exists z \vdash (x, z) \in f^{- 1} \wedge (z, x) \in
      f\\
      & \Rightarrow & (x, x) \in f \circ f^{- 1}\\
      & \Rightarrowlim_{x = y} & (x, y) \in f \circ f^{- 1}
    \end{eqnarray*}
    \item $f^{- 1} \circ f : A \rightarrow A$ is a function as a composition
    of two functions. Further we have
    \begin{eqnarray*}
      (x, y) \in f^{- 1} \circ f & \Rightarrow & \exists z \vdash (x, z) \in f
      \wedge (z, y) \in f^{- 1}\\
      & \Rightarrow & \exists z \vdash (x, z) \in f \wedge (y, z) \in f\\
      & \Rightarrowlim_{f \tmop{is} \tmop{injective}} & x = y\\
      & \Rightarrowlim_{\tmop{dom} (f^{- 1} \circ f) = A} & (x, y) \in 1_A\\
      (x, y) \in 1_A & \Rightarrow & x \in A \wedge x = y\\
      & \Rightarrowlim_{f \tmop{is} a \tmop{function}} & \exists z \vdash (x,
      z) \in f\\
      & \Rightarrow & \exists z \vdash (x, z) \in f \wedge (z, x) \in f^{-
      1}\\
      & \Rightarrow & (x, x) \in f^{- 1} \circ f\\
      & \Rightarrowlim_{x = y} & (x, y) \in f^{- 1} \circ f
    \end{eqnarray*}
  \end{enumerate}
\end{proof}

\begin{theorem}
  \label{characterization of a bijective mapping}If $f : A \rightarrow B$ and
  $g : B \rightarrow A$ are two functions such that $f \circ g = 1_B$ and $g
  \circ f = 1_A$ then $f$ is bijective (hence invertible) and $g = f^{- 1}$.
\end{theorem}

\begin{proof}[injectivity][surjectivity]
  
  \begin{enumerate}
    \item
    \begin{eqnarray*}
      (x, y), (x', y) \in f & \Rightarrowlim_{y \in B, g \tmop{is} a
      \tmop{function}} & \exists z \vdash (y, z) \in g\\
      & \Rightarrow & (x, z), (x', z) \in g \circ f = 1_A\\
      & \Rightarrow & x = z = x' \Rightarrow x = x'
    \end{eqnarray*}
    \item
    \begin{eqnarray*}
      y \in B & \Rightarrow & (y, y) \in 1_B = f \circ g\\
      & \Rightarrow & \exists z \vdash (y, z) \in g \wedge (z, y) \in f\\
      & \Rightarrow & y \in \tmop{range} (f)\\
      & \tmop{so} \tmop{we} \tmop{have} & \\
      & B \subseteq \tmop{range} (f) \subseteq B & 
    \end{eqnarray*}
    and thus $B = \tmop{range} (f)$
    
    \item We have
    \begin{eqnarray*}
      (x, y) \in g & \Rightarrowlim_{g \subseteq B \times A} & y \in A\\
      & \Rightarrowlim_{\tmop{dom} (f) = A} & \exists z \vdash (y, z) \in f\\
      & \Rightarrow & (x, z) \in f \circ g = 1_B\\
      & \Rightarrow & x = z\\
      & \Rightarrow & (y, x) \in f\\
      & \Rightarrow & (x, y) \in f^{- 1}\\
      (x, y) \in f^{- 1} & \Rightarrow & (y, x) \in f\\
      & \Rightarrowlim_{f \subseteq A \times B} & x \in B\\
      & \Rightarrowlim_{\tmop{dom} (g) = B} & \exists z \vdash (x, z) \in g\\
      & \Rightarrow & (y, z) \in g \circ f = 1_A\\
      & \Rightarrow & y = z\\
      & \Rightarrow & (x, y) \in g
    \end{eqnarray*}
  \end{enumerate}
  proving that $g = f^{- 1}$
\end{proof}

We can summarize the two previous theorems in the following

\begin{theorem}
  \label{condition for beeing a inversible function}A function $f : A
  \rightarrow B$ is invertible (or bijective) iff there exists a function $g :
  B \rightarrow A$ such that $f \circ g = 1_B$ and $g \circ f = 1_A$. If such
  a $g$ exists then $g = f^{- 1}$. Note that for $g = f^{- 1}$ we have $g
  \circ f = 1_A \wedge f \circ g = 1_B$ so that $(f^{- 1})^{- 1} = f$
\end{theorem}

\begin{lemma}
  \label{image restriction of a function is a function}If $f : A \rightarrow
  B$ is a function and $C \supseteq \tmop{range} (f)$ then $f : A \rightarrow
  C$ is a function
\end{lemma}

\begin{proof}
  We use \ref{alternative definition of partial function} to prove that $f : A
  \rightarrow C$ is a function
  \begin{enumerate}
    \item $\tmop{range} (f) \subseteq C$ by the assumption of the lemma.
    
    \item $\tmop{dom} (f) = A$ because $f : A \rightarrow B$ is a function.
    
    \item If $(x, y), (x, y') \in f \Rightarrowlim_{f : A \rightarrow B
    \tmop{is} a \tmop{function}} y = y'$
  \end{enumerate}
\end{proof}

\begin{theorem}
  \label{union of bijections}If $f : A \rightarrow B$ and $g : C \rightarrow
  D$ are bijections and $A \bigcap C = \emptyset = B \bigcap D$ then $f
  \bigcup g : A \bigcup C \rightarrow B \bigcup D$ is a bijection.
\end{theorem}

\begin{proof}[injectivity][$(x, y) \in f \wedge (x', y) \in f$][$(x, y) \in f
\wedge (x', y) \in g$][$(x, y) \in g \wedge (x', y) \in f$][$(x, y) \in g
\wedge (x', y) \in g$][surjectivity]
  First from the two given functions we form the functions $f : A \rightarrow
  B \bigcup D$ and $g : C \rightarrow B \bigcup D$ so using \ref{union of
  functions} and $A \bigcap C$ we have that $f \bigcup g : A \bigcup C
  \rightarrow B \bigcup D$ is a function. We prove now that it is a bijection
  \begin{enumerate}
    \item If $(x, y), (x', y) \in f \bigcup g$ then we have for $y$ the
    following exclusive possibilities then we can have the following
    possibilities
    \begin{enumerate}
      \item then as $f$ is injective we have $x = x'$
      
      \item then $y \in B \wedge y \in D$ which is impossible as $B \bigcap D
      = \emptyset$ so this case can not occur.
      
      \item then $y \in D \wedge y \in B$ which is impossible as $B \bigcap D
      = \emptyset$ so this case can not occur.
      
      \item then as $g$ is injective we have $x = x'$
    \end{enumerate}
    this proves that in all possible cases we have $x = x'$ thus proving
    injectivity.
    
    \item If $y \in B \bigcup D$ then either $y \in B \Rightarrowlim_{f
    \tmop{is} \tmop{bijective}} \exists x \in A \vdash (x, y) \in f \subseteq
    f \bigcup g \Rightarrow (x, y) \in f \bigcup g$ or $y \in D
    \Rightarrowlim_{g \tmop{is} \tmop{bijective}} \exists x \in C \vdash (x,
    y) \in g \subseteq f \bigcup g \Rightarrow (x, y) \in f \bigcup g$
  \end{enumerate}
\end{proof}

\begin{theorem}
  \label{injective function implies function in other directory}If $f : A
  \rightarrow B$ is a function, $A \neq \emptyset$ then $f : A \rightarrow B$
  is injective iff there exists a function $g : B \rightarrow A$ such that $g
  \circ f = 1_A$
\end{theorem}

\begin{proof}
  \
  
  $\Rightarrow$
  
  Suppose $f$ is injective. Take $C = \tmop{range} (f) = f (A)$ then by the
  previous lemma we have that $f : A \rightarrow C$ is a function which
  moreover is surjective (for $C = \tmop{range} (f)$). Hence $f$ is a
  bijection. Thus by \ref{condition for beeing a inversible function} there
  exists a function $g' : f (A) \rightarrow A$ such that $g' \circ f = 1_A$.
  If $f (A) = B$ then $g = g' : B \rightarrow A$ is the required function, if
  $f (A) \neq B$ then as $A \neq \emptyset$ there exists a $x \in A$ then $g :
  B \rightarrow A$ defined by $g (y) = \left\{ \begin{array}{l}
    g' (y) \tmop{if} y \in f (A)\\
    x
  \end{array} \right.$ (see \ref{union definition of functions}) is the
  desired function.
  
  $\Leftarrow$
  
  Assume that there exists a function $g : B \rightarrow A$ such that $g \circ
  f = i_A$ then
  \begin{eqnarray*}
    (x, y), (x', y) \in f & \Rightarrowlim_{y \in B, \tmop{dom} (g) = B} &
    \exists z \vdash (y, z) \in g\\
    & \Rightarrow & (x, z), (x', z) \in g \circ f = 1_A\\
    & \Rightarrow & x = z = x'\\
    & \Rightarrow & x = x'
  \end{eqnarray*}
\end{proof}

\begin{theorem}
  \label{surjection implies function in other directory}A function $f : A
  \rightarrow B$ is surjective if there exists a $g : B \rightarrow A$ such
  that $f \circ g = 1_B$.
\end{theorem}

\begin{proof}
  
  \begin{eqnarray*}
    x \in B & \Rightarrow & (x, x) \in 1_B = f \circ g\\
    & \Rightarrow & \exists z \vdash (x, z) \in g \wedge (z, x) \in f\\
    & \Rightarrow & x \in \tmop{range} (f)
  \end{eqnarray*}
  So we have $B \subseteq \tmop{range} (f) \subseteq B \Rightarrow
  \tmop{range} (f) = B$ proving surjectivity.
\end{proof}

\begin{theorem}
  \label{properties of composition of functions}If $f : A \rightarrow B$ and
  $g : B \rightarrow C$ are functions then if $\tmop{range} (f) \subseteq B =
  \tmop{dom} (g)$ we have that $g \circ f : A \rightarrow C$ is a function. In
  addition we have the following
  \begin{enumerate}
    \item If $f : A \rightarrow B$ and $g : B \rightarrow C$ are injective
    then $g \circ f : A \rightarrow C$ is injective.
    
    \item If $f : A \rightarrow B$ and $g : B \rightarrow C$ are surjective
    then $g \circ f : A \rightarrow C$ is surjective.
    
    \item If $f : A \rightarrow B$ and $g : B \rightarrow C$ are bijective
    then $g \circ f : A \rightarrow C$ is bijective. Or equivalent if $f, g$
    are invertible we have that $g \circ f$ is invertible. Furthermore by
    \ref{properties of composition of mappings} we have that $(g \circ f)^{-
    1} = f^{- 1} \circ g^{- 1}$.
  \end{enumerate}
\end{theorem}

\begin{proof}
  
  \begin{enumerate}
    \item Let $(x, y), (x', y) \in g \circ f \Rightarrow \exists z, z' \vdash
    (x, z), (x', z') \in f, (z, y), (z', y) \in g \Rightarrowlim_{g \tmop{is}
    \tmop{injective}} z = z' \Rightarrow (x, z), (x', z) \in f
    \Rightarrowlim_{f \tmop{is} \tmop{injective}} x = x'$
    
    \item If $z \in C \Rightarrowlim_{g \tmop{is} \tmop{surjective}} \exists y
    \vdash (y, z) \in g \Rightarrowlim_{g \subseteq B \times C} y \in B
    \Rightarrowlim_{f \tmop{is} \tmop{surjective}} \exists x \vdash (x, y) \in
    f \Rightarrow (x, z) \in g \circ f \Rightarrow \tmop{range} (g \circ f)
    \supseteq C \Rightarrow \tmop{range} (g \circ f) = C$
    
    \item This follows from (1) and (2)
  \end{enumerate}
\end{proof}

Note: \ From now on we start using the notation $y = f (x)$ instead of $(x, y)
\in f$

\subsection{Images and preimages of functions}

\begin{theorem}
  \label{image of a restricted function}Let $f : A \rightarrow B$ be a
  function, $D \subseteq C \subseteq A$ then $f_{|C} (D) = f (D)$ and $f_{|D}
  = (f_{|C})_{|D}$
\end{theorem}

\begin{proof}
  First
  \begin{eqnarray*}
    y \in f_{|C} (D) & \Rightarrow & \exists x \in D \vdash (x, y) \in
    f_{|C}\\
    & \Rightarrow & \exists x \in D \vdash (x, y) \in f \wedge x \in C\\
    & \Rightarrow & \exists x \in D \bigcap C \vdash (x, y) \in f\\
    & \Rightarrow & y \in f \left( D \bigcap C \right)\\
    & \Rightarrowlim_{D \subseteq C} & y \in f (D)\\
    y \in f \left( D \bigcap C \right) & \Rightarrow & \exists x \in D \bigcap
    C \vdash (x, y) \in f\\
    & \Rightarrowlim & \exists x \in D \vdash x \in C \wedge (x, y) \in f\\
    & \Rightarrow & \exists c \in D \vdash (x, y) \in f_{|C}\\
    & \Rightarrow & y \in f_{|C} (D)
  \end{eqnarray*}
  Also if $(x, y) \in f_{|D}$ then $x \in D \wedge (x, y) \in f
  \Rightarrowlim_{D \subseteq C} x \in D \wedge (x \in C \wedge (x, y) \in f)
  \Rightarrow x \in D \wedge (x, y) \in f_{|C} \Rightarrow (x, y) \in
  (f_{|C})_{|D} \Rightarrow f_{|D} \subseteq (f_{|C})_{|D}$, proving that
  $f_{|D} \subseteq (f_{|C})_{|D}$, if $(x, y) \in (f_{|C})_{|D} \Rightarrow x
  \in D \wedge (x, y) \in f_{|C} \Rightarrow x \in D \wedge (x \in C \wedge
  (x, y) \in f) \Rightarrow x \in D \bigcap C \wedge (x, y) \in f
  \Rightarrowlim_{D \subseteq} x \in D \wedge (x, y) \in f \Rightarrow (x, y)
  \in f_{|D}$ proving that $(f_{|C})_{|D} \subseteq f_{|D}$. So we have
  $f_{|D} = (f_{|C})_{|D}$.
\end{proof}

\begin{theorem}
  \label{composition of a restricted function}Let $f : A \rightarrow B$ be a
  function, $g : C \rightarrow D$ a function, \ $E \subseteq A \bigcap f^{- 1}
  (C)$ then $(g \circ f)_{|E} = (g_{|f (E)}) \circ (f_{|E})$ so that \ $(g
  \circ f)_{|E} : E \rightarrow D$ is equal to the function $g_{|f (E)} \circ
  f_{|E} : E \rightarrow D$.
\end{theorem}

\begin{proof}
  First if $(x, y) \in (g \circ f)_{|E} \Rightarrow x \in E \wedge (x, y) \in
  (g \circ f) \Rightarrow \exists z \vdash x \in E \wedge (x, z) \in f \wedge
  (z, y) \in g = \exists z \vdash x \in E \wedge (x, z) \in E \wedge z \in f
  (E) \wedge (z, x) \in g \Rightarrow \exists z \vdash (x, z) \in f_{|E}
  \wedge (z, x) \in f_{|f (E)} \Rightarrow (x, y) \in g_{|f (E)} \circ f_{|E}$
  proving that $(g \circ f)_{|E} \subseteq g_{|f (E)} \circ f_{|E}$.
  
  Second if $(x, y) \in g_{|f (E)} \circ f_{|E}$ then $\exists z \vdash (x,
  z) \in f_{|E} \wedge (z, y) \in g_{|f (E)} \Rightarrow \exists z|x \in E
  \wedge (x, z) \in f \wedge (z, y) \in g \Rightarrow \exists z|x \in E \wedge
  (x, z) \in f \wedge (z, y) \in g \Rightarrow x \in E \wedge (x, y) \in g
  \circ f \Rightarrow (x, y) \in (g \circ f)_{|E}$ proving that $g_{|f (E)}
  \circ f_{|E} \subseteq (g \circ f)_{|E}$. And thus we have $g_{|f (E)} \circ
  f_{|E} = (g \circ f)_{|E}$.
  
  As $E \subseteq A \bigcap f^{- 1} (C)$ we have that $f (E) \subseteq C$ so
  that $\tmop{dom} (g_{|f (E)}) = f (E)$ and thus as $\tmop{dom} (f_{|E}) = E$
  we have that $\tmop{dom} (g_{|f (E)} \circ f_{|E}) = \tmop{dom} (f_{|E})
  \bigcap f^{- 1} (\tmop{dom} (g_{|f (E)})) = E \bigcap f^{- 1} (f (E))
  \equallim_{f^{- 1} (f (E)) \subseteq E} E$ proving that $(g \circ f)_{|E} :
  E \rightarrow D$.
\end{proof}

\begin{theorem}
  \label{f|C is bijective if f is injective}If $f : A \rightarrow B$ is a
  injective function and $C \subseteq A$ then $f_{|C} : C \rightarrow f (C)$
  is a bijection. Further we have that $(f_{|C})^{- 1} = (f^{- 1})_{|f (C)}$
  so that the following bijections are equal $(f_{|C})^{- 1} : f (C)
  \rightarrow C, (f^{- 1})_{|f (C)} : f (C) \rightarrow C$
\end{theorem}

\begin{proof}[injectivity][surjectivity]
  First we prove bijectivity
  \begin{enumerate}
    \item If $(x, y), (x, y') \in f_{|C} \subseteq f \Rightarrow (x, y), (x,
    y') \in f \Rightarrow y = y' \Rightarrow f_{|C}$ is injective
    
    \item If $y \in f (C) \Rightarrow \exists x \in C \vdash (x, y) \in f
    \Rightarrow (x, y) \in f \bigcap (C \times f (C)) \Rightarrow (x, y) \in
    f_{|C} \Rightarrow f_{|C}$ is surjective.
  \end{enumerate}
  Next if $(x, y) \in (f_{|C})^{- 1} \Rightarrow (y, x) \in f_{|C} \Rightarrow
  y \in C \wedge (y, x) \in f \Rightarrow y \in C \wedge (x, y) \in f^{- 1}
  \Rightarrow (x, y) \in (f^{- 1})_{|C}$ proving that $(f_{|C})^{- 1}
  \subseteq (f^{- 1})_{|C}$, if $(x, y) \in (f^{- 1})_{|C} \Rightarrow x \in C
  \wedge (x, y) \in f^{- 1} \Rightarrow x \in C \wedge (y, x) \in f
  \Rightarrow (y, x) \in f_{|C} \Rightarrow (x, y) \in (f_{|C})^{- 1}$ proving
  that $(f^{- 1})_{|C} \subseteq (f_{|C})^{- 1}$. So we conclude that
  $(f_{|C})^{- 1} = (f^{- 1})_{|C}$.
\end{proof}

\begin{theorem}
  \label{preimage of a bijection}Let $f : A \rightarrow B$ be a bijection then
  \begin{enumerate}
    \item If $C \subseteq B$ then $f^{- 1} (C) = (f^{- 1}) (C)$
    
    \item If $C \subseteq A$ then $(f^{- 1})^{- 1} (C) = f (C)$
  \end{enumerate}
\end{theorem}

\begin{proof}
  
  \begin{enumerate}
    \item If $x \in f^{- 1} (C)$ then $f (x) \in C$ and thus $x = (f^{- 1}) (f
    (x)) \in (f^{- 1}) (C)$. If $x \in (f^{- 1}) (C)$ then there exists a $y
    \in C$ with $x = (f^{- 1}) (y) \Rightarrow f (x) = f ((f^{- 1}) (y)) = (f
    \circ f^{- 1}) (y) = y \in C \Rightarrow x \in f^{- 1} (C)$.
    
    \item As $f^{- 1} : B \rightarrow A$ is also a bijection take $C \subseteq
    A$ then using (1) we have $(f^{- 1})^{- 1} (C) = ((f^{- 1})^{- 1}) (C) = f
    (C)$
  \end{enumerate}
\end{proof}

\begin{theorem}
  \label{bijections and exclusions}If $f : A \rightarrow B$ is a bijection and
  $C \subseteq A$ then $f_{| (A \backslash C)} : A \backslash C \rightarrow B
  \backslash f (C)$ is a bijection
\end{theorem}

\begin{proof}[injective][surjective]
  
  \begin{enumerate}
    \item If $x, x' \in A \backslash C$ is such that $f_{| (A \backslash C)}
    (x) = f_{| (A \backslash C)} (x') \Rightarrow f (x) = f (x') \Rightarrow x
    = x'$.
    
    \item If $y \in B \backslash f (C)$ then as $f$ is surjective there exists
    a $x \in A$ such that $f (x) = y$, if now $x \in C \Rightarrow y = f (x)
    \in f (C)$ contradicting $y \in B \backslash f (C)$ so we have $x \nin C
    \Rightarrow x \in A \backslash C$ proving surjectivity.
  \end{enumerate}
\end{proof}

\begin{theorem}
  Let $f : A \rightarrow B$ be a function then if $C, D$ are classes with
  \begin{enumerate}
    \item If $C \subseteq A \wedge D \subseteq A$ and $C = D \Rightarrow f (C)
    = f (D)$
    
    \item If $C \subseteq B \wedge D \subseteq B$ and $C = D \Rightarrow f^{-
    1} (C) = f^{- 1} (D)$
  \end{enumerate}
\end{theorem}

\begin{proof}
  
  \begin{enumerate}
    \item If $C = D$ then
    \begin{eqnarray*}
      y \in f (C) & \Leftrightarrow & \exists x \in C \vdash y = f (x)\\
      & \Leftrightarrowlim_{C = D} & \exists x \in D \vdash y = f (x)\\
      & \Leftrightarrow & y \in f (D)
    \end{eqnarray*}
    \item If $C = D$ then
    \begin{eqnarray*}
      y \in f^{- 1} (C) & \Leftrightarrow & f (y) \in C\\
      & \Leftrightarrowlim_{C = D} & f (y) \in D\\
      & \Leftrightarrow & y \in f^{- 1} (D)
    \end{eqnarray*}
  \end{enumerate}
\end{proof}

Note that the opposite is not true so from $f (C) = f (D)$ (or $f^{- 1} (C) =
f^{- 1} (D)$) it does not follow that $C = D$

\begin{theorem}
  \label{surjective function and image/preimage}If $f : A \rightarrow B$ is a
  function and $C \subseteq B$ then $f (f^{- 1} (C)) \subseteq C$ if $f$ is
  surjective then $f (f^{- 1} (C)) = C$
\end{theorem}

\begin{proof}
  If $x \in f (f^{- 1} (C)) \Rightarrowlim \exists y \in f^{- 1} (C)$ so that
  $x = f (y) \Rightarrowlim_{y \in f^{- 1} (C)} x = f (y) \in C \Rightarrow x
  \in C$. If $x \in C \Rightarrowlim_{f \tmop{is} \tmop{surjective}} \exists y
  \in A$ such that $C \ni x = f (y) \Rightarrow y \in f^{- 1} (C) \Rightarrow
  x = f (y) \in f (f^{- 1} (C))$
\end{proof}

\begin{theorem}
  \label{injective function and preimage/image}If $f : A \rightarrow B$ is a
  injective function and $C \subseteq A$ then $(f^{- 1}) (f (C)) = C$
\end{theorem}

\begin{proof}
  If $x \in f^{- 1} (f (C)) \Rightarrow f (x) \in f (C) \Rightarrow \exists y
  \in C$ such that $f (x) = f (y) \Rightarrowlim_{f \tmop{is}
  \tmop{injective}} x = y \Rightarrow x \in C$. If $x \in C \Rightarrow f (x)
  \in f (C) \Rightarrow x \in f^{- 1} (f (C))$
\end{proof}

\begin{theorem}
  Let $A$ and $B$ be sets (so $\mathcal{P} (A)$ and $\mathcal{P} (B)$ are sets
  by the axiom of power sets \ref{axiom of power sets}) and $f : A \rightarrow
  B$ a function. If we define now $\bar{f} = \{ (C, D) | C \in \mathcal{P} (A)
  \wedge D = f (C) \nobracket \} \equallim_{\tmop{shorthand}} \{ (C, f (C)) |
  C \in \mathcal{P} (A) \nobracket \}$ and $\breve{f} = \{ (C, D) | C \in
  \mathcal{P} (B) \wedge D = f^{- 1} (C) \nobracket \}
  \equallim_{\tmop{shorthand}} \{ (C, f^{- 1} (C)) | C \in \mathcal{P} (B)
  \nobracket \}$ then we have that the following are functions
  \begin{eqnarray*}
    \bar{f} : \mathcal{P} (A) \rightarrow \mathcal{P} (B) &  & \\
    \breve{f} : \mathcal{P} (B) \rightarrow \mathcal{P} (A) &  & 
  \end{eqnarray*}
\end{theorem}

where $\tilde{f}$ and $\check{f}$ are defined by $\bar{f} (C) = f (C)$ and
$\breve{f} (D) = f^{- 1} (D)$

\begin{proof}
  We use \ref{alternative definition of partial function} to prove this
  \begin{enumerate}
    \item
    \begin{enumerate}
      \item $D \in \tmop{range} (\bar{f}) \Rightarrow \exists C \vdash (C, D)
      \in \bar{f} \Rightarrow D = f (C) \subseteq B \Rightarrow D \in
      \mathcal{P} (B) \Rightarrow \tmop{range} (\bar{f}) \subseteq \mathcal{P}
      (B)$
      
      \item $D \in \tmop{range} (\breve{f}) \Rightarrow \exists C \vdash (C,
      D) \in \breve{f} \Rightarrow D = f^{- 1} (C) \subseteq A \Rightarrow D
      \in \mathcal{P} (A) \Rightarrow \tmop{range} (\breve{f}) \subseteq
      \mathcal{P} (A)$
    \end{enumerate}
    \item
    \begin{enumerate}
      \item $D \in \mathcal{P} (A) \Rightarrow (D, f (D)) \in \bar{f}
      \Rightarrow D \in \tmop{dom} (\bar{f}) \Rightarrow \mathcal{P} (A)
      \subseteq \tmop{dom} (\bar{f})$
      
      \item $D \in \mathcal{P} (B) \Rightarrow (D, f^{- 1} (D)) \in \breve{f}
      \Rightarrow D \in \tmop{dom} (\breve{f}) \Rightarrow \mathcal{P} (B)
      \subseteq \tmop{dom} (\breve{f})$
    \end{enumerate}
    \item
    \begin{enumerate}
      \item If $(C, D), (C, D') \in \bar{f} \Rightarrow D = f (C) \wedge D' =
      f (C)$ and by the previous theorem we have $D = D'$
      
      \item If $(C, D), (C, D') \in \breve{f} \Rightarrow D = f^{- 1} (C)
      \wedge D' = f^{- 1} (C)$ and by the previous theorem we have $D = D'$
    \end{enumerate}
  \end{enumerate}
\end{proof}

\begin{theorem}
  \label{properties of image and preimage}If $f : A \rightarrow B$ is a
  function then we have
  \begin{enumerate}
    \item If $C, D \subseteq A$ with $C \subseteq D$ then $f (C) \subseteq f
    (D)$
    
    \item If $C, D \subseteq B$ with $C \subseteq D$ then $f^{- 1} (C)
    \subseteq f^{- 1} (D)$
    
    \item If $C \subseteq A \wedge D \subseteq B \vdash f (C) \subseteq D$
    then $C \subseteq f^{- 1} (D)$
    
    \item If $C \subseteq B$ then $f^{- 1} (B\backslash C) = A\backslash f^{-
    1} (C)$
    
    \item If $f$ is injective so that $f^{- 1} : f (A) \rightarrow A$ is a
    function and $C \subseteq A$ then $(f^{- 1})^{- 1} (C) = f (C)$
  \end{enumerate}
\end{theorem}

\begin{proof}
  
  \begin{enumerate}
    \item If $y \in f (C)$ then $\exists x \in C$ such that $y = f (x)
    \Rightarrowlim_{C \subseteq D} x \in D$ such that $y = f (x) \Rightarrow x
    \in f (D)$
    
    \item If $x \in f^{- 1} (C)$ then $f (x) \in C \Rightarrowlim_{C \subseteq
    D} f (x) \in D \Rightarrow x \in f^{- 1} (D)$
    
    \item If $x \in C$ then $f (x) \in f (C) \subseteq D \Rightarrow f (x) \in
    D \Rightarrow x \in f^{- 1} (D)$
    
    \item If $x \in f^{- 1} (B\backslash C)$ then $f (x) \in B\backslash C
    \Rightarrowlim_{f^{- 1} (B) \subseteq A} x \in A \wedge f (x) \nin C
    \Rightarrow x \in A \wedge x \nin f^{- 1} (C) \Rightarrow x \in
    A\backslash f^{- 1} (C)$. If $x \in A\backslash f^{- 1} (C) \Rightarrow x
    \in A \wedge x \nin f^{- 1} (C) \Rightarrow f (x) \in B \wedge f (x) \nin
    C \Rightarrow f (x) \in B\backslash C \Rightarrow x \in f^{- 1}
    (B\backslash C)$
    
    \item If $y \in (f^{- 1})^{- 1} (C)$ then $f^{- 1} (y) \in C$ so that $y =
    f (f^{- 1} (y)) \in f (C) \Rightarrow (f^{- 1})^{- 1} (C) \subseteq f
    (C)$. If $y \in f (C)$ then there exists a $x \in C$ such that $y = f (x)
    \Rightarrow C \ni x = f^{- 1} (y) \Rightarrow y \in (f^{- 1})^{- 1} (C)$
  \end{enumerate}
\end{proof}

\begin{definition}
  If $f : A \rightarrow B$ is a function and $\{ C_i \}_{i \in I}$ is a family
  so that $\forall i \in I$ we have $C_i \subseteq A$ [in other words $\{ C_i
  \}_{i \in I}$ is a family of sub-classes of $A$]. The we define $\{ f (C_i)
  \}_{i \in I}$ to be the family formed by $\tmop{graph} (\{ f (C_i) \}_{i \in
  I}) = \{ (i, x) |i \in I \wedge x \in f (C_i) \}$.
\end{definition}

The following theorem motivates why we use the notation $\{ f (C_i)_{i \in I}
\}$

\begin{theorem}
  If $f : A \rightarrow B$ is a function and $\{ C_i \}_{i \in I}$ is a family
  of sub-classes of $A$ and $G = \langle \tmop{graph} (\{ f (C_i) \}_{i \in
  I}), I \rangle$ then $\forall i \in I$ we have that $G_i = f (C_i)$
\end{theorem}

\begin{proof}
  If $i \in I$ then if $x \in G_i \Rightarrow (i, x) \in G \Rightarrow i \in I
  \wedge x \in f (C_i) \Rightarrow x \in f (C_i) \Rightarrow G_i \subseteq f
  (C_i)$. If $x \in f (C_i) \Rightarrowlim_{i \in I} (i, x) \in G \Rightarrow
  x \in G_i \Rightarrow f (C_i) \subseteq G_i \Rightarrow f (C_i) = G_i$
\end{proof}

Once we have defined $\{ f (C_i) \}_{i \in I}$ we can prove the following
theorem.

\begin{theorem}
  \label{image (preimage) of union , intersections}If $f : A \rightarrow B$ is
  a function, let $\{ C_i \}_{i \in I}$ be a family of sub-classes of $A$ and
  let $\{ D_i \}_{i \in I}$ be a family of sub-classes of $B$ then we have
  \begin{enumerate}
    \item $f \left( \bigcup_{i \in I} C_i \right) = \bigcup_{i \in I} f (C_i)$
    
    \item $f^{- 1} \left( \bigcup_{i \in I} D_i \right) = \bigcup_{i \in I}
    f^{- 1} (D_i)$
    
    \item $f \left( \bigcap_{i \in I} C_i \right) \subseteq \bigcap_{i \in I}
    f (C_i)$, if $f$ is injective then $f \left( \bigcap_{i \in I} C_i \right)
    = \bigcap_{i \in I} f (C_i)$
    
    \item $f^{- 1} \left( \bigcap_{i \in I} D_i \right) = \bigcap_{i \in I}
    f^{- 1} (D_i)$
  \end{enumerate}
\end{theorem}

\begin{proof}[$I = \emptyset$][$I \neq 0$]
  
  \begin{enumerate}
    \item
    \begin{eqnarray*}
      y \in f \left( \bigcup_{i \in I} C_i \right) & \Leftrightarrow & \exists
      x \in \bigcup_{i \in I} C_i \vdash y = f (x)\\
      & \Leftrightarrow & \exists x \vdash \exists i \in I \vdash x \in C_i\\
      & \Leftrightarrow & \exists i \in I \vdash \exists x \in C_i \vdash y =
      f (x)\\
      & \Leftrightarrow & \exists i \in I \vdash y \in f (C_i)\\
      & \Leftrightarrow & y \in \bigcup_{i \in I} f (C_i)
    \end{eqnarray*}
    \item
    \begin{eqnarray*}
      x \in f^{- 1} \left( \bigcup_{i \in I} D_i \right) & \Leftrightarrow & f
      (x) \in \bigcup_{i \in I} D_i\\
      & \Leftrightarrow & \exists i \in I \vdash f (x) \in D_i\\
      & \Leftrightarrow & \exists i \in I \vdash x \in f^{- 1} (D_i)\\
      & \Leftrightarrow & \bigcup_{i \in I} f^{- 1} (D_i)
    \end{eqnarray*}
    \item
    \begin{eqnarray*}
      y \in f \left( \bigcap_{i \in I} C_i \right) & \Leftrightarrow & \exists
      x \in \bigcap_{i \in I} C_i \vdash y = f (x)\\
      & \Leftrightarrow & \exists x \vdash \forall i \in I \vDash x \in C_i
      \wedge y = f (x)\\
      & \Rightarrow & \forall i \in I \vdash \exists x \in C_i \wedge y = f
      (x)\\
      & \Rightarrow & \forall i \in I \vdash y \in f (C_i)\\
      & \Rightarrow & y \in \bigcap_{i \in I} f (C_i)
    \end{eqnarray*}
    If $f$ is injective then we have to prove that $\bigcap_{i \in I} f (C_i)
    \subseteq f \left( \bigcap_{i \in I} C_i \right)$. Consider the following
    cases for $I$
    \begin{enumerate}
      \item then $\bigcap_{i \in I} C_i \equallim_{\text{\ref{union and
      intersection of emptyset}}} \mathcal{U} \equallim_{\text{\ref{union and
      intersection of emptyset}}} f \left( \bigcap_{i \in I} C_i \right)
      \text{}$
      
      \item as $f \left( \bigcap_{i \in I} C_i \right) \subseteq \bigcap_{i
      \in I} f (C_i)$ we must prove that $\bigcap_{i \in I} f (C_i) \backslash
      f \left( \bigcap_{i \in I} C_i \right) = \emptyset$. We prove this by
      contradiction, so let $y \in \bigcap_{i \in I} f (C_i) \backslash f
      \left( \bigcap_{i \in I} C_i \right)$ \ then as $I \neq \emptyset$ there
      exists a $i_0 \in I$ with $y \in f (C_{i_0}) \Rightarrow \exists x_{i_0}
      \in C_{i_0} \vdash y = (x_{i_0})$, as $y \nin f \left( \bigcap_{i \in I}
      C_i \right)$ we must have that $x_{i_0} \nin \bigcap_{i \in I} C_i$ so
      that there exists a $i_1 \in I$ with $x_{i_0} \nin C_{i_1}$ As $y \in
      \bigcap_{i \in I} f (C_i)$ there exists a $x_{i_1} \in C_{i_1}$ with $y
      = f (x_{i_1})$, so we have that $f (x_{i_0}) = y = f (x_{i 1})
      \Rightarrowlim_{f \tmop{is} \tmop{injective}} x_{i_0} = x_{i_1}$ but
      this would mean that $x_{i_0} \in C_{i_1}$ a contradiction.
    \end{enumerate}
    
    
    \item
    \begin{eqnarray*}
      y \in f^{- 1} \left( \bigcap_{i \in I} C_i \right) & \Leftrightarrow & f
      (y) \in \bigcap_{i \in I} C_i\\
      & \Leftrightarrow & \forall i \in I \vdash f (y) \in C_i\\
      & \Leftrightarrow & \forall i \in I \vdash y \in f^{- 1} (C_i)\\
      & \Leftrightarrow & y \in \bigcap f^{- 1} (C_i)
    \end{eqnarray*}
  \end{enumerate}
  
\end{proof}

\begin{theorem}
  \label{injective functions preserve set difference}Let $f : A \rightarrow B$
  be injective then if $X, Y \subseteq A$ we have $f (X\backslash Y) = f (X)
  \backslash f (Y)$ 
\end{theorem}

\begin{proof}
  Let $y \in f (X\backslash Y)$ then we have $y = f (x)$ where $x \in
  X\backslash Y$ then $x \in X$ so that $y = f (x) \in f (X)$, if now $y \in f
  (Y)$ then there exists a $z \in Y$ such that $y = f (z) = f (x)
  \Rightarrowlim_{\tmop{injectivity}} z = x \nin Y$ a contradiction, so we
  have $y \nin f (Y)$. This proves that $f (X\backslash Y) \subseteq f (X)
  \backslash f (Y)$. If $y \in f (X) \backslash f (Y)$ then $y \in f (X)
  \wedge y \nin f (Y)$ so there exists a $x \in X$ such that $y = f (x) \wedge
  y \nin f (Y)$. If now $x \in Y$ then $y = f (x) \in f (Y)$ a contradiction
  so we must have $x \nin Y$ but then $y \in f (X\backslash Y)$ proving that
  $f (X) \backslash f (Y) \subseteq f (X\backslash Y)$. Finally we conclude
  from the above that $f (X\backslash Y) = f (X) \backslash f (Y)$
\end{proof}

The following theorem is going to be very important for manifolds.

\begin{theorem}
  \label{f@g^-1}If $f : A \rightarrow E$ and $g : C \rightarrow E$ be
  injections (so that $f : A \rightarrow f (A)$ and $g : C \rightarrow g (C)$
  are bijections (see \ref{f|C is bijective if f is injective})) then
  \begin{enumerate}
    \item $\tmop{dom} (f \circ g^{- 1}) = g \left( A \bigcap C \right)$
    
    \item $f \circ g^{- 1} = (f \circ g^{- 1})_{|g \left( A \bigcap C \right)}
    = f_{| \left( A \bigcap C \right)} \circ (g^{- 1})_{|g \left( A \bigcap C
    \right)} = f_{| \left( A \bigcap C \right)} \circ \left( g_{|A \bigcap C}
    \right)^{- 1}$
    
    \item If $D \subseteq A \bigcap C$ then $(f \circ g^{- 1})_{|g (D)} =
    f_{|D} \circ (g_{|D})^{- 1}$ and $(f \circ g^{- 1})_{|g (D)} : g (D)
    \rightarrow f (D)$ is a bijection that is the composition of the
    bijections $(g^{- 1})_{|g (D)} : g (D) \rightarrow D$ and $f_{|D} : D
    \rightarrow f (D)$. Note that if we take $D = A \bigcap C$ then using (1)
    we have that $f \circ g^{- 1} : g \left( A \bigcap C \right) \rightarrow f
    \left( A \bigcap C \right)$ is a bijection that is the composition of
    $\left( g_{|A \bigcap C} \right)^{- 1} : g \left( A \bigcap C \right)
    \rightarrow A \bigcap C$ and $f_{|A \bigcap C} : A \bigcap C \rightarrow f
    \left( A \bigcap C \right)$.
  \end{enumerate}
\end{theorem}

\begin{proof}
  
  \begin{enumerate}
    \item Using \ref{dom and range of a composition of partial functions} we
    have that $\tmop{dom} (f \circ g^{- 1}) = \tmop{dom} (g^{- 1}) \bigcap
    (g^{- 1})^{- 1} (\tmop{dom} (f)) \equallim_{f : A \rightarrow f (A)
    \tmop{and} g^{- 1} : g (C) \rightarrow C \tmop{are} \tmop{bijections}} g
    (C) \bigcap (g^{- 1})^{- 1} (A) \equallim_{\text{\ref{preimage of a
    bijection}}} g (C) \bigcap g (A) \equallim_{\text{\ref{image (preimage) of
    union , intersections}}} g \left( C \bigcap A \right) = g \left( A \bigcap
    C \right)_{}$
    
    \item If $(x, y) \in f \circ g^{- 1} \Rightarrow \exists z \vdash (x, z)
    \in g^{- 1} \wedge (z, y) \in f \Rightarrow \exists z \vdash (z, x) \in g
    \wedge (z, y) \in g \Rightarrowlim_{f \subseteq A \times E \wedge g
    \subseteq C \times E} \exists z \vdash z \in C \wedge (z, x) \in g \wedge
    z \in A \wedge (z, y) \in f \Rightarrow \exists z \vdash z \in A \bigcap C
    \wedge (z, x) \in g \wedge (z, y) \in f \Rightarrowlim_{z \in A \bigcap C
    \wedge (z, x) \in g \Rightarrow x \in g \left( A \bigcap C \right)} x \in
    g \left( A \bigcap C \right) \wedge \exists z \vdash (z, x) \in g \wedge
    (z, y) \in f \Rightarrow x \in g \left( A \bigcap C \right) \wedge \exists
    z \vdash (x, z) \in g^{- 1} \wedge (z, y) \in f \Rightarrow x \in g \left(
    A \bigcap C \right) \wedge (x, y) \in f \circ g^{- 1} \Rightarrow (x, y)
    \in (f \circ g^{- 1})_{|g \left( A \bigcap C \right)} \Rightarrow f \circ
    g^{- 1} \in (f \circ g^{- 1})_{|g \left( A \bigcap C \right)}$ proving
    that $f \circ g^{- 1} \subseteq (f \circ g^{- 1})_{|g \left( A \bigcap C
    \right)}$, as $(f \circ g^{- 1})_{|g \left( A \bigcap C \right)} \subseteq
    f \circ g^{- 1}$ we have thus $f \circ g^{- 1} = (f \circ g^{- 1})_{|g
    \left( A \bigcap C \right)}$. Now $f \circ g^{- 1} = (f \circ g^{- 1})_{|g
    \left( A \bigcap C \right)} \equallim_{\text{\ref{composition of a
    restricted function}}} f_{| (g^{- 1}) \left( g \left( A \bigcap C \right)
    \right)} \circ (g^{- 1})_{|g \left( A \bigcap C \right)}
    \equallim_{\text{\ref{injective function and preimage/image}}} f_{| \left(
    A \bigcap C \right)} \circ (g^{- 1})_{|g \left( A \bigcap C \right)}
    \equallim_{\text{\ref{f|C is bijective if f is injective}}} f_{| \left( A
    \bigcap C \right)} \circ \left( g_{| \left( A \bigcap C \right)}
    \right)^{- 1}$.
    
    \item First note that by \ref{f|C is bijective if f is injective} and
    injectivity we have that $g_{|D} : D \rightarrow g (D)$ is a bijection
    with \ $(g_{|D})^{- 1} = (g^{- 1})_{|g (D)} : g (D) \rightarrow D$ as it's
    inverse and that $f_{|D} : D \rightarrow f (D)$ is a bijection. Using
    \ref{properties of composition of functions} we have then that the
    composition of these two functions \ $(f_{|D}) \circ (g_{|D})^{- 1} =
    (f_{|D}) \circ (g^{- 1})_{|g (D)} : g (D) \rightarrow f (D)$ is a
    bijection. Finally $(f_{|D}) \circ (g_{|D})^{- 1} = (f_{|D}) \circ (g^{-
    1})_{|g (D)} \equallim_{\text{\ref{composition of a restricted function}}}
    (f \circ g^{- 1})_{|g (D)}$ proving that $(f_{| D |}) \circ (g_{|D})^{- 1}
    : g (D) \rightarrow f (D)$ is a bijection that is the composition of the
    bijections $(g_{|D})^{- 1} : g (D) \rightarrow D$ and $f_{|D} : D
    \rightarrow f (D)$.
  \end{enumerate}
\end{proof}

\begin{theorem}
  \label{image restricted function}If $f : A \rightarrow B$ is a function and
  $C \subseteq B$ then $_{C|} f : f^{- 1} (C) \rightarrow C$ defined by $_{C|}
  f = f \bigcap (A \times C)$ is a function. Furthermore if $f$ is injective,
  surjective or bijective then $_{C|} f$ is also injective, surjective or
  bijective
\end{theorem}

\begin{proof}
  First if $(x, y) \in_{C|} f$ then $(x, y) \in f$ and $(x, y) \in A \times C
  \Rightarrow x \in f^{- 1} (C) \Rightarrow (x, y) \in f^{- 1} (C) \times C
  \Rightarrow_{C|} f \subseteq f^{- 1} (C) \times C$. Second if $(x, y), (x,
  y') \in_{C|} f \Rightarrow (x, y), (x, y') \in f \Rightarrowlim_{f \tmop{is}
  a \tmop{function} \tmop{thus} a \tmop{partial} \tmop{function}} y = y'$ so
  $_{C|} f : f^{- 1} (C) \rightarrow C$ is a partial function. If $x \in f^{-
  1} (C)$ then there exists a $y \in C$ such that $(x, y) \in f \Rightarrow
  (x, y) \in f \bigcap (f^{- 1} (C) \times C) \Rightarrow (x, y) \in_{C|} f$
  so $f^{- 1} (C) \subseteq \tmop{dom} (_{C|} f)$ and thus $_{C|} f$ is a
  function. Third if $f$ is injective and $(x, y), (x', y) \in_{C|} f$ then
  $(x, y), (x', y) \in f \Rightarrow x = x'$ and thus $_{C|} f$ is also
  injective. Fourth if $f$ is surjective then if $y \in C$ there exists a $x
  \in A$ such that $(x, y) \in f$ and thus $x \in f^{- 1} (C) \Rightarrow (x,
  y) \in f \bigcap (f^{- 1} (C) \times C) =_{C|} f$ and thus $_{C|} f$ is
  surjective. Finally if $f$ is bijective then it is injective and surjective
  and we have just proved that then $_{C|} f$ is injective and surjective and
  thus bijective.
\end{proof}

\begin{notation}
  If $f : A \rightarrow B$ is a function defined by $f = \{ (x, y) | y = r (x)
  \nobracket \}$ where $r$ is a rule that produces one single $y$ depending on
  $x$ then we say the $f : A \rightarrow B$ is defined by $x \rightarrow r
  (x)$
\end{notation}

\subsection{Constructing families of sets}

\begin{theorem}
  \label{family of sets defined by set function}If $I, A$ are classes and $f :
  I \rightarrow \mathcal{P} (A)$ a function. Define then the graph $F = \{ (i,
  x) \in I \times A|x \in f (i) \} \subseteq I \times A$ which has
  $\tmop{domain} (F) \subseteq I$ and defines thus a family $\{ F_i \}_{i \in
  I}$ such that $\forall i \in I$ that $F_i = f (i)$. So we can write $\{ F_i
  \}_{i \in I} = \{ f (i) \}_{i \in I}$
\end{theorem}

\begin{proof}
  If $x \in F_i = \{ x| (i, x) \in F \} \Rightarrow (i, x) \in F = x \in f (i)
  \Rightarrow F_i \subseteq f (i)$. If $x \in f (i) \Rightarrow (i, x) \in F
  \Rightarrow x \in F_i$.
\end{proof}

We show now how we can construct from a family of classes such function

\begin{theorem}
  \label{family of classes as a set function}If $\{ F_i \}_{i \in I}$ is a
  family of classes then there is a graph $F$ with $\tmop{dom} (F) = I$.
  Define then $f \subseteq I \times \mathcal{P} \left( \bigcup_{i \in I} F_i
  \right)$ by $f = \left\{ (i, A) \in I \times \mathcal{P} \left( \bigcup_{i
  \in I} F_i \right) |A = F_i \right\}$ then $f : I \rightarrow \mathcal{P}
  \left( \bigcup_{i \in I} F_i \right)$ is a function such that $f (i) = F_i$
  . In other words $\{ F_i \}_{i \in I} = \{ f (i) \}_{i \in I} = \{ f_i \}_{i
  \in I}$
\end{theorem}

\begin{proof}
  If $(i, A), (i, A') \in f$ then $A = F_i = A'$ so $f : I \rightarrow
  \mathcal{P} \left( \bigcup_{i \in I} F_i \right)$ is indeed a function. \
  Also $f (i) = A_i$ by definition.
\end{proof}

\begin{theorem}
  \label{reindexing of a family}I{\index{reindexing of a family}}f $\{ F_i
  \}_{i \in I}$ is a family of classes defined and $g : J \rightarrow I$ a
  surjection then by the above theorem we have the existence of a $f : I
  \rightarrow \mathcal{P} \left( \bigcup_{i \in I} F_i \right)$ such that $\{
  F_i \}_{i \in I} = \{ f (i) \}_{i \in I}$. We can form then $f \circ g : J
  \rightarrow \mathcal{P} \left( \bigcup_{j \in I} F_i \right)$ which defines
  then $\{ (f \circ g) (j) \}_{j \in J} = \{ f (g (j)) \}_{j \in J}$ which is
  noted by $\{ F_{g (j)} \}_{j \in J}$ [here $F_{g (j)} \equiv f (g (j)) = (f
  \circ g) (j)$]. We have then that $\bigcup_{i \in I} F_i = \bigcup_{j \in J}
  F_{g (j)}$
\end{theorem}

\begin{proof}
  If $x \in \bigcup_{i \in I} F_i$ then $\exists i \in I$ such that $x \in f
  (i) = F_i \Rightarrowlim_{g \tmop{is} a \tmop{bijection}} \exists j \in J$
  such that $g (j) = i \Rightarrow x \in f (g (j)) = F_{g (j)} \Rightarrow x
  \in \bigcup_{j \in J} F_{g (j)}$. If $x \in \bigcup_{j \in J} F_{g (j)}$
  there exists a $j \in J$ such that $x \in F_{g (j)} = f (g (j))
  \Rightarrowlim_{i = g (j) \in I} x \in f (i) \Rightarrow x \in \bigcup_{i
  \in I} F_i$
\end{proof}

\subsection{Product, Union and Intersection of a family of sets}

\begin{axiom}[Axiom of Replacement]
  \label{axiom of replacements}{\index{axiom of replacement}}If $A$ is a set
  and $f : A \rightarrow B$ is a \tmtextbf{surjective} function then $B$ is a
  set.
\end{axiom}

\begin{corollary}
  If $A$ is a set and $f : A \rightarrow B$ a bijection then $B$ is a set.
\end{corollary}

\begin{lemma}
  If $\{ A_i \}_{i \in I}$ is a family of sets then $\{ A_i | i \in I
  \nobracket \} \equallim^{\tmop{definition}} \{ x | \exists i \in I \vdash x
  = A_i \nobracket \}$ (which as $A_i$ is a set and thus a element is a valid
  definition of a class) then
  \begin{eqnarray*}
    \bigcup_{i \in I} A_i & = & \bigcup_{A \in \{ A_i | i \in I \nobracket \}}
    A\\
    \bigcap_{i \in I} A_i & = & \bigcap_{A \in \{ A_i | i \in I | \}} A
  \end{eqnarray*}
\end{lemma}

\begin{proof}
  
  \begin{eqnarray*}
    x \in \bigcup_{i \in I} A_i & \Leftrightarrow & \exists i \in I \vdash x
    \in A_i\\
    & \Leftrightarrow & \exists A \in \{ A_i | i \in I \nobracket \} \vdash x
    \in A\\
    & \Leftrightarrow & x \in \bigcup_{A \in \{ A_i | i \in I \nobracket \}}
    A\\
    x \in \bigcap_{i \in I} A_i & \Leftrightarrow & \forall i \in I \vdash x
    \in A_i\\
    & \Leftrightarrow & \forall A \in \{ A_i | i \in I \nobracket \} \vdash x
    \in A\\
    & \Leftrightarrow & \bigcap_{A \in \{ A_i | i \in I \nobracket \}} A
  \end{eqnarray*}
  
\end{proof}

\begin{theorem}
  \label{union of a family of sets is a set}If $\{ A_i \}_{i \in I}$ is a
  family of sets with $I$ a set then $\bigcup_{i \in I} A_i$ is a set. 
\end{theorem}

\begin{proof}
  Define $\varphi : I \rightarrow \{ A_i | i \in I \nobracket \}$ by $\varphi
  = \{ (i, A_i) | i \in I \nobracket \}$ then $\varphi$ is a surjective
  function and thus by the axiom of replacement \ref{axiom of replacements} we
  have that $\{ A_i | i \in I \nobracket \}$ is a set. Using then \ref{axiom
  of unions} we have that $\bigcup_{i \in I} A_i = \bigcup_{A \in \{ A_i | i
  \in I \nobracket \}} A_{}$ is a set.
\end{proof}

\begin{definition}
  \label{function space}{\index{$B^A$}}If $A$ and $B$ are sets then the class
  $B^A = \{ f|f \subseteq A \times B \wedge \langle f, A, B \rangle \tmop{is}
  a \tmop{function} \}$, as $A \times B$ is a set (see \ref{product of sets is
  a set}) and thus by the axiom of subsets (see \ref{axiom of subsets}) we
  have that $f$ is a set so $B^A$ is indeed a valid class by the axiom of
  class construction (see \ref{axiom of construction}).
\end{definition}

\begin{theorem}
  \label{B^A is a set if B and A are sets}If $A$ and $B$ are sets then $B^A$
  is a set
\end{theorem}

\begin{proof}
  \
  
  If $f \in B^A$ then $f \subseteq A \times B$ and thus $f \in \mathcal{P} (A
  \times B)$ so that we have $B^A \subseteq \mathcal{P} (A \times B)$. Now by
  \ref{product of sets is a set} we have that $A \times B$ is a set, using the
  axiom of power sets (see \ref{axiom of power sets}) we have that
  $\mathcal{P} (A \times B)$ is a set and finally by the axiom of subsets (see
  \ref{axiom of subsets}) we have that $B^A$ is a set.
\end{proof}

\begin{theorem}
  \label{P(A) and 2^A are bijective}If $A$ is a set then there exists a
  bijection between $2^A$ and $\mathcal{P} (A)$
\end{theorem}

\begin{proof}
  Define $\gamma : \mathcal{P} (A) \rightarrow 2^A$ by $\gamma = \{ (B, C_B) |
  B \in \mathcal{P} (A) \nobracket \}$ (here $C_B$ \ is the graph of the
  characteristic function (see \ref{characteristics function}) if $B \in
  \mathcal{P}_A$ then $C_B = \left( B \times \{ 1 \} \bigcup ((A\backslash B)
  \times \{ 0 \}) \right)$ then we have that $\gamma$ is a function by
  applying \ref{alternative definition of partial function} because:
  \begin{enumerate}
    \item $f \in \tmop{range} (\gamma) \Rightarrow \exists B \vdash (B, f) \in
    \gamma \Rightarrow C \in \mathcal{P} (A) \wedge f = C_B \subseteq B \times
    \{ 0, 1 \} = B \times 2 \Rightarrow f = C_B : A \rightarrow 2 = \{ 0, 1
    \}_{} \tmop{is} a \tmop{function} \left( \tmop{see}
    \text{\ref{characteristics function})} \right) \Rightarrow f \in 2^A$
    
    \item If $B \in \mathcal{P} (A) \Rightarrow B \subseteq A \Rightarrow (B,
    C_B) \in \gamma \Rightarrow C \in \tmop{dom} (\gamma)$
    
    \item If $(B, f), (B, f') \in \gamma$ then $f = C_B = f'$ 
  \end{enumerate}
  so $\gamma$ is a function, next we have to prove that it is bijective
  \begin{enumerate}
    \item $\gamma$ is injective. For if $(D, C_D), (E, C_E) \in \gamma$ with
    $C_D = C_E$ then we have
    \begin{eqnarray*}
      x \in D & \Leftrightarrow & C_D (x) = 1\\
      & \Leftrightarrowlim_{C_D = C_E} & C_E (x) = 1\\
      & \Leftrightarrow & x \in E
    \end{eqnarray*}
    and thus $D = E$
    
    \item $\gamma$ is surjective. For if $f \in 2^A$ take then $B = \{ x \in A
    | f (x) = 1 \} \nobracket$ then we have for $x \in A$
    \begin{enumerate}
      \item $x \in B \Rightarrow f (x) = 1$
      
      \item $x \nin B \Rightarrow x \in A \backslash B \Rightarrow f (x) \neq
      1 \Rightarrowlim_{\tmop{range} (f) = \{ 0, 1 \} = 2} f (x) = 0$
    \end{enumerate}
    and thus $f = C_B = \gamma (B)$
  \end{enumerate}
\end{proof}

The classical way that a general product of a family of sets $\{ A_i \}_{i \in
I}$ is as the set of functions from $I \rightarrow \bigcup_{i \in I} A_i$. The
problem with this definitions is that subsets of a product of a family of sets
can not be itself be a product of a family of $\{ B_i \}_{i \in I}$ where $B_i
\subseteq A_i$ as this would be functions from $I \rightarrow \bigcup_{i \in
I} B_i$ not from $I \rightarrow \bigcup_{i \in I} A_i$. So we would need a
kind of function that has no defined destination. This is the idea of a
pretuple

\begin{definition}[pretuple]
  \label{pretuple}{\index{pretuple}}A pretuple $\langle f, I \rangle$ is a
  pair of a class and a function graph [see \ref{function graph} [so $\forall
  (x, y), (x, y') \in f$ we have $y = y'$] and $\tmop{dom} (f) = I$. Note that
  because of this definition we have that $\forall i \in I$ there exists a
  unique $y$ such that $(i, y) \in f$ (just like with functions) we use also
  the same notation to specify this unique value as $f (i)$, essential saying
  $y = f (i)$ is the same as saying $(i, y) \in f$.
\end{definition}

\begin{example}
  $\langle \emptyset, \emptyset \rangle$ is a pretuple for $\forall (x, y),
  (x, y') \in \emptyset$ we have vacuously that $y = y'$ and $\tmop{dom}
  (\emptyset) = \emptyset$
\end{example}

\begin{theorem}
  \label{subtuple}Let $\langle f, I \rangle$ be a pretuple and let $J
  \subseteq I$ then if we define $f_{|J} = \{ (i, x) \in f|i \in J \}$ then
  $\langle f_{|J}, J \rangle$ is a pretuple. Note that if $(i, x) \in f_{|J}
  \Rightarrow (i, x) \in f \Rightarrow f_{|J} (i) = f (i)$.
\end{theorem}

\begin{proof}
  If $(i, x), (i, y) \in f_{|J} \Rightarrow (i, x), (i, y) \in f \Rightarrow x
  = y$ so $f_{|J}$ is a function graph. Also if $i \in \tmop{dom} (f_{|J})
  \Rightarrow \exists y$ such that $(i, y) \in f_{|J} \Rightarrow i \in J
  \Rightarrow \tmop{dom} (f_{|J}) \subseteq J$. If $i \in J \Rightarrow i \in
  I = \tmop{dom} (f) \Rightarrow \exists y$ such that $(i, y) \in f
  \Rightarrowlim_{i \in J} (i, y) \in f_{|J} .$
\end{proof}

Now we are ready to define a product of a family of sets.

\begin{definition}
  \label{generalized product of sets}Let $\{ A_i \}_{i \in I}$ be a family of
  sets where $I$ is also a set (see \ref{family of classes}) then we define
  $\prod_{i \in I} A_i = \left\{ f|f \subseteq I \times \left( \bigcup_{i \in
  I} A_i \right) \wedge \langle f, I \rangle \tmop{is} a \tmop{pretuple}
  \wedge \forall i \in I \vDash f (i) \in A_i \right\}$. Note that $\prod_{i
  \in I} A_i$ is well defined as a class as by \ref{union of a family of sets
  is a set} $\bigcup_{i \in I} A_i$ is a set so that $I \times \left(
  \bigcup_{i \in I} A_i \right)$ is a set which by the axiom of subsets (see
  \ref{axiom of subsets}) gives that $f$ is a set. The elements of $\prod_{i
  \in \{ 1, \ldots, n \}} A_i$ are called tuples.
\end{definition}

We prove now that $\prod_{i \in I} A_i$ is also a set

\begin{theorem}
  \label{general product of sets is a set}Let $\{ A_i \}_{i \in I}$ be a
  family of sets where $I$ is also a set (see \ref{family of classes}) then
  $\prod_{i \in I} A_i$ is a set
\end{theorem}

\begin{proof}
  \
  
  If $f \in \prod_{i \in I} A_i$ then $f \subseteq I \times \left( \bigcup_{i
  \in I} A_i \right)$ so that $f \in \mathcal{P} \left( I \times \left(
  \bigcup_{i \in I} A_i \right) \right)$, from this it follows that $\prod_{i
  \in I} A_i \subseteq \mathcal{P} (A \times B)$. Now $I$ is a set, using
  \ref{union of a family of sets is a set} we have that $\bigcup_{i \in I}
  A_i$ is a set and by \ref{product of sets is a set} we have that $I \times
  (A_i)$ is a set, next using the axiom of power sets (see \ref{axiom of power
  sets}) we have that $\mathcal{P} \left( I \times \left( \bigcup_{i \in I}
  A_i \right) \right)$ is a set and finally by the axiom of subsets (see
  \ref{axiom of subsets}) we have that $\prod_{i \in I} A_i$ is a set. 
\end{proof}

\begin{example}
  \label{product of a empty family}If $\{ A_i \}_{i \in \emptyset}$ is the
  empty family of sets then $\bigcup_{i \in \emptyset} A_i = \emptyset$ [if $x
  \in \bigcup_{i \in \emptyset} A_i$ there exists a $i \in I$ such that $x \in
  A_i$ which as $I = \emptyset$ is impossible], so in this cases $\prod_{i \in
  I} A_i = \{ \emptyset \}$ (it only contains the empty graph and $\langle
  \emptyset, \emptyset \rangle$ is a pretuple and of course $\forall i \in I
  \vdash \emptyset (i) \in A_i$ is satisfied vacuously).
\end{example}

\begin{notation}
  Let $\{ A_i \}_{i \in I}$ be a family of sets where $I$ is also a set then
  $\forall i \in I$ and $x \in \prod_{i \in I} A_i$ we have that $x \in \left(
  I \times \bigcup_{i \in I} A_i \right)$ is a pretuple (so $\forall (x, y),
  (x, y') \in f$ we have $y = y'$) so $\forall i \in I$ we have a unique $x
  (i) \in A_i$, we note then $x (i)$ as $x_i$, so $x_i = x (i)$ or $(i, x (i))
  \in x$.
\end{notation}

\begin{theorem}
  \label{condition to belong to a product of sets}Let $\{ A_i \}_{i \in I}$ be
  a family of sets where $I$ is also a set then if $f : I \rightarrow B$ is a
  function ($f$ is the graph of our function) such that $\forall i \in I$ we
  have $f (i) \in A_i$ then $f \in \prod_{i \in I} A_i$
\end{theorem}

\begin{proof}
  As $\forall i \in I$ we have $f (i) \in A_i$ we have that $f \subseteq I
  \times \left( \bigcup_{i \in I} A_i \right)$ and as $f : I \rightarrow B$ is
  a function with domain $I$ we have that $\forall (x, y), (x, y') \in f
  \Rightarrow y = y'$ and $\tmop{dom} (f) = i$ so $\langle f, I \rangle$ is
  also a pretuple, with $\forall i \in I$ that $f (i) \in A_i$ and thus that
  $f \in \prod_{i \in I} A_i$
\end{proof}

\begin{definition}
  \label{projection function}{\index{projection function}}{\index{$\pi_i$}}Let
  $\{ A_i \}_{i \in I}$ be a family of sets where $I$ is a set then $\forall i
  \in I$ we define $\pi_i : \prod_{i \in I} A_i \rightarrow A_i$ by $\pi_i =
  \left\{ (x, x (i)) \left| x \in \prod_{i \in I} A_i \right. \right\}$. We
  have then that $\pi_i$ is a function called the projection function.
\end{definition}

\begin{proof}
  We still have to prove that $\pi_i$ is a function, for this we use
  \ref{alternative definition of partial function}
  \begin{enumerate}
    \item $y \in \tmop{range} (\pi_i) \Rightarrow \exists x \vdash (x, y) \in
    \pi_i \Rightarrow y = \pi_i (x) = x (i) \in A_i \Rightarrow \tmop{range}
    (\pi_i) \subseteq A_i$
    
    \item $x \in \tmop{dom} (\pi_i) \Rightarrow \exists y \vdash (x, y) \in
    \pi_i \Rightarrow x \in \Pi_{i \in I} A_i \Rightarrow \tmop{dom} (\pi_i)
    \subseteq \Pi_{i \in I} A_i$
    
    \item If $(x, y), (x, y') \in \pi_i$ then $y = \pi_i (x) = x (i)$ and $y'
    = \pi_i (x) = x (i) \Rightarrow y = y'$
  \end{enumerate}
  proving that $\pi_i : \Pi_{j \in I} A_j \rightarrow A_i$ is a partial
  function. To prove that it is a function consider $x \in \Pi_{j \in I} A_j$
  and $i \in \{ 1, \ldots, n \}$ then $(x, x_i) \in \pi_i \Rightarrow x \in
  \tmop{dom} (\pi_i)$ and thus $\Pi_{j \in I} A_j \subseteq \tmop{dom}
  (\pi_i)$ proving that $\pi_i : \prod_{j \in I} A_j \rightarrow A_i$ is a
  function.
\end{proof}

\begin{theorem}
  \label{power of a family of sets}Let $\{ A_i \}_{i \in I}$ be a family of
  sets where $\forall i \in I$ we have $A_i = A$ then $\prod_{i \in I} A_i =
  A^I$
\end{theorem}

\begin{proof}
  If $f \in \prod_{i \in I} A_i$ then $f \subseteq I \times \bigcup_{i \in I}
  A_i = I \times A$, $\langle f, I \rangle$ is a pretuple and $\forall i \in
  I$ we have $f (i) \in A_i = A$. Then we have
  \begin{description}
    \item[$\tmop{range} (f) \subseteq A$] If $y \in \tmop{range} (f)$ then
    $\exists i \in I$ with $(i, y) \in f \Rightarrow y = f (i) \in A$ proving
    that $\tmop{range} (f) \subseteq A$
    
    \item[$\tmop{dom} (f) = I$] this follows from the fact that $\langle f, I
    \rangle$ is a pretuple
    
    \item[$\forall (x, y), (x, y') \in f \Rightarrow y = y'$] this follows
    from the fact that $\langle f, I \rangle$ is a pretuple 
  \end{description}
  So using the definition of a partial function (see \ref{alternative
  definition of partial function}) and a function (see \ref{alternative
  definition of a function (1)} we have that $\langle f, I, A \rangle$ is a
  function which means that by definition of $A^I$ (see \ref{function
  space})we have $f \in A^I$. This proves that A $\prod_{i \in I} A_i
  \subseteq A^I$. Next if $f \in A^I$ then $f \subseteq I \times A = I \times
  \bigcup_{i \in I} A_i$ and $\langle f, I, A \rangle$ is a function, as also
  $\forall i \in I \vDash f (i) \in A$ [as $\tmop{range} (f) \subseteq A$] we
  have by \ref{condition to belong to a product of sets} that $f \in \prod_{i
  \in I} A_i$ proving that $A^I \subseteq \prod_{i \in I} A_i$. So finally we
  conclude that $A^I = \prod_{i \in I} A_i$.
\end{proof}

\begin{theorem}
  \label{general product of sets and subsets}If $I$ is a set and $\{ B_i \}_{i
  \in I}$, $\{ A_i \}_{i \in I}$ families of sets such that $\forall i \in I
  \vDash B_i \subseteq A_i$ then we have $\prod_{i \in I} B_i \subseteq
  \prod_{i \in I} A_i$
\end{theorem}

\begin{proof}
  If $f \in \prod_{i \in I} B_i$ then $f \subseteq I \times \left( \bigcup_{i
  \in I} B_i \right) \subseteq I \times \left( \bigcup_{i \in I} A_i \right)
  \Rightarrow f \in I \times \left( \bigcup_{i \in I} A_i \right)$, $\langle
  f, I \rangle$ is a pretuple and $\forall i \in I$ we have $x (i) \in B_i
  \subseteq A_i \Rightarrow x (i) \in A_i$ proving that $f \in \prod_{i \in I}
  A_i$. 
\end{proof}

\begin{theorem}
  \label{intersection of general product of sets}If $I$ is a set and $\{ A_i
  \}_{i \in I}$, $\{ B_i \}_{i \in I}$ families of sets then $\left( \prod_{i
  \in I} A_i \right) \bigcap \left( \prod_{i \in I} B_i \right) = \prod_{i \in
  I} \left( A_i \bigcap B_i \right)$
\end{theorem}

\begin{proof}
  
  \begin{enumerate}
    \item If $f \in \left( \prod_{i \in I} A_i \right) \bigcap \left( \prod_{i
    \in I} B_i \right)$ then $f \in \prod_{i \in I} A_i \wedge f \in \prod_{i
    \in I} B_i$ so $f \subseteq I \times \left( \bigcup A_i \right) \wedge f
    \subseteq I \times \left( \bigcup_{i \in I} B_i \right)$ and $(\forall i
    \in I \vDash f (i) \in A_i) \wedge (\forall i \in I \vDash f (i) \in
    B_i)$. So if $(i, x) \in f$ then $i \in I$ and $x = f (i) \in A_i \wedge x
    = f (i) \in B_i \Rightarrow x \in A_i \bigcap B_i$ so that $f \in I \times
    \left( A_i \bigcap B_i \right) \subseteq I \times \left( \bigcup_{i \in I}
    \left( A_i \bigcap B_i \right) \right) \wedge \forall i \in I \vDash f (i)
    \in A_i \bigcap B_i$, this together with the fact that $\langle f, I
    \rangle$ is a pretuple [as $f \in \prod_{i \in I} A_i$ or $f \in \prod_{i
    \in I} B_i$] proves that $f \in \prod_{i \in I} \left( A_i \bigcap B_i
    \right)$ proving that \
    \begin{equation}
      \label{eq 2.1} \left( \prod_{i \in I} A_i \right) \bigcap \left(
      \prod_{i \in I} B_i \right) \subseteq \prod_{i \in I} \left( A_i \bigcap
      B_i \right)
    \end{equation}
    .
    
    \item If $f \in \prod_{i \in I} \left( A_i \bigcap B_i \right)$ then $f
    \subseteq I \times \left( \bigcup_{i \in I} \left( A_i \bigcap B_i \right)
    \right) \equallim_{\text{\ref{generalized distributive laws}}} I \times
    \left( \left( \bigcup_{i \in I} A_i \right) \bigcap \left( \bigcup_{i \in
    I} B_i \right) \right) \equallim_{\text{\ref{product of sets properties}}}
    \left( I \times \left( \bigcup_{i \in I} A_i \right) \right) \bigcap
    \left( I \times \left( \bigcup_{i \in I} B_i \right) \right)$ proving that
    $f \subseteq I \times \left( \bigcup_{i \in I} A_i \right) \wedge f
    \subseteq I \times \left( \bigcup_{i \in I} B_i \right)$, from $f \in
    \prod_{i \in I} (A_i \times B_i)$ it follows also that $\langle f, I
    \rangle$ is a pretuple. Finally $\forall i \in I$ we have $f (i) \in A_i
    \bigcap B_i \Rightarrow f (i) \in A_i \wedge f (i) \in B_i$ so that $f \in
    \prod_{i \in I} A_i \wedge f \in \prod_{i \in I} B_i \Rightarrow f \in
    \left( \prod_{i \in I} A_i \right) \bigcap \left( \prod_{i \in I} B_i
    \right)$ giving
    \begin{equation}
      \label{eq 2.6.1} \prod_{i \in I} \left( A_i \bigcap B_i \right)
      \subseteq \left( \prod_{i \in I} A_i \right) \bigcap \left( \prod_{i \in
      I} B_i \right)
    \end{equation}
  \end{enumerate}
  From \ref{eq 2.1} and \ref{eq 2.6.1} it follows then that $\left( \prod_{i
  \in I} A_i \right) \bigcap \left( \prod_{i \in I} B_i \right) = \prod_{i \in
  I} \left( A_i \bigcap B_i \right)$
\end{proof}

\begin{theorem}
  \label{union of general product of sets}If $I$ is a set and $\{ A_i \}_{i
  \in I}$, $\{ B_i \}_{i \in I}$ families of sets then $\left( \prod_{i \in I}
  A_i \right) \bigcup \left( \prod_{i \in I} B_i \right) \subseteq \prod_{i
  \in I} \left( A_i \bigcup B_i \right)$
\end{theorem}

\begin{proof}[$f \in \prod_{i \in I} A_i$][$f \in \prod_{i \in I} B_i$]
  If $f \in \left( \prod_{i \in I} A_i \right) \bigcup \left( \prod_{i \in I}
  B_i \right)$ then we have either :
  \begin{enumerate}
    \item then $f \subseteq I \times \left( \bigcup_{i \in I} A_i \right)
    \wedge \langle I, f \rangle$ is a pretuple $\wedge$ $\forall i \in I
    \vDash f (i) \in A_i$. As $\bigcup_{i \in I} A_i \subseteq \bigcup_{i \in
    I} \left( A_i \bigcup B_i \right)$ we have that $f \subseteq I \times
    \left( \bigcup_{i \in I} \left( A_i \bigcup B_i \right) \right) \wedge
    \langle I, f \rangle$ is a pretuple $\wedge$ $\forall i \in I$ we have $f
    (i) \in A_i \subseteq A_i \bigcup B_i$. This means that $f \in \prod_{i
    \in I} \left( A_i \bigcup B_i \right)$.
    
    \item then $f \subseteq I \times \left( \bigcup_{i \in I} B_i \right)
    \wedge \langle f, I \rangle$ is a function $\wedge$ $\forall i \in I
    \vDash f (i) \in B_i$. As $\bigcup_{i \in I} B_i \subseteq \bigcup_{i \in
    I} \left( A_i \bigcup B_i \right)$ we have that $f \subseteq I \times
    \left( \bigcup_{i \in I} \left( A_i \bigcup B_i \right) \right) \wedge
    \langle f, I \rangle$ is a pretuple $\wedge$ $\forall i \in I$ we have $f
    (i) \in B_i \subseteq A_i \bigcup B_i$. This means that $f \in \prod_{i
    \in I} \left( A_i \bigcup B_i \right)$.
  \end{enumerate}
  so we have always $f \in \prod_{i \in I} \left( A_i \bigcup B_i \right)$ and
  thus $\left( \prod_{i \in I} A_i \right) \bigcup \left( \prod_{i \in I} B_i
  \right) \subseteq \prod_{i \in I} \left( A_i \bigcup B_i \right)$
\end{proof}

Note that $(\Pi_{i \in I} A_i) \bigcup \left( \prod_{i \in I} B_i \right)
\subseteq \prod_{i \in I} \left( A_i \bigcup B_i \right) $ but the opposite is
not true as the following example shows.

\begin{example}
  If $f \in \prod_{i \in \{ 0, 1 \}} \left( A_i \bigcup B_i \right)$ is such
  that $f (0) \in A_0$ and $f (1) \in B_1$ then if $A_0 \bigcap B_1 =
  \emptyset$ we can not have $f \in \prod_{i \in \{ 0, 1 \}} A_i$ (because $f
  (1) \nin A_1$) and we can not have $f \in \prod_{i \in \{ 0, 1 \}} B_i$
  because $f (0) \in A_0$
\end{example}

\begin{theorem}
  \label{intersection of a product}If $I, J$ are sets and $\{ A_{i, j} \}_{(i,
  j) \in I \times J}$ a family of sets then $\bigcap_{i \in I} \left( \prod_{j
  \in J} A_{i, j} \right) = \prod_{j \in J} \left( \bigcap_{i \in I} A_{i, j}
  \right)$
\end{theorem}

\begin{proof}
  If $x \in \bigcap_{i \in I} \left( \prod_{j \in J} A_{i, j} \right)$ then
  $\forall i \in I$ we have $x \in \prod_{j \in J} A_{i, j}$ so $x \subseteq J
  \times \left( \bigcup_{j \in J} A_{i, j} \right)$, $\langle x, I \rangle$ is
  a pretuple and $\forall j \in J$ we have $x (j) \in A_{i, j}$, so that we
  have $\forall j \in J \vDash x (j) \in \bigcap_{i \in I} A_{i, j}$. This
  means that if $(j, y) \in x \Rightarrow y = x (j) \in \bigcap_{i \in I}
  A_{i, j} \Rightarrow (j, y) \in J \times \left( \bigcap_{j \in J} A_{i, j}
  \right) \subseteq J \times \left( \bigcup_{j \in J} \left( \bigcap_{i \in I}
  A_{i, j} \right) \right)$ so we have that $x \subseteq J \times \left(
  \bigcup_{j \in J} \left( \bigcap_{i \in I} A_{i, j} \right) \right)$,
  $\langle x, I \rangle$ is still a pretuple and $\forall j \in J$ we have $x
  (j) \in \bigcap_{i \in I} A_{i, j}$ or in other words $x \in \prod_{j \in J}
  \left( \bigcap_{i \in I} A_i \right)$. If $x \in \prod_{j \in J} \left(
  \bigcap_{i \in I} A_{i, j} \right)$ then $x \subseteq J \times \left(
  \bigcup_{j \in J} \left( \bigcap_{i \in I} A_{i, j} \right) \right)$,
  $\langle x, J \rangle$ is a pretuple and $\forall j \in J$ we have $x (j)
  \in \bigcap_{i \in I} A_{i, j} \Rightarrow \forall i \in I \vDash x (j) \in
  A_{i, j}$. So $\forall i \in I$ we have $x \subseteq J \times \left(
  \bigcup_{j \in J} \left( \bigcap_{i \in I} A_{i, j} \right) \right)
  \subseteq J \times \left( \bigcup_{j \in J} A_{i, j} \right)$, \ $\langle x,
  J \rangle$ is a pretuple and $\forall j \in J$ we have $x (j) \in A_{i, j}$
  or in other words $\forall i \in I$ we have $x \in \prod_{j \in J} A_{i, j}$
  and thus $x \in \bigcap_{i \in I} \left( \prod_{j \in J} A_{i, j} \right)$.
\end{proof}

\subsection{Family of elements in a set}

\begin{definition}
  \label{family of elements}{\index{family of elements}}If $I, A$ are sets and
  $x \in A^I$ then we note $x$ as $\{ x_i \}_{i \in I}$ where $x_i = x (i)$
  and call $\{ x_i \}_{i \in I}$ a family of elements of $A$ (or family in $A$
  in short). We say that the family is non empty if $I \neq \emptyset$.
\end{definition}

\begin{example}
  \label{selfindexed family}If $A$ is a set then $1_A : A \rightarrow A \in
  A^A$ defines a family of elements in $A$ which we note by $\{ x \}_{x \in
  A}$ (so $\{ x \}_{x \in A} = \{ (1_A)_i \}_{i \in A}$) and call the family
  obtained by self indexing $A$.
\end{example}

\begin{definition}
  If $\{ x_i \}_{i \in I}$ is a family of elements of $A$ (so $\{ x_i \}_{i
  \in I} \in A^I$) and $J \subseteq I$ then $\{ x_i \}_{i \in J}$ is defined
  by $(\{ x_i \}_{i \in I})_{|J} \in B^J$.
\end{definition}

\begin{notation}
  If $A$ is a set and $x = \{ x_i \}_{i \in I}$ is a family of elements in
  $A$, $J \subseteq I$ and $\sigma : J \rightarrow J$ a bijection then as $x$
  is a function $x : I \rightarrow A$ we can consider the function $x \circ
  \sigma : J \rightarrow A$ which represent again a family of elements in $A$
  which will be noted as $\{ x_{\sigma_i} \}_{i \in J}$. Here $\forall i \in
  J$ we have $x_{\sigma_i} \equiv (x \circ \sigma) (i) = x (\sigma (i)) \equiv
  x_{\sigma (i)} = (x \circ \sigma)_i$
\end{notation}

\begin{definition}
  \label{set of family of elements}If $I, A$ are sets and $\{ x_i \}_{i \in
  I}$ a family of elements in $A$ then $\{ x_i : i \in I \}$ is defined to be
  equal to $\{ x_i \}_{i \in I} (I) = \{ y \in A| \exists i \in I \vdash y =
  \{ x_i \}_{i \in I} (i) \} \subseteq A$ (so that $\{ x_i : i \in I \}$ must
  be a set). 
\end{definition}

\begin{example}
  \label{selfindexed family image}If $A$ is a set then $\{ x \}_{x \in A} (I)
  = A$ and this is consistent with the notation $A = \{ x|x \in A \}$)
\end{example}

\begin{theorem}
  \label{permutation of a family gives same set as the family}If $A, I$ are
  sets and $\sigma : I \rightarrow I$ a bijection (this is called a
  permutation) then if $\{ x_i \}_{i \in I}$ is a family of elements in $A$
  then for the family $\{ x_{\sigma_i} \}_{i \in I}$($= \{ x_i \}_{i \in I}
  \circ \sigma$) we have $\{ x_i |i \in I \} = \{ x_{\sigma_i} |i \in I \}$
\end{theorem}

\begin{proof}
  
  \begin{eqnarray*}
    y \in \{ x_i |i \in I \} & \Rightarrow & \exists i \in I \vdash y = \{ x_i
    \}_{i \in I} (i)\\
    & \Rightarrowlim_{\sigma \tmop{is} a \tmop{bijection}} & \exists j \in I
    \vdash i = \sigma (j)\\
    & \Rightarrow & y = \{ x_i \}_{i \in I} (\sigma (j))\\
    & \Rightarrow & y \in \{ x_{\sigma_i} |i \in I \}\\
    y \in \{ x_{\sigma_i} |i \in I \} & \Rightarrow & \exists i \in I \vdash y
    = (\{ x_i \}_{i \in I} \circ \sigma) (i)\\
    & \Rightarrow & y = \{ x_i \}_{i \in I} (\sigma (i)) \tmop{where} \sigma
    (i) = j \in I\\
    & \Rightarrow & y = \{ x_i \}_{i \in I} (j)\\
    & \Rightarrow & y \in \{ x_i |i \in I \}
  \end{eqnarray*}
\end{proof}

We have now two ways of defining a family of sets that are subsets of a class:
\begin{enumerate}
  \item If $A$ is a set then $\{ A_i \}_{i \in I}$ is a family in $\mathcal{P}
  (A)$ so $\{ A_i \}_{i \in I}$ is a function from $I$ to $\mathcal{P} (A)$
  
  \item If $A$ is a set then $\{ A_i \}_{i \in I}$ is a family of sets (see
  \ref{family of classes}) such that $\forall i \in I \vDash A_i \subseteq A$
  and thus a $\tmop{graph}$ with $\tmop{domain} (\tmop{graph} (\{ A_i \}_{i
  \in I})) \subseteq I$
\end{enumerate}
However \ref{family of sets defined by set function} and \ref{family of
classes as a set function} shows that the two definitions (1),(2) are
essentially the same if it's about subset's of a certain set.

\begin{definition}
  \label{family and subset}{\index{$\subseteq$}}Let $I, A$ be sets, $B
  \subseteq A$ and $\{ x_i \}_{i \in I}$ a family of elements in $A$ then we
  say that $\{ x_i \}_{i \in I} \subseteq B$ if $\forall i \in I$ we have $\{
  x_i |i \in I \} \subseteq B$ (or $\{ x_i \}_{i \in I} (I) \subseteq B$)
\end{definition}

\section{Relations}

\subsection{Relations}

\begin{definition}
  {\index{relation}}Let $A$ be a class then a relation in $A$ is any subclass
  of $A \times A$
\end{definition}

\begin{definition}[reflexive][symmetric][anti-symmetric][transitive]
  {\index{reflexitivity}}{\index{symmetric}}{\index{anti-symmetric}}{\index{transitive}}If
  $A$ is a class then a relation $R$ in $A$ is
  \begin{enumerate}
    \item iff
    \begin{eqnarray*}
      \forall x \in A \vdash (x, x) \in R &  & 
    \end{eqnarray*}
    \item  iff
    \begin{eqnarray*}
      (x, y) \in R & \Rightarrow & (y, x) \in R
    \end{eqnarray*}
    \item  iff
    \begin{eqnarray*}
      (x, y) \in R \wedge (y, x) \in R & \Rightarrow & x = y
    \end{eqnarray*}
    \item  iff
    \begin{eqnarray*}
      (x, y) \in R \wedge (y, z) \in R & \Rightarrow & (x, z) \in R
    \end{eqnarray*}
  \end{enumerate}
\end{definition}

\begin{notation}
  Let $A$ be a class and $R$ is a relation in $A$ then we note $(x, y) \in R$
  as $\tmop{xRy}$
\end{notation}

\begin{definition}
  If $A$ is a class then $I_A = \{ (x, x) | x \in A \nobracket \}$ is the
  diagonal graph.
\end{definition}

\subsection{Equivalence relations}

\begin{definition}
  \label{equivalence relation}{\index{equivalence relation}}If $A$ is a class
  then a relation in $A$ is a equivalence relation if it is reflexive,
  symmetric and transitive.
\end{definition}

\begin{definition}
  If $A$ is a set then a partition of $A$ is a family of nonempty subsets of
  $A$ $\{ A_i \}_{i \in I}$ with the following properties
  \begin{enumerate}
    \item $\bigcup_{i \in I} A_i = A$
    
    \item $\forall i, j \in I$ we have $A_i \bigcap A_j = \emptyset \vee A_i =
    A_j$
  \end{enumerate}
\end{definition}

\begin{theorem}
  \label{condition of partition}If $A$ is a set and $\{ A_i \}_{i \in I}$ is a
  family of non empty subsets of $A$ then $\{ A_i \}_{i \in I}$ is a partition
  of $A$ iff
  \begin{enumerate}
    \item $x \in A \Rightarrow \exists i \in I \vdash x \in A_i$
    
    \item $\forall i, j \in I \vdash x \in A_i \bigcap A_j \Rightarrow A_i =
    A_j$
  \end{enumerate}
\end{theorem}

\begin{proof}
  
  
  $\Rightarrow$
  \begin{enumerate}
    \item $x \in A = \bigcup_{i \in I} A_i \Rightarrow \exists i \in I \vdash
    x \in A_i$
    
    \item $\forall i, j \in I$ we have $x \in A_i \bigcap A_j \Rightarrow A_i
    \bigcap A_j \neq \emptyset \Rightarrow A_i = A_j$
  \end{enumerate}
  
  
  $\Leftarrow$
  \begin{enumerate}
    \item As $\forall i \in I$ we have $A_i \subseteq A \Rightarrow \bigcup_{i
    \in I} A_i \subseteq A$. If $x \in A \Rightarrow \exists i \in I \vdash x
    \in A_i \Rightarrow x \in \bigcup_{i \in I} A_i \Rightarrow A \subseteq
    \bigcup_{i \in I} A_i \Rightarrow A = \bigcup_{i \in I} A_i$
    
    \item $\forall i, j \in I$ if $A_i \bigcap A_j \neq \emptyset \Rightarrow
    \exists x \in A_i \bigcap A_j \Rightarrow A_i = A_j$
  \end{enumerate}
\end{proof}

\begin{definition}
  If $A$ is a set and $R$ a equivalence relation in $A$ then given a $x \in A$
  the equivalence class $R [x]$ of $x$ modulo $R$ is
  \begin{eqnarray*}
    R [x] & = & \{ y \in A | (y, x) \in R \nobracket \} \subseteq A
  \end{eqnarray*}
\end{definition}

(note that because $R [x] \subseteq A$ and $A$ is a set we have by the axiom
of subset \ref{axiom of subsets} that $R [x]$ is a set)

\begin{theorem}
  \label{condition for R[x]=R[y]}If $R$ is a equivalence relation in a set $A$
  then
  \begin{eqnarray*}
    \tmop{xRy} & \Leftrightarrow & R [x] = R [y]
  \end{eqnarray*}
\end{theorem}

\begin{proof}
  \
  
  $\Rightarrow$
  
  If $\tmop{xRy}$ then if $z \in R [x] \Rightarrow \tmop{zRx} \wedge
  \tmop{xRy} \Rightarrow \tmop{zRy} \Rightarrow z \in R [y]$. If $z \in R [y]
  \Rightarrow \tmop{zRy} \wedge \tmop{xRy} \Rightarrow \tmop{zRy} \wedge
  \tmop{yRx} \Rightarrow \tmop{zRx} \Rightarrow R [x] = R [y]$
  
  $\Leftarrow$
  
  If $R [x] = R [y]$ then as $\tmop{xRx} \Rightarrow x \in R [x] = R [y]
  \Rightarrow \tmop{xRy}$
\end{proof}

\begin{theorem}
  Let $A$ be a set and $R$ a equivalence relation in $A$ then $\{ R [x] \}_{x
  \in A}$ is a partition of $A$. $\{ R [x] \}_{x \in A}$ is the partition
  corresponding with $R$
\end{theorem}

\begin{proof}
  First for every $x \in A$ we have $\tmop{xRx}$ so that $x \in R [x]$ and
  thus $\{ R [x] \}_{x \in A}$ is a family of non empty subsets of $A$. We use
  \ref{condition of partition} to prove that $\{ R [x] \}_{x \in A}$ is a
  partition
  \begin{enumerate}
    \item If $x \in A \Rightarrow \tmop{xRx} \Rightarrow x \in R [x]$
    
    \item $\forall x, y \in A \vdash z \in R [x] \bigcap R [y] \Rightarrow
    \tmop{zRx} \wedge \tmop{zRy} \Rightarrow \tmop{xRz} \wedge \tmop{zRy}
    \Rightarrow \tmop{xRy} \Rightarrowlim_{\tmop{previous} \tmop{theorem}} R
    [x] = R [y]$ 
  \end{enumerate}
\end{proof}

The above theorem has also a opposite

\begin{theorem}
  If $\{ A_i \}_{i \in I}$ is a partition of $A$ then define $R = \{ (x, y) |
  \exists i \in I \nobracket \vdash x \in A_i \wedge y \in A_i \}$ then $R$ is
  a equivalence relation, $\forall i \in I \vdash \exists x \in A \vdash A_i =
  R [x]$ and $\forall x \in A \vdash \exists i \in I \vdash R [x] = A_i$. We
  call $R$ to be the equivalence relation corresponding with the partition $\{
  A_i \}_{i \in I}$
\end{theorem}

\begin{proof}
  First we prove that $R$ is a equivalence relation
  \begin{enumerate}
    \item If $x \in A$ then $\exists i \in I \vdash x \in A_i \Rightarrow (x,
    x) \in R$
    
    \item If $x, y \in A$ with $\tmop{xRy} \Rightarrow \exists i \vdash x \in
    A_i^{} \wedge y \in A_i \Rightarrow y \in A_i \wedge x \in A_i \Rightarrow
    (y, x) \in R$
    
    \item If $\tmop{xRy}$ and $\tmop{yRz}$ then $\exists i, j \in I \vdash x
    \in A_i \wedge y \in A_i \wedge y \in A_j \wedge z \in A_j
    \Rightarrowlim_{y \in A_i \bigcap A_j} A_i = A_j \Rightarrow x \in A_i
    \wedge z \in A_i \Rightarrow \tmop{xRz}$
  \end{enumerate}
  Next if $i \in I$ then as $A_i \neq \emptyset$ there exists a $x \in A_i$ we
  have then
  \begin{eqnarray*}
    z \in R [x] & \Leftrightarrow & \tmop{zRx}\\
    & \Leftrightarrow & \exists j \in I \vdash z \in A_j \wedge x \in A_j\\
    & \Leftrightarrowlim_{j = i \tmop{as} x \in A_i \bigcap A_j} & z \in A_i
    \wedge x \in A_i\\
    & \Leftrightarrowlim_{x \in A_i} & z \in A_i
  \end{eqnarray*}
  Finally if $x \in A$ then $\exists i \in I$ with $x \in A_i$ and we can use
  the above again to prove that $A_i = R [x]$.
\end{proof}

\begin{definition}
  \label{A/R R is a equivalence relation}{\index{$A / R$}}If $A$ is set and
  $R$ is a equivalence relation on $A$ then the set $A / R$ is defined to be
  $\{ R [x] | x \in A \nobracket \}$ (note that as $R [x] \subseteq A$ we have
  that $A / R \subseteq \mathcal{P} (A)$ which because of axiom of subsets
  \ref{axiom of subsets} and \ref{axiom of power sets} that $A / R$ is a set)
\end{definition}

\begin{theorem}
  \label{preimage of a relation}If $f : A \rightarrow B$ is a function and $R$
  is a equivalence relation on $B$ then $f \langle R \rangle = \{ (x, y) | f
  (x) \tmop{Rf} (y) \nobracket \}$ is a equivalence relation. $f \langle R
  \rangle$ is called the preimage of $R$ by $f$
\end{theorem}

\begin{proof}[reflexive][symmetry][transitive]
  
  \begin{enumerate}
    \item If $x \in A \Rightarrow f (x) \in B \Rightarrow f (x) \tmop{Rf} (x)
    \Rightarrow \tmop{xf} \langle R \rangle x$
    
    \item If $\tmop{xf} \langle R \rangle y \Rightarrow f (x) \tmop{Rf} (y)
    \Rightarrow f (y) \tmop{Rf} (x) \Rightarrow \tmop{yf} \langle R \rangle x$
    
    \item If $\tmop{xf} \langle R \rangle y \wedge \tmop{yf} \langle R \rangle
    z \Rightarrow f (x) \tmop{Rf} (y) \wedge f (y) \tmop{Rf} (z) \Rightarrow f
    (x) \tmop{Rf} (z) \Rightarrow \tmop{xf} \langle R \rangle z$
  \end{enumerate}
\end{proof}

\begin{theorem}
  \label{restriction of a equivalence relation}{\index{restriction of a
  equivalence relation}}If $A$ is a class with a equivalence relation $R$ then
  if $B \subseteq A$ then $R_{| B \nobracket} = \{ (x, y) | x \in B \wedge y
  \in B \wedge (x, y) \in R \nobracket \}$ is a equivalence relation.
\end{theorem}

\begin{proof}[reflexive][symmetry]
  
  \begin{enumerate}
    \item $x \in B \Rightarrow \tmop{xRx} \Rightarrow \tmop{xR}_{| B
    \nobracket} x$
    
    \item $\tmop{xR}_{| B \nobracket} y \Rightarrow x \in B \wedge y \in B
    \wedge \tmop{xRy} \Rightarrow x \in B \wedge y \in B \wedge \tmop{yRx}
    \Rightarrow \tmop{yR}_{| B \nobracket} x$
    
    \item transitivity. If $\tmop{xR}_{| B \nobracket} y \wedge \tmop{yR}_{| B
    \nobracket} z \Rightarrow x \in B \wedge y \in B \wedge z \in B \wedge
    \tmop{xRy} \wedge \tmop{yRz} \Rightarrow x \in B \wedge z \in B \wedge
    \tmop{xRz} \Rightarrow \tmop{xR}_{| B \nobracket} z$
  \end{enumerate}
\end{proof}

\begin{definition}
  Let $R, R'$ be equivalence relations on $A$ then $R$ is a refinement of $R'$
  if $R \subseteq R'$ (we express this also by saying that $R$ is finer then
  $R'$ and that $R'$ is coarser then $R$
\end{definition}

\begin{theorem}
  If $R, R'$ are equivalence relations on $A$ with $R \subseteq R'$ then we
  have $z \in R' [x] \Rightarrow R [z] \subseteq R' [x]$
\end{theorem}

\begin{proof}
  If $z \in R' [x]$ then if $y \in R [z] \Rightarrow \tmop{yRz} \Rightarrow
  \tmop{yR}' z \Rightarrowlim_{z \in R' [x]} \tmop{yR}' z \wedge \tmop{zR}' x
  \Rightarrow \tmop{yR}' x \Rightarrow y \in R' [x]$
\end{proof}

\begin{corollary}
  If $R, R'$ are equivalence relations on $A$ with $R \subseteq R'$ then
  $\forall x \in A \vdash R [x] \subseteq R' [x]$
\end{corollary}

\begin{proof}
  If $x \in A$ then as $x \in R' [x] \Rightarrowlim_{\tmop{previous}
  \tmop{theorem}} R [x] \subseteq R' [x]$
\end{proof}

\begin{corollary}
  If $P, R$ are equivalence relations on a set $A$ with $P \subseteq R$. Then
  the quotient of $\frac{R}{P}$ is a relation in $A / P$ defined as follows
  \begin{eqnarray*}
    \frac{R}{P} & = & \{ (P [x], P [y]) | (x, y) \in R \nobracket \}
  \end{eqnarray*}
  We have then that $\frac{R}{P}$ is a equivalence relation on $A / P$
\end{corollary}

\begin{proof}[reflexive][symmetry][transitive]
  This is proved by
  \begin{enumerate}
    \item If $P [x] \in A / P$ then from $(x, x) \in R$ we have $(P [x], P
    [x]) \in \frac{R}{P}$
    
    \item If $P [x], P [y] \in A / P$ then if $(P [x], P [y]) \in \frac{R}{P}
    \Rightarrow (x, y) \in R \Rightarrow (y, x) \in R \Rightarrow (P [y], P
    [x]) \in \frac{R}{P}$
    
    \item If $(P [x], P [y]) \in \frac{R}{P} \wedge (P [y], P [z]) \in
    \frac{R}{P}$ then $(x, y) \in R \wedge (y, z) \in R \Rightarrow (x, z) \in
    R \Rightarrow (P [x], P [z]) \in \frac{R}{P}$
  \end{enumerate}
\end{proof}

\subsection{Equivalence relations and functions}

\begin{theorem}
  \label{equivalence relation determined by f}If $f : A \rightarrow B$ is a
  function then $R_f \subseteq A \times A$ defined by $R_f = \{ (x, y) \in A
  \times A | f (x) = f (y) \} \nobracket$ is a equivalence relation in $A$, it
  is called the equivalence relation determined by $f$
\end{theorem}

\begin{proof}[reflexivity][symmetry][transitive]
  
  \begin{enumerate}
    \item If $x \in A \Rightarrow f (x) = f (x) \Rightarrow (x, x) \in R_f$
    
    \item If $\tmop{xR}_f y \Rightarrow f (x) = f (y) \Rightarrow f (y) = f
    (x) \Rightarrow \tmop{yR}_f x$
    
    \item If $\tmop{xR}_f y$ and $\tmop{yR}_f z$ then $f (x) = f (y) \wedge f
    (y) = f (z) \Rightarrow f (x) = f (z) \Rightarrow \tmop{xR}_f z$
  \end{enumerate}
\end{proof}

So with a function is associated a equivalence relation, we can also do the
opposite, given a equivalence relation then associate this equivalence
relation with a function.

\begin{theorem}
  \label{canonical function of a equivalence relation}If $R$ is a equivalence
  relation in a set $A$ then $f_R : A \rightarrow A / R$ \ defined by $f_R =
  \{ (x, R [x]) | x \in A \nobracket \}$ is a function and we have then
  $R_{f_R} = R$. Furthermore we have that $f_R$ is surjective. We call $f_R$
  the canonical function of $R$
\end{theorem}

\begin{proof}
  First we use \ref{alternative definition of partial function} to prove that
  $f_R$ is a function
  \begin{enumerate}
    \item $\tmop{range} (f_R) \subseteq A / R$. If $y \in \tmop{range} (f_R)
    \Rightarrow \exists x \vdash y = R [x] \in A / R \Rightarrow \tmop{range}
    (f_R) \subseteq A / R$
    
    \item $\tmop{dom} (f_R) = A$ If $x \in \tmop{dom} (f_R) \Rightarrow
    \exists y \vdash (x, y) \in f_R \Rightarrow x \in A \wedge y = R [x]
    \Rightarrow x \in A \Rightarrow \tmop{dom} (f_R) \subseteq A$. If $x \in A
    \Rightarrow (x, R [x]) \in f_R \Rightarrow x \in \tmop{dom} (f)
    \Rightarrow A \subseteq \tmop{dom} (f)$
    
    \item If $(x, y), (x, y') \in f \Rightarrow y = y'$. If $(x, y), (x, y')
    \in f_R$ then $x \in A \wedge y = R [x] \wedge y' = R [x] \Rightarrow y =
    y'$
  \end{enumerate}
  This proves that $f_R : A \rightarrow A / R$ is a function. Next
  \begin{eqnarray*}
    (x, y) \in R & \Leftrightarrowlim_{\text{\ref{condition for R[x]=R[y]}}} &
    R [x] = R [y]\\
    & \Leftrightarrow & f_R (x) = f_R (y)\\
    & \Leftrightarrow & (x, y) \in R_{f_R}
  \end{eqnarray*}
  proving that $R = R_{f_R}$.
  
  Finally we must prove that $f_R $ is surjective, so if $y \in A / R
  \Rightarrow \exists x \vdash y = R [x] \Rightarrow \exists x \vdash y = f_R
  (x) = R [x] \Rightarrow y \in \tmop{range} (f) \Rightarrow \tmop{range} (f)
  = A / R$
\end{proof}

\begin{theorem}
  \label{canonical decomposition of a function}{\index{canonical decomposition
  of a function}}If $f : A \rightarrow B$ is a function where $A, B$ are sets,
  define then $s_f : A / R_f \rightarrow f (A)$ with $s_f = \{ (R_f [x], f
  (x)) | x \in A \nobracket \}$ then we have $f = e_{f (A)} \circ s_f \circ
  f_{R_f}$ where
  \begin{enumerate}
    \item $f_{R_f} : A \rightarrow A / R_f$ is defined in \ref{canonical
    function of a equivalence relation} and is proved there to be surjective
    
    \item $e_{f (A)} : f (A) \rightarrow B$ is the inclusion function (see
    \ref{inclusion function}) which is injective (see \ref{inclusion
    function})
    
    \item $s_f$ is bijective
  \end{enumerate}
\end{theorem}

This is called the canonical decomposition of a function. So we can consider
every function as the composition of a surjective, bijective and injective
function

\begin{proof}[injective][surjective]
  First we prove that $s_f$ is a function. Again we use \ref{alternative
  definition of partial function}
  \begin{enumerate}
    \item $\tmop{range} (s_f) \subseteq f (A)$. If $y \in \tmop{range} (s_f)
    \Rightarrow \exists x \vdash (x, y) \in s_f \Rightarrow \exists z \in A
    \vdash x = R_f [z], y = f (z) \Rightarrow y \in f (A) \Rightarrow
    \tmop{range} (s_f) \subseteq f (A)$
    
    \item $\tmop{dom} (s_f) = A / R_f$. If $x \in \tmop{dom} (s_f) \Rightarrow
    \exists y \vdash (x, y) \in s_f \Rightarrow \exists z \in A \vdash x = R_f
    [z] \wedge y = f (z) \Rightarrow x \in A / R_f \Rightarrow \tmop{dom}
    (s_f) \subseteq A / R_f$. Now if $x \in A / R_f \Rightarrow \exists z \in
    A \vdash x = R_f [z] \Rightarrow (R_f [z], f (z)) \in s_f \Rightarrow (x,
    f (z)) \in s_f \Rightarrow x \in \tmop{dom} (s_f) \Rightarrow A / R_f
    \subseteq \tmop{dom} (s_f)$ so we conclude that $\tmop{dom} (s_f) = A /
    R_f$
    
    \item If $(x, y), (x, y') \in s_f \Rightarrow y = y'$. This is the most
    important one to prove, so let $(x, y), (x, y') \in s_f$ then there exists
    a $z, z' \in A \vdash x = R_f [z] \wedge y = f (z) \wedge x = R_f [z']
    \wedge y' = f (z')$ now from $R_f [z] = x = R_f [z'] \Rightarrow R_f [z] =
    R_f [z']$ we have by \ref{condition for R[x]=R[y]} that $\tmop{zR}_f z'
    \Rightarrow f (z) = f (z') \Rightarrow y = y'$
  \end{enumerate}
  Next we must prove that $s_f$ is a bijection
  \begin{enumerate}
    \item If $(x, y), (x', y) \in s_f$ then $\exists z, z' \in A \vdash x =
    R_f [z] \wedge x' = R_f [z'] \wedge f (z) = y = f (z') \Rightarrow
    \tmop{zR}_f z' \Rightarrow R_f [z] = R_f [z'] \Rightarrow x = x'$
    
    \item If $y \in f (A) \Rightarrow \exists z \in A \vdash y = f (z)
    \Rightarrow (R_f [z], f (z)) = (R_f [z], y) \in s_f \Rightarrow y \in
    \tmop{range} (s_f) \Rightarrow f (A) \subseteq \tmop{range} (s_f)$ proving
    surjectivity.
  \end{enumerate}
  Finally we must prove that $f = e_{f (A)} \circ s_f \circ f_{R_f}$ now if $x
  \in A$ then we have by \ref{composition of functions and function
  application} that
  \begin{eqnarray*}
    (e_{f (A)} \circ s_f \circ f_{R_f}) (x) & = & (e_{f (A)} \circ s_f)
    (f_{R_f} (x))\\
    & = & e_{f (A)} (s_f (f_{R_f} (x)))\\
    & = & e_{f (A)} (s_f (R_f [x]))\\
    & = & e_{f (A)} (f (x))\\
    & = & f (x)
  \end{eqnarray*}
  by \ref{condition for equality of functions} we have then that $f = e_{f
  (A)} \circ s_f \circ f_{R_f}$
  
  \ 
\end{proof}

So we conclude that $A / R_f \approx f (A)$. Further if $f$ is surjective then
$f (A) = \tmop{range} (f) = B$ and then $e_{f (A)}$ is $i_{f (A)}$ which is a
bijection so that by \ref{properties of composition of functions} we have that
$e_{f (A)} \circ s_f : A / R_f \rightarrow B$ is a bijection so $A / R_f
\approx B$.

\begin{theorem}
  Let $f : A \rightarrow B$ be a function between sets $A$ and $B$. Let $R
  \subseteq R_f$ be a equivalence relation in $A$ which is finer then $R_f$
  then we can define $f / R : A / R \rightarrow B$ by $f / R = \{ (R [x], f
  (x)) | x \in A \nobracket \}$. Then we have that $f$ is a function \ (called
  the \tmtextbf{quotient of f by R}). Furthermore we have that $\frac{R_f}{R}
  = R_{f / R}$
\end{theorem}

\begin{proof}
  First we prove that $f / R : A / R \rightarrow B$ is a function (using
  \ref{alternative definition of partial function}):
  \begin{enumerate}
    \item $\tmop{range} (f / R) \subseteq B$. If $y \in \tmop{range} (f / R)
    \vdash \exists x \vdash (x, y) \in f / R \Rightarrow \exists z \in A
    \vdash x = R [x] \wedge y = f (z) \in B \Rightarrow \tmop{range} (f)
    \subseteq B$
    
    \item $\tmop{dom} (f / R) = A / R$. If $x \in \tmop{dom} (f / R)$ then
    $\exists y \vdash (x, y) \in f / R \Rightarrow \exists z \in A \vdash x =
    R [z] \wedge y = f (z) \Rightarrow x \in A / R \Rightarrow \tmop{dom} (f /
    R) \subseteq A / R$. If $x \in A / R \Rightarrow \exists z \vdash x = R
    [z] \Rightarrow (R [z], f (z)) \in f / R \Rightarrow x = R [z] \in
    \tmop{dom} (f / R) \Rightarrow A / R \subseteq \tmop{dom} (f / R)$
    
    \item $(x, y), (x, y') \in f / R \Rightarrow y = y'$. So if $(x, y), (x,
    y') \in f / R \Rightarrow \exists z, z' \in A \vdash x = R [z] \wedge y =
    f (z) \wedge x = R [z'] \wedge y' = f (z') \Rightarrow (x, z) \in R \wedge
    (x, z') \in R \wedge y = f (z) \wedge y = f (z') \Rightarrowlim_{R
    \subseteq R_f} (x, z) \in R_f \wedge (x, z') \in R_f \wedge y = f (z)
    \wedge y' = f (z) \Rightarrow f (x) = f (z) \wedge f (x) = f (z') \wedge y
    = f (z) \wedge y' = f (z') \Rightarrow y = f (z) = f (x) = f (z') = y'
    \Rightarrow y = y'$. Note that $R \subseteq R_f$ is essential to prove
    that $R_f$ is a function.
  \end{enumerate}
  Now for the last part
  \begin{eqnarray*}
    (x, y) \in R_{f / R} & \Leftrightarrow & x \in A / R \wedge y \in A / R
    \wedge (f / R) (x) = (f / R) (y)\\
    & \Leftrightarrow & \exists z, z' \in A \wedge x = R [z] \wedge y = R
    [z'] \wedge (f / R) (R [z]) = (f / R) (R [z'])\\
    & \Leftrightarrow & \exists z, z' \in A \wedge x = R [z] \wedge y = R
    [z'] \wedge f (z) = f (z')\\
    & \Leftrightarrow & \exists z, z' \in A \wedge x \in R [z] \wedge y = R
    [z'] \wedge (z, z') \in R_f\\
    & \Leftrightarrow & (x, y) \in \frac{R_f}{R}
  \end{eqnarray*}
  
\end{proof}

\section{Partially ordered classes}

\subsection{Order relation}

\begin{definition}[Pre-order]
  \label{pre-order}{\index{pre-order}}If $A \tmop{is} a \tmop{class}
  \tmop{then} a \tmop{relation}$R in $A$ is a pre-order if it is reflexive and
  transitive or in other words
  \begin{enumerate}
    \item $\forall x \in A \vdash \tmop{xRx}$ (reflexitivity)
    
    \item $\forall x, y \in A \vdash \tmop{xRy} \wedge \tmop{yRz} \Rightarrow
    \tmop{xRz}$ (transitivity)
  \end{enumerate}
\end{definition}

\begin{definition}
  \label{pre-ordered class}{\index{pre-ordered class}}A class $A$ together
  with a pre-order $R$ on the class is called a pre-ordered class and noted by
  $\langle A, R \rangle$ (if $A$ is a set then $\langle A, R \rangle$ is
  called a pre-ordered set)
\end{definition}

We can extend the concept of a pre-order to a order using the following
definition.

\begin{definition}[Order relation]
  \label{order relation}{\index{order relation}}If $A$ is a class then a
  relation $R$ in $A$ is a order relation if it is reflexive, anti-symmetric
  and transitive or in other words
  \begin{enumerate}
    \item $\forall x \in A \vdash \tmop{xRx}$ (reflexivity)
    
    \item $\tmop{xRy} \wedge \tmop{yRx} \Rightarrow x = y$ (anti-symmetry)
    
    \item $\tmop{xRy} \wedge \tmop{yRz} \Rightarrow \tmop{xRz}$ (transitivity)
  \end{enumerate}
\end{definition}

\begin{example}
  \label{inclusion is partial order}If $\mathcal{A}$ is a class of classes
  then $\langle A, \subseteq \rangle$ is a partially ordered class
\end{example}

\begin{proof}[reflexive][symmetry][transitive]
  
  \begin{enumerate}
    \item If $A \in \mathcal{A}$ then $A \subseteq A$
    
    \item If $A \subseteq B$ and $B \subseteq A$ then $A = B$
    
    \item If $A \subseteq B$ and $B \subseteq C$ then $A \subseteq C$ 
  \end{enumerate}
\end{proof}

\begin{definition}
  \label{partial ordered set}{\index{partial ordered set}}{\index{poset}}If we
  have a class $A$ together with a order relation $R$ in $A$ then the pair
  $\langle A, R \rangle$ (see \ref{pair of classes}) is called a partially
  ordered class. If $A$ is a set then $\langle A, R \rangle$ is called a
  partially ordered set.
\end{definition}

\begin{notation}
  If $\langle A, R \rangle$ is a partially ordered class (set) then $R$ is
  noted as $\leqslant$ so $x \leqslant y$ is the same as $(x, y) \in R$. In
  words we say that $x$ is smaller then $y$. Also $y \geqslant x$ ($y$ is
  greater then $x$) is another way of saying $x \leqslant y$. Finally $x < y$
  ($x$ is strictly less then $y$) is a abbreviation for $x \leqslant y \wedge
  x \neq y$ and $y > x$ ($y$ is strictly greater then $x$) means $x < y \wedge
  x \neq y$.
\end{notation}

\begin{theorem}
  \label{property of stritly less}If $\langle A, \leqslant \rangle$ is a
  partially ordered set then
  \begin{enumerate}
    \item $x \leqslant y \wedge y < z \Rightarrow x < z$
    
    \item $x < y \wedge y \leqslant z \Rightarrow x < z$
    
    \item $x < y \wedge y < z \Rightarrow x < z$
    
    \item $(x < y \vee x = y) \Leftrightarrow (x \leqslant y)$
  \end{enumerate}
\end{theorem}

\begin{proof}
  
  \begin{enumerate}
    \item $x \leqslant y \wedge y < z \Rightarrow x \leqslant y \wedge y
    \leqslant z \wedge y \neq z \Rightarrow x \leqslant z \wedge y \neq z$ if
    now $x = z \Rightarrow z \leqslant y \Rightarrow z = y$ a contradiction so
    we must have $x \leqslant z \wedge x \neq z \Rightarrow x < z$
    
    \item $x < y \wedge y \leqslant z \Rightarrow x \leqslant y \wedge y
    \leqslant z \wedge x \neq y \Rightarrow x \leqslant z \wedge x \neq y$
    then if $x = z \Rightarrow y \leqslant x \Rightarrow x = y$ a
    contradiction so we must have $x \leqslant z \wedge x \neq z \Rightarrow x
    < z$
    
    \item $x < y \wedge y < z \Rightarrow x \leqslant y \wedge y < z
    \Rightarrowlim_{(1)} x < z$
    
    \item We have
    \begin{eqnarray*}
      (x < y \vee x = y) & \Leftrightarrow & ((x \leqslant y \wedge x \neq y)
      \vee x = y)\\
      & \Leftrightarrow & ((x \leqslant y \vee x = y) \wedge (x \neq y \vee x
      = y))\\
      & \Leftrightarrow & x \leqslant y \vee x = y\\
      & \Rightarrowlim_{y \leqslant y \wedge x = y \Rightarrow x \leqslant y}
      & x \leqslant y \vee x \leqslant y\\
      & \Rightarrow & x \leqslant y\\
      x \leqslant y & \Rightarrow & x \leqslant y \wedge (x = y \vee x \neq
      y)\\
      & \Rightarrow & (x \leqslant y \wedge x = y) \vee (x \leqslant y \wedge
      x \neq y)\\
      & \Rightarrow & (x = y) \vee (x \leqslant y \wedge x \neq y)\\
      & \Rightarrow & x = y \vee x < y
    \end{eqnarray*}
  \end{enumerate}
  
\end{proof}

\begin{theorem}
  \label{pre-order to order}If $\langle A, \leqslant \rangle$ is a pre-ordered
  class then
  \begin{enumerate}
    \item $\sim \in A \times A$ defined by $x \sim y$ if and only if $x
    \leqslant y \wedge y \leqslant x$ is a equivalence relation.
    
    \item If $\sim [x], \sim [y] \in A / \sim$ with $\sim [x] \neq \sim [y]$
    then we define $\sim [x] < \sim [y]$ iff $x < y$, this definition is well
    defined (so independent of the choice of the representation).
    
    \item If we define $\sim [x] \leqslant \sim [y]$ by $\sim [x] < \sim [y]
    \vee \sim [x] = \sim [y]$ then \ $\langle A / \sim, \leqslant \rangle$ is
    a ordered class.
    
    \item If \ $x \leqslant y \Rightarrow \sim [x] \leqslant \sim [y]$
    
    \item If $\sim [x] \leqslant \sim [y]$ and $x' \in \sim [x], y' \in \sim
    [y]$ then $x' \leqslant y'$
  \end{enumerate}
\end{theorem}

\begin{proof}
  
  \begin{enumerate}
    \item To prove that $\sim$ is a equivalence relation note that
    \begin{enumerate}
      \item If $x \in A$ then $x \leqslant x \wedge x \leqslant x \Rightarrow
      x \sim x$
      
      \item If $x, y \in A$ and $x \sim y$ then $x \leqslant y \wedge y
      \leqslant x \Rightarrow y \leqslant x \wedge x \leqslant y \Rightarrow y
      \sim x$
      
      \item If $x, y, z \in A$ and $x \sim y \wedge y \sim z \Rightarrow x
      \leqslant y \wedge y \leqslant x \wedge y \leqslant z \wedge z \leqslant
      y \Rightarrow x \leqslant z \wedge z \leqslant x \Rightarrow x \sim z$
    \end{enumerate}
    \item If $\sim [x'] = \sim [x] \neq \sim [y] = \sim [y']$ with $y'
    \leqslant x'$ then $y' \leqslant x' \wedge x' \leqslant x \wedge x
    \leqslant x' \wedge y \leqslant y' \wedge y' \leqslant y \Rightarrow y
    \leqslant x$ which together with $x < y \Rightarrow x \leqslant y$ means
    that $x \sim y$ contradicting $\sim [x] \neq \sim [y]$ so we must have $x'
    < y'$ proving that the definition is well defined.
    
    \item 
    \begin{enumerate}
      \item If $\sim [x] \in A / \sim$ then $\sim [x] = \sim [x] \Rightarrow
      \sim [x] \leqslant \sim [x]$
      
      \item If $\sim [x] \leqslant \sim [y]$ and $\sim [y] \leqslant \sim [x]$
      then we have either
      \begin{enumerate}
        \item $\sim [x] = \sim [y]$ proving anti-symmetry
        
        \item $\sim [x] < \sim [y]$ \ and $\sim [y] < \sim [x] \Rightarrow x <
        y \wedge y < x \Rightarrow x \leqslant y \wedge y \leqslant x
        \Rightarrow x \sim y \Rightarrow \sim [x] = \sim [y]$ proving
        anti-symmetry
      \end{enumerate}
      \item If $\sim [x] \leqslant \sim [y] \wedge \sim [y] \leqslant \sim
      [z]$ then we have the following cases to consider
      \begin{enumerate}
        \item $\sim [x] = \sim [y] \wedge \sim [y] = \sim [z] \Rightarrow \sim
        [x] = \sim [z] \Rightarrow \sim [x] \leqslant \sim [z]$
        
        \item $\sim [x] = \sim [y] \wedge \sim [y] < \sim [z] \Rightarrow \sim
        [x] < \sim [z] = \sim [x] \leqslant \sim [z]$
        
        \item $\sim [x] < \sim [y] \wedge \sim [y] = \sim [z] \Rightarrow \sim
        [x] < \sim [z] \Rightarrow \sim [x] \leqslant \sim [z]$
        
        \item $\sim [x] < \sim [y] \wedge \sim [y] < \sim [z] \Rightarrow x <
        y \wedge y < z \Rightarrow x < z \Rightarrow \sim [x] < \sim [y]
        \Rightarrow \sim [x] \leqslant \sim [y]$
      \end{enumerate}
    \end{enumerate}
    \item If $x \leqslant y \Rightarrow x < y \vee x = y \Rightarrow \sim [x]
    < \sim [y] \vee \sim [x] = \sim [y] \Rightarrow \sim [x] \leqslant \sim
    [y]$
    
    \item If $x' \in \sim [x], y' \in \sim [y]$ then $x' \leqslant x \wedge x
    \leqslant x'$ and $y' \leqslant y \wedge y \leqslant y'$. If $\sim [x]
    \leqslant \sim [y] \Rightarrow \sim [x] < \sim [y] \vee \sim [x] = \sim
    [y]$ giving the following possibilities
    \begin{enumerate}
      \item $x < y \Rightarrow x' \leqslant x < y \leqslant y' \Rightarrow x'
      \leqslant y'$
      
      \item $x \leqslant y \wedge y \leqslant x \Rightarrow x' \leqslant x
      \leqslant y \leqslant y' \Rightarrow x' \leqslant y'$
    \end{enumerate}
  \end{enumerate}
\end{proof}

\

\begin{theorem}
  \label{induced order relation}{\index{induced order relation}}If $\langle A,
  \leqslant \rangle$ is a partial ordered (pre-ordered) class and $B \subseteq
  A$ then we can define the relation $\leqslant_{| B \nobracket} = \leqslant
  \bigcap (B \times B) \subseteq B \times B$ we have then that $\langle B,
  \leqslant_{| B \nobracket} \rangle$ is a partial ordered (pre-ordered)
  class. In general we use the simpler notation $\langle B, \leqslant \rangle$
  for the order relation (pre-order) on $B$.
\end{theorem}

\begin{proof}
  The proof is quite trivial
  \begin{enumerate}
    \item $x \in B \Rightarrow x \in A \Rightarrow x \leqslant x \Rightarrow
    (x, x) \in \leqslant \bigcap (B \times B) = \leqslant_{| B \nobracket}$
    
    \item $x \leqslant_{| B \nobracket} y \wedge y \leqslant_{| B \nobracket}
    x \Rightarrow x \leqslant y \wedge y \leqslant x \Rightarrow x = y$
    
    \item $x \leqslant_{| B \nobracket} y \wedge y \leqslant_{| B \nobracket}
    z \Rightarrow x, y, z \in B \wedge x \leqslant y \wedge y \leqslant z
    \Rightarrow x, z \in B \wedge x \leqslant z \Rightarrow x \leqslant_{| B
    \nobracket} z$
  \end{enumerate}
\end{proof}

\begin{definition}
  If $\langle A, \leqslant_A \rangle$ and $\langle B, \leqslant_B \rangle$ are
  partially ordered sets then the lexical order $\leqslant_{A, B} \subseteq (A
  \times B) \times (A \times B)$ is defined by $\leqslant_{A, B} = \left\{
  ((x, y), (x', y')) \in (A \times B) \times (A \times B) \left| \left( \left(
  x \neq x' \right) \wedge (x \leqslant_A x') \right) \vee ((x = x') \wedge (y
  \leqslant_B y')) \right| \right\}$
\end{definition}

\begin{theorem}
  \label{lexical order}If $\langle A, \leqslant_A \rangle$ and $\langle B,
  \leqslant_B \rangle$ are partially ordered sets then $\langle A \times B,
  \leqslant_{A, B} \rangle$ is a partially ordered set
\end{theorem}

\begin{proof}
  
  \begin{enumerate}
    \item If $(x, y) \in A \times B$ then $x = x \wedge y \leqslant_B y
    \Rightarrow (x, y) \leqslant_{A, B} (x, y)$
    
    \item If $(x, y) \leqslant_{A, B} (x', y')$ and $(x', y') \leqslant_{A, B}
    (x, y)$ then if $x \neq x'$ we must have $x \leqslant_A x'$ and $x'
    \leqslant_A x$ giving $x = x'$ a contradiction. So we must have $x = x'$
    but then we have $y \leqslant_B y'$ and $y' \leqslant_B y$ and thus $y =
    y'$
    
    \item If $(x, y) \leqslant_{A, B} (x', y')$ and $(x', y') \leqslant_{A, B}
    (x'', y'')$ then consider the following possible cases for $x, x'$
    \begin{enumerate}
      \item $x = x' \Rightarrow y \leqslant_B y'$ now we have the following
      possibilities for $x'$ and $x''$
      \begin{enumerate}
        \item $x' = x'' \Rightarrow y' \leqslant_B y'' \Rightarrow x = x''
        \wedge y \leqslant_B y'' \Rightarrow (x, y) \leqslant_{A, B} (x'',
        y'')$
        
        \item $x' \neq x'' \Rightarrow x' \leqslant_A x'' \Rightarrowlim_{x =
        x' \Rightarrow x \leqslant_A x'} x \neq x'' \wedge x \leqslant_A x''
        \Rightarrow (x, y) \leqslant_{A, B} (x'', y'')$
      \end{enumerate}
      \item $x \neq x' \Rightarrow x \leqslant_A x'$ now we have the following
      possibilities for $x'$ and $x''$
      \begin{enumerate}
        \item $x' = x'' \Rightarrow x' \leqslant_A x'' \wedge x \neq x''
        \Rightarrow x \neq x'' \wedge x \leqslant_A x'' \Rightarrow (x, y)
        \leqslant_{A, B} (x'', y'')$
        
        \item $x' \neq x'' \Rightarrow x' \leqslant_A x'' \Rightarrow x
        \leqslant_A x''$ now we have the following cases
        \begin{enumerate}
          \item $x = x'' \Rightarrow x' \leqslant_A x \Rightarrow x = x'$ a
          contradiction so this case does not occur
          
          \item $x \neq x'' \Rightarrow (x, y) \leqslant_{A, B} (x'', y'')$
        \end{enumerate}
      \end{enumerate}
    \end{enumerate}
  \end{enumerate}
\end{proof}

\begin{definition}
  {\index{comparability}}\label{comparability}If $\langle A, \leqslant
  \rangle$ is a partially ordered class then $x, y \in A$ are comparable if $x
  \leqslant y$ or $y \leqslant x$ 
\end{definition}

\begin{theorem}
  \label{comparable elements and relations}If $\langle A, \leqslant \rangle$
  is a partially ordered class and $x, y \in A$ are comparable then we have
  either $x \leqslant y$ or $y < x$
\end{theorem}

\begin{proof}
  If $x, y$ are comparable then we have $(x \leqslant y) \vee (y \leqslant x)$
  now we have the following exclusive cases
  \begin{enumerate}
    \item $x \leqslant y \Rightarrow x \leqslant y$
    
    \item $\neg (x \leqslant y) \Rightarrow y \leqslant x$ now if $x = y
    \Rightarrow x \leqslant y$ contradicting $\neg (x \leqslant y)$ so we have
    $x \neq y$ and this together with $y \leqslant x$ gives $y < x$ 
  \end{enumerate}
\end{proof}

\

\begin{definition}
  \label{linear ordered class}{\index{linear ordered class}}{\index{fully
  ordered class}}{\index{chain}}{\index{chain}}A partially ordered
  (pre-ordered) class $\langle A, \leqslant \rangle$ is called a
  \tmtextbf{fully ordered class} (\tmtextbf{fully pre-ordered)} or a
  \tmtextbf{linear ordered class} (\tmtextbf{linear pre-ordered class)} if
  $\forall x, y \in A \vDash x \leqslant y \vee y \leqslant x$. A subclass $B
  \subseteq A$ is called a \tmtextbf{fully ordered (pre-ordered) subclass of}
  $A$ or a \tmtextbf{linear ordered (pre-ordered) subclass} or a
  \tmtextbf{chain} of $A$ if $\forall x, y \in B \vDash x \leqslant y \vee y
  \leqslant x$. So a fully ordered (sub)class is a (sub)class where every
  element is comparable with every other element. Also in a fully ordered
  class we have for every $x, y$ either $x \leqslant y$ or $y < x$ (see
  previous theorem).
\end{definition}

\begin{definition}
  \label{initial segment}{\index{initial segment}}{\index{$S_a$}}If $\langle
  A, \leqslant \rangle$ is a partially ordered class and $a \in A$ then the
  \tmtextbf{initial segment of A determined by a} $S_a$ is defined by
  \begin{eqnarray*}
    S_a & = & \{ x \in A | x < a \nobracket \}
  \end{eqnarray*}
\end{definition}

\begin{theorem}
  If $\langle A. \leqslant \rangle$ and $P$ is a initial segment of $A$ and
  $Q$ a initial segment of $P$ (using the induced order relation on $P$) then
  $Q$ is a initial segment of $A$
\end{theorem}

\begin{proof}
  By the hypothesis of the theorem we have $\exists a \in A \vdash P = \{ x
  \in A | x < a \nobracket \}$ and $\exists b \in P \vdash Q = \{ x \in P
  \vdash x <_{| P \nobracket} b \}$ then we have as $b \in P \Rightarrow b <
  a$
  \begin{eqnarray*}
    x \in S_b = \{ x \in A | x < b \nobracket \} & \Leftrightarrow & x \in A
    \wedge x < b\\
    & \Rightarrowlim_{x < b < a \Rightarrow x < a} & x \in A \wedge x < a
    \wedge x < b\\
    & \Rightarrow & x \in P \wedge x < b\\
    & \Rightarrow & x \in P \wedge x <_{| P \nobracket} b\\
    & \Rightarrow & x \in Q\\
    x \in Q & \Rightarrow & x \in P \wedge x <_{| p \nobracket} b\\
    & \Rightarrow & x \in P \wedge x < b\\
    & \Rightarrowlim_{P \subseteq A} & x \in A \wedge x < b\\
    & \Rightarrow & x \in S_b
  \end{eqnarray*}
\end{proof}

\begin{definition}
  If $\langle A, \leqslant \rangle$ is a partially ordered class then a
  \tmtextbf{cut} of $A$ is a pair $\langle C, D \rangle$ such that
  \begin{enumerate}
    \item $C \neq \emptyset \wedge D \neq \emptyset$
    
    \item $C \bigcap D = \emptyset$
    
    \item $x \in C \wedge y \leqslant x \Rightarrow y \in C$
    
    \item $x \in D \wedge y \geqslant x \Rightarrow y \in D$
  \end{enumerate}
\end{definition}

\begin{definition}
  {\index{increasing function}}\label{increasing function}If $\langle A,
  \leqslant_A \rangle$ and $\langle B, \leqslant_B \rangle$ are partially
  ordered classes then a function $f : A \rightarrow B$ is
  \tmtextbf{increasing} if $\forall x, y \in A$ we have
  \begin{eqnarray*}
    x \leqslant y & \Rightarrow & f (x) \leqslant f (y)
  \end{eqnarray*}
  Such a function is also called \tmtextbf{order preserving}.
\end{definition}

\begin{theorem}
  \label{composition of increasing functions}If $\langle A, \leqslant_A
  \rangle$, $\langle B, \leqslant_B \rangle$ and $\langle C, \leqslant_C
  \rangle$ are partially ordered classes then $f : A \rightarrow B$ and $g : B
  \rightarrow C$ are increasing functions then $g \circ f$ is a increasing
  function. Further if $f$ and $g$ is strictly increasing then $g \circ f$ is
  strictly increasing.
\end{theorem}

\begin{proof}
  If $x \leqslant y \Rightarrow f (x) \leqslant f (y) \Rightarrow g (f (x))
  \leqslant g (f (y)) \Rightarrow (g \circ f) (x) \leqslant (g \circ f) (y)$.
  If $g$ is strictly increasing then $x < y \Rightarrow f (x) < f (y)
  \Rightarrow g (f (x)) < g (f (y)) \Rightarrow (g \circ f) (x) < (g \circ f)
  (y)$
\end{proof}

\begin{definition}
  {\index{strictly increasing function}}\label{strictly increasing function}If
  $\langle A, \leqslant_A \rangle$ and $\langle B, \leqslant_B \rangle$ are
  partially ordered classes then a function $f : A \rightarrow B$ is
  \tmtextbf{strictly increasing} if $\forall x, y \in A$ we have
  \begin{eqnarray*}
    x < y & \Rightarrow & f (x) < f (y)
  \end{eqnarray*}
\end{definition}

\begin{definition}
  {\index{isomorphism}}\label{isomorphism}If $\langle A, \leqslant_A \rangle$
  and $\langle B, \leqslant_B \rangle$ are partially ordered classes then a
  function $f : A \rightarrow B$ is a \tmtextbf{isomorphism} iff $f$ is
  bijective and \ $\forall x, y \in A$ we have
  \begin{eqnarray*}
    x \leqslant_A y & \Leftrightarrow & f (x) \leqslant_B f (y)
  \end{eqnarray*}
\end{definition}

\begin{notation}
  \label{isomorph classes}{\index{$A \simeq B$}}{\index{$\cong$}}If $\langle
  A, \leqslant_A \rangle$ and $\langle B, \leqslant_B \rangle$ are partially
  ordered class and there exists a isomorphism $f : A \rightarrow B$ between
  $A$ and $B$ then we say that $A$ and $B$ are isomorphic noted by $A \cong B$
\end{notation}

\begin{theorem}
  \label{isomorphism is strictly increasing}If $\langle A, \leqslant_A
  \rangle$ and $\langle B, \leqslant_B \rangle$ are partially ordered classes
  and $f : A \rightarrow B$ \ a \tmtextbf{isomorphism} then
  \begin{eqnarray*}
    x <_A y & \Leftrightarrow & f (x) <_B f (y)
  \end{eqnarray*}
\end{theorem}

\begin{proof}
  
  
  $\Rightarrow$
  
  If $x <_A y \Rightarrow x \leqslant_A y \wedge x \neq y$ $\Rightarrow f (x)
  \leqslant_B f (y)$. If $f (x) = f (y) \Rightarrowlim_{f \tmop{is}
  \tmop{injective}} x = y$ contradicting $x \neq y$ so we have $f (x) \neq f
  (y)$ and thus $f (x) <_B f (y)$
  
  $\Leftarrow$
  
  If $f (x) <_B f (y) \Rightarrow f (x) \leqslant_B f (y) \Rightarrowlim_{f
  \tmop{is} \tmop{isomorphism}} x \leqslant_A y$ now if $x = y \Rightarrow f
  (x) = f (y)$ contradicting $f (x) <_B f (y) \Rightarrow x <_A y$
\end{proof}

\begin{theorem}
  \label{condition for bijection to be isomorph}If $\langle A, \leqslant_A
  \rangle$ and $\langle B, \leqslant_B \rangle$ are partially ordered classes
  then a bijective function $f : A \rightarrow B$ is a isomorphism iff $f : A
  \rightarrow B$ and $f^{- 1} : B \rightarrow A$ are increasing functions
\end{theorem}

\begin{proof}
  First as $f$ is bijective $f^{- 1}$ exists and
  \begin{enumerate}
    \item If $f$ is a isomorphism then $x \leqslant_A y \Rightarrow f (x)
    \leqslant_B f (y) \Rightarrow f$ is a increasing function. If $x, y \in B$
    with $x \leqslant_A y$ then $f (f^{- 1} (x)) = x \leqslant_A y = f (f^{-
    1} (y)) \Rightarrowlim_{f \tmop{is} a \tmop{isomorphism}} f^{- 1} (x)
    \leqslant_B f^{- 1} (y)$ so $f^{- 1}$ is a increasing function.
    
    \item Suppose $f, f^{- 1}$ are increasing functions then if $x \leqslant_A
    y \Rightarrowlim_{f \tmop{is} \tmop{increasing}} f (x) \leqslant_B f (y)$.
    Suppose that $f (x) \leqslant_B f (y) \Rightarrowlim_{f^{- 1} \tmop{is}
    \tmop{increasing}} f^{- 1} (f (x)) \leqslant_A f^{- 1} (f (y)) \Rightarrow
    x \leqslant y$
  \end{enumerate}
\end{proof}

\begin{theorem}
  If $\langle A, \leqslant_A \rangle$, $\langle B, \leqslant_B \rangle$ and
  $\langle C, \leqslant_C \rangle$ are partially ordered classes then
  \begin{enumerate}
    \item $1_A : A \rightarrow A$ is a isomorphism
    
    \item If $f : A \rightarrow B$ is a isomorphism then $f^{- 1} : B
    \rightarrow A$ is a isomorphism
    
    \item If $f : A \rightarrow B$ and $g : B \rightarrow C$ are isomorphism's
    then $g \circ f$ is a isomorphism
  \end{enumerate}
\end{theorem}

\begin{proof}
  
  \begin{enumerate}
    \item By \ref{identity function} we have that $1_A : A \rightarrow A$ is a
    bijection then as $x = 1_A (x)$ and $y = 1_A (y)$ so that $x \leqslant y
    \Leftrightarrow 1_A (x) \leqslant 1_A (y)$.
    
    \item If $f : A \rightarrow B$ is a isomorphism then by \ref{bijection and
    its inverse} we have that $f^{- 1} : B \rightarrow A$ is a bijection. By
    the previous theorem we have that $f^{- 1}$ is increasing. So assume $f^{-
    1} (x) \leqslant f^{- 1} (y) \Rightarrowlim_{f \tmop{is}
    \tmop{increasing}} f (f^{- 1} (x)) \leqslant f (f^{- 1} (y)) \Rightarrow x
    \leqslant y$.
    
    \item Using \ref{properties of composition of functions} we have that $g
    \circ f$ is a bijection and that $(g \circ f)^{- 1} = f^{- 1} \circ g^{-
    1}$. Now using \ref{composition of increasing functions} and the fact that
    $f, g, f^{- 1}, g^{- 1}$ are increasing we have that $g \circ f$ and $(g
    \circ f)^{- 1} = f^{- 1} \circ g^{- 1}$ are increasing. Using
    \ref{condition for bijection to be isomorph} we have then that $g \circ f$
    is a isomorphism.
  \end{enumerate}
\end{proof}

\begin{theorem}
  \label{properties of the isomorph relation}If $\langle A, \leqslant_A
  \rangle$, $\langle B, \leqslant_B \rangle$ and $\langle C, \leqslant_C
  \rangle$ are partially ordered classes then we have
  \begin{enumerate}
    \item $A \cong A$
    
    \item If $A \cong B$ then $B \cong A$
    
    \item If $A \cong B$ and $B \cong D$ then $B \cong D$
  \end{enumerate}
\end{theorem}

\begin{proof}
  This follows easily from the previous theorem.
\end{proof}

\begin{theorem}
  \label{condition for isomorphism in a full ordered set}If $\langle A,
  \leqslant_A \rangle$ is a fully ordered class and $\langle B, \leqslant_B
  \rangle$ is a partially ordered class then a bijective and increasing
  function $f : A \rightarrow B$ is a isomorphism
\end{theorem}

\begin{proof}
  Suppose that $f (x) \leqslant_B f (y)$ then since $A$ is fully ordered we
  have that $x, y$ are comparable therefore by \ref{comparable elements and
  relations} we have the following exclusive cases
  \begin{enumerate}
    \item $x \leqslant_A y$ in this case or theorem is proved
    
    \item $y <_A x$ in this case we would have $f (y) \leqslant_B f (x)
    \Rightarrow f (y) = f (x) \Rightarrowlim_{f \tmop{is} \tmop{injective}} x
    = y$ a contradiction. So this case does not occurs.
  \end{enumerate}
\end{proof}

\begin{definition}
  \label{maximal (minimal element in a pre-ordered class}If $\langle X,
  \leqslant \rangle$ is a pre-ordered class and $A \leqslant X$ then
  \begin{enumerate}
    \item $m$ is a maximal element of $A$ iff $m \in A$ and if $x \in A$ with
    $m \leqslant x$ then $x \leqslant m$
    
    \item $m$ is a minimal element of $A$ iff $m \in A$ and if $x \in A$ with
    $x \leqslant m$ then $m \leqslant x$
  \end{enumerate}
\end{definition}

\begin{definition}
  \label{maximal, minimal, greatest, least}{\index{maximal
  element}}{\index{minimal element}}{\index{greatest element}}{\index{least
  element}}If $\langle X, \leqslant \rangle$ is a partially ordered class and
  $A \subseteq X$ then
  \begin{enumerate}
    \item $m$ is a maximal element of $A$ iff $m \in A$ and $\forall x \in A$
    if $x \geqslant m \Rightarrow x = m$
    
    \item $m$ is a minimal element of $A$ iff $m \in A$ and $\forall x \in A$
    if $x \leqslant m \Rightarrow x = m$
    
    \item m is the greatest element of $A$ iff $m \in A$ and $\forall x \in A
    \vdash x \leqslant m$
    
    \item $m$ is the least element of $A$ iff $m \in A$ and $\forall x \in A
    \vdash m \leqslant x$
  \end{enumerate}
  note that (3) and (4) suggests that there is only one greatest and least
  element, this is indeed true as is proved in the next theorem.
\end{definition}

\begin{note}
  There is a difference between the definition of a maximal element and a
  greatest element. If $m$ is a greatest element of $A$ then $m$ is comparable
  with every element in $A$. This is not needed if $m$ is a maximal element of
  $A$. The same goes for minimal element and least element.
\end{note}

\begin{theorem}
  \label{maximum and minimum}If $\langle X, \leqslant \rangle$ is a partially
  ordered class and $A \subseteq X$ then if $A$ has a greatest element then
  this is unique. If $A$ has a least element then it is unique. The unique
  greatest element of $A$ if it exists is called the maximum of $A$ and noted
  by $\max (A)$. The unique least element if it exists is called the minimum
  of $A$ and noted by $\min (A)$.
  
  Note: The existence of a maximum is not the same as the existence of a
  maximal element, the existence of a minimum is not the same as the existence
  of a minimal element.
\end{theorem}

\begin{proof}
  
  \begin{enumerate}
    \item If $m, m'$ are greatest elements then $m, m' \in A \Rightarrow m
    \leqslant m' \wedge m' \leqslant m \Rightarrow m = m'$
    
    \item If $m, m'$ are least elements then $m, m' \in A \Rightarrow m
    \leqslant m' \wedge m' \leqslant m \Rightarrow m = m'$
  \end{enumerate}
\end{proof}

\begin{theorem}
  \label{maximum of class with bigger elements then another class}Let $\langle
  X, \leqslant \rangle$ be a partially ordered class and let $B, C \subseteq
  X$ subclasses such that $\max (B), \max (C)$ exists [or $\min (B) \nocomma
  \min (C)$ exists] then if $\forall x \in B$ there exists a $y \in C$ with $x
  \leqslant y$ [or $\forall x \in B$ there exists a $y \in C$ such that $y
  \leqslant x$] then $\max (B) \leqslant \max (C)$ [or $\min (C) \leqslant
  \min (B)$] 
\end{theorem}

\begin{proof}
  If $\max (B), \max (C)$ exists then $\max (B) \in B$ and thus there exist a
  $c \in C$ such that $\max (B) \leqslant c \leqslant \max (C) \Rightarrow
  \max (B) \leqslant \max (C)$. If $\min (B), \min (C)$ exists then as $\min
  (B) \in B$ there exists a $c \in C$ such that $\min (C) \leqslant c
  \leqslant \min (B) = \min (C) \leqslant \min (B)$.
\end{proof}

\begin{definition}
  \label{upper bound in pre-ordered set}If $\langle A, \leqslant \rangle$ is a
  pre-ordered class and $B \subseteq A$ then
  \begin{enumerate}
    \item $u \in A$ is a upper bound of $B$ iff $\forall x \in B \vdash x
    \leqslant u$
    
    \item $B$ is bounded above if it has a upper bound
    
    \item $l \in A$ is a lower bound of $B$ iff $\forall x \in B \vdash l
    \leqslant x$
    
    \item $B$ is bounded below if it has a lower bound
  \end{enumerate}
\end{definition}

\begin{definition}
  \label{upper bound}{\index{upper bound}}{\index{$\sup (A)$}}{\index{$\inf
  (A)$}}{\index{lower bound}}If $\langle A, \leqslant \rangle$ is a partially
  ordered class and $B \subseteq A$ a subclass of $A$ then
  \begin{enumerate}
    \item $u \in A$ is a upper bound of $B$ iff $\forall x \in B \vdash x
    \leqslant u$
    
    \item $B$ is bounded above if it has a upper bound
    
    \item $l \in A$ is a lower bound of $B$ iff $\forall x \in B \vdash l
    \leqslant x$
    
    \item $B$ is bounded below if it has a lower bound
    
    \item $\upsilon (B) = \{ u \in A | u \tmop{is} a \tmop{upper} \tmop{bound}
    \tmop{of} B \nobracket \}$ (the class of all the upper bounds of $B$)
    
    \item $\lambda (B) = \{ l \in A | l \tmop{is} a \tmop{lower} \tmop{bound}
    \tmop{of} B \nobracket \}$ (the class of all the lower bounds of $B$)
    
    \item If $\upsilon (B)$ has a least element then this is called the
    \tmtextbf{least upper bound} or \tmtextbf{supremum} of $B$ which is noted
    as $\sup (B)$ (it is unique by the previous theorem).
    
    \item If $\lambda (B)$ has a greatest element then this is called the
    \tmtextbf{greatest lower bound} or \tmtextbf{infinum} of $B$ which is
    noted as $\inf (B)$ (it is unique by the previous theorem).
  \end{enumerate}
\end{definition}

\begin{theorem}
  \label{property of inf and sup}If $\langle A, \leqslant \rangle$ is a
  partially ordered class and $B \subseteq A$ then we have
  \begin{enumerate}
    \item If $\inf (B)$ exists then $\forall a \in A$ with $\inf (B) < a$
    there exists a $b \in B$ such that $\inf (B) \leqslant b < a$
    
    \item If $\sup (B)$ exists then $\forall a \in A$ with $a < \sup (B)$
    there exists a $b \in B$ such that $a < b \leqslant \sup (B)$
  \end{enumerate}
\end{theorem}

\begin{proof}
  
  \begin{enumerate}
    \item We proceed by contradiction so assume that $\exists a \in A$ with
    $\inf (B) < a$ so that $\forall b \in B$ we have that $\neg (\inf (B)
    \leqslant b < a) = \neg (\inf (B) \leqslant b \wedge b < a) = b < \inf (B)
    \vee a \leqslant b \Rightarrowlim_{\inf (B) \tmop{is} a \tmop{lower}
    \tmop{bound} \tmop{of} B \Rightarrow \inf (B) \leqslant b} a \leqslant b
    \Rightarrow a$ is a lower bound of $B$ and thus by definition of $\inf
    (B)$ we have $a \leqslant \inf (B)$ contradicting $\inf (B) < a$. So we
    must have that $\forall a \in A$ with $\inf (B) < a$ there exist a $b \in
    B$ such that $\inf (B) \leqslant b < a$.
    
    \item We proceed by contradiction so assume that $\exists a \in A$ with $a
    < \sup (B)$ so that $\forall b \in B$ we have that $\neg (a < b \leqslant
    \sup (B)) = \neg (a < b \wedge b \leqslant \sup (B)) = b \leqslant a \vee
    \sup (B) < b \Rightarrowlim_{\sup (B) \tmop{is} a \tmop{upper}
    \tmop{bound} \tmop{of} B \Rightarrow b \leqslant \sup (B)} b \leqslant a
    \Rightarrow a$ is a upper bound of $B$ and thus by definition of $\sup
    (B)$ we have $\sup (B) \leqslant a$ contradicting $a < \sup (B)$. So we
    must have that $\forall a \in A$ with $a < \sup (B)$ there exist a $b \in
    B$ such that $a < b \leqslant \sup (B)$.
  \end{enumerate}
\end{proof}

\begin{theorem}
  \label{inf, sup in class ordered by inclusion}Let $\mathcal{A}$ be a class
  of classes partially ordered by inclusion ($\langle \mathcal{A}, \subseteq
  \rangle$) then if $\mathcal{B} \subseteq \mathcal{A}$ then
  \begin{enumerate}
    \item If $\bigcap_{B \in \mathcal{B}} B \in \mathcal{A} \Rightarrow \inf
    (\mathcal{B}) = \bigcap_{B \in \mathcal{B}} B$
    
    \item If $\bigcup_{B \in \mathcal{B}} B \in \mathcal{A} \Rightarrow \sup
    (\mathcal{B}) = \bigcup_{B \in \mathcal{B}} B$
  \end{enumerate}
\end{theorem}

\begin{proof}
  
  \begin{enumerate}
    \item If $B \in \mathcal{B} \Rightarrow \bigcap_{B \in \mathcal{B}} B
    \subseteq B \Rightarrow \bigcap_{B \in \mathcal{B}} B \in \lambda
    (\mathcal{B})$ now if $C \in \lambda (\mathcal{B})$ then $\forall B \in
    \mathcal{B}$ we have $C \subseteq B \Rightarrow C \subseteq \bigcap_{B \in
    \mathcal{B}} B$ so $\bigcap_{B \in \mathcal{B}} B$ is the greatest element
    of $\lambda (B)$
    
    \item If $B \in \mathcal{B} \Rightarrow B \subseteq \bigcup_{B \in
    \mathcal{B}} B \Rightarrow \bigcup_{B \in \mathcal{B}} B \in \upsilon
    (\mathcal{B})$ now if $C \in \upsilon (\mathcal{B})$ then $\forall B \in
    \mathcal{B}$ we have $B \subseteq C \Rightarrow \bigcup_{B \in
    \mathcal{B}} B \subseteq C$ so $\bigcup_{B \in \mathcal{B}} B$ is the
    least element of $\upsilon (\mathcal{B})$
  \end{enumerate}
\end{proof}

\begin{theorem}
  \label{inclusion and greatest and least element}If $\langle A, \leqslant
  \rangle$ is a partially ordered class then if $B, C \subseteq A$ with $B
  \subseteq C$ we have
  \begin{enumerate}
    \item If $B$ has a greatest element $b$ and $C$ has a greatest element $c$
    then $b \leqslant c$
    
    \item If $B$ has a least element $b$ and $C$ has a least element $c$ then
    $c \leqslant b$
  \end{enumerate}
\end{theorem}

\begin{proof}
  
  \begin{enumerate}
    \item $b \in B \Rightarrow b \in C \Rightarrowlim_{\forall x \in C \vdash
    x \leqslant c} b \leqslant c$
    
    \item $b \in B \Rightarrow b \in C \Rightarrowlim_{\forall x \in C \vdash
    c \leqslant x} c \leqslant b$
  \end{enumerate}
\end{proof}

\begin{theorem}
  \label{inclusion and upper and lower bounds}If $\langle A, \leqslant
  \rangle$ is a partially ordered class then if $B, C \subseteq A$ with $B
  \subseteq C$ we have
  \begin{enumerate}
    \item $\upsilon (C) \subseteq \upsilon (B)$
    
    \item $\lambda (C) \subseteq \lambda (B)$
  \end{enumerate}
\end{theorem}

\begin{proof}
  
  \begin{enumerate}
    \item If $x \in \upsilon (C)$ then $\forall b \in B \vdash b \in C
    \Rightarrow b \leqslant x \Rightarrow x \in \upsilon (B)$
    
    \item If $x \in \lambda (C)$ then $\forall b \in B \vdash b \in C
    \Rightarrow x \leqslant b \Rightarrow x \in \lambda (B)$
  \end{enumerate}
\end{proof}

\begin{theorem}
  \label{inclusion and sup and inf}If $\langle A, \leqslant \rangle$ is a
  partially ordered class then if $B, C \subseteq A$ with $B \subseteq C$
\end{theorem}
\begin{enumerate}
  \item If the supremum exists for $B, C$ then $\sup (B) \leqslant \sup (C)$
  
  \item If the infinum exists for $B, C$ then $\inf (C) \leqslant \inf (B)$
\end{enumerate}
\begin{proof}
  
  \begin{enumerate}
    \item By \ref{inclusion and upper and lower bounds} we have $\upsilon (C)
    \subseteq \upsilon (B)$ so using \ref{inclusion and greatest and least
    element} we have $\sup (B)$ = least element of $v (B) \leqslant
    \tmop{least} \tmop{element} \tmop{of} \upsilon (C) = \sup (C)$
    
    \item By \ref{inclusion and upper and lower bounds} we have $\lambda (C)
    \subseteq \lambda (B)$ so using \ref{inclusion and greatest and least
    element} we have $\inf (C) = \tmop{greatest} \tmop{element} \tmop{of}
    \lambda (C) \leqslant \tmop{greatest} \tmop{element} \tmop{of} B = \inf
    (B)$
  \end{enumerate}
\end{proof}

\begin{theorem}
  \label{sup of set of bigger elements}If $\langle A, \leqslant \rangle$ is a
  partially ordered class, $B, C \subseteq A$ are subsets of $A$ \ such that \
  $\forall b \in B$ there exists a $c \in C$ such that $b \leqslant c$ \ then
  $\sup (B) \leqslant \sup (C)$ (assuming that the supremums of $B, C$
  exists). On the other hand if $\forall b \in B$ there exists a $c \in C$
  such that $c \leqslant b$ then $\inf (C) \leqslant \inf (B)$ (assuming that
  the infinums of $B, C$ exits).
\end{theorem}

\begin{proof}
  If $b \in B$ then there exists a $c \in C$ such that $b \leqslant c
  \leqslant \sup (C)$ ($\sup (C)$ is a upper bound of $C$) so that $\sup (C)$
  is a upper bound of $B$ and thus we have $\sup (B) \leqslant \sup (C)$.
  
  On the other hand if $b \in B$ then there exists a $c \in C$ such that
  $\inf (C) \leqslant c \leqslant b$ so that $\inf (C)$ is a lower bound of
  $B$ so that $\inf (C) \leqslant \inf (B)$
\end{proof}

\begin{theorem}
  \label{set is included in its set of upper and lower bounds}If $\langle A,
  \leqslant \rangle$ is a partially ordered class with $B \subseteq A$ then $B
  \subseteq \upsilon (\lambda (B))$ and $B \subseteq \lambda (\upsilon (B))$
\end{theorem}

\begin{proof}
  
  \begin{enumerate}
    \item If $x \in B \Rightarrow \forall y \in \lambda (B) \vdash y \leqslant
    x \Rightarrow x \in \upsilon (\lambda (B))$
    
    \item If $x \in B \Rightarrow \forall y \in \upsilon (B) \vdash x
    \leqslant y \Rightarrow x \in \lambda (\upsilon (B))$
  \end{enumerate}
\end{proof}

\begin{theorem}
  \label{relation inf and sup}If $\langle A, \leqslant \rangle$ is a partially
  ordered class with $B \subseteq A$ a subclass then
  \begin{enumerate}
    \item If $\lambda (B)$ has a supremum then $B$ has a infinum and $\sup
    (\lambda (B)) = \inf (B)$
    
    \item If $\upsilon (B)$ has a infinum then $B$ has a supremum and $\inf
    (\upsilon (B)) = \sup (B)$
  \end{enumerate}
\end{theorem}

\begin{proof}
  
  \begin{enumerate}
    \item Let $s = \sup (\lambda (B))$ then if $b \in B$ we have $\forall y
    \in \lambda (B)$ that $y \leqslant b$ hence $b \in \upsilon (\lambda (B))$
    thus ($s$ is least upper bound of $\lambda (B)$) $s \leqslant b$. As $b$
    has been chosen arbitrary we have that $s \in \lambda (B)$. Now if $d \in
    \lambda (B)$ then $d \leqslant s$ [because $s = \sup (\lambda (B))
    \Rightarrow s$ is the least element of $\upsilon (\lambda (B)) \Rightarrow
    s \in \upsilon (\lambda (B))$] so $s$ is the greatest element of $\lambda
    (B)$ hence $s = \inf (B)$.
    
    \item Let $s = \inf (\upsilon (B))$ then if $b \in B$ we have $\forall y
    \in \upsilon (B)$ that $b \leqslant y$ hence $b \in \lambda (\upsilon
    (B))$ thus $(s \tmop{is} \tmop{greatest} \tmop{lower} \tmop{bound}
    \tmop{of} \upsilon (B))$ $b \leqslant s$. As $b$ has been chosen arbitrary
    we have $s \in \upsilon (B)$. Now if $d \in \upsilon (B)$ then $s
    \leqslant d$ [because $s = \inf (\upsilon (B)) \Rightarrow s$ is the
    greatest element of $\lambda (\upsilon (B)) \Rightarrow s \in \lambda
    (\upsilon (B))$] so $s$ is the least element of $\upsilon (B)$ hence $s =
    \sup (B)$
  \end{enumerate}
\end{proof}

\begin{definition}
  \label{conditonal complete classes}{\index{conditional complete classes}}If
  $\langle A, \leqslant \rangle$ is a partially ordered class. If every
  nonempty subclass of $A$ that is bounded above has a supremum then $A$ is
  said to be \tmtextbf{conditional complete}.
\end{definition}

Next theorem shows that conditional completeness can be defined also with
lower bounds and infinum

\begin{theorem}
  \label{alternative definition for conditional completeness}If $\langle A,
  \leqslant \rangle$ is a partially ordered class then the following are
  equivalent
  \begin{enumerate}
    \item Every nonempty subclass of $A$ that is bounded above has a supremum
    
    \item Every nonempty subclass of $A$ that is bounded below has a infinum
  \end{enumerate}
\end{theorem}

\begin{proof}
  
  
  $1 \Rightarrow 2$
  
  Let $B \subseteq A$ which is bounded below$\Rightarrow \lambda (B) \neq
  \emptyset$. Now as $B \neq \emptyset$ there exists a $b \in B$ and by
  definition of $\lambda (B)$ we have $\forall y \in \lambda (B)$ that $y
  \leqslant b$. So $\lambda (B)$ is nonempty and bounded above. From the
  hypothesis we have then that $\sup (\lambda (B))$ exists but then by
  \ref{relation inf and sup} we have that $B$ has a infinum.
  
  $2 \Rightarrow 1$
  
  Let $B \subseteq A$ which is bounded above$\Rightarrow \upsilon (B) \neq
  \emptyset$. Now as $B \neq \emptyset$ there exists a $b \in B$ and by
  definition of $\upsilon (B)$ we have $\forall y \in \upsilon (B)$ that $b
  \leqslant y$. So $\upsilon (B)$ is nonempty and bounded below. From the
  hypothesis we have then that $\tmop{in} (\upsilon (B))$ exists and then by
  \ref{relation inf and sup} we have that $B$ has a supremum.
\end{proof}

\begin{definition}
  \label{generalized interval}{\index{generalized interval}}Let $\langle A,
  \leqslant \rangle$ be a pre-ordered set then if $a, b \in A$ with $a
  \leqslant b$ we define
  \begin{enumerate}
    \item $[a, b] = \{ x \in A|a \leqslant x \wedge x \leqslant b \}$
    
    \item $] a, b] = \{ x \in A|a < x \wedge x \leqslant b \}$
    
    \item $[a, b [= \{ x \in A|a \leqslant x \wedge x < b \}$
    
    \item $] a, b [= \{ x \in A|a < x \wedge x < b \}$
    
    \item $] - \infty, b] = \{ x \in A|x \leqslant b \}$
    
    \item $] - \infty, b [= \{ x \in A|x < b \}$
    
    \item $[a, \infty [= \{ x \in A|a \leqslant x \}$
    
    \item $] a, \infty [= \{ x \in A|a < x \}$
    
    \item $] - \infty, \infty [= A$
    
    \item $] a, a [= \emptyset$
    
    \item $[a, a] = \{ a \}$
  \end{enumerate}
  If a set takes one of the forms (1)-(11) it is called a
  \tmtextbf{generalized interval}.
\end{definition}

We have a much simpler definition for a generalized interval if $X$is
conditional complete.

\begin{theorem}
  \label{generalized intervals condition}Let $\langle A, \leqslant \rangle$ be
  a partial ordered class that is conditional complete then the following are
  equivalent for a subset $B \subseteq A$
  \begin{enumerate}
    \item $B$ is a generalized interval
    
    \item $\forall x, y \in B$ with $x < y$ we have $[x, y] \subseteq A$
    
    \item $\forall x, y \in B$ with $x < y$ we have that $\forall z \in A$
    with $x \leqslant z \leqslant y$ that $z \in B$
    
    \item $\forall x, y \in B$ with $x < y$ we have that $\forall z \in A$
    with $x < z < y$ that $z \in B$
  \end{enumerate}
\end{theorem}

\begin{proof}[$1 \Rightarrow 2$][$B = [a, b]$][$B = [a, b [$][$B =] a, b]$][$B
=] a, b [$][$B =] - \infty, a]$][$B =] - \infty, a [$][$B = [a, \infty [$][$B
=] a, \infty [$][$B =] - \infty, \infty [$][$B = \emptyset$][$B = \{ a \}$][$2
\Rightarrow 3$][$3 \Rightarrow 4$][$4 \Rightarrow 1$][$B = \emptyset$][$B = \{
b \}$][$B \tmop{contains} \tmop{at} \tmop{least} \tmop{two}
\tmop{elements}$][$B$ has no upper/lower bound][$x \in [a_1, a_2]$][$x =
a_1$][$x = a_2$][a\tmrsub{1}<x{\wedge}x<a\tmrsub{2}][$x < a_1$][$a_2 < x$][$B
\tmop{has} a \tmop{upper} \tmop{and} \tmop{lower} \tmop{bound}$][$i, s \in
B$][$i \nin B \wedge s \in B$][$i \in B \wedge s \nin B$][$i, s \nin B$][$B
\tmop{has} \tmop{only} a \tmop{upper} \tmop{bound}$][$s \in B$][$s \nin B$][$B
\tmop{has} \tmop{only} a \tmop{lower} \tmop{bound}$][$i \in B$][$i \nin B$]
  The proof is easy but rather tedious
  \begin{enumerate}
    \item Let $B$ be a generalized interval then we have either
    \begin{enumerate}
      \item if $x, y \in [a, b]$ then $a \leqslant x, y \leqslant b$ so if $t
      \in [x, y] \Rightarrow x \leqslant t \wedge t \leqslant y$ so that $a
      \leqslant t \wedge t \leqslant b \Rightarrow t \in [a, b] = B
      \Rightarrow [x, y] \subseteq B$
      
      \item if $x, y \in [a, b [$ then $a \leqslant x, y < b$ so if $t \in [x,
      y] \Rightarrow x \leqslant t \wedge t \leqslant y$ so that $a \leqslant
      t \wedge t < b \Rightarrow t \in [a, b [= B \Rightarrow [x, y] \subseteq
      B$
      
      \item if $x, y \in] a, b]$ then $a < x, y \leqslant b$ so if $t \in [x,
      y] \Rightarrow x \leqslant t \wedge t \leqslant y$ so that $a < t \wedge
      t \leqslant b \Rightarrow t \in] a, b] = B \Rightarrow [x, y] \subseteq
      B$
      
      \item if $x, y \in] a, b [$ then $a < x, y < b$ so if $t \in [x, y]
      \Rightarrow x \leqslant t \wedge t \leqslant y$ so that $a < t \wedge t
      < b \Rightarrow t \in] a, b [= B \Rightarrow [x, y] \subseteq B$
      
      \item if $x, y \in] - \infty, a]$ then $x, y \leqslant a$ so if $t \in
      [x, y] \Rightarrow x \leqslant t \wedge t \leqslant y \Rightarrow t
      \leqslant a \Rightarrow t \in] - \infty, a] = B \Rightarrow [x, y]
      \subseteq B$
      
      \item if $x, y \in] - \infty, a [$ then $x, y < a$ so if $t \in [x, y]
      \Rightarrow x \leqslant t \wedge t \leqslant y \Rightarrow t < a
      \Rightarrow t \in] - \infty, a [= B \Rightarrow [x, y] \subseteq B$
      
      \item if $x, y \in [a, \infty [$ then $a \leqslant x, y$ so if $t \in
      [x, y] \Rightarrow x \leqslant t \wedge t \leqslant y \Rightarrow a
      \leqslant t \Rightarrow t \in [a, \infty [= B \Rightarrow [x, y]
      \subseteq B$
      
      \item if $x, y \in] a, \infty [$ then $a < x, y$ so if $t \in [x, y]
      \Rightarrow x \leqslant t \wedge t \leqslant y \Rightarrow a < t
      \Rightarrow t \in] a, \infty [= B \Rightarrow [x, y] \subseteq B$
      
      \item this is trivial as $[x, y] = \{ t \in A|x \leqslant t \wedge t
      \leqslant y \}$
      
      \item then (2) is satisfied vacuously
      
      \item then if $x, y \in B$ then $x = y$ so there is not $x < y$ and
      again (2) is satisfied vacuously
    \end{enumerate}
    \item If $x, y \in B$ with $x < y$ then by (2) we have $[x, y] \subseteq B
    \Rightarrow \forall z \in B$ such that $x \leqslant z \wedge z \leqslant
    y$ we have $z \in [x, y] \Rightarrow z \in B$
    
    \item If $x, y \in B$ with $x < y$ and $x < z < y \Rightarrow x \leqslant
    z \leqslant y \Rightarrowlim_{(3)} z \in B$
    
    \item Consider the following cases for $B$
    \begin{enumerate}
      \item then $B$ is a generalized interval
      
      \item then $B$ is a generalized interval
      
      \item then there exists $b_1, b_2$ with $b_1 < b_2$ then we have the
      following possibilities for $B$:
      \begin{enumerate}
        \item Then if $x \in A$ we have the following possibilities related to
        $a_1, a_2$
        \begin{enumerate}
          \item then we have either
          \begin{enumerate}
            \item $\Rightarrow x \in B$
            
            \item $\Rightarrow x \in B$
            
            \item $\Rightarrowlim_{(4)} x \in B$
          \end{enumerate}
          \item then as there is no lower bound for $B$ $\exists b \in B$ such
          that $b < x$ (otherwise $x$ is a lower-bound of $B$) so $b < a_1$
          and $b < x < a_1 \Rightarrowlim_{(4)} x \in B$
          
          \item then as there is no lower bound for $B$ $\exists b \in B$ such
          that $x < b$ then as $a_2 < b$ and $a_2 < x < b$ we have by (4) that
          $x \in B$
        \end{enumerate}
        So in all cases we have $x \in B$ proving that $A \subseteq B$ and as
        $B \subseteq A$ we have $B = A$ a generalized interval.
        
        \item As $A$ is conditionally complete there exists then $s = \sup
        (B)$ and $i = \inf (B)$ and as $i \leqslant b_1$, $b_1 \leqslant s$ we
        have $i \leqslant s$. Consider now the following possibilities.
        \begin{enumerate}
          \item If $x \in [i, s]$ then either $x = i \in B$, $x = s \in B$ or
          $i < x < s \Rightarrow i < s$ and $i < x < s \Rightarrowlim_{(4)} x
          \in B$, proving that $[i, s] \subseteq B$. If $x \in B$ then $i
          \leqslant x \wedge x \leqslant s$ so that $B \subseteq [i, s]$. This
          proves that $B = [i, s]$ a generalized interval.
          
          \item If $x \in] i, s]$ then either $x = s \in B$ or $i < x < s
          \Rightarrow i < s \wedge i < x < s \Rightarrowlim_{(4)} x \in B$
          proving that $] i, s] \subseteq B$. If $x \in B$ then $i \leqslant x
          \wedge x \leqslant s$ and as $i \nin B$ we must have $i < x
          \leqslant s \Rightarrow x \in] i, s]$ so that $B \subseteq] i, s]$.
          This proves that $B =] i, s]$ a generalized interval.
          
          \item If $x \in [i, s [$ then either $x = i \in B$ or $i < x < s
          \Rightarrow i < s \wedge i < x < s \Rightarrowlim_{(4)} x \in B$,
          proving that $[i, s [\subseteq B$. If $x \in B$ then $i \leqslant x
          \wedge x \leqslant s$ as $s \nin B$ we have $x < s \Rightarrow i
          \leqslant x < s \Rightarrow x \in [i, s [\Rightarrow B \subseteq [i,
          s [$. This proves that $B = [i, s [$ a generalized interval.
          
          \item If $x \in] i, s [$ then $i < x < s \Rightarrow i < s \wedge i
          < x < s \Rightarrowlim_{(4)} x \in B$ proving that $] i, s
          [\subseteq B$. On the other hand if $x \in B$ then $i \leqslant x
          \leqslant s$ and as $i, s \nin B$ we must have $i < x < s
          \Rightarrow x \in] i, s [\Rightarrow B \subseteq] i, s [$. So $B =]
          i, s [$ a generalized interval.
        \end{enumerate}
        \item then $s = \sup (B)$ exists by conditional completion. We have
        then the following cases
        \begin{enumerate}
          \item If $x \in] - \infty, s]$ then $x \leqslant s$ and either $x =
          s \in B$ or $x < s$ and as there is no lower bound there exists a $b
          \in B$ such that $b \leqslant x < s$ so that either $x = b \in B$ or
          $b < x < s \Rightarrowlim_{(4)} x \in B$. We conclude thus that $] -
          \infty, s] \subseteq B$. If $x \in B$ then $x \leqslant s
          \Rightarrow x \in] - \infty, s] \Rightarrow B \subseteq] - \infty,
          s]$. So we conclude that $B =] - \infty, s]$ a generalized interval.
          
          \item If $x \in] - \infty, s [$ then $x < s$ and as there is no
          lower bound there exists a $b \in B$ with $b \leqslant x$. Then
          either $x = b \in B$ or $b < x < s \Rightarrowlim_{(4)} x \in B$
          proving that $] - \infty, s [\subseteq B$. If $x \in B \Rightarrow x
          \leqslant s$ and as $s \nin B$ we have $x < s \Rightarrow x \in] -
          \infty, s [\Rightarrow B \subseteq] - \infty, s [\nosymbol$. So we
          conclude that $B =] - \infty, s [$ a generalized interval.
        \end{enumerate}
        \item then $i = \inf (B)$ exists by conditional completion. We have
        then the following cases
        \begin{enumerate}
          \item If $x \in [i, \infty [$ then $i \leqslant x$ and either $x = i
          \in B$ or $i < x$ and as there is no upper bound there exists a $b
          \in B$ with $i < x \leqslant b$ so that either $x = b \in B$ or $i <
          x < b \Rightarrowlim_{(4)} x \in B$. So we conclude that $[i, \infty
          [\subseteq B$. If $x \in B$ then $i \leqslant x \Rightarrow x \in
          [i, \infty [\Rightarrow B \subseteq [i, \infty [$. This finally
          means that $B = [i, \infty [$ a generalized interval.
          
          \item If $x \in] i, \infty [$ then $i < x$ and as there is no upper
          bound there exists a $b \in B$ with $i < x \leqslant b$ so that
          either $x = b \in B$ or $i < x < b \Rightarrowlim_{(4)} x \in B
          \nocomma$, thus we have $] i, \infty [\subseteq B$. If $x \in B$
          then $i \leqslant x$ and as $i \nin B$ we must have $i < x
          \Rightarrow x \in] i, \infty [\Rightarrow B \subseteq] i, \infty [$.
          So we have $B =] i, \infty [$ a generalized interval
        \end{enumerate}
      \end{enumerate}
    \end{enumerate}
  \end{enumerate}
\end{proof}

\subsection{Well-ordered classes}

\begin{definition}
  \label{well ordered class}{\index{well-ordered class}}A partially ordered
  class $\langle A, \leqslant \rangle$ is \tmtextbf{well-ordered} if every
  nonempty subclass of $A$ has a least element.
\end{definition}

\begin{theorem}
  \label{subset of a well-ordered (fully-ordered) set is well-ordered
  (fully-ordered)}If $\langle A, \leqslant \rangle$ is a partially ordered
  class and $B \subseteq A$ then we have the following for $\langle B,
  \leqslant \rangle$ (see \ref{induced order relation})
  \begin{enumerate}
    \item If $\langle A, \leqslant \rangle$ is fully ordered then $\langle B,
    \leqslant \rangle$ is full ordered
    
    \item If $\langle A, \leqslant \rangle$ is well-ordered then $\langle B,
    \leqslant \rangle$ is well-ordered
  \end{enumerate}
\end{theorem}

\begin{proof}
  
  \begin{enumerate}
    \item If $x, y \in B \Rightarrow x, y \in A \Rightarrow x \leqslant y \vee
    y \leqslant x \Rightarrow x \leqslant_{| B \nobracket} y \vee y
    \leqslant_{| B \nobracket} x$
    
    \item If $C \subseteq B$ is a nonempty set then $C \subseteq A$ then there
    exists a least element $c$ of $C$. So $c \in C$ and $\forall x \in C
    \vdash c \leqslant x \Rightarrow c \in C$ and $\forall x \in C \vdash c
    \leqslant_{| B \nobracket} x$
  \end{enumerate}
\end{proof}

\begin{theorem}
  \label{well ordering implies fully ordering}A well-ordered class $\langle A,
  \leqslant \rangle$ is fully ordered
\end{theorem}

\begin{proof}
  If $x, y \in A \Rightarrow \{ x, y \}$ is a nonempty subclass and thus it
  has a least element. We have then the following cases
  \begin{enumerate}
    \item $x$ is the least element then $x \leqslant y$
    
    \item $y$ is the least element then $y \leqslant x$
  \end{enumerate}
\end{proof}

\begin{theorem}
  \label{well ordering implies conditional completeness}A well-ordered class
  $\langle A, \leqslant \rangle$ is conditional complete.
\end{theorem}

\begin{proof}
  If $B$ is a non empty subclass that is bounded above then $\upsilon (B) \neq
  \emptyset$ and thus $\upsilon (B)$ has a least element which is by
  definition $\sup (B)$
\end{proof}

\begin{definition}
  \label{immediate successor}{\index{immediate successor}}Let $\langle A,
  \leqslant \rangle$ be a partially ordered class and $x, y \in B$ then $y$ is
  a \tmtextbf{immediate successor} of $x$ if $x < y$ and there is no element
  $a \in A$ with $x < a < y$
\end{definition}

\begin{theorem}
  \label{in a well ordered set every element has a immediate succesor}Let
  $\langle A, \leqslant \rangle$ be a well-ordered class then every element in
  $A$ (with the exception of the greatest element) has a immediate successor 
\end{theorem}

\begin{proof}
  If $x \in A$ and $x$ is not the greatest element then $B = \{ y \in A | y >
  x \nobracket \}$ is nonempty and thus by well-ordering there exists a least
  element $b \in B$. Then $x < b$ and if $x < a < b$ we would have $a \in B$
  and $a < b$ now $b$ is the least element and thus $b \leqslant a \Rightarrow
  a < a \Rightarrow a \neq a$ a contradiction,
\end{proof}

\begin{definition}
  \label{section of a well ordered class}{\index{section}}Let $\langle A,
  \leqslant \rangle$ be a partially ordered class and $B \subseteq A$ a
  subclass then $B$ is a \tmtextbf{section} of $A$ iff
  \begin{eqnarray*}
    & \forall x \in A \vdash \tmop{if} y \in B \wedge x \leqslant y
    \Rightarrow x \in B & 
  \end{eqnarray*}
\end{definition}

\begin{theorem}
  \label{section is a initial segement in a well-ordered class}Let $\langle A,
  \leqslant \rangle$ be a well-ordered class then $B \subseteq A$ is a section
  of $A$ iff $A = B$ or $B$ is a initial segment of $A$
\end{theorem}

\begin{proof}
  \
  
  $\Rightarrow$
  
  If $B$ is a section of $A$ then if $B = A$ we are done, so consider $B \neq
  A$ then $A \backslash B \neq \emptyset$. Because $A$ is well-ordered $A
  \backslash B$ has a least element $l$ we prove then that $B = S_l$
  \begin{eqnarray*}
    x \in S_l & \Rightarrow & x \in A \wedge x < l\\
    & \Rightarrow & x \in B [\tmop{if} x \nin B \Rightarrow x \in A
    \backslash B \Rightarrow l \tmop{is} \tmop{not} \tmop{the} \tmop{least}
    \tmop{element}]\\
    x \in B & \Rightarrow & \tmop{if} l \leqslant x \Rightarrowlim_{B
    \tmop{is} a \tmop{section}} l \in B \tmop{contradicting} l \in A
    \backslash B \Rightarrow x < l \Rightarrow x \in S_l
  \end{eqnarray*}
  $\Leftarrow$
  
  If $A = B$ or $B$ is a initial segment of $A$ then either
  \begin{enumerate}
    \item $A = B$ then if $\forall x \in A \vdash y \in B \wedge x \leqslant
    y$ we have $x \in A$
    
    \item $\exists b \in A \vdash B = S_b = \{ x \in A | x < b \nobracket \}$
    so if $x \in A$ and there exists a $y \in B$ with $x \leqslant y
    \Rightarrowlim_{y < b} x < b \Rightarrow x \in B$
  \end{enumerate}
\end{proof}

\begin{theorem}[Transfinite Induction]
  \label{transfinite induction}{\index{transfinite induction}}Let $\langle A,
  \leqslant \rangle$ be a well-ordered class and let $P (x)$ be a statement
  which is either false or true for a element $x \in A$ which satisfies
  \begin{eqnarray*}
    & \tmop{If} P (y) \tmop{is} \tmop{true} \tmop{for} \tmop{every} y < x
    \tmop{then} P (x) \tmop{is} \tmop{true} & 
  \end{eqnarray*}
  then $P (x)$ is true for every $x \in A$
\end{theorem}

\begin{proof}
  Suppose $P (x)$ is not true for every $x \in A$ then $B = \{ y \in A | P (x)
  \tmop{is} \tmop{false} \nobracket \}$ is not empty. As $A$ is well-ordered
  there exists a least element $b$ of $B$. Now if $x < b$ then if $x \in B
  \Rightarrow x < b \leqslant x \Rightarrow x < x$ a contradiction. So if $x <
  b$ then $x \nin B$ and thus $P (x)$ is true. By the hypothesis we have then
  that $P (b)$ is true which means $b \nin B$ contradicting the fact that $b$
  is the least element of $B$. So we must have that $P (x)$ is true for every
  $x \in A$.
\end{proof}

\begin{theorem}
  If $\langle A, \leqslant \rangle$ is a well-ordered class, $B \subseteq A$
  and $f : A \rightarrow B$ a isomorphism then $\forall x \in A \vdash x
  \leqslant f (x)$
\end{theorem}

\begin{proof}
  We prove this by contradiction so assume that there exists a $x \in A$ such
  that $f (x) < x$. Then we have that $C = \{ x \in A | f (x) < x \nobracket
  \} \neq \emptyset$ and by well-ordering it has a least element $c$. As $c
  \in C$ we have $f (c) < c$ and so using the fact that $f$ is a isomorphism
  we have by \ref{isomorphism is strictly increasing} that $f (f (c)) < f (c)$
  so $f (c) \in C$ but as $c$ is the least element we have $f (c) < c
  \leqslant f (c) \Rightarrow f (c) < f (c)$ a contradiction. So we must
  conclude that $C = \emptyset$ proving our theorem.
\end{proof}

\begin{theorem}
  \label{a class is not isomorph to a subclass of a initial segment of
  itself}Let $\langle A, \leqslant \rangle$ be a well-ordered class then there
  does not exists a isomorphism from $A$ to a subclass of an initial segment
  of $A$.
\end{theorem}

\begin{proof}
  
  
  We prove this again by contradiction. So assume that there exists a $a \in
  A$, a $B \subseteq S_a$ and a isomorphism $f : A \rightarrow B$. Then by the
  previous theorem we have $a \leqslant f (a) \Rightarrow \neg (f (a) < a)$
  and thus $f (a) \nin S_a \Rightarrow f (a) \nin B$ but this contradicts the
  fact that $\tmop{range} (f) = B$. Hence the theorem must be true. 
\end{proof}

As a initial segment is a subclass of itself so we have the following
corollary

\begin{corollary}
  \label{in a well-ordered class there is no isomorphism to a initial
  segment}Let $\langle A, \leqslant \rangle$ be a well-ordered class then
  there does not exists a isomorphism from $A$ to a initial segment of $A$.
\end{corollary}

\begin{theorem}
  \label{if A is isomorph with a segment of B then B is not isomorph with a
  sublcass of A}If $\langle A, \leqslant_A \rangle$, $\langle B, \leqslant_B
  \rangle$ be well-ordered classes then if $A$ is isomorphic with an initial
  segment of $B$ we have that $B$ is not isomorphic with any subclass of $A$ 
\end{theorem}

\begin{proof}
  Let $b \in B$ and let $f : A \rightarrow S_b$ be a isomorphism from $A$ to a
  initial segment of $B$. We proceed now by contradiction, so assume that
  there exists a $C \subseteq A$ and a isomorphism $g : B \rightarrow C$ then
  $g : B \rightarrow A$ is a injective increasing function hence $f \circ g :
  B \rightarrow S_b$ is a injective increasing function (see \ref{properties
  of composition of functions} and \ref{composition of increasing functions})
  and thus $f \circ g : B \rightarrow (f \circ g) (B) \subseteq S_b$ is a
  injective increasing function that by \ref{condition for isomorphism in a
  full ordered set}, \ref{well ordering implies fully ordering} is a
  isomorphism. So we have a isomorphism of $B$ to a subclass of one of its
  initial segments which by \ref{a class is not isomorph to a subclass of a
  initial segment of itself} is impossible. Proving the theorem by
  contradiction.
\end{proof}

\begin{lemma}
  Let $\langle A, \leqslant \rangle$ be a well-ordered and $a, b \in A$ with
  $a < b$ then $S_a$ is a initial segment of $S_b$ (using the induced order
  relation on $S_b$).
\end{lemma}

\begin{proof}
  First
  \begin{eqnarray*}
    x \in S_a & \Rightarrow & x < a\\
    & \Rightarrowlim_{a < b} & x < b\\
    & \Rightarrow & x \in S_b
  \end{eqnarray*}
  so we have $S_a \subseteq S_b$. Now if $x \in S_b$ and there is a $y \in
  S_a$ such that $x \leqslant y \Rightarrow x \leqslant y < a \Rightarrow x <
  a \Rightarrow x \in S_a$. So $S_a$ is a section of $S_b$ and as $a \nin S_a
  \wedge a \in S_b$ we have $S_a \neq S_b$ so that using \ref{section is a
  initial segement in a well-ordered class} $S_a$ is a initial segment of
  $S_b$.
\end{proof}

\begin{theorem}
  \label{relation of well ordered classes}Let $\langle A, \leqslant_A
  \rangle$, $\langle B, \leqslant_B \rangle$ be well-ordered classes then
  exactly one of the following cases holds
  \begin{enumerate}
    \item $A$ is isomorphic with $B$
    
    \item $A$ is isomorphic with an initial segment of $B$
    
    \item $B$ is isomorphic with an initial segment of $A$
  \end{enumerate}
\end{theorem}

\begin{proof}[injective][surjective][increasing][injective][surjective][increasing][injective][surjective][increasing]
  Let's define $C = \{ x \in A | \exists y \in B \vdash S_x \cong S_y
  \nobracket \}$ we have then the following
  \begin{eqnarray*}
    x \in C & \Rightarrow & \exists !y \vdash S_x \approx S_y  (\tmop{there}
    \tmop{is} \tmop{only} \tmop{on} y \tmop{such} \tmop{that} S_x \cong S_y)
  \end{eqnarray*}
  Suppose that given $x \in C$ there exists $y, y'$ with $y \neq y'$ such that
  $S_x \cong S_y$ and $S_x \cong S_{y'}$ giving $S_y \cong S_{y'}$ and $S_{y'}
  \cong S_y$(see \ref{properties of the isomorph relation}) then we have as
  $B$ is well-ordered and thus fully ordered the following cases
  \begin{enumerate}
    \item $y \leqslant_B y' \Rightarrow y <_B y'$ so by the previous lemma we
    have that $S_y$ is a initial segment of $S_{y'}$ and then from $S_{y'}
    \cong S_y$ we have that $S_{y'}$ is isomorphic to a initial segment of
    itself which is forbidden by \ref{in a well-ordered class there is no
    isomorphism to a initial segment}.
    
    \item $y' \leqslant_B y \Rightarrow y' <_B y$ so by the previous lemma we
    have that $S_{y'}$ is a initial segment of $S_y$ and then from $S_y \cong
    S_{y'}$ we have that $S_y$ is isomorphic to a initial segment of itself
    which is forbidden by \ref{in a well-ordered class there is no isomorphism
    to a initial segment}.
  \end{enumerate}
  So we must conclude that $y = y'$ proving our assessment.
  
  Using the above we can define a function $F : C \rightarrow B$ where $F = \{
  (x, y) | S_x \approx S_y \nobracket \}$ so if $x \in C$ then $S_x \cong S_{F
  (x)}$. We prove now that $F : C \rightarrow D$ where $D = \tmop{range} (F)
  \subseteq B$ is a isomorphism.
  \begin{enumerate}
    \item If $F (x) = F (x') = d$ then $S_x \cong S_d$ and $S_{x'} \cong S_d
    \Rightarrow S_x \cong S_{x'}$ now by fully ordering we have if $x \neq x'$
    \begin{enumerate}
      \item $x \leqslant_A x' \Rightarrow x <_A x'$ so $S_x$ is a initial
      segment of $S_{x'}$ which by \ref{in a well-ordered class there is no
      isomorphism to a initial segment} conflicts with $S_x \cong S_{x'}$
      
      \item $x' \leqslant_A x \Rightarrow x' <_A x$ so $S_{x'}$ is a initial
      segment of $S_x$ which by \ref{in a well-ordered class there is no
      isomorphism to a initial segment} conflicts with $S_{x'} \cong S_x$
    \end{enumerate}
    proving that $x \neq x'$ leads to a contradiction, so we must have $x =
    x'$.
    
    \item This is trivial as $D = \tmop{range} (F)$
    
    \item Suppose $x \leqslant_A y$ then $S_x \cong S_{F (x)}$ and $S_y \cong
    S_{F (y)}$. Proceed now by contradiction so assume $\neg (F (x)
    \leqslant_B F (y))$ or $F (y) <_B F (x)$ then $S_{F (y)}$ is a initial
    segment of $S_{F (x)}$ and \ from $x \leqslant_A y$ we have $S_x \subseteq
    S_y$ which proves that
    \begin{enumerate}
      \item $S_y$ is isomorphic \ with $S_{F (y)}$ a initial segment of $S_{F
      (x)}$
      
      \item $S_{F (x)}$ is isomorphic with $S_x$ a subclass of $S_y$ 
    \end{enumerate}
    which by \ref{if A is isomorph with a segment of B then B is not isomorph
    with a sublcass of A} is forbidden. So we are left with $F (x) \leqslant_B
    F (y)$ and thus $F$ is increasing.
  \end{enumerate}
  From 1,2 and 3 and \ref{condition for isomorphism in a full ordered set} we
  have that $F : C \rightarrow D$ is a isomorphism.
  
  Next we show that \tmtextbf{$C$ is a section of $A$}. So if $x \in A$ and
  there is a $c \in C$ such that $x \leqslant_A c$ then $S_c \cong S_{F (c)}$
  and thus there exists a isomorphism $g : S_c \rightarrow S_{F (c)}$ we prove
  now that
  \begin{eqnarray*}
    g_{| S_x \nobracket} : S_x \rightarrow S_{g (x)} &  & \tmop{is} a
    \tmop{isomorphism}
  \end{eqnarray*}
  First $x \leqslant_A c \Rightarrow S_x \subseteq S_c$ and if $y \in S_x
  \Rightarrow y <_A x \Rightarrowlim_{g \tmop{is} a \tmop{isomorphism}} g (y)
  <_B g (x) \Rightarrow g (y) \in S_{g (x)}$ and thus $g_{| S_x \nobracket} :
  S_x \rightarrow S_{g (x)}$ is a function. We prove now that it is a
  isomorphism
  \begin{enumerate}
    \item If $g_{| S_x \nobracket} (y) = g_{| S_x \nobracket} (y') \Rightarrow
    g (y) = g (y') \Rightarrowlim_{g \tmop{is} a \tmop{isomorphism}} y =
    y'_{}$
    
    \item If $y \in S_{g (x)} \Rightarrow y <_B g (x) \in S_{F (c)}
    \Rightarrow g (x) <_B F (c) \Rightarrow y <_B F (c) \Rightarrowlim_{g
    \tmop{is} \tmop{surjective}} \exists r \in S_c$ such that $g (r) = y$. Now
    if $x \leqslant_A r$ then $g (x) \leqslant_B g (r) = y \text{
    contradicting $y <_B g (x)$ so we must have } r <_A x \Rightarrow r \in
    S_x$ and $y = g (r) = g_{| S_x \nobracket} (r) \Rightarrow \tmop{range}
    (g_{| S_x \nobracket}) = S_{g (x)}$
    
    \item If $s, r \in S_x$ and $s \leqslant_A r \Rightarrow g_{| S_x
    \nobracket} (s) = g (s) \leqslant_B g (r) = g_{| S_x \nobracket} (r)$
  \end{enumerate}
  So by using \ref{condition for isomorphism in a full ordered set} we have
  that $g_{| S_x \nobracket}$ is a isomorphism. Thus $S_x \cong S_{g (x)}$ and
  thus $x \in C$ proving that \tmtextbf{C is a section of A.} \
  
  Next we prove that \tmtextbf{D is a section of B}, so if $y \in B$ is such
  that there is a $d \in D$ with $y \leqslant_B d$. Then as $d \in D =
  \tmop{range} (F)$ there exists a $c \in C \vdash F (c) = d$ so that $S_c
  \cong S_d \Rightarrow S_d \approx S_c$. So there exists a isomorphism $g :
  S_d \rightarrow S_c$. Now from $y \leqslant_B d$ we have that $S_y \subseteq
  S_d$ and if $x \in S_y \Rightarrow x <_B y \Rightarrowlim_{g \tmop{is} a
  \tmop{isomorphism}} g (x) <_A g (y) \Rightarrow g (x) \in S_{g (y)}$ so
  $g_{| S_y \nobracket} : S_y \rightarrow S_{g (y)}$ is a function, we prove
  now that
  \begin{eqnarray*}
    g_{| S_y \nobracket} : S_y \rightarrow S_{g (y)} &  & \tmop{is} a
    \tmop{isomorphism}
  \end{eqnarray*}
  \begin{enumerate}
    \item If $g_{| S_y \nobracket} (x) = g_{| S_y \nobracket} (x') \Rightarrow
    g (x) = g (x') \Rightarrowlim_{g \tmop{is} \tmop{injeective}} x = x'$
    
    \item If $z \in S_{g (y)} \Rightarrow z <_A g (y) \in S_c \Rightarrow g
    (y) <_A c \Rightarrow z <_A c \Rightarrow z \in S_c$ so there exists a $x
    \in S_d$ such that $g (x) = z$. Now if $y \leqslant_B x \Rightarrow g (y)
    \leqslant_A g (x) = z \Rightarrow g (y) \leqslant_A z \Rightarrow z \nin
    S_{g (y)}$ contradicting $z \in S_{g (y)}$ so we must have $x <_B y
    \Rightarrow x \in S_y$ and $g_{| S_y \nobracket} (x) = g (x) = z
    \Rightarrow \tmop{range} (g_{| S_y \nobracket}) = S_{g (y)}$
    
    \item If $s, r \in S_y$ and $s \leqslant_B r \Rightarrow g_{| S_y
    \nobracket} (s) = g (s) \leqslant_A g (r) = g_{| S_y \nobracket} (r)$
  \end{enumerate}
  So using \ref{condition for isomorphism in a full ordered set} we have that
  $g_{| S_y \nobracket} : S_y \rightarrow S_{g (y)}$ is a isomorphism. Thus
  $S_y \cong S_{g (y)}$ and thus as $g (y) \in S_c \subseteq A$ we have $g (y)
  \in C$ and $F (g (y)) = y \Rightarrow y \in D$ proving that \tmtextbf{D is a
  section of B.}
  
  We have thus proved that $F : C \rightarrow D$ is a isomorphism where $C$
  is a section of $A$ and $D$ is a section of $B$. Using \ref{section is a
  initial segement in a well-ordered class} we have then the following cases
  for the isomorphism $F : C \rightarrow D$
  \begin{enumerate}
    \item $C = A \wedge D = B$ then $A$ is isomorphic with $B$
    
    \item $C = A \wedge D$ is a initial segment of $B$ then $A$ is isomorphic
    with a initial segment of $B$
    
    \item $C$ is a initial segment of $A \wedge D = B$ then $B$ is isomorphic
    with a initial segment of $B$
    
    \item $C$ is a initial segment of $A$ and $D$ is a initial segment of $B$.
    In this case there exists a $a \in A$ and a $b \in B$ so that $C = S_a$
    and $D = S_b$ and $S_a \cong S_b$ but then $a \in C \Rightarrow a \in S_a
    \Rightarrow a < a$ a contradiction. So this case does not occur.
  \end{enumerate}
  We conclude that or theorem is proved if we prove that 1,2 and 3 can not
  occur at the same time but this follows from \ref{if A is isomorph with a
  segment of B then B is not isomorph with a sublcass of A}.
\end{proof}

\begin{corollary}
  \label{every subclass of a well ordered class is isomorphic to the class or
  a segment}Let $\langle A, \leqslant_A \rangle$ be a well-ordered class then
  every subclass of $A$ is isomorphic with $A$ or an initial segment of $A$
\end{corollary}

\begin{proof}
  If $B \subseteq A$ then $\langle B, \leqslant \rangle$ is a well-ordered
  class and using the previous theorem we have the following exclusive cases
  \begin{enumerate}
    \item $A$ is isomorphic with B
    
    \item $B$ is isomorphic with a initial segment of $A$ then as $B \cong B$
    we use \ref{if A is isomorph with a segment of B then B is not isomorph
    with a sublcass of A} to reach a contradiction. So this can not occur.
    
    \item $B$ is isomorphic with a initial segment of $A$
  \end{enumerate}
\end{proof}

\subsection{The Axiom of Choice}

\

\begin{definition}
  Let $A$ be a set then a \tmtextbf{choice} function for $A$ is a function $c
  : \mathcal{P}' (A) \rightarrow A$ such that $\forall B \in \mathcal{P}' (A)
  \vdash c (B) \in B$ (see \ref{P'(X)}
\end{definition}

We can then state the axiom of choice as follows

\begin{axiom}[Axiom of Choice]
  \label{axiom of choice}{\index{axiom of choice}}Every set has a choice
  function.
\end{axiom}

Using the Axiom of Choice we can prove the opposite from \ref{surjection
implies function in other directory}

\begin{theorem}
  \label{surjective function implies injective function in opposite
  direction}If $f : A \rightarrow B$ is a surjective function then there
  exists a function $g : B \rightarrow A$ such that $f \circ g = 1_B$
\end{theorem}

\begin{proof}
  If $f : A \rightarrow B$ is surjective. Then $\forall y \in B$ we have that
  $f^{- 1} (\{ y \})$ is a non empty subset of $A \Rightarrow f^{- 1} (\{ y
  \}) \in \mathcal{P}' (A)$. By the axiom of choice there exists a choice
  function $c : \mathcal{P}' (A) \rightarrow A$. Define now $g : B \rightarrow
  A$ such that $g (y) = c (f^{- 1} (y))$. Then if $y \in B$ we have $g (y) = c
  (f^{- 1} (y)) \in f^{- 1} (\{ y \}) \Rightarrow f (g (y)) = y$ and thus $f
  \circ g = 1_B .$ we prove now that $g$ is injective. If $g (y) = g (y')$
  then we have $f (g (y)) = f (g (y')) \Rightarrow 1_B (y) = 1_B (y')
  \Rightarrow y = y'$.
\end{proof}

Using the above theorem and \ref{surjection implies function in other
directory} we have

\begin{theorem}
  A function $f : A \rightarrow B$ is surjective if and only if there exists a
  function $g : B \rightarrow A$ such that $f \circ g = 1_B$
\end{theorem}

We have the following equivalent definitions of the axiom of choice

\begin{theorem}
  \label{equivalences of axiom of choice}The following are equivalent
  \begin{enumerate}
    \item The Axiom of Choice
    
    \item Let $\mathcal{A}$ be a set of sets such that $A \in \mathcal{A}
    \Rightarrow A \neq \emptyset$ and if $A, B \in \mathcal{A}$ with $A \neq
    B$ then $A \bigcap B = \emptyset$ (in other words $\mathcal{A}$ is a set
    of nonempty mutually disjoint sets) then there exists a set $C$ (called
    the choice set for $\mathcal{A}$) such that $\forall A \in \mathcal{A}
    \vdash \exists !a \in A$ so that $a \in C$ and if $a \in C$ then $\exists
    A \in \mathcal{A}$ such that $a \in A$ (in other words a choice set
    consists of exactly one element of every element of $\mathcal{A}$)
    
    \item If $\{ A_i \}_{i \in I}$ is a family of nonempty set then $\exists c
    : I \rightarrow \bigcup_{i \in I} A_i$ such that $\forall i \in I \vdash c
    (i) \in A_i$ (in other words $\prod_{i \in I} A_i \neq \emptyset$)
  \end{enumerate}
\end{theorem}

\begin{proof}
  
  
  $1 \Rightarrow 2$
  
  So let $\mathcal{A}$ be a set of nonempty mutually disjoint sets and let
  $\mathbbm{A}= \bigcup_{X \in \mathcal{A}} X$ then we have $\mathcal{A}
  \subseteq \mathcal{P}' (\mathbbm{A})$, now by the axiom of choice there is a
  function $c : \mathcal{P}' (\mathcal{A}) \rightarrow \mathbbm{A}$. Take then
  $C = c (\mathcal{A})$ then if $A \in \mathcal{A} \subseteq \mathcal{P}'
  (\mathbbm{A}) \Rightarrow c (A) \in A$ and also $c (A) \in C$ so $C$ consist
  of a element of all sets in $\mathcal{A}$. If now $A \in \mathcal{A}$ and $X
  \in \mathcal{A} \vdash c (X) \in A \Rightarrowlim_{c (X) \in X} X \bigcap A
  \neq \emptyset \Rightarrow X = A$ (as we have mutually disjoint sets). So if
  there exists a $A \in \mathcal{A}$ with $a, a' \in A$ and \ $a, a' \in C = c
  (\mathcal{A})$ then $\exists X, X' \in \mathcal{A} \vdash c (X) = a \in A, c
  (X') = a' \in A \Rightarrow X = A = X' \Rightarrowlim_{X = X' = A \wedge (X,
  a) \in c, (X', a') \in c} (A, a), (A, a') \in c \Rightarrowlim_{c \tmop{is}
  a \tmop{function}} a = a'$ proving that $C$ is indeed the choice set for
  $\mathcal{A}$ \ 
  
  $2 \Rightarrow 1$
  
  Let $A$ be a set and let $B \subseteq A$ be a nonempty subset of $A$. Form
  then $P_B = \{ (B, x) | x \in B \nobracket \}$, Then we have that $P_B$ is
  nonempty as $B$ is nonempty. Also if $z \in P_B \bigcap P_{B'} \Rightarrow z
  = (B, x) \wedge x \in B \wedge z = (B', x') \wedge x' \in B' \Rightarrow B =
  B' \Rightarrow P_B = P_{B'}$. So if $P_B \neq P_{B'}$ then $P_B \bigcap
  P_{B'} = \emptyset$. From this it follows that $\mathcal{A}= \{ P_B | B
  \subseteq A \wedge B \neq \emptyset \nobracket \}$ is set of nonempty
  mutually disjoint sets (note that $P_B \subseteq \mathcal{P} (A) \times A$
  and thus $\mathcal{A} \subseteq \mathcal{P} (\mathcal{P} (A) \times A)$ is
  indeed a set). From the hypothesis we have then the existence of a choice
  set $C$. If $B \in \mathcal{P}' (A)$ then $P_B \in \mathcal{A} \Rightarrow
  \exists (B, b) \in P_B \vdash (B, b) \in C$. If $c \in C$ then $\exists B
  \in \mathcal{P}' (A) \vdash c \in P_B \Rightarrow \exists b \in B \subseteq
  A \vdash (B, b) = c \Rightarrow C \subseteq \mathcal{P}' (A) \times A$. Also
  if $(B, b) \in C \Rightarrow \exists P_{B'} \vdash (B, b) \in P_{B'}
  \Rightarrow B = B' \wedge b \in B' \Rightarrow b \in B$. So if $(B, b), (B,
  b') \in C \Rightarrow b, b' \in B \Rightarrow (B, b), (B, b') \in P_B
  \Rightarrowlim_{C \tmop{is} a \tmop{choice} \tmop{set}} (B, b) = (B, b')
  \Rightarrow b = b'$. So we have proved that $C : \mathcal{P}' (A)
  \rightarrow A$ is a function such that $\forall B \in \mathcal{P}' (A)
  \vdash C (B) \in B$. So $C$ is the choice function we would search for.
  
  $1 \Rightarrow 3$
  
  If $\{ A_i \}_{i \in I}$ is a family of nonempty sets, form then $A =
  \bigcup_{i \in I} A_i$ by the axiom of choice there exists a $c :
  \mathcal{P}' (A) \rightarrow A$ such that $\forall B \in \mathcal{P}' (A)$
  we have $c (B) \in B$. Define then the function $x : I \rightarrow
  \mathcal{A}$ with $x = \{ (i, y) | i \in I \wedge y = c (A_i) | \} \subseteq
  I \times A$ [then $\tmop{dom} (x) = I$ and $(i, y), (i, y') \in x
  \Rightarrow y = c (A_i) = y' \Rightarrow y = y'$ so $x$ is a function and
  $\forall i \in I \vdash x (i) = c (A_i) \in A_i$] and thus $x \in \prod_{i
  \in I} A_i$
  
  $3 \Rightarrow 1$
  
  If $A$ is a set define then the graph $\mathcal{B}= \{ (B, b) |B \in
  \mathcal{P}' (A) \wedge b \in B \}$. We have then $\tmop{dom} (\mathcal{B})
  \subseteq \mathcal{P}' (A)$ and if $B \in \mathcal{P}' (A)$ there exists a
  $b \in B$ and thus $(B, b) \in \mathcal{B}$ proving that $\tmop{dom}
  (\mathcal{B}) =\mathcal{P}' (A)$. This defines then the family $\{
  \mathcal{B}_B \}_{B \in \mathcal{P}' (A)}$. Now given $B \in \mathcal{P}'
  (A)$ we have
  \begin{eqnarray*}
    x \in B & \Leftrightarrow & B \in \mathcal{P}' (A) \wedge x \in B\\
    & \Leftrightarrow & (B, x) \in \mathcal{B}\\
    & \Leftrightarrow & x \in \{ y| (B, y) \in \mathcal{B} \}\\
    & \Leftrightarrow & x \in \mathcal{B}_B
  \end{eqnarray*}
  proving that $\mathcal{B}_B = B$. So \ $\bigcup_{B \in \mathcal{P}' (A)}
  \mathcal{B}_B \subseteq A$ and as $x \in A \Rightarrow \{ x \} \in
  \mathcal{P}' (A) \Rightarrow x \in \{ x \} =\mathcal{B}_{\{ x \}}
  \Rightarrow x \in \bigcup_{B \in \mathcal{P}' (A)} \mathcal{B}_B$ so $A =
  \bigcup_{B \in \mathcal{P}' (A)} \mathcal{B}_B$ and by the hypothesis there
  exists a $c : \mathcal{P}' (A) \rightarrow A$ with $\forall B \in
  \mathcal{P}' (A) \vdash c (B) \in B$ and $c$ is the sought for choice
  function.
\end{proof}

\begin{corollary}
  \label{construction of a function using Axiom of choice}Let $A$ and $B$ be
  sets and assume that $\forall x \in A$ there exists a $y \in B$ such that $P
  (x, y)$ is true then there exists a function $f : A \rightarrow B$ such that
  $\forall x \in A$ we have $P (f (x))$ is true 
\end{corollary}

\begin{proof}
  Define the graph $G = \{ (x, y) \in A \times B|P (x, y) \}$ then this
  defines (see \ref{family of classes}) a family $\{ F_a \}_{a \in A}$ with
  $F_a = \{ y \in |P (a, y) \} \neq \emptyset$ so using the equivalences of
  the Axiom of Choice (see \ref{equivalences of axiom of choice}) there exists
  a function $f : A \rightarrow \bigcup_{a \in A} F_a$ with $\forall a \in A$
  that $f (a) \in F_a \Rightarrow P (a, f (a))$ is true.
\end{proof}

\

As a direct application of the Axiom of Choice we prove the following

\begin{theorem}
  \label{application of axiom of choice}Let $\langle A, \leqslant \rangle$ be
  a partially ordered set such that
  \begin{enumerate}
    \item $A$ has a least element $p$
    
    \item Every chain of $A$ has a supremum
  \end{enumerate}
  then there is a element $x \in A$ which has no immediate successor
\end{theorem}

\begin{proof}
  We will prove this by contradiction. So assume that $\forall x \in A$ there
  exists a immediate successor, we will then derive a contradiction from this.
  
  First $\forall x \in A$ define $T_x = \{ y : y \tmop{is} a \tmop{immediate}
  \tmop{successor} \tmop{of} x \}$ then $T_x \neq \emptyset \in \mathcal{P}'
  (A)$. By the axiom of choice there exists a choice function from $c :
  \mathcal{P}' (A) \rightarrow A$ such that $\forall A \in \mathcal{P}' (A)
  \vDash c (A) \in A$. This lets us define a function $f : A \rightarrow A$
  where $f = \{ (x, y) : x \in A \wedge y = c (T_x) \} \subseteq A \times A$
  [if $x \in A \Rightarrow T_x \neq \emptyset \Rightarrow (x, c (T_x)) \in f
  \Rightarrow \tmop{dom} (f) = A$ and if $(x, y), (x, y') \in f \Rightarrow y
  = c (T_x) = y' \Rightarrow y = y'$ thus $f : A \rightarrow A$ is indeed a
  function] and if $x \in A \Rightarrow f (x) \in T_x$ so $f (x)$ is a
  immediate successor of $x$. So we have defined a function
  \begin{eqnarray*}
    f : A \rightarrow A &  & \forall x \in A \vDash f (x) \tmop{is} a
    \tmop{immediate} \tmop{successor} \tmop{of} x
  \end{eqnarray*}
  Given the least element $p$ of $A$ we define the concept of a
  \tmtextbf{p-sequence} as follows
  
  \begin{definition}
    A subset $B \subseteq A$ is called a \tmtextbf{p-sequence} iff
    \begin{enumerate}
      \item $p \in B$
      
      \item If $x \in B \Rightarrow f (x) \in B$
      
      \item If $C$ is a chain of $B$ (and thus of $A$) then $\sup (C) \in B$
    \end{enumerate}
    
  \end{definition}
  
  Note that $A$ is trivially a p-sequence. We have now the following lemma
  
  \begin{lemma}
    Every intersection of p-sequences is a p-sequence
  \end{lemma}
  
  \begin{proof}
    If $\mathcal{A}$ is a collection of $p \um \tmop{sequences}$ then
    \begin{enumerate}
      \item $\forall A \in \mathcal{A} \vDash p \in A \Rightarrow p \in
      \bigcap_{A \in \mathcal{A}} A$
      
      \item If $x \in \bigcap_{A \in \mathcal{A}} A \Rightarrow \forall A \in
      \mathcal{A} \vDash x \in A \Rightarrow \forall A \in \mathcal{A} \vDash
      f (x) \in A \Rightarrow f (x) \in \bigcap_{A \in \mathcal{A}} A$
      
      \item If $C$ is a chain in $\bigcap_{A \in \mathcal{A}} A$ then every
      element in $C$ is comparable and $C \subseteq \bigcap_{A \in A} A
      \Rightarrow \forall A \in \mathcal{A} \vDash C \subseteq A \Rightarrow
      \forall A \in \mathcal{A} \vDash C \tmop{is} a \tmop{chain} \tmop{in} A
      \Rightarrow \forall A \in \mathcal{A} \vDash \sup (C) \in A \Rightarrow
      \sup (C) \in \bigcap_{A \in \mathcal{A}} A$
    \end{enumerate}
    So we have that $\bigcap_{A \in \mathcal{A}} A$ is a p-sequence
  \end{proof}
  
  Take now $\mathcal{A}= \{ B \in \mathcal{P} (A) : B \tmop{is} a p \um
  \tmop{sequence} \}$ which is nonempty as it contains $A$ and define the
  intersection of all p-sequences
  
  \begin{theorem}
    Let $\{ A_i \}_{i \in I}$ be a family of sets and $b : J \rightarrow I$ a
    bijection then
    \begin{eqnarray*}
      \bigcap_{i \in I} A_i & = & \bigcap_{j \in J} A_{b (j)}\\
      \bigcup_{i \in I} A_i & = & \bigcup_{j \in J} A_{b (j)}
    \end{eqnarray*}
  \end{theorem}
  
  \begin{proof}
    First we prove that
    
    \ 
  \end{proof}
  \begin{eqnarray*}
    P & = & \bigcap_{B \in \mathcal{A}} B
  \end{eqnarray*}
  then $P$ is a p-sequence and (as $p \in P$ we have $P \neq \emptyset$) and
  if $B$ is a p-sequence then $P \subseteq B$. Next we define the concept of
  \tmtextbf{select} elements in P.
  
  \begin{definition}
    A element $x \in P$ is called \tmtextbf{select} if it is comparable with
    every element y of P
  \end{definition}
  
  We have then the following lemma
  
  \begin{lemma}
    Suppose $x$ is \tmtextbf{select} then if $y \in P$ with $y < x$ we have $f
    (y) \leqslant x$
  \end{lemma}
  
  \begin{proof}
    If $y \in P$ then as $P$ is a p-sequence we have by (2) that $f (y) \in
    P$. Now $x$ is select so we have either $f (y) \leqslant x$ or $x < f
    (y)$. If $x < f (y)$ then from $y < x$ we have $y < x < f (y)$
    contradicting that $f (y)$ is the immediate successor of $y$, so we must
    have $f (y) \leqslant x$
  \end{proof}
  
  Next we prove the following lemma
  
  \begin{lemma}
    Suppose $x$ is select then $B_x = \{ y \in P | y \leqslant x \tmop{or} f
    (x) \leqslant y \nobracket \}$ is a p-sequence.
  \end{lemma}
  
  \begin{proof}
    
    \begin{enumerate}
      \item Since $p$ is the least element of $A$ we have $p \leqslant x
      \Rightarrow p \in B_x$
      
      \item Suppose $y \in B_x$ then $y \leqslant x$ or $y \geqslant f (x)$
      lets break this in the following three cases
      \begin{enumerate}
        \item $y < x$. Then by the previous lemma we have $f (y) \leqslant x
        \Rightarrow f (y) \in B_x$
        
        \item $y = x$. Then $f (y) = f (x)$ thus $f (x) \leqslant f (y)
        \Rightarrow f (y) \in B_x$
        
        \item $y \geqslant f (x) \nosymbol$ Then as $f (y)$ is the immediate
        successor of $y$ we have $y < f (y) \Rightarrow f (x) < f (y)
        \Rightarrow f (x) \leqslant f (y) \Rightarrow f (y) \in B_x$
      \end{enumerate}
      So in all cases we have $f (y) \in B_x$
      
      \item If $C$ is a chain in $B_x$ (and thus in $A$) let $s = \sup (C)$
      then we have the following excluding cases
      \begin{enumerate}
        \item $\exists y \in C \vdash f (x) \leqslant y \Rightarrowlim_{y
        \leqslant s} f (x) \leqslant s \Rightarrow s \in B_x$
        
        \item $\forall y \in C \vDash \neg (f (x) \leqslant y)$. Now $\forall
        y \in C$ as $y \in C \subseteq B_x$ we have $y \leqslant x$ or $f (x)
        \leqslant y$, as the last contradicts $\neg (f (x) \leqslant y)$, we
        have always $y \leqslant x$ and thus $x$ is a upper bound of $C$. As
        $s$ is the least upper bound we must have $s \leqslant x \Rightarrow s
        \in B_x$
      \end{enumerate}
      So in all valid cases $s \in B_x$
    \end{enumerate}
  \end{proof}
  
  We can easily derive the following corollary.
  
  \begin{corollary}
    If $x$ is select then $\forall y \in P$ we have $y \leqslant x$ or $f (x)
    \leqslant y$
  \end{corollary}
  
  \begin{proof}
    As $P \subseteq B_x$ ($P$ is the intersection of all the p-sequences) and
    by definition of $B_x$ we have $B_x \subseteq P \Rightarrow P = B_x$
  \end{proof}
  
  We can use this to prove the following lemma
  
  \begin{lemma}
    The set of all select elements $\{ x \in P | x \tmop{is} \tmop{select}
    \nobracket \}$ is a p-sequence
  \end{lemma}
  
  \begin{proof}
    
    \begin{enumerate}
      \item $p$ is select because $\forall x \in P \subseteq A$ we have $p
      \leqslant x$ so it is comparable with every element of $p$
      
      \item If $x$ is select then, by the previous corollary, $\forall y \in
      P$ we have either
      \begin{enumerate}
        \item $y \leqslant x$. Then as $x < f (x) \Rightarrow y < f (x)
        \Rightarrow y \leqslant f (x)$
        
        \item $f (x \leqslant y)$
      \end{enumerate}
      So $f (x)$ is comparable with every element of $P$ and $f (x)$ is thus
      select.
      
      \item Let $C$ be a chain of select elements. Then as C is a chain in $A$
      we have that $s = \sup (C) \tmop{exists} \nosymbol$. Then $\forall y \in
      P$ we have the following possibilities for $C$
      \begin{enumerate}
        \item $\exists x \in C \vdash y \leqslant x$ then as $x \leqslant s$
        we have $y \leqslant s$
        
        \item $\forall x \in C \vdash \neg (y \leqslant x)$ then as $y \in P$
        we have $\forall x \in C$ either $y \leqslant x$ (contradicted by
        $\neg (y \leqslant x)$) or $f (x) \leqslant y$ so $\forall x \in C$ we
        have $f (x) \leqslant y \Rightarrowlim_{x < f (x)} x \leqslant y$ and
        thus $y$ is a upper bound of $C$. Ans as $s$ is the least upper bound
        of S we have $s \leqslant y$
      \end{enumerate}
      So $s$ is comparable with every $y \in P$ and thus $s$ is select.
    \end{enumerate}
  \end{proof}
  
  \begin{corollary}
    $P$ is fully-ordered
  \end{corollary}
  
  \begin{proof}
    $S = \{ x \in P | x \tmop{is} \tmop{select} \nobracket \} \subseteq P$ and
    by the previous lemma we have that $S$ is a p-sequence thus $P \subseteq
    S$ so $P = S = \{ x \in P | x \tmop{is} \tmop{select} \nobracket \}$. Or
    every element in $P$ is select and thus comparable with every other
    element so $P$ is fully-ordered.
  \end{proof}
  
  Now for the finally proof, we show that this corollary causes a
  contradiction. As $P$ is by the corollary a chain we have that $s = \sup
  (P)$ exists by the hypothesis. But as $P$ is a p-sequence we find by (2)
  that $f (s) \in P$ hence $f (s) \leqslant s$ but as $f (s)$ is the successor
  or $s$ we have $s < f (s) \Rightarrow s < s$ a contradiction. So we have
  derived a contradiction and the theorem must be true.
\end{proof}

We show now that the following theorem follows from the Axiom of Choice (via
\ref{application of axiom of choice})

\begin{theorem}[Hausdorff's Maximal Principle]
  \label{Hausdorff's maximal principle}{\index{Hausdorff's maximal
  principle}}Every partially ordered set $\langle A, \leqslant \rangle$ has a
  maximal chain $C$ (see \ref{maximal, minimal, greatest, least} if $D$ is a
  chain and if $C \subseteq D \Rightarrow C = D$)
\end{theorem}

\begin{proof}
  Define $\mathcal{C}= \{ B \in \mathcal{P} (A) | B \tmop{is} a \tmop{chain}
  \nobracket \}$ be the set [is $\subseteq \mathcal{P} (A)$ and applying
  \ref{axiom of power sets} and \ref{axiom of subsets} gives $\mathcal{C}$ is
  a set] of all chains in $A$ and order $\mathcal{C}$ by inclusion (see
  \ref{inclusion is partial order}) so $\langle \mathcal{C}, \subseteq
  \rangle$ is a partially ordered set. Notice that $\mathcal{C}$ \tmtextbf{has
  a least element} namely $\emptyset$. Let now $\mathcal{D} \subseteq
  \mathcal{C}$ be a chain in $\mathcal{C}$ and define
  \begin{eqnarray*}
    & K = \bigcup_{C \in \mathcal{D}} C & 
  \end{eqnarray*}
  then we prove that $K$ is a chain in $A$ and thus $K \in \mathcal{C}$.
  Indeed if $x, y \in K$ then $\exists C_1, C_2 \in \mathcal{D}$ such that $x
  \in C_1$ and $y \in C_2$. Now as $\mathcal{D}$ is a chain in $\mathcal{C}$
  we have either
  \begin{enumerate}
    \item $C_1 \subseteq C_2 \Rightarrow x, y \in C_2 \Rightarrowlim_{C_2
    \tmop{is} a \tmop{chain} \tmop{in} A} x$ and $y$ are comparable
    
    \item $C_2 \subseteq C_1 \Rightarrow x, y \in C_1 \Rightarrowlim_{C_1
    \tmop{is} a \tmop{chain} \tmop{in} A} x$ and $y$ are comparable
  \end{enumerate}
  \begin{theorem}
    Let $\{ A_i \}_{i \in I}$ be a family of sets and $b : J \rightarrow I$ a
    bijection then
    \begin{eqnarray*}
      \bigcap_{i \in I} A_i & = & \bigcap_{j \in J} A_{b (j)}\\
      \bigcup_{i \in I} A_i & = & \bigcup_{j \in J} A_{b (j)}
    \end{eqnarray*}
  \end{theorem}
  
  \begin{proof}
    First we prove that
    
    \ 
  \end{proof}
  
  proving that $K$ is a chain. Using \ref{inf, sup in class ordered by
  inclusion} we have that \tmtextbf{$K = \sup (\mathcal{D})$} exists. So we
  can apply \ref{application of axiom of choice} (which is derived from the
  Axiom of Choice) to find a element $C \in \mathcal{C}$ which has no
  immediate successor. $C$ is (by definition of $C$) a chain. We prove now by
  contradiction that $C$ is maximal. If there exists a chain $D$ in $A$ with
  $C \subseteq D$ and $D \neq C$ then $\exists d \in D \vdash d \nin C$. Then
  $C \bigcup \{ d \}$ is a chain for if $x, y \in C \bigcup \{ d \} \subseteq
  D$ we have as $D$ is a chain that $x, y$ are comparable. Now if there is a
  set $X$ with $C \subset X \subset C \bigcup \{ d \}$ then $X = X \bigcup (X
  \backslash C) \subseteq C \bigcup \{ d \} \Rightarrow (X \backslash C)
  \subseteq C \bigcup \{ d \}$ and if $x \in (X \backslash C) \Rightarrow x
  \nin C \wedge x \in C \bigcup \{ d \} \Rightarrow x = d$ so $(X \backslash
  C) = \{ d \}$ so $X = C \bigcup \{ d \}$ contradicting $X \subset C \bigcup
  \{ d \}$. So for every chain $D$ in $A$ with $C \subseteq D$ we must have $C
  = D$ proving maximality of $C$.
\end{proof}

\begin{definition}
  A partially ordered set $\langle A, \leqslant \rangle$ is said to be
  \tmtextbf{inductive} when every chain in A has an upper bound.
\end{definition}

\

\begin{lemma}[Zorn's Lemma]
  \label{Zorn's lemma}{\index{Zorn's Lemma}}Every inductive set has at least
  one maximal element. In other words, if $\langle A, \leqslant \rangle$ is a
  partially ordered set such that every chain in $A$ has a upper bound, then
  $A$ has a maximal element.
\end{lemma}

We prove now that from the Hausdorff's maximal theorem follows directly Zorn's
lemma, without using either \ref{equivalences of axiom of choice} or
\ref{application of axiom of choice} ,so it does not depends indirectly on the
Axiom of Choice, \ but directly on Hausdorff's maximal theorem.

\begin{theorem}
  Hausdorff's maximal theorem implies $\tmop{Zorn}' s$ lemma
\end{theorem}

\begin{proof}
  Let $\langle A, \leqslant \rangle$ be a partially ordered set such that
  every chain in $A$ has a upper bound. By the Hausdorff's Maximal Principle
  (see \ref{Hausdorff's maximal principle}) there exists a maximal chain $C$
  in $A$, by the hypothesis there exists a upper bound $u$ for $C$. We prove
  now by contradiction that $u$ is a maximal element of $A$. So assume that
  $u$ is not a maximal element of $A$ then there exists a $x \in A \vdash u <
  x$. If now $x \in C \Rightarrowlim_{u \tmop{is} a \tmop{upper} \tmop{bound}}
  x \leqslant u < x \Rightarrow x < x$ a contradiction so we must have that $x
  \nin C$. Now if $r, s \in C \bigcup \{ x \}$ then we have the following
  cases
  \begin{enumerate}
    \item $r = s = x \Rightarrowlim_{\tmop{reflexitivity}} r \leqslant s$
    
    \item $r = x, s \in C \Rightarrow s \leqslant u < x = r \Rightarrow s
    \leqslant r$
    
    \item $r \in C, s = x \Rightarrow r \leqslant u < x = s \Rightarrow r
    \leqslant s$
    
    \item $r, s \in C \Rightarrowlim_{C \tmop{is} a \tmop{chain}} s \leqslant
    r$ or $r \leqslant s$
  \end{enumerate}
  proving that $C \bigcup \{ x \}$ is a chain with (because $x \nin C$) $C
  \subset C \bigcup \{ x \}$ contradicting that $C$ is a maximal chain. So we
  must have that $u$ is a maximal element of $A$.
\end{proof}

We state now the Well-Ordering Theorem

\begin{theorem}[Well-Ordering Theorem]
  \label{well-ordering theorem}{\index{well-ordering theorem}}Given a set $A$
  then there exists a order relation $\leqslant$ on $A$ such that $\langle A,
  \leqslant \rangle$ forms a well-ordered set.
\end{theorem}

We prove then that from the Zorn's lemma follows directly the Well-Ordering
Theorem without using either \ref{equivalences of axiom of choice},
\ref{Hausdorff's maximal principle} or \ref{application of axiom of choice} so
it depends directly on Zorn's Lemma.

\begin{theorem}
  Zorn's Lemma implies the Well-Ordering Theorem
\end{theorem}

\begin{proof}[reflectivity][anti-symmetry][transitivity][reflectivity][anti-symmetry][transitivity][reflexive][anti-symmetry][transitive]
  Let $A$ be a arbitrary set and let $B \subseteq A$ a subset. A relation $G$
  on $B$ is a subclass of the set $B \times B$ (see \ref{product of sets is a
  set}) and we have by \ref{axiom of subsets} that $G$ is a set. So the pair
  $(B, G)$ is a element (see \ref{pair}) we define then the class
  $\mathcal{A}= \{ (B, G) | B \in \mathcal{P} (A) \wedge G \tmop{is} a
  \tmop{order} \tmop{relation} \tmop{on} B \nobracket \tmop{so} \tmop{that}
  \langle B, G \rangle \tmop{is} \tmop{well} \um \tmop{ordered} \}$. We define
  now $\prec \subseteq \mathcal{A} \times \mathcal{A}$ by
  \begin{equation}
    (B, G) \prec (B', G') \Leftrightarrow B \subseteq B' \tmop{and} G
    \subseteq G' \tmop{and} [x \in B \wedge y \in B' \backslash B \Rightarrow
    (x, y) \in G']
  \end{equation}
  
  
  \begin{lemma}
    $\langle \mathcal{A}, \prec \rangle$ forms a order relation
    
    \begin{proof}
      
      \begin{enumerate}
        \item If $(B, G) \in \mathcal{A}$ then
        \begin{enumerate}
          \item $B \subseteq B$
          
          \item $G \subseteq G$
          
          \item $x \in B \wedge y \in B \backslash B = \emptyset \Rightarrow
          (x, y) \in G$ is satisfied because $y \in \emptyset$ can not occur
          so $x \in B \wedge y \in B \backslash B$ is always false.
        \end{enumerate}
        proving $(B, G) \prec (B, G)$
        
        \item If $ (B, G) \prec (B', G') \wedge (B', G') \prec (B, G)
        \Rightarrow B \subseteq B' \subseteq B \wedge G \subseteq G' \subseteq
        G \Rightarrow B = B' \wedge G = G' \Rightarrow (B, G) = (B', G')$
        proving anti-symmetry.
        
        \item If $(B, G) \prec (B', G') \wedge (B', G') \prec (B'', G'')$ then
        we have
        \begin{enumerate}
          \item $B \subseteq B' \wedge B' \subseteq B'' \Rightarrow B
          \subseteq B''$
          
          \item $G \subseteq G' \wedge G' \subseteq G'' \Rightarrow G
          \subseteq G''$
          
          \item If $x \in B \wedge y \in B'' \backslash B$ we have then for
          $y$ the following cases
          \begin{enumerate}
            \item $y \in B' \Rightarrow y \in B' \backslash B \Rightarrow (x,
            y) \in G' \Rightarrowlim_{G' \subseteq G''} (x, y) \in G''$
            
            \item $y \nin B' \Rightarrow y \in B'' \backslash B' \Rightarrow
            (x, y) \in G''$
          \end{enumerate}
          So in all cases we have $(x, y) \in G''$
        \end{enumerate}
        proving $(B, G) \prec (B'', G'')$
      \end{enumerate}
    \end{proof}
  \end{lemma}
  
  We now have the following lemma.
  
  \begin{lemma}
    If $\mathcal{C} \subseteq \mathcal{A}$ is a chain in $\langle \mathcal{A},
    \prec \rangle$ then if
    \begin{eqnarray*}
      B_{\mathcal{C}} & = & \bigcup_{(B, G) \in \mathcal{C}} B\\
      G_{\mathcal{C}} & = & \bigcup_{(B, G) \in \mathcal{C}} G
    \end{eqnarray*}
    we have $(B_{\mathcal{C}}, G_{\mathcal{C}}) \in \mathcal{A}$
  \end{lemma}
  
  \begin{proof}
    
    \begin{enumerate}
      \item $\forall (B, G) \in \mathcal{C}$ we have $B \subseteq A$ so if $x
      \in B_{\mathcal{C}}$ then $\exists (B, G) \in \mathcal{C} \vdash x \in B
      \subseteq A \Rightarrow x \in A$ and thus $B_{\mathcal{C}} \subseteq A$
      
      \item We must prove now that $G_{\mathcal{C}}$ is a order relation on
      $B_{\mathcal{C}}$
      \begin{enumerate}
        \item If $x \in B_{\mathcal{C}} \Rightarrow \exists (B, G) \in
        \mathcal{C} \vdash x \in B \Rightarrow (x, x) \in G \Rightarrow (x, x)
        \in G_{\mathcal{C}}$
        
        \item If $(x, y) \in G_{\mathcal{C}} \wedge (y, x) \in
        G_{\mathcal{C}}$ then $\exists (B, G), (B', G') \in \mathcal{C} \vdash
        (x, y) \in G \wedge (y, x) \in G'$ now because $\mathcal{C}$ is a
        chain we have either
        \begin{enumerate}
          \item $(B, G) \prec (B', G')$ then $G \subseteq G' \Rightarrow (x,
          y), (y, x) \in G' \Rightarrow x = y$
          
          \item $(B', G') \prec (B, G)$ then $G' \subseteq G \Rightarrow (x,
          y), (y, x) \in G \Rightarrow x = y$
        \end{enumerate}
        \item If $(x, y), (y, z) \in G_{\mathcal{C}}$ then $\exists (B, G),
        (B', G') \in \mathcal{C} \vdash (x, y) \in G \wedge (y, z) \in G'$.
        Now because $\mathcal{C}$ is a chain we have either
        \begin{enumerate}
          \item $(B, G) \prec (B', G') \Rightarrow G \subseteq G' \Rightarrow
          (x, y), (y, z) \in G' \Rightarrow (x, z) \in G' \Rightarrow (x, z)
          \in G_{\mathcal{C}}$
          
          \item $(B', G') \prec (B, G) \Rightarrow G' \subseteq G \Rightarrow
          (x, y), (y, z) \in G \Rightarrow (x, z) \in G \Rightarrow (x, z) \in
          G_{\mathcal{C}}$
        \end{enumerate}
      \end{enumerate}
      \item Now we must prove well-ordering. Suppose that $D \subseteq
      B_{\mathcal{C}}$ and $D \neq \emptyset$ then there exists a $x \in D
      \Rightarrow x \in B_{\mathcal{C}} \Rightarrow \exists (B, G) \in
      \mathcal{C} \vdash x \in B \Rightarrow D \bigcap B \neq \emptyset$. Now
      $D \bigcap B \subseteq B$ hence by well-ordering of $\langle B, G
      \rangle$ we have that there exists a least element $b \in B$ in $B
      \Rightarrow \forall y \in B \vDash (b, y) \in G \Rightarrow \forall y
      \in B \bigcap D \vDash (b, y) \in G$. We prove now that $b$ is the least
      element of $D$.
      
      \begin{proof}
        If $x \in D \Rightarrow \exists (B', G')$ with $x \in B'$ then we have
        the following cases
        \begin{enumerate}
          \item $x \in B \Rightarrow x \in B \bigcap D \Rightarrow (b, x) \in
          G \Rightarrow (b, x) \in G_{\mathcal{C}}$
          
          \item $x \nin B \Rightarrow x \in B' \backslash B \wedge b \in B$
          now we have two cases
          \begin{enumerate}
            \item $(B, G) \prec (B', G') \Rightarrowlim_{b \in B \wedge x \in
            B' \backslash B} (b, x) \in B' \Rightarrow (b, x) \in
            G_{\mathcal{C}}$
            
            \item $(B', G') \prec (B, G) \Rightarrow B' \subseteq B
            \Rightarrow x \in B$ contradicting $x \nin B$ so this case does
            not apply.
          \end{enumerate}
        \end{enumerate}
        So in all cases that are valid we have $(b, x) \in G_{\mathcal{C}}$
        proving that $b$ is the least element of $D$ and thus proving that
        $\langle B_{\mathcal{C}}, G_{\mathcal{C}} \rangle$ is a well-ordered
        set.
      \end{proof}
    \end{enumerate}
  \end{proof}
  
  Next we have the finally lemma before the Well-Ordering theorem
  
  \begin{lemma}
    If $\mathcal{C}$ is a chain in $\langle \mathcal{A}, \prec \rangle$ then
    $(B_{\mathcal{C}}, G_{\mathcal{C}})$ is a upper bound of $\mathcal{C}$
  \end{lemma}
  
  \begin{proof}
    If $(B, G) \in \mathcal{C}$ then clearly $B \subseteq B_{\mathcal{C}}$ and
    $G \subseteq G_{\mathcal{C}}$ so to prove that $(B, G) \prec
    (B_{\mathcal{C}}, G_{\mathcal{C}})$ we must prove that if $x \in B$ and $y
    \in B_{\mathcal{C}} \backslash B$ that $(x, y) \in G_{\mathcal{C}}$. So
    assume that $x \in B$ and $y \in B_{\mathcal{C}} \backslash B$ then
    $\exists (B', G') \vdash y \in B'$. We have then as $y \nin B$ that $B'
    \nsubseteq B$ hence as $\mathcal{C}$ is a chain and we can't have $(B',
    G') \prec (B, G)$ we must have $(B, G) \prec (B', G')$ and thus as $B
    \subseteq B'$ we have $y \in B' \backslash B \Rightarrow (x, y) \in G'
    \Rightarrow (x, y) \in B_{\mathcal{C}}$ proving that $(B, G) \prec
    (B_{\mathcal{C}}, G_{\mathcal{C}})$.
  \end{proof}
  
  Finally we can prove our well-ordering theorem. Using Zorn's lemma
  (\ref{Zorn's lemma}) we have a maximal element $(B_m, G_m) \in \mathcal{A}$.
  We prove now that $B_m = A$ proving that $(A, G_m) \in \mathcal{A}$ and that
  $\langle A, G_m \rangle$ is well-ordered.
  
  \begin{proof}
    Assume that $B_m \neq A$ then as $B_m \subseteq A$ we have $\exists x \in
    A \backslash B_m$ define then $G^{\ast} = G_m \bigcup \{ (b, x) |b \in B_m
    \} \bigcup \{ (x, x) \}$ (a disjoint union), we prove then that
    $\left\langle B_m \bigcup \{ x \}, G^{\ast} \right\rangle$ is a
    well-ordered set
    
    \begin{proof}
      Note that if $(x, r) \in G^{\ast}$ then we have the following
      possibilities
      \begin{enumerate}
        \item $(x, r) \in G_m$ which is impossible because $G_m \subseteq B_m
        \times B_m$ and $x \nin B_m$
        
        \item $(x, r) \in \{ (b, x) | b \in B_m \nobracket \}$ which is
        impossible as $x \nin B_m$
        
        \item $(x, r) \in \{ (x, x) \} \Rightarrow r = x$
      \end{enumerate}
      So from $(x, r) \in G^{\ast}$ it follows that $(x, r) = (x, x) \in \{
      (x, x) \}$ we now prove that $G^{\ast}$ is a well-ordering.
      \begin{enumerate}
        \item If $y \in B_m \bigcup \{ x \}$ then we have the following cases
        \begin{enumerate}
          \item $y \in B_m \Rightarrow (y, y) \in G_m \Rightarrow (y, y) \in
          G^{\ast}$
          
          \item $y \in \{ x \} \Rightarrow (y, y) = (x, x) \in G^{\ast}$
        \end{enumerate}
        \item If $(r, s), (s, r) \in G^{\ast}$ we have either
        \begin{enumerate}
          \item $(r, s) \in G_m \Rightarrowlim_{G_m \subseteq B_m \times B_m}
          r, s \in B_m \Rightarrowlim_{r, s \neq x \wedge (s, r) \in G^{\ast}}
          (s, r) \in G_m \Rightarrow r = s$
          
          \item $(r, s) \in \{ (b, x) |b \in B_m \} \Rightarrow s = x
          \Rightarrowlim_{(x, r) \in G^{\ast}} r = x \Rightarrow r = s$
          
          \item $(r, s) \in \{ (x, x) \} \Rightarrow r = x = s$
        \end{enumerate}
        \item If $(r, s), (s, t) \in G^{\ast}$ then we have the following
        cases
        \begin{enumerate}
          \item $ (r, s) \in G_m$ we have then the following sub cases
          \begin{enumerate}
            \item $(s, t) \in G_m \Rightarrow (r, t) \in G_m \Rightarrow (r,
            t) \in G^{\ast}$
            
            \item $(s, t) = (b, x) \Rightarrow t = x \Rightarrow (r, t) = (r,
            x) \in G^{\ast} \Rightarrow (r, t) \in G^{\ast}$
            
            \item $(s, t) = (x, x) \Rightarrow t = x \Rightarrow (r, t) = (r,
            x) \in G^{\ast} \Rightarrow (r, t) \in G^{\ast}$
          \end{enumerate}
          \item $(r, s) \in \{ (b, x) | b \in \nobracket B_m \} \Rightarrow s
          = x \Rightarrow (x, t) \in G^{\ast} \Rightarrow t = x \Rightarrow
          (r, t) \in \{ (b, x) | b \in B_m \nobracket \} \Rightarrow (r, t)
          \in G^{\ast}$
          
          \item $(r, s) \in \{ (x, x) \} \Rightarrow r = s = x \Rightarrow (x,
          t) \in G^{\ast} \Rightarrow t = x \Rightarrow (r, t) = (x, x) \in
          G^{\ast}$
        \end{enumerate}
      \end{enumerate}
      so $G^{\ast}$ is indeed a order relation. Now if $C \subseteq B_m
      \bigcup \{ x \}$ is a nonempty set then we have the following
      possibilities
      \begin{enumerate}
        \item $C \bigcap B_m \neq \emptyset \Rightarrow \emptyset \neq C
        \bigcap B_m \subseteq B_m \Rightarrowlim_{G_m \tmop{is} \tmop{well}
        \um \tmop{ordered}} \exists l \in C \bigcap B_m \vdash \forall y \in C
        \bigcap B_m \vDash (l, y) \in G_m$. Now if $y \in C$ then we have the
        following possibilities
        \begin{enumerate}
          \item $y \in B_m \Rightarrow y \in C \bigcap B_m \Rightarrow (l, y)
          \in G_m \Rightarrow (l, y) \in G^{\ast}$
          
          \item $y \nin B_m \Rightarrow y = x \Rightarrow (l, y) \in \{ (b, x)
          | b \in B_m \nobracket \} \Rightarrow (l, y) \in G^{\ast}$
        \end{enumerate}
        So we conclude that $C$ has a least element.
        
        \item $C \bigcap B_m = \emptyset \Rightarrow C = \{ x \} \Rightarrow
        x$ is the least element of $C$.
      \end{enumerate}
      So we have that $\left( B_m \bigcup \{ x \}, G^{\ast} \right) \in
      \mathcal{A}$
      
      Next we have $(B_m, G_m) \prec \left( B_m \bigcup \{ x \}, G^{\ast}
      \right)$ as
      \begin{enumerate}
        \item $B_m \subseteq B_m \bigcup \{ x \}$
        
        \item $G_m \subseteq G^{\ast}$
        
        \item If $r \in B_m$ and $s \in \left( B_m \bigcup \{ x \} \right)
        \backslash B_m \Rightarrow s = x \Rightarrow (r, s) \in \{ (b, x) | b
        \in B_m \nobracket \} \Rightarrow (r, s) \in G^{\ast}$
      \end{enumerate}
      But as $x \nin B_m$ we have \ $B_m \neq B_m \bigcup \{ x \}$ and thus
      $(B_m, G_m) \neq \left( B_m \bigcup \{ x \}, G^{\ast} \right)$ so we
      have a contradiction as by maximality of $(B_m, G_m)$ in $\langle
      \mathcal{A}, \prec \rangle$ we \ would have $(B_m, G_m) = \left( B_m
      \bigcup \{ x \}, G^{\ast} \right)$. So we must conclude that $B_m = A$
    \end{proof}
    
    So finally $(A, G_m) \in \mathcal{A} \Rightarrow \langle A, G_m \rangle$
    is a well-ordered set.
  \end{proof}
  
  \ 
\end{proof}

Using the above we have proved that

\

Axiom Of Choice $\Rightarrow$ Hausdorff's Maximal Principle $\Rightarrow$
Zorn's Lemma $\Rightarrow$ Well-Ordering Theorem.

\

We prove now that the Axiom of Choice follows from the Well-Ordering Theorem.

\

\begin{theorem}
  The Well-Ordering Theorem implies the Axiom of Choice
\end{theorem}

\begin{proof}
  If $A$ is any set then choose a well-ordering $\langle A, \leqslant \rangle$
  of $A$. \ Define then $c : \mathcal{P}' (A) \rightarrow A$ by $c = \{ (B, b)
  | B \in \mathcal{P}' (A) \wedge b \tmop{is} \tmop{least} \tmop{element}
  \tmop{of} B \nobracket \} \subseteq \mathcal{P}' (A) \times A$. We have then
  that if $(B, b), (B, b') \in c \Rightarrowlim_{\tmop{least} \tmop{element}
  \tmop{is} \tmop{unique}} b = b'$ proving that $c$ is a partial function. And
  if $B \in \mathcal{P}' (A) \Rightarrow \emptyset \neq B \subseteq A
  \Rightarrowlim_{\tmop{well} \um \tmop{ordering}} \exists b \vdash b
  \tmop{is} \tmop{least} \tmop{element} \tmop{in} B \Rightarrow (B, b) \in c$
  proving that $\tmop{dom} (c) =\mathcal{P}' (A)$ and thus that $c :
  \mathcal{P}' (A) \rightarrow A$ is a function. Finally if $B \in
  \mathcal{P}' (A)$ we have $c (B) \in B$ as the least element of a set is in
  the set itself.
\end{proof}

So we have proved the following equivalences.

\begin{theorem}
  The following statements are equivalent
  \begin{enumerate}
    \item Axiom of Choice
    
    \item Hausdorff's Maximal Principle
    
    \item Zorn's Lemma
    
    \item Every set can be well-ordered
  \end{enumerate}
\end{theorem}

\

For the rest of this document we assume that the Axiom of Choice is valid and
thus also that the Hausdorff's Maximal Principle, Zorn's Lemma and 'Every set
can be well-ordered' holds true.

As an example of applying the Axiom of Choice let's prove the following
theorem.

\begin{theorem}
  \label{use domain restriction to make a function injective}Assume $X, Y$
  sets and $f : X \rightarrow Y$ a function then there exists a $Z \subseteq
  X$ such that $f_{|Z} : X \rightarrow Y$ is injective and $f_{|Z} (X) = f
  (X)$ (which ensures that $f_{|Z} : Z \rightarrow f (X)$ is a bijection).
\end{theorem}

\begin{proof}
  First define $\mathcal{A}= \{ f^{- 1} (\{ y \}) |y \in f (X) \}$. If $A \in
  \mathcal{A}$ there exists a $y \in f (X)$ such that $A = f^{- 1} (\{ y \})
  \Rightarrow A \subseteq X$, as $y \in f (X)$ there exists a $x \in X$ such
  that $y = f (x) \Rightarrow x \in f^{- 1} (\{ y \}) = A \Rightarrow A \neq
  \emptyset \Rightarrow A \in \mathcal{P}' (X)$ so we conclude that
  $\mathcal{A} \subseteq \mathcal{P}' (X)$. By the Axiom of Choice (see
  \ref{axiom of choice}) there exists a choice function $c : \mathcal{P}' (X)
  \rightarrow X$ such that $\tmop{if} A \in \mathcal{P}' (X)$ then $c (A) \in
  A$. Take now $Z = c (\mathcal{A})$ and consider $f_{|Z} : Z \rightarrow Y$
  and prove that is injective. Let $z_1, z_2 \in Z$ is such that $f_{|Z} (z_1)
  = f_{|Z} (z_2) \Rightarrow f (z_1) = f (z_2)$. As $z_1, z_2 \in Z = c
  (\mathcal{A})$ there exists $A_1, A_2 \in \mathcal{A}$ such that $z_1 = c
  (A_1) \in A_1, z_2 = c (A_2) \in A_2$ and thus as $A_1, A_2 \in \mathcal{A}$
  there exists $y_1, y_2 \in f (X)$ such that $A_1 = f^{- 1} (\{ y_1 \}), A_2
  = f^{- 1} (\{ y_2 \})$ so $z_1 \in f^{- 1} (\{ y_1 \}), z_2 \in f^{- 1} (\{
  y_2 \}) \Rightarrow f (z_1) = y_1, f (z_2) = y_2 \Rightarrowlim_{f (z_1) = f
  (z_2)} y_1 = y_2 \Rightarrow A_1 = f^{- 1} (\{ y_1 \}) = f^{- 1} (\{ y_2 \})
  = A_2 \Rightarrow z_1 = c (A_1) = c (A_2) = z_2 \Rightarrow z_1 = z_2$
  proving that $f_{|Z} : Z \rightarrow Y$ is injective.
  
  If now $y \in f (X)$ then $f^{- 1} (\{ y \}) \in \mathcal{A} \Rightarrow x =
  c (f^{- 1} (\{ y \})) \in c (\mathcal{A}) = Z \Rightarrow x \in Z$ and as
  also $x = c (f^{- 1} (\{ y \})) \in f^{- 1} (\{ y \})$ we have $f (x) = y$
  so we have fount a $x \in Z$ such that $y = f (x) \Rightarrow f (X)
  \subseteq f (Z)$. From $Z \subseteq X$ we have $f (Z) \subseteq f (X)$ and
  we conclude thus that $f (X) = f (Z)$ as must be proved.
\end{proof}

Finally we prove that Zorn's lemma is also valid if we replace partial ordered
sets by pre-ordered set.

\begin{theorem}
  \label{Zorn' lemma for pre-ordered sets}Let $\langle A, \leqslant \rangle$
  be a pre-ordered set (see \ref{pre-ordered class}) such that every chain $C
  \subseteq A$ has a upper bound (see \ref{upper bound in pre-ordered set})
  then $A$ has a maximal element (see \ref{maximal (minimal element in a
  pre-ordered class})
\end{theorem}

\begin{proof}
  Use \ref{pre-order to order} to construct the partial ordered set $\langle A
  / \sim, \leqslant \rangle$ where $x \sim y$ iff $x \leqslant y \wedge y
  \leqslant x$ and where $\sim [x] < \sim [y]$ iff $x < y$. Let now $C
  \subseteq A / \sim$ \ be a chain in $A / \sim$ (in the pre-order sense) .
  Form then $C' = \bigcup_{S \in C} S \subseteq A$. If $x, y \in C'$ then
  there exists $\sim [x'], \sim [y'] \in C$ such that $x \in \sim [x'], y \in
  \sim [y']$ or $x \leqslant x' \wedge x' \leqslant x$ and $y \leqslant y'
  \wedge y' \leqslant y$. As $C$ is a chain we have either:
  \begin{enumerate}
    \item $\sim [x'] \leqslant \sim [y'] \Rightarrowlim_{\text{\ref{pre-order
    to order}}} x' \leqslant y' \Rightarrow x \leqslant y$
    
    \item $\sim [y'] \leqslant \sim [x'] \Rightarrowlim_{\text{\ref{pre-order
    to order}}} y' \leqslant x' \Rightarrow y \leqslant x$
  \end{enumerate}
  proving that $C'$ is a chain in $A$. By the hypothesis there exists a upper
  bound $u$ of $C'$ in $A$, in other words $\exists u \in U$ such that
  $\forall a \in A$ we have $a \leqslant u$. Consider now $\sim [u]$ and take
  $\sim [z] \in C$ then as $z \in \sim [z]$ we have $z \in C'$ and thus $z
  \leqslant u$ proving that $\sim [z] \leqslant \sim [u]$. This proves that
  every chain in $A / \sim$ has a upper bound and thus by Zorn's lemma there
  exists a maximal element $\sim [m] \in A$ of $A$. If now $x \in A$ with $m
  \leqslant x$ then $\sim [m] \leqslant \sim [x]$ so by maximality of $\sim
  [m]$ we have then that $\sim [m] = \sim [x]$ giving $x \leqslant m$ and thus
  by \ref{maximal (minimal element in a pre-ordered class} that $m$ is a
  maximal element in the pre-order of $A$.
\end{proof}

\

\chapter{Algebraic Constructs}

\section{Groups}

\begin{definition}[Neutral Element][Associativity]
  {\index{semi-group}}\label{semi-group}\label{semi-group}A semi-group is a
  pair $\langle G, \odot \rangle$ where $G$ is a set and $\odot : G \times G
  \rightarrow G$ a function such that (here $\odot (x, y)$ is noted as $x
  \odot y$)
  \begin{enumerate}
    \item $\exists e \in G \vdash \forall x \in G \vdash x \odot e = e \odot
    x$
    
    \item $\forall x, y, z \in G$ we have $x \odot (y \odot z) = (x \odot y)
    \odot z$
  \end{enumerate}
\end{definition}

\begin{lemma}
  \label{group is not empty}If $\langle G, \odot \rangle$ is a semi-group then
  $G \neq \emptyset$
\end{lemma}

\begin{proof}
  This follows from (1) in the definition.
\end{proof}

\begin{theorem}
  If $\langle G, \odot \rangle$ is a semi-group then there exists only one
  neutral element. 
\end{theorem}

\begin{proof}
  If $e, e'$ are neutral elements then $e = e' \odot e = e'$
\end{proof}

\begin{definition}[Inverse Element]
  {\index{group}}\label{group}A group is a semi-group $\langle G, \odot
  \rangle$ with the extra condition
  \begin{enumerate}
    \item $\forall x \in G \vdash \exists y \in G \vDash x \odot y = e = y
    \odot x$
  \end{enumerate}
\end{definition}

\begin{theorem}
  If $\langle G, \odot \rangle$ is a group then $\forall x \in G \vdash
  \exists !y \in G \Rightarrow x \circ y = e = y \odot x$ this unique element
  is noted as $x^{\um 1}$
\end{theorem}

\begin{proof}
  Take $x \in G$ and suppose that $y, y'$ is such that $x \odot y = e = y
  \odot x$ and $x \odot y' = e = y' \odot x$ then $y = y \odot e = y \odot (x
  \odot y') = (y \odot x) \odot y' = e \odot y' = y'$
\end{proof}

\begin{theorem}
  \label{inverse of inverse}If $\langle G, \odot \rangle$ is a group then
  $\forall x \in G$ we have $(x^{- 1})^{- 1} = x$
\end{theorem}

\begin{proof}
  If $x \in G$ then $x \odot x^{- 1} = e = x^{- 1} \odot x$ and $(x^{- 1})^{-
  1} \odot x^{- 1} = e = x^{- 1} \odot (x^{- 1})^{- 1}$, using the previous
  theorem we have then $x = (x^{- 1})^{- 1}$
\end{proof}

\begin{definition}[Commutativity]
  {\index{abelian group}}{\index{abelian semi-group}}A group or semi-group
  $\langle G, \odot \rangle$ is \tmtextbf{abelian} or
  \tmtextbf{commutative}.If the following is satisfied
  \begin{enumerate}
    \item $\forall x, y \in G \vdash x \odot y = y \odot x$
  \end{enumerate}
\end{definition}

\begin{definition}
  \label{sub-semi-group}{\index{sub-semi-group}}If $\langle G, \odot \rangle$
  is a semi-group then $F \subseteq G$ is a sub-semi-group of $G$ if
  \begin{enumerate}
    \item $\forall x, y \in F \vDash x \odot y \in F$
    
    \item $e \in F$
  \end{enumerate}
\end{definition}

\begin{theorem}
  \label{sub-semi-group is a semi-group}If $\langle G, \odot \rangle$ is a
  semi-group then if $F \subseteq G$ is a sub-semi-group then $\langle F,
  \odot_{| F \times F \nobracket} \rangle$ is a semi-group. For simplicity we
  note this as $\langle F, \odot \rangle$
\end{theorem}

\begin{proof}
  
  \begin{enumerate}
    \item If $x, y \in F \Rightarrow x \odot y \in F \Rightarrow x \odot_{| F
    \times F \nobracket} y \in F \Rightarrow \odot_{| F \times F \nobracket} :
    F \times F \rightarrow F$ is a function
    
    \item $e \in F \Rightarrow x \odot_{| F \times F \nobracket} e = x \odot e
    = x = e \odot x = e \odot_{| F \times F \nobracket} x$
    
    \item If $x, y, z \in F \Rightarrow x \odot_{| F \times F \nobracket} (y
    \odot_{| F \times F \nobracket} z) = x \odot (y \odot z) = (x \odot y)
    \odot z = (x \odot_{| X \times Y \nobracket} y) \odot_{| F \times F
    \nobracket} z$
  \end{enumerate}
\end{proof}

\begin{definition}
  {\index{sub-group}}\label{sub-group}If $\langle G, \odot \rangle$ is a group
  then $F \subseteq G$ is a sub-group of $G$ if
  \begin{enumerate}
    \item $\forall x, y \in F \vDash x \odot y \in F$
    
    \item $\forall x \in G \vDash x^{\um 1} \in F$
    
    \item $e \in F$ where $e$ is the unit in $G$.
  \end{enumerate}
\end{definition}

\begin{theorem}
  If $\langle G, \odot \rangle$ is a group then if $F \subseteq G$ is a
  sub-group we have that $\langle F, \odot_{| F \times F \nobracket} \rangle$
  is a group noted as $\langle F, \odot \rangle$ for simplicity.
\end{theorem}

\begin{proof}
  As $e \in F$ we have using \ref{sub-semi-group is a semi-group} that
  $\langle F, \odot \rangle$ is a semi-group. Also if $x \in F \Rightarrow
  x^{\um 1} \in F = x \odot_{| F \times F \nobracket} x^{- 1} = x \odot x^{-
  1} = e = x^{- 1} \odot x = x^{- 1} \odot_{| F \times F \nobracket} x$
\end{proof}

\begin{definition}
  \label{group homorphism}{\index{group homeorphism}}If $\langle F, \odot
  \rangle$ and $\langle G, \oplus \rangle$ are semi-groups then a
  \tmtextbf{group homeomorphism} is a function $f : F \rightarrow G$ such that
  $\forall x, y \in F$ we have $f (x \odot y) = f (x) \oplus f (y)$
\end{definition}

\begin{theorem}
  \label{group homomorphism and neutral and inverse elements}If $\langle F,
  \odot \rangle$ and $\langle G, \oplus \rangle$ are groups with units $e_F,
  e_G$ and $f : F \rightarrow G$ a group homeomorphism then we have
  \begin{enumerate}
    \item $f (e_F) = f (e_G)$
    
    \item $\forall a \in F$ we have $f (a^{- 1}) = f (a)^{- 1}$
  \end{enumerate}
\end{theorem}

\begin{proof}
  
  \begin{enumerate}
    \item $f (e_F) = f (e_F \cdot e_F) = f (e_F) \cdot f (e_F)
    \Rightarrowlim_{\text{multiply both sides by $f (e_F)^{- 1}$}} e_G = f
    (e_F)$
    
    \item If $a \in F \Rightarrow f (a^{- 1} \cdot a) = f (e_F)
    \equallim_{(1)} e_G \equallim_{(2)} f (e_F) = f (a \cdot a^{- 1})
    \Rightarrow f (a^{- 1}) \cdot f (a) = e_G = f (a) \cdot f (a^{- 1})
    \Rightarrow f (a^{- 1}) = f (a)^{- 1}$
  \end{enumerate}
\end{proof}

\begin{definition}
  \label{group ismorphism}{\index{group isomorphism}}If $\langle F, \odot
  \rangle$ and $\langle G, \oplus \rangle$ are semi-groups then a
  \tmtextbf{group isomorphism} is a function $f : F \rightarrow G$ that is a
  group homeomorphism and a bijection.
\end{definition}

\begin{theorem}
  \label{product of a family of groups}If $\{ \langle F_i, \odot_i \rangle
  \}_{i \in I}$ is a family of (abelian) groups (or semi-groups) then we have:
  \begin{enumerate}
    \item $\forall x, y \in \prod_{i \in I} F_i$ we have:
    \begin{enumerate}
      \item $\forall i \in I$ that $x (i) \odot_i y (i) \in F_i$
      
      \item $x \odot y : I \rightarrow \bigcup_{i \in I} F_i$ defined by $i
      \rightarrow x (i) \odot_i y (i)$ is a element of $\prod_{i \in I} F_i$
    \end{enumerate}
    \item $\odot : \prod_{i \in I} F_i \times \prod_{i \in I} F_i \rightarrow
    \prod_{i \in I} F_i$ defined by $(x, y) \rightarrow x \odot y$ is a
    function [here we use $x \odot y$ to actually note the function
    $\left\langle x \odot y, I, \bigcup_{i \in I} F_i \right\rangle$]
    
    \item $\left\langle \prod_{i \in I} F_i, \odot \right\rangle$ is a
    (abelian) group (or semi group) with neutral element $0 : I \rightarrow
    \bigcup_{i \in I} F_i$ defined by $i \rightarrow 0_i \in F_i$ (where $0_i$
    is the neutral element in $F_i$) and if $x \in \prod_{i \in I} F_i$ then
    $- x : I \rightarrow \bigcup_{i \in I} F_i$ defined by $i \rightarrow - x
    (i) \in F_i$ (where $- x (i)$ is the inverse element of $x (i)$)
  \end{enumerate}
\end{theorem}

\begin{proof}[associativity][neutral element][inverse element][commutativity]
  
  \begin{enumerate}
    \item If $x, y \in \prod_{i \in I} F_i$ then we have
    \begin{enumerate}
      \item As $x (i), y (i) \in F_i$ by the definition of $\prod_{i \in I}
      F_i$ we have as $\odot : F_i \odot_i F_i \rightarrow F_i$ is a function
      we have that $x (i) \odot_i y (i) \in F_i$
      
      \item As $\forall i \in I$ we have by $x (i) \odot_i y (i) \in F_i$ we
      have by definition of $\prod_{i \in I} F_i$ that $x \odot y \in \prod_{i
      \in I} F_i$
    \end{enumerate}
    \item This follows from the definition of $\odot$ and the fact that $x
    \odot y \in F_i$
    
    \item We have if $\{ \langle F_i, \odot_i \rangle \}_{i \in I}$ is a
    family of semi-groups then we have
    \begin{itemizedot}
      \item $\forall i \in I$ we have $((x \odot y) \odot z) (i) = (x \odot y)
      (i) \odot_i z (i) = (x (i) \odot_i y (i)) \odot_i z (i)
      \equallim_{\langle F_i, \odot_i \rangle \tmop{is} a \tmop{semi} -
      \tmop{group}} x (i) \odot_i (y (i) \odot_i z (i)) = x (i) \odot_i (y
      \odot z) (i) = (x \odot (y \odot z)) (i)$ resulting in $(x \odot y)
      \odot z = x \odot (y \odot z)$
      
      \item $\forall i \in I$ we have $(x \odot 0) (i) = x (i) \odot_i 0 (i) =
      x (i) \odot_i 0_i = x (i) = 0_i \odot_i x (i) = (0 \odot x)$ proving
      that $0 \odot x = x = x \odot 0$
    \end{itemizedot}
    If $\{ \langle F_i, \odot_i \rangle \}_{i \in I}$ is also a group then we
    have
    \begin{itemize}
      \item $\forall i \in I$ we have $(x \odot (- x)) (i) = x (i) \odot_i ((-
      x) (i)) = x (i) \odot (- (x (i))) = 0_i = 0 (i) = 0_i = (- x (i)) \odot
      x (i) = (- x) (i) \circ_i x (i) = ((- x) \odot x) (i)$ giving $x \circ
      (- x) = 0 = (- x) \odot x$
    \end{itemize}
    and in case of a abelian family we have
    \begin{itemize}
      \item $\forall i \in I$ we have $(x \odot y) (i) = x (i) \odot_i y (i) =
      y (i) \odot_i x (i) = (y \odot x) (i)$ giving $x \odot y = y \odot x$
    \end{itemize}
  \end{enumerate}
\end{proof}

\begin{definition}
  \label{left (right action)}{\index{left action}}{\index{right action}}Let
  $\langle G, \odot \rangle$ be a group with neutral element $0_G$ and let $X$
  be a set then we have the following definitions:
  \begin{enumerate}
    \item A {\tmstrong{left group action}} is a function $\vartriangleright :
    G \times X \rightarrow X$ where $\vartriangleright (g, x)
    \equallim_{\tmop{noted}} g \vartriangleright x$ such that
    \begin{enumerate}
      \item $\forall x \in X$ we have $0_G \vartriangleright x = x$
      
      \item $\forall g, g' \in G$ and $\forall x \in X$ we have $(g \odot g')
      \vartriangleright x = g \vartriangleright (g' \vartriangleright x)$
    \end{enumerate}
    \item A {\tmstrong{right group action }} is a function $\vartriangleleft :
    X \times G \rightarrow X$ where $\vartriangleleft (x, g)
    \equallim_{\tmop{noted}} x \vartriangleleft g$ such that
    \begin{enumerate}
      \item $\forall x \in X$ we have $x \vartriangleleft 0_G = x$
      
      \item $\forall g, g' \in G$ and $\forall x \in X$ we have $x
      \vartriangleleft (g \odot g') = (x \vartriangleleft g) \vartriangleleft
      g'$
    \end{enumerate}
  \end{enumerate}
\end{definition}

\begin{note}
  If $\langle G, \odot \rangle$ is a group and $X$ as set with a left group
  action $\vartriangleright$ then you think that we can always define a right
  group action $\vartriangleleft$ as follows: $\vartriangleleft : X \times G
  \rightarrow X$ is defined by $(x, g) \rightarrow x \vartriangleleft g \equiv
  g \vartriangleright g$, however this will in general not give you a right
  group action as we have not in general that 2 (b) is not fulfilled for $x
  \vartriangleleft (g \odot g') \equallim_{\tmop{definition}} (g \odot g')
  \vartriangleright x = g \vartriangleright (g' \vartriangleright x)
  \equallim_{\tmop{definition}} (g' \vartriangleright x) \vartriangleleft g
  \equallim_{\tmop{definition}} (x \vartriangleleft g') \vartriangleleft g$
  which in general is not the same as $(x \vartriangleleft g) \vartriangleleft
  g'$.
\end{note}

\begin{definition}
  Let $\langle G, \odot \rangle$ be a group, $X$ a set then if
  $\vartriangleright$ ($\vartriangleleft$) is a left (right) group action then
  we define if $g \in G$
  \begin{enumerate}
    \item $g_{\vartriangleright} : X \rightarrow X$ by $x \rightarrow
    g_{\vartriangleright} (x) = g \vartriangleright x$
    
    \item $_{} g_{\vartriangleleft} : X \rightarrow X$ by $x \rightarrow
    g_{\vartriangleleft} (x) = x \vartriangleleft g$
  \end{enumerate}
\end{definition}

\begin{definition}
  \label{faithful, trasitive action}{\index{faithful
  action}}{\index{transitive action}}Let $\langle G, \odot \rangle$ be a group
  with neutral element $0_G$ and let $X$ be a set then we have the following
  definitions for a left (right) group action $\vartriangleright
  (\vartriangleleft)$:
  \begin{enumerate}
    \item $\vartriangleright$ (or $\vartriangleleft$) is {\tmstrong{faithful}}
    iff $g_{\vartriangleright} = 1_X$ (or $g_{\vartriangleleft} = 1_X$) if and
    only if $g = 0_G$where $1_X : X \rightarrow X$ is the identity mapping.
    Equivalently this means that $\{ g \in G| \forall x \in X|g
    \vartriangleright x = x \} = \{ 0_G \}$ (or $\{ g \in G| \forall x \in X|x
    \vartriangleleft g = x \} = \{ 0_G \}$)
    
    \item $\vartriangleright$ (or $\vartriangleleft$) is
    {\tmstrong{transitive}} iff $\forall x_1, x_2$ there exists a $g \in G$
    such that $g \vartriangleright x_1 = x_2$ (or $x_1 \vartriangleright g =
    x_2$)
    
    \item $\vartriangleright$ (or $\vartriangleleft$) is {\tmstrong{free}} iff
    $\forall x \in X$ we have $\{ g \in G|g \vartriangleright x = x \} = \{
    0_G \}$ (or $\{ g \in |x \vartriangleleft g \} = \{ 0_G \}$)
  \end{enumerate}
\end{definition}

\section{Rings}

\begin{definition}[Ring][Associative][Neutral
element][Inverse][Commutative][Distributive][Neutral
element][Commutative][Associative]
  \label{ring}{\index{ring}}A triple $\langle R, \oplus, \odot \rangle$ is a
  ring iff
  \begin{enumerate}
    \item $R$ is a set
    
    \item $\oplus : R \times R \rightarrow R$ is a function
    
    \item $\odot : R \times R \rightarrow R$ is a function
    
    \item $\langle R, \oplus \rangle$ is a abelian group
    \begin{enumerate}
      \item $\forall a, b, c \in R \vDash a \oplus (b \oplus c) = (a \oplus b)
      \oplus c$
      
      \item $\exists 0 \in R \vdash \forall a \in R \vDash a \oplus 0 = a = 0
      \oplus a$
      
      \item $\forall a \in R$ there $\exists \um a \in R \vDash a \oplus (\um
      a) = 0 = (\um a) \oplus a$
      
      \item $\forall a, b \in R \vdash a \oplus b = b \oplus a$
    \end{enumerate}
    \item $\forall a, b, c \in R \vDash a \odot (b \oplus c) = (a \odot b)
    \oplus (a \odot c)$
    
    \item $\exists 1 \in R \vdash \forall a \in R \vDash a \odot 1 = a = 1
    \odot a$
    
    \item $\forall a, b \in R \vDash a \odot b = b \odot a$
    
    \item $\forall a, b, c \in R \vDash a \odot (b \odot c) = (a \odot b)
    \odot c$
  \end{enumerate}
  Note that in a ring we have also that $\langle R, \odot \rangle$ is a
  abelian semi-group.
\end{definition}

\begin{definition}
  {\index{zero divisor}}If $\langle R, \oplus, \odot \rangle$ is a ring then a
  {\tmstrong{zero divisor}} is a $x \in R \backslash \{ 0 \}$ so that there
  exists a $b \in R \backslash \{ 0 \}$ such that $a \cdot b = 0$. 
\end{definition}

\begin{definition}
  {\index{integral domain}}A ring $\langle R, \oplus, \odot \rangle$ is a
  {\tmstrong{integral domain}} if it does not contain a {\tmstrong{zero
  divisor}}
\end{definition}

\begin{definition}[Sub-Ring]
  \label{sub-ring}{\index{sub-ring}}If $\langle R, \oplus, \odot \rangle$ is a
  ring then a subset $S \subseteq R$ is a sub ring iff
  \begin{enumerate}
    \item $\forall a, b \in S$ we have $a \oplus b \in S$
    
    \item $\forall a, b \in S$ we have $a \odot b \in S$
    
    \item $\forall a \in S$ we have $- a \in S$
    
    \item $1 \in S$ where $1$ is the unit of multiplication $\odot$
    
    \item $0 \in S$ where $0$ is the unit of multiplication $\oplus$
  \end{enumerate}
\end{definition}

\begin{theorem}
  \label{subring is a ring}If $\langle R, \oplus, \odot \rangle$ is a ring and
  $S \subseteq R$ a sub-ring then $\langle S, \oplus_{| S \times S
  \nobracket}, \odot_{| S \times S \nobracket} \rangle$ is ring. For
  simplicity we note this ring as $\langle S, \oplus, \odot \rangle .$
\end{theorem}

\begin{proof}[Associative][Neutral element][Inverse
element][Commutativity][Distributivity][Neutral
Element][Commutativity][Associative]
  
  \begin{enumerate}
    \item As $S \subseteq R$ then by \ref{axiom of subsets} S is a set.
    
    \item As $\oplus : R \times R \rightarrow R$ is a function we have by
    \ref{restricted partial function} that $\oplus_{| S \times S \nobracket} :
    S \times S \rightarrow S$ is a function.
    
    \item As $\odot : R \times R \rightarrow R$ is a function we have by
    \ref{restricted partial function} that $\odot_{| S \times S \nobracket} :
    S \times S \rightarrow S$ is a function.
    
    \item $\langle S, \oplus_{| S \times S \nobracket} \rangle$ is a abelian
    group
    \begin{enumerate}
      \item $\forall a, b, c \in S$ we have $(a \oplus_{| S \times S
      \nobracket} b) \oplus_{| S \times S \nobracket} c = (a \oplus b) \oplus
      c = a \oplus (b \oplus c) = a \oplus_{| S \times S \nobracket} (b
      \oplus_{| S \times S \nobracket} c)$
      
      \item  As $0 \in S$ we have $\forall a \in S \Rightarrow a \oplus_{| S
      \times S \nobracket} 0 = a \oplus 0 = a = 0 \oplus a = 0 \oplus_{| S
      \times S \nobracket} a$
      
      \item  If $a \in S$ then there exists a $- a \in S$ with $a \oplus (- a)
      = 0 = (- a) \oplus a \Rightarrow a \oplus_{| S \times S \nobracket} (-
      a) = a \oplus (- a) = 0 = (- a) \oplus a = (- a) \oplus_{| S \times S
      \nobracket} a$
      
      \item If $a, b \in S$ then $a \oplus_{| S \times S \nobracket} b = a
      \oplus b = b \oplus a = b \oplus_{| S \times S \nobracket} a$
    \end{enumerate}
    \item $\forall a, b, c \in S$ we have $a \odot_{| S \times S \nobracket}
    (b \oplus_{| S \times S \nobracket} c) = a \odot (b \oplus c) = (a \odot
    b) \oplus (a \odot c) = (a \odot_{| S \times S \nobracket} b) \oplus_{| S
    \times S \nobracket} (a \odot_{| S \times S \nobracket} c)$
    
    \item As $1 \in S$ then we have if $a \in S$ that $1 \odot_{| S \times S
    \nobracket} a = 1 \odot a = a \odot 1 = a \odot_{| S \times S \nobracket}
    1$
    
    \item If $a, b \in S$ then $a \odot_{| S \times S \nobracket} b = a \odot
    b = b \odot a = b \odot_{| S \times S \nobracket} a$
    
    \item If $a, b, c \in S$ then $a \odot_{|S \times S} (b \odot_{|S \times
    S} c) = a \odot (b \odot c) = (a \odot b) \odot c = (a \odot_{|S \times S}
    b) \odot_{|S \times S} c$
  \end{enumerate}
\end{proof}

\begin{theorem}
  \label{absorbing element}If $\langle R, \oplus, \odot \rangle$ is a ring
  with neutral element $0$ in $\langle R, + \rangle$ then $\forall x \in R$ we
  have $x \odot 0 = 0 \odot x = 0$
\end{theorem}

\begin{proof}
  If $x \in R$ then $0 = (0 \odot x) \oplus (- (0 \odot x)) \equallim_{0
  \oplus 0 = 0} ((0 \oplus 0) \odot x) \oplus (- (0 \odot x)) = [(0 \odot x)
  \oplus (0 \odot x)] \oplus (- (0 \odot x)) = 0 \odot x \oplus [(0 \odot x) +
  (- (0 \odot x))] = 0 \odot x + 0 = 0 \odot x$
\end{proof}

\begin{definition}
  \label{ring homorphism}{\index{ring homomorphism}}If $\langle A, \oplus_A,
  \odot_A \rangle$ and $\langle B, \oplus_B, \odot_B \rangle$ are rings then a
  function $f : A \rightarrow B$ is a ring homeomorphism iff
  \begin{enumerate}
    \item $\forall a, b \in A$ we have $f (a \oplus_A b) = f (a) \oplus_B f
    (b)$
    
    \item $\forall a, b \in A$ we have $f (a \odot_A b) = f (a) \odot_B f (b)$
    
    \item $f (1_A) = 1_B$ where $1_A$ is the multiplicative inverse in $A$ and
    $1_B$ is the multiplicative inverse in $B$.
  \end{enumerate}
\end{definition}

Note that a ring homeomorphism $f : A \rightarrow B$ for the rings $\langle A,
\oplus_A, \odot_A \rangle, \langle B, \oplus_B, \odot_B \rangle$ is
automatically a group homeomorphism for the groups $\langle A, \oplus_A
\rangle, \langle B, \oplus_B \rangle$. Using \ref{group homomorphism and
neutral and inverse elements} \ we have then the following theorem

\begin{theorem}
  If $\langle A, \oplus_A, \odot_A \rangle$ and $\langle B, \oplus_B, \odot_B
  \rangle$ are rings with additive units $0_A, 0_B$ and $f : A \rightarrow B$
  a ring homeomorphism then we have
  \begin{enumerate}
    \item $f (0_A) = 0_B$
    
    \item $\forall a \in A$ we have $f (- a) = - f (a)$
  \end{enumerate}
\end{theorem}

\begin{definition}
  \label{ring isomorphism}{\index{ring isomorphism}}If $\langle A, \oplus_A,
  \odot_A \rangle$ and $\langle B, \oplus_B, \odot_B \rangle$ are rings then a
  function $f : A \rightarrow B$ is a ring homeomorphism if it is a ring
  homeomorphism and a bijection.
\end{definition}

\section{Fields}

\begin{definition}[Field][Associative][Neutral
element][Inverse][Commutative][Distributive][Neutral
element][Commutative][Associative][Inverse for non zero element]
  \label{field}{\index{field}}A ring $\langle F, \oplus, \odot \rangle$ is a
  field if \
  \begin{enumerate}
    \item $F$ is a set
    
    \item $\oplus : F \times F \rightarrow F$ is a function
    
    \item $\odot : F \times F \rightarrow F$ is a function
    
    \item $\langle F, \oplus, \odot \rangle$ is a Ring
    \begin{enumerate}
      \item $\langle F, \oplus \rangle$ is a abelian group
      \begin{enumerate}
        \item $\forall a, b, c \in F \vDash a \oplus (b \oplus c) = (a \oplus
        b) \oplus c$
        
        \item $\exists 0 \in F \vdash \forall a \in F \vDash a \oplus 0 = a =
        0 \oplus a$
        
        \item $\forall a \in F$ there $\exists \um a \in F \vDash a \oplus
        (\um a) = 0 = (\um a) \oplus a$
        
        \item $\forall a, b \in F \vdash a \oplus b = b \oplus a$
      \end{enumerate}
      \item $\forall a, b, c \in F \vDash a \odot (b \oplus c) = (a \odot b)
      \oplus (a \odot c)$
      
      \item $\exists 1 \in F \vdash \forall a \in F \vDash a \odot 1 = a = 1
      \odot a$
      
      \item $\forall a, b \in F \vDash a \odot b = b \odot a$
      
      \item $\forall a, b, c \in F \vDash a \odot (b \odot c) = (a \odot b)
      \odot c$
    \end{enumerate}
    \item If $a \in F \backslash \{ 0 \}$ then $\exists a^{\um 1} \in F \vdash
    a^{\um 1} \odot a = 1$
  \end{enumerate}
\end{definition}

So a field is a ring that has also for every non-zero element a multiplicative
inverse.

\begin{definition}
  \label{sub-field}{\index{sub field}}If $\langle F, \oplus, \odot \rangle$ is
  a field then a subset $S \subseteq F$ is a sub-field iff the following is
  full filled
  \begin{enumerate}
    \item $\forall a, b \in S \vDash a \oplus b \in S$
    
    \item $\forall a, b \in S \vDash a \odot b \in S$
    
    \item $\forall a \in S \vDash - a \in S$
    
    \item $\forall a \in S \backslash \{ 0 \} \vDash a^{- 1} \in S$
    
    \item $0 \in S$ (where $0$ is the additive neutral element of $F$)
    
    \item $1 \in S$ (where $1$ is the multiplicative neutral element of $F$)
  \end{enumerate}
\end{definition}

\begin{theorem}
  If $\langle F, \oplus, \odot \rangle$ is a field and $S \subseteq F$ a sub
  field then $\langle S, \oplus_{| S \nobracket}, \odot_{| S \nobracket}
  \rangle$ (for simplicity of notation we note this as $\langle S, \oplus,
  \odot \rangle$) 
\end{theorem}

\begin{proof}
  Using \ref{subring is a ring} we have that $\langle S, \oplus_{| S
  \nobracket}, \odot_{| S \nobracket} \rangle$ is a sub ring so we only have
  to prove that every non-zero element in $S$ has a multiplicative inverse. If
  $x \in S \backslash \{ 0 \} \subseteq F\backslash \{ 0 \} \Rightarrow x^{-
  1}$ exists, \ then by the definition of a sub-field $x^{- 1} \in S$ and we
  have $x \odot_{| S \nobracket} x^{- 1} = x \odot x^{- 1} = 1 = x^{- 1} \odot
  x = x^{- 1} \odot_{| S \nobracket} x$.
\end{proof}

\begin{definition}
  \label{field homomorphism}If $\langle F_1, \odot_1, \oplus \rangle$ and
  $\langle F_2, \odot_2, \oplus_2 \rangle$ are fields with multiplicative
  units $1_1, 1_2$ then a function $f : F_1 \rightarrow F_2$ is a field
  homeomorphism iff
  \begin{enumerate}
    \item $\forall a, b \in F_1$ we have $f (a \odot_1 b) = f (a) \odot_2 f
    (b)$
    
    \item $\forall a, b \in F_1$ we have $f (a \oplus_1 b) = f (a) \oplus_2 f
    (b)$
    
    \item $f (1_1) = f (1_2)$
  \end{enumerate}
  If $f$ is also a bijection then we call $f$ a field isomorphism.
\end{definition}

\

Note that a field homeomorphism $f : A \rightarrow B$ for the fields $\langle
A, \oplus_A, \odot_A \rangle, \langle B, \oplus_B, \odot_B \rangle$ is
automatically a group homeomorphism for the groups $\langle A, \oplus_A
\rangle, \langle B, \oplus_B \rangle$. Using \ref{group homomorphism and
neutral and inverse elements} \ we have then the following theorem

\begin{theorem}
  If $\langle A, \oplus_A, \odot_A \rangle$ and $\langle B, \oplus_B, \odot_B
  \rangle$ are rings with additive units $0_A, 0_B$ and $f : A \rightarrow B$
  a ring homeomorphism then we have
  \begin{enumerate}
    \item $f (0_A) = 0_B$
    
    \item $\forall a \in A$ we have $f (- a) = - f (a)$
  \end{enumerate}
\end{theorem}

\

\chapter{The Natural Numbers}

\section{Definition of the Natural Numbers}

Recall the definition of successor (see \ref{successor of a set}), successor
set (see \ref{successor set}) and the Axiom of Infinity (\ref{axiom of
infinity}).

\begin{definition}
  If $A$ is a set then the successor of $A$ is defined to be the set $s (A) =
  A \bigcup \{ A \}$
\end{definition}

\begin{definition}
  A set $A$ is a successor set if
  \begin{enumerate}
    \item $\emptyset \in A$
    
    \item If $X \in A \Rightarrow s (X) \in A$
  \end{enumerate}
\end{definition}

\begin{axiom}[Axiom of Infinity]
  There exists a successor set
\end{axiom}

We can now define the set of natural numbers $\mathbbm{N}$ as follows

\begin{definition}[Natural Numbers]
  \label{natural numbers}{\index{natural numbers}}Let $\mathbbm{S}= \{ S : S
  \tmop{is} a \tmop{successor} \tmop{set} \}$ then $\mathbbm{N}= \bigcap_{S
  \in \mathbbm{S}} S$ is the set of natural numbers. In other words the set of
  Natural Numbers is the intersection of all successor sets. Note that there
  exists a successor set $S$ and $\mathbbm{N} \subseteq S$ so we have by
  \ref{axiom of subsets} that $\mathbbm{N}$ is a set.
\end{definition}

As for all successor sets $S$ we have $\emptyset \in S, s (\emptyset) \in S, s
(s (\emptyset)) \in S, \ldots$ so thus $\emptyset, s (\emptyset), s (s
(\emptyset)) \ldots \in \mathbbm{N}$

\begin{definition}
  The successor function $s : \mathbbm{N} \rightarrow \mathbbm{N}$ is the
  function defined by $n \rightarrow s (n)$
\end{definition}

\begin{definition}
  \label{0 is a natural number}We define the numbers \
  \begin{enumerate}
    \item $0 = \emptyset \in \mathbbm{N}$
    
    \item $1 = s (\emptyset) = \emptyset \bigcup \{ \emptyset \} = \{
    \emptyset \} = \{ 0 \} \in \mathbbm{N}$
    
    \item $2 = s (1) = 1 \bigcup \{ 1 \} = \{ 0 \} \bigcup \{ 1 \} = \{ 0, 1
    \} = \{ \emptyset, \{ \emptyset \} \} \in \mathbbm{N}$
    
    \item $3 = s (2) = 2 \bigcup \{ 2 \} = \{ 0, 1 \} \bigcup \{ 2 \} = \{ 0,
    1, 2 \} = \{ \emptyset, \{ \emptyset \}, \{ \emptyset, \{ \emptyset \} \}
    \}$
    
    \item ...
  \end{enumerate}
\end{definition}

\begin{definition}
  $\mathbbm{N}_0 =\mathbbm{N} \backslash \{ 0 \}$ (the set of non-zero natural
  numbers)
\end{definition}

\begin{theorem}
  \label{successor of zero is not zero}If $n \in \mathbbm{N} \Rightarrow s (n)
  \neq 0$
\end{theorem}

\begin{proof}
  By definition $s (n) = n \bigcup \{ n \} \Rightarrow n \in s (n) \Rightarrow
  s (n) \neq \emptyset = 0$
\end{proof}

\

\begin{theorem}
  \label{successor of a natural number is a natural number}If $n \in
  \mathbbm{N} \Rightarrow s (n) \in \mathbbm{N}$
\end{theorem}

\begin{proof}
  If $n \in \mathbbm{N}$ then if $S$ is a successor set we have that $n \in S
  \Rightarrow s (n) \in S \Rightarrow s (n) \in \bigcap_{S \in \mathbbm{S}} S
  =\mathbbm{N}$
\end{proof}

\begin{theorem}[Mathematical Induction]
  \label{mathematical induction}{\index{mathematical induction}}If $X
  \subseteq \mathbbm{N}$ has the following properties
  \begin{enumerate}
    \item $0 \in X$
    
    \item $n \in X \Rightarrow s (n) \in X$
  \end{enumerate}
  then we have $X =\mathbbm{N}$
\end{theorem}

\begin{proof}
  By (1) and (2) we have that $X$ is a successor set and thus by \ref{natural
  numbers} we have $\mathbbm{N} \subseteq X$ so that from $X \subseteq
  \mathbbm{N}$ we have $X =\mathbbm{N}$
\end{proof}

\begin{theorem}
  \label{n element of successor}Let $m, n \in \mathbbm{N}$ be natural numbers
  then if $m \in s (n)$ then $m \in n \vee m = n$
\end{theorem}

\begin{proof}
  As $s (n) = n \bigcup \{ n \}$ then from $m \in s (n)$ we have either $m \in
  n \vee m \in \{ n \}$ so we have either $m \in n \vee m = n$
\end{proof}

\begin{definition}
  A set $A$ is \tmtextbf{transitive} if $\forall x \in A \vdash x \subseteq A$
\end{definition}

\begin{theorem}
  \label{every natural number is transitive}$\forall n \in \mathbbm{N}$ we
  have that $n$ is transitive
\end{theorem}

\begin{proof}
  We prove this by mathematical induction so let $S = \{ n \in \mathbbm{N} | n
  \tmop{is} \tmop{transitive} \nobracket \} \subseteq \mathbbm{N}$. We have
  then
  \begin{enumerate}
    \item $0 = \emptyset$ so $\forall x \in \emptyset \vdash x \subseteq
    \emptyset$ is satisfied vacuously and thus $0 \in S$
    
    \item If $n \in S$ then $\forall m \in s (n)$ we have by the previous
    theorem (\ref{n element of successor})
    \begin{enumerate}
      \item $m \in n \Rightarrowlim_{n \tmop{is} \tmop{transitive}} m
      \subseteq n \subseteq n \bigcup \{ n \} = s (n) \Rightarrow m \subseteq
      s (n)$
      
      \item $m = n \Rightarrow m \subseteq n \bigcup \{ n \} = s (n)
      \Rightarrow m \subseteq s (n)$
    \end{enumerate}
  \end{enumerate}
  Using mathematical induction we have then that $S =\mathbbm{N}$ so if $n \in
  \mathbbm{N} \Rightarrow n \in S \Rightarrow n$ is transitive.
\end{proof}

\begin{theorem}
  \label{a natural number is different from its successor}If $n \in
  \mathbbm{N} \Rightarrow n \neq s (n)$
\end{theorem}

\begin{proof}
  We prove this by mathematical induction, so let $S = \{ n \in \mathbbm{N} |
  n \neq s (n) \nobracket \}$ then we have
  \begin{enumerate}
    \item $0 \neq \emptyset \bigcup \{ \emptyset \} = s (0)$ [as $\emptyset
    \nin \emptyset = 0 \text{]} \Rightarrow 0 \in S$
    
    \item If $n \in S$ then $n \neq s (n)$ we prove now by contradiction that
    $s (n) \neq s (s (n))$. So assume that $s (n) = s (s (n))$ now as $s (s
    (n)) = s (n) \bigcup \{ s (n) \}$ we have $s (n) \in s (s (n)) = s (n)
    \Rightarrow s (n) \in s (n) = n \bigcup \{ n \} \Rightarrowlim_{n \neq s
    (n)} s (n) \in n \Rightarrowlim_{\text{\ref{every natural number is
    transitive}}} s (n) \subseteq n \Rightarrow n \bigcup \{ n \} \subseteq n
    \subseteq n \bigcup \{ n \} \Rightarrow n = n \bigcup \{ n \} = s (n)$
    contradicting $n \neq s (n)$. So we must have $s (n) \neq s (s (n))
    \Rightarrow s (n) \in S$
  \end{enumerate}
  Using \ref{mathematical induction} we have then that $S =\mathbbm{N}
  \Rightarrow$ if $n \in \mathbbm{N} \Rightarrow n \in S \Rightarrow n \neq s
  (n)$
\end{proof}

\begin{theorem}
  \label{if successors are equal numbers are equal}If $n, m \in \mathbbm{N}$
  is such that $s (n) = s (m)$ then $n = m$
\end{theorem}

\begin{proof}
  If $s (n) = s (m)$ then as $n \in n \bigcup \{ n \} = s (n) \Rightarrow n
  \in s (m)$ then using \ref{n element of successor} we have either
  \begin{enumerate}
    \item $n \in m \Rightarrowlim_{m \tmop{is} \tmop{transitive}} n \subseteq
    m$. Now $m \in s (m) = s (n) \Rightarrow m \in s (n)$ and by \ref{n
    element of successor} we have then
    \begin{enumerate}
      \item $m \in n$ now we have by transitivity of $n$ that $m \subseteq n$
      and thus $n = m$
      
      \item $m = n$ in this case the theorem is proved.
    \end{enumerate}
    \item $n = m$ in this case the theorem is proved.
  \end{enumerate}
\end{proof}

We have proved now that $\mathbbm{N}$ full fills the \tmtextbf{Peano} axiom's

\begin{theorem}[Peano Axioms]
  \label{peano axioms}{\index{peano axioms}}$\mathbbm{N}$ satisfies the
  following so-called Peano Axioms
  \begin{enumerate}
    \item $0 \in \mathbbm{N}$ (see \ref{0 is a natural number})
    
    \item If $n \in \mathbbm{N} \Rightarrow s (n) \in \mathbbm{N}$ (see
    \ref{successor of a natural number is a natural number})
    
    \item $\forall n \in \mathbbm{N}$ we have $s (n) \neq 0$ (see
    \ref{successor of zero is not zero})
    
    \item If $X \subseteq \mathbbm{N}$ is such that
    \begin{enumerate}
      \item $0 \in X$
      
      \item $n \in X \Rightarrow s (n) \in X$
    \end{enumerate}
    then $X =\mathbbm{N}$ (see \ref{mathematical induction})
    
    \item If $n, m \in \mathbbm{N}$ is such that $s (n) = s (m)$ then $n = m$
    (see \ref{if successors are equal numbers are equal})
  \end{enumerate}
\end{theorem}

\begin{theorem}
  \label{non zero element is a successor}If $n \in \mathbbm{N} \wedge n \neq
  0$ then $\exists !m \in \mathbbm{N}$ such that $n = s (m)$
\end{theorem}

\begin{proof}
  Define $A = \{ n \in \mathbbm{N} | (n = 0) \vee (\exists !m \in \mathbbm{N}
  \vdash n = s (m)) \nobracket \}$ then we have
  \begin{enumerate}
    \item $0 \in A$
    
    \item If $n \in A \Rightarrow n \in \mathbbm{N}$ and $s (n) = s (n)$
    further if $m \in \mathbbm{N}$ such that $s (n) = s (m) \Rightarrow n = m
    \Rightarrow \exists !r \in \mathbbm{N} [\tmop{just} \tmop{take} r = n]
    \vdash s (n) = s (r)$ and thus $s (n) \in a$
  \end{enumerate}
  So if $n \in \mathbbm{N} \wedge n \neq 0 \Rightarrow n \in A \wedge n \neq 0
  \Rightarrow \exists !m \in \mathbbm{N} \vdash n = s (m)$
\end{proof}

\section{Recursion}

\begin{theorem}[Recursion]
  \label{recursion}{\index{recursion}}If $A$ is a set, $a \in A$ and $f : A
  \rightarrow A$ is a function then there exists a unique function $\lambda :
  \mathbbm{N} \rightarrow A$ such that
  \begin{enumerate}
    \item $\lambda (0) = a$
    
    \item $\forall n \in \mathbbm{N} \vDash \lambda (s (n)) = f (\lambda (n))$
  \end{enumerate}
\end{theorem}

\begin{proof}
  Let $\mathcal{G}= \{ G | G \subseteq \mathbbm{N} \times A \nobracket
  \tmop{such} \tmop{that} (0, a) \in G \tmop{and} \forall n \in \mathbbm{N}
  \vDash (n, x) \in G \Rightarrow (s (n), f (x)) \in G \}$. We have then that
  $\mathcal{G} \neq \emptyset$ as for $\mathbbm{N} \times A$ we have $(0, a)
  \in \mathbbm{N} \times A$ and if $(n, x) \in \mathbbm{N} \times A$ then as
  $s (n) \in \mathbbm{N} \wedge f (x) \in A \Rightarrow (s (n), f (x)) \in
  \mathbbm{N} \times A \Rightarrow \mathbbm{N} \times A \in \mathcal{G}$. We
  prove now that
  \begin{eqnarray*}
    \lambda = \bigcap_{G \in \mathcal{G}} G & \in & \mathcal{G}
  \end{eqnarray*}
  \begin{proof}
    
    \begin{enumerate}
      \item $\lambda \subseteq N \times A$ as $\forall G \in \mathcal{G}$ we
      have $G \subseteq \mathbbm{N} \times A$
      
      \item $\forall G \in \mathcal{G}$ we have $(0, a) \in G \Rightarrow (0,
      a) \in \lambda$
      
      \item If $(n, x) \in \lambda \Rightarrow \forall G \in \mathcal{G}$ we
      have $(n, x) \in G \Rightarrow (s (n), f (x)) \in G \Rightarrow (s (n),
      f (x)) \in \lambda$
    \end{enumerate}
    So we have that $\lambda \in \mathcal{G}$
  \end{proof}
  
  We shown now that $\tmop{dom} (\lambda) =\mathbbm{N}$. We have
  \begin{enumerate}
    \item $\tmop{dom} (\lambda) \subseteq \mathbbm{N}$
    
    \item $0 \in \tmop{dom} (\lambda)$ as $(0, a) \in \lambda$
    
    \item If $n \in \tmop{dom} (\lambda) \Rightarrow \exists x \in A \vdash
    (n, x) \in \lambda \Rightarrow (s (n), f (x)) \in \lambda \Rightarrow s
    (n) \in \tmop{dom} (\lambda)$
  \end{enumerate}
  Using mathematical induction (see \ref{mathematical induction}) we have then
  that $\tmop{dom} (\lambda) =\mathbbm{N}$. We prove now using mathematical
  induction that $\lambda$ is a partial function (from $\tmop{dom} (\lambda)
  =\mathbbm{N}$ we have then that $\lambda$ is a function). So let $S = \{ n
  \in \mathbbm{N} | \exists !x \vdash (n, x) \in \lambda \nobracket \}
  \subseteq \mathbbm{N}$ then we have
  \begin{enumerate}
    \item $0 \in S$. We prove this by contradiction, so let $(0, a), (0, x)
    \in \lambda$ with $a \neq x \Rightarrow (0, a) \neq (0, x)$ Take then
    $\lambda' = \lambda \backslash (0, x) \subset \lambda$. We have then
    \begin{enumerate}
      \item $(0, a) \in \lambda \Rightarrowlim_{(0, a) \neq (0, x)} (0, a) \in
      \lambda'$
      
      \item If $(n, y) \in \lambda' \Rightarrowlim_{\lambda' \subseteq
      \lambda} (n, y) \in \lambda \Rightarrow (s (n), f (y)) \in \lambda
      \Rightarrowlim_{\text{\ref{successor of zero is not zero}} \Rightarrow s
      (n) \neq 0} (s (n), f (y)) \in \lambda'$
    \end{enumerate}
    this proves that $\lambda' \in \mathcal{G}$ but then we have $\lambda
    \subseteq \lambda' \Rightarrow \lambda \subset \lambda'$ a contradiction.
    So we must have that $0 \in S$
    
    
    
    \item If $n \in S \Rightarrow s (n) \in S$. Again we prove this by
    contradiction. So assume that $n \in S \Rightarrow \exists !x \vdash (n,
    x) \in \lambda \Rightarrow (s (n), f (x)) \in \lambda$, assume now that
    $\exists y \neq f (x) \vdash (s (n), y) \in \lambda$. Then we can form
    $\lambda' = \lambda \backslash (s (n), y)$ where because $y \neq f (x)$ we
    have $\lambda' \subset \lambda$. For $\lambda'$ we have then that
    \begin{enumerate}
      \item $(0, a) \in \lambda \Rightarrowlim_{0 \neq s (n) \Rightarrow (0,
      a) \neq (s (n), y)} (0, a) \in \lambda'$
      
      \item If $ (m, z) \in \lambda' \Rightarrow (m, z) \in \lambda
      \Rightarrow (s (m), f (z)) \in \lambda$ we have then two sub cases
      \begin{enumerate}
        \item $s (m) = s (n)$ then by \ref{if successors are equal numbers are
        equal} we have that $m = n \Rightarrow (m, z) = (n, z)
        \Rightarrowlim_{n \in S \Rightarrow (n, z), (n, x) \in \lambda
        \Rightarrow z = x} z = x \Rightarrow (s (m), f (z)) = (s (n), f (x))
        \neq (s (n), y) \Rightarrow (s (m), f (z)) \in \lambda'$
        
        \item $s (m) \neq s (n)$ then $ (s (m), f (z)) \neq (s (n), y)
        \Rightarrow (s (m), f (z)) \in \lambda'$
      \end{enumerate}
      or in all cases $(s (m), f (z)) \in \lambda'$
    \end{enumerate}
    This proves that $\lambda' \in \mathcal{G}$ but then we have $\lambda
    \subseteq \lambda' \Rightarrowlim_{\lambda' \subset \lambda} \lambda
    \subset \lambda$ a contradiction. So we must conclude that $\forall y
    \vdash (s (n), y) \in \lambda$ that $y = f (x)$ and thus that $\exists !y
    (= f (x)) \vdash (s (n), y) \in \lambda$ proving that $s (n) \in S$
  \end{enumerate}
  Using mathematical induction we have then that $S =\mathbbm{N}$ and thus
  that $\lambda$ is a partial function and because we have already proved that
  $\tmop{dom} (\lambda) =\mathbbm{N}$ we have that $\lambda : \mathbbm{N}
  \rightarrow A$ is a function. By the fact that $\lambda \in \mathcal{G}$ we
  have also that
  \begin{enumerate}
    \item $(0, a) \in \lambda$
    
    \item if $(n, x) \in \lambda \Rightarrow (s (n), f (x)) \in \lambda$
  \end{enumerate}
  proving that $\lambda$ is the function we search for. Now to prove that
  $\lambda$ is unique suppose that there exists a $\lambda'$ satisfying (1)
  and (2) above, define $T = \{ n| \lambda (n) = \lambda' (n) \}$, then we
  have
  \begin{enumerate}
    \item From $\lambda (0) = a = \lambda' (0)$ that $0 \in T$
    
    \item If $n \in T \Rightarrow \lambda (n) = \lambda' (n) = d \Rightarrow
    (n, d) \in \lambda \wedge (n, d) \in \lambda' \Rightarrowlim_{(2) \noplus
    f \tmop{is} a \tmop{fucntion}} (s (n), f (d)) \in \lambda \wedge (s (n), f
    (d)) \in \lambda' \Rightarrowlim \lambda (s (n)) = f (d) = \lambda' (s
    (n)) \Rightarrow s (n) \in T$
  \end{enumerate}
  By using mathematical induction again we have then that $T =\mathbbm{N}$ so
  if $n \in \mathbbm{N} \Rightarrow n \in T \Rightarrow \lambda (n) = \lambda'
  (n) \Rightarrow \lambda = \lambda'$
\end{proof}

As a corollary of the above theorem we have the following

\begin{corollary}
  \label{recursive injective function}If $A$ is a set, $a \in A$ and $f : A
  \rightarrow A$ is a injective function then there exists a unique function
  $\lambda : \mathbbm{N} \rightarrow A$ such that
  \begin{enumerate}
    \item $\lambda (0) = a$
    
    \item $\forall n \in \mathbbm{N} \vDash \lambda (s (n)) = f (\lambda (n))$
    
    \item If $a \nin f (A)$ then $\lambda$ is injective
  \end{enumerate}
\end{corollary}

\begin{proof}
  By recursion (see \ref{recursion}) we have that there exists a unique
  function $\lambda : \mathbbm{N} \rightarrow A$ such that (1) and (2) are
  satisfied. Now if $a \nin f (A)$ then we must show that if $\lambda (m) =
  \lambda (n)$ then $m = n$. We do this by induction so let $n \in
  \mathbbm{N}$ and define $S = \{ m \in \mathbbm{N} | \tmop{if} n \in
  \mathbbm{N} \tmop{is} \tmop{such} \tmop{that} \lambda (m) = \lambda (n)
  \Rightarrow m = n \nobracket \}$ then we have
  \begin{enumerate}
    \item Then $0 \in S$ as we have the following if $\lambda (n) = \lambda
    (0)$
    \begin{enumerate}
      \item $n = 0$ then $0 \in S$
      
      \item $n \neq 0$ then by \ref{non zero element is a successor} there
      exists a $k \in \mathbbm{N}$ such that $n = s (k)$ but then $\lambda (n)
      = \lambda (0) = a$ implies $\lambda (s (k)) = a \Rightarrow f (\lambda
      (k)) = a$ which is false as $a \nin f (A)$ and thus $\lambda (n) =
      \lambda (0) \Rightarrow n = 0$ is true or $0 \in S$ again.
    \end{enumerate}
    \item If $m \in S$ then if $k \in \mathbbm{N} \vdash \lambda (m) = \lambda
    (k) \Rightarrow m = k$. Suppose now that there exists a $n \in \mathbbm{N}
    \vdash \lambda (s (m)) = \lambda (n)$ then we have the following
    possibilities
    \begin{enumerate}
      \item $n = 0$ then $a = \lambda (0) = \lambda (s (m)) = f (\lambda (m))$
      is not allowed because $a \nin f (A)$ so $\lambda (s (m)) = \lambda (n)$
      is false and thus $\lambda (s (m)) = \lambda (n) \Rightarrow s (m) = n$
      is true and thus $s (m) \in S_n$
      
      \item $n \neq 0$ then by \ref{non zero element is a successor} we have
      that $\exists k \in \mathbbm{N} \vdash n = s (k)$ so from $\lambda (s
      (m)) = \lambda (n)$ we have $\lambda (s (m)) = \lambda (s (k))
      \Rightarrow f (\lambda (m)) = f (\lambda (k)) \Rightarrowlim_{f
      \tmop{is} \tmop{injective}} \lambda (m) = \lambda (k) \Rightarrowlim_{m
      \in S} m = k \Rightarrow s (m) = s (k) \Rightarrow s (m) = n \Rightarrow
      s (m) \in S$
    \end{enumerate}
  \end{enumerate}
  Using \ref{mathematical induction} we have then that $S =\mathbbm{N}$ and
  thus if there exists a $n \in \mathbbm{N}$ and a $m \in \mathbbm{N}$ with $s
  (m) = s (n) \Rightarrow n \in S \Rightarrow m = n$ proving injectivity.
\end{proof}

\

\begin{theorem}
  \label{universal property of natural numbers}{\index{universal property of
  $\mathbbm{N}$}}Suppose $A$ is a set $a \in A$ and $f : A \rightarrow A$ a
  function, then there exists a \tmtextbf{unique} function $\varphi :
  \mathbbm{N} \rightarrow A$ such that $\varphi (0) = a$ and $\varphi \circ s
  = f \circ \varphi$.
\end{theorem}

\begin{proof}
  Define the function $\sigma : \mathbbm{N} \times A \rightarrow A$ by $(n, x)
  \rightarrow (s (n), f (x))$. Given $a \in A$ we define a subset $R \subseteq
  \mathbbm{N} \times A$ to be a-closed if the following is true
  \begin{enumerate}
    \item $(0, a) \in R$
    
    \item $\sigma (R) \subseteq R$
  \end{enumerate}
  We have that $\mathbbm{N} \times A$ is a-closed for
  \begin{enumerate}
    \item $(0, a) \in \mathbbm{N} \times A$
    
    \item If $(m, y) \in \sigma (\mathbbm{N} \times A) \Rightarrow \exists (n,
    x) \in \mathbbm{N} \times A \vdash (m, y) = (s (n), f (x)) \in \mathbbm{N}
    \times A \Rightarrow \sigma (\mathbbm{N} \times A) \subseteq \mathbbm{N}
    \times A$.
  \end{enumerate}
  If $\mathcal{R}$ is a set of a-closed subsets of $\mathbbm{N} \times A$ then
  we have
  \begin{enumerate}
    \item $\forall R \in \mathcal{R}$ we have $(0, a) \in R \Rightarrow (0, a)
    \in \bigcap_{R \in \mathcal{R}} R$
    
    \item if $y \in \sigma \left( \bigcap_{R \in \mathcal{R}} R \right)
    \Rightarrow \exists x \in \bigcap_{R \in \mathcal{R}} R \vdash y = \sigma
    (x) \Rightarrow \exists x \vdash (\forall R \in \mathcal{R} \vdash x \in
    R) \wedge y = \sigma (x) \Rightarrow \forall R \in \mathcal{R} \vdash y
    \in \sigma (R) \Rightarrow y \in \bigcap_{R \in \mathcal{R}} \sigma (R)
    \Rightarrow \sigma \left( \bigcap_{R \in \mathcal{R}} R \right) \subseteq
    \bigcap_{R \in \mathcal{R}} \sigma (R) \Rightarrowlim_{\sigma (R)
    \subseteq R} \sigma \left( \bigcap_{R \in \mathcal{R}} R \right) \subseteq
    \bigcap_{R \in \mathcal{R}} R$
  \end{enumerate}
  proving that $\bigcap_{R \in \mathcal{R}} R$ is also a-closed. Define now
  the set $\mathcal{R}_a = \{ R \in \mathcal{P} (\mathbbm{N} \times A) | R
  \text{is a-closed} \nobracket \}$ set (because $\subseteq \mathcal{P}
  (\mathbbm{N} \times A)$ which is set because $\mathbbm{N} \times A$ is a set
  and \ref{axiom of power sets}) and $R_a = \bigcap_{R \in \mathcal{R}_a} R$
  then we have by the above that $R_a$ is a-closed and is further the smallest
  a-closed set. We have then
  \begin{enumeratealpha}
    \item $(0, a) \in \{ (0, a) \} \bigcup \sigma (R_a)$
    
    \item $\sigma \left( \{ (0, a) \} \bigcup \sigma (R_a) \right) \subseteq
    \sigma (\{ (0, a) \}) \bigcup \sigma (\sigma (R_a)) \subseteq_{\sigma (0,
    a) \in R_a} R_a \bigcup \sigma (\sigma (R_a)) \subseteq_{\sigma (R_a)
    \subseteq R_a} R_a \bigcup \sigma (R_a) \subseteq_{\sigma (R_a) \subseteq
    R_a} R_a \bigcup R_a = R_a$
  \end{enumeratealpha}
  proving that $\{ (0, a) \} \bigcup R_a$ is a-closed and then from the
  minimality of $R_a$ we have
  \begin{eqnarray*}
    & R_{\alpha} \subseteq \{ (0, a) \} \bigcup \sigma (R_a) & 
  \end{eqnarray*}
  As $\{ (0, a) \} \subseteq R_a$ and $\sigma (R_a) \subseteq R_a$ we have
  \begin{eqnarray*}
    \{ (0, a) \} \bigcup \sigma (R_a) \subseteq R_a &  & 
  \end{eqnarray*}
  and thus we finally have
  \begin{eqnarray*}
    & R_a = \{ (0, a) \} \bigcup \sigma (R_a) & 
  \end{eqnarray*}
  We say now that $n \in \mathbbm{N}$ is a-paired-up iff
  \begin{enumeratealpha}
    \item There exists $y \in A$ such that $(n, y) \in R_a$
    
    \item If also $(n, y') \in R_a \Rightarrow y = y'$
  \end{enumeratealpha}
  Let's define now the set of a-paired-up numbers $P_a = \{ n \in \mathbbm{N}
  | n \text{is a-paired-up} \nobracket \} \subseteq \mathbbm{N}$ then we proof
  that
  \begin{enumerate}
    \item $0 \in P_a$
    
    \begin{proof}
      First $(0, a) \in R_a$. Second if $(0, a') \in R_a$ with $a \neq a'
      \Rightarrow (0, a) \neq (0, a') \Rightarrowlim_{(0, a') \in R_a = \{ (0,
      a) \} \bigcup \sigma (R_a)} (0, a') \in \sigma (R_a)$ so there exists a
      $(n, a'') \in R_a$ with $(0, a') = \sigma (n, a'') = (s (n), f (a''))$
      giving $0 = s (n)$ a contradiction. So we must conclude that $a = a'$.
    \end{proof}
    
    \item If $n \in P_a \Rightarrow s (n) \in P_a$
    
    \begin{proof}
      As $n \in P_a$ then $n$ is a-paired-up so there exists a
      \tmtextbf{unique} $y \in A$ such that $(n, y) \in R_a
      \Rightarrowlim_{R_a \tmop{is} \text{a-closed}} \sigma (n, y) \in R_a
      \Rightarrow (s (n), f (y)) \in R_a$. Suppose now that $(s (n), y') \in
      R_a = \{ (0, a) \} \bigcup \sigma (R_a) \Rightarrowlim_{s (n) \neq 0} (s
      (n), y') \in \sigma (R_a) \Rightarrow \exists (m, c) \in R_a \vdash (s
      (n), y') = \sigma (m, c) = (s (m), f (c)) \Rightarrow s (n) = s (m)
      \wedge y' = f (c) \Rightarrowlim_{\text{\ref{if successors are equal
      numbers are equal}}} n = m \wedge y' = f (c) \Rightarrowlim (n, c) \in
      R_a \wedge y' = f (c) \Rightarrowlim_{(n, y) \in R_a} (n, c), (n, y) \in
      R_a \wedge y' = f (c) \Rightarrowlim_{n \text{is a-paired-up}} c = y
      \wedge y' = f (c) \Rightarrow y' = f (y)$. So $s (n)$ is a-paired-up and
      thus $s (n) \in P_a$
    \end{proof}
  \end{enumerate}
  Using mathematical induction (see \ref{mathematical induction} ) we have
  then that $P_a =\mathbbm{N}$. Define now $\varphi : \mathbbm{N} \rightarrow
  A$ with $\varphi = R_a$ then we have that if $n \in \mathbbm{N} \Rightarrow
  n \in P_a$ so there exists a $y \in A \vdash (n, a) \in R_a = \varphi$ and
  thus $\tmop{dom} (\varphi) =\mathbbm{N}$. Second if $(n, y), (n, y') \in
  \varphi$ then as $n \in \mathbbm{N} \Rightarrow n \in P_a$ we have that $n$
  is a-paired-up and thus $y = y'$. So we conclude that $\varphi : \mathbbm{N}
  \rightarrow A$ is indeed a function. Now as $R_a$ is a-closed we have $(0,
  a) \in R_a = \varphi \Rightarrow \varphi (0) = a$ and thus
  \begin{eqnarray*}
    & \varphi (0) = a & 
  \end{eqnarray*}
  Second if $n \in \mathbbm{N}$ then $n \in P_a$ so there exists a $y \in A$
  such that $(n, y) \in R_a = \varphi \Rightarrow y = \varphi (n)$ \ then as
  $R_a$ is a-closed we have $(s (n), f (y)) \in R_a = \varphi$ and thus
  $(\varphi \circ s) (n) = \varphi (s (n)) = f (y) = f (\varphi (n)) = (f
  \circ \varphi) (n)$ proving that
  \begin{eqnarray*}
    \varphi \circ s & = & f \circ \varphi
  \end{eqnarray*}
  So we have found that the function $\varphi : \mathbbm{N} \rightarrow A$
  satisfies the requirements of the theorem. Let's now prove that it is
  unique. Assume that there exists also a $\varphi' : \mathbbm{N} \rightarrow
  A$ satisfying
  \begin{eqnarray*}
    \varphi' (0) & = & a\\
    \varphi' \circ s & = & f \circ \varphi
  \end{eqnarray*}
  Take then $E = \{ n \in \mathbbm{N} | \varphi (n) = \varphi' (n) \nobracket
  \}$ then we have
  \begin{enumerate}
    \item $\varphi (0) = a = \varphi' (0) \Rightarrow 0 \in A$
    
    \item If $n \in E$ then for $s (n)$ we have $\varphi (s (n)) = (\varphi
    \circ s) (n) = (f \circ \varphi) (n) = f (\varphi (n)) \equallim_{n \in E
    \Rightarrow \varphi (n) = \varphi' (n)} f (\varphi' (n)) = (f \circ
    \varphi') (n) = (\varphi' \circ s) (n) = \varphi' (s (n))$ and thus $s (n)
    \in E$
  \end{enumerate}
  Using mathematical induction (see \ref{mathematical induction}) we have then
  that $E =\mathbbm{N}$ so $\forall n \in \mathbbm{N}$ we have $n \in E
  \Rightarrow \varphi (n) = \varphi' (n)$ and thus $\varphi = \varphi'$
  
  \ 
\end{proof}

This theorem allows us to iterate the application of a function in the
following way: If $a \in A$ and $f : A \rightarrow A$ is a function then there
exists a function $\varphi : \mathbbm{N} \rightarrow A$ such that
\begin{eqnarray*}
  \varphi (0) & = & a\\
  \varphi (1) = \varphi (s (0)) & = & f (\varphi (0)) = f (a)\\
  \varphi (2) = \varphi (s (1)) & = & f (\varphi (1)) = f (f (a))\\
  \varphi (3) = \varphi (s (2)) & = & f (\varphi (2)) = f (f (f (a)))\\
  & \ldots & \\
  \varphi (n) & = & \overbrace{^{} f (f (\ldots . (f (a))))}^{n \tmop{times}}
\end{eqnarray*}
As \ application of the above theorem we have the following iteration theorem

\begin{theorem}[Iteration]
  \label{iteration}{\index{iteration}}Let $A$ be a set and $f : A \rightarrow
  A$ be a function. Then $\forall n \in \mathbbm{N}$ there exists a function
  $(f)^n : A \rightarrow A$ such that
  \begin{enumerate}
    \item $(f)^0 = 1_A$
    
    \item $(f)^{s (n)} = f \circ (f)^n$
  \end{enumerate}
\end{theorem}

\begin{proof}
  If $A = \emptyset$ then $\emptyset : \emptyset \rightarrow \emptyset$ then
  as $1_{\emptyset} : \emptyset \rightarrow \emptyset$ is also the empty
  function $\emptyset$ then $(\emptyset)^n = \emptyset$ satisfies (1), (2). If
  $A \neq \emptyset$ then let $n \in \mathbbm{N}$ and $a \in A$ and use the
  previous theorem \ref{universal property of natural numbers} to find a
  function $\varphi_a : \mathbbm{N} \rightarrow A$ such that $\varphi_a (0) =
  a$ and $\varphi_a \circ s = f \circ \varphi_a$. Define now $(f)^n : A
  \rightarrow A$ by $a \rightarrow \varphi_a (n)$ then we have
  \begin{enumerate}
    \item $\forall a \in A$ that $(f)^0 (a) = \varphi_a (0) = a \Rightarrow
    (f)^0 = i_A$
    
    \item $\forall a \in A$ then $(f)^{s (n)} (a) = \varphi_a (s (n)) = f
    (\varphi_a (n)) = f ((f)^n (a)) = (f \circ (f)^n) (a) \Rightarrow (f)^{s
    (n)} = f \circ (f)^n$
  \end{enumerate}
\end{proof}

\

As a illustration of this iteration let $f : A \rightarrow A$ then we have
\begin{eqnarray*}
  (f)^0 & = & 1_A\\
  (f)^1 = (f)^{s (0)} & = & f \circ (f)^0 = f \circ i_A = f\\
  (f)^2 = (f)^{s (1)} & = & f \circ (f)^1 = f \circ f\\
  (f)^3 = (f)^{s (2)} & = & f \circ (f)^2 = f \circ f \circ f\\
  & \ldots & \\
  (f)^n & = & \overbrace{f \circ \ldots \circ f}^{n \tmop{times}}
\end{eqnarray*}
\begin{example}
  \label{iteration over a group}Let $\langle A, \oplus \rangle$ be a group and
  $a \in A$ define then $\oplus_a : A \rightarrow A$ by $x \rightarrow
  \oplus_a (x) = x \oplus a$ we define then given $n \in \mathbbm{N}$ $a
  \langle \oplus \rangle n = (\oplus_a)^n (\nu)$ where $\nu$ is the neutral
  element in the group . We have then that
  \begin{eqnarray*}
    a \langle \oplus \rangle 0 & = & (\oplus_a)^0 (\nu) = i_A (\nu) = \nu\\
    a \langle \oplus \rangle s (n) & = & ((\oplus_a)^n (\nu)) \oplus a\\
    & = & (a \langle \oplus \rangle n) \oplus a
  \end{eqnarray*}
  
  
  Sometimes we consider a group to be additive or multiplicative this is
  either noted as $\langle A, + \rangle$ with neutral element $0$ or $\langle
  A, \cdot \rangle$ with neutral element $1$ then we note $a \langle + \rangle
  n \text{ as } a \cdot n$ as and $a \langle \cdot \rangle n$ as $a^n$ we have
  then
  \begin{enumerate}
    \item Additive group $\langle A, + \rangle$ with neutral element $0$
    \begin{eqnarray*}
      a \cdot 0 & = & 0\\
      a \cdot s (n) & = & (a \cdot n) \upl a
    \end{eqnarray*}
    so we have
    \begin{eqnarray*}
      a \cdot 0 & = & 0\\
      a \cdot 1 & = & a \cdot s (0) = (a \cdot 0) \upl a = 0 \upl a = a\\
      a \cdot 2 & = & a \cdot s (1) = (a \cdot 1) \upl a = a \upl a\\
      a \cdot 3 & = & a \cdot s (2) = (a \cdot 2) \upl a = (a \upl a) \upl a\\
      & \ldots & 
    \end{eqnarray*}
    \item Multiplicative group $\langle A, \cdot \rangle$ with neutral element
    $1$
    \begin{eqnarray*}
      a^0 & = & 1\\
      a^{s (n)} & = & (a^n) \cdot a
    \end{eqnarray*}
    so we have
    \begin{eqnarray*}
      a^0 & = & 1\\
      a^1 & = & a^{s (0)} = (a^0) \cdot a = 1 \cdot a = a\\
      a^2 & = & a^{s (1)} = (a^1) \cdot a = a \cdot a\\
      a^3 & = & a^{s (2)} = (a \cdot a) \cdot a\\
      & \ldots & 
    \end{eqnarray*}
  \end{enumerate}
\end{example}

\begin{definition}
  If $\langle F, +, \cdot \rangle$ is a field with multiplicative unit $u$ and
  additive neutral element $e$, let $n \in \mathbbm{N}_0$ define take then $f
  : F \rightarrow F$ defined by $x \rightarrow x + u$ then $n \cdot u = (f)^n
  (e)$. 
\end{definition}

\begin{example}
  If $\langle F, \upl, \cdot \rangle$ is a field with multiplicative unit $u$
  then we have
  \begin{eqnarray*}
    0 \cdot u = (f)^0 (e) & = & 1_F (e) = e\\
    1 \cdot u = (f)^1 (e) & = & f^{(s (0))} (e) = f (f^0 (e)) = f (e) = e + u
    = u\\
    2 \cdot u = (f)^2 (e) & = & f^{s (1)} (e) = f (f^1 (e)) = f (u) = u + u\\
    3. u = (f)^3 (e) & = & f^{s (2)} (e) = f (f^2 (e)) = f (u + u) = u + u +
    u\\
    & \ldots & 
  \end{eqnarray*}
\end{example}

\begin{definition}
  \label{field with characterization zero}{\index{field with characterization
  zero}}If $\langle F, +, \cdot \rangle$ is a field with multiplicative unit
  $u$ and neutral element $e$ then $F$ is of characterization zero if $\forall
  n \in \mathbbm{N}_0$ we have $n \cdot u \neq e$
\end{definition}

\begin{theorem}[Recursion on $\mathbbm{N}$ - Step Form]
  \label{Recursion step form}{\index{recursion on $\mathbbm{N}$ step form}}Let
  $A$ be a set, $a \in A$ and $g : \mathbbm{N} \times A \rightarrow A$ a
  function. Then there exists a unique function $f : \mathbbm{N} \rightarrow
  A$ satisfying
  \begin{eqnarray*}
    f (0) & = & a\\
    \forall n \in \mathbbm{N} \tmop{we} \tmop{have} f (s (n)) & = & g (n, f
    (n))
  \end{eqnarray*}
\end{theorem}

\begin{proof}
  First define the projection functions $\pi_1$ and $\pi_2$ as follows
  \begin{eqnarray*}
    \pi_1 : \mathbbm{N} \times A \rightarrow \mathbbm{N} & \tmop{by} & (n, x)
    \rightarrow \pi_1 (n, x) = n\\
    \pi_2 : \mathbbm{N} \times A \rightarrow \mathbbm{N} & \tmop{by} & (n, x)
    \rightarrow \pi_2 (n, x) = x
  \end{eqnarray*}
  we haven then $\forall (n, x) \in \mathbbm{N} \times A$ that $(n, x) =
  (\pi_1 (n, x), \pi_2 (n, x))$. Next define $\gamma$ by
  \begin{eqnarray*}
    \gamma : \mathbbm{N} \times A \rightarrow \mathbbm{N} \times A & \tmop{by}
    & (n, x) \rightarrow \gamma (n, x) = (s (n), g (n, x))
  \end{eqnarray*}
  By using \ref{iteration} we have $\forall n \in \mathbbm{N}$ the existence
  of a function $(\gamma)^n : \mathbbm{N} \times A \rightarrow \mathbbm{N}
  \times A$ such that
  \begin{eqnarray*}
    (\gamma)^0 & = & 1_{\mathbbm{N} \times A}\\
    (\gamma)^{s (n)} & = & \gamma \circ (\gamma)^n
  \end{eqnarray*}
  We prove now by induction that $\forall n \in \mathbbm{N}$ we have $\pi_1
  ((\gamma)^n (0, a)) = n$. So define $B = \{ n \in \mathbbm{N}| \pi_1
  ((\gamma)^n (0, a)) = n \}$ then we have
  \begin{enumerate}
    \item $\pi_1 ((\gamma)^0 (0, a)) = \pi_1 (i_{\Nu \times A} (0, a)) = \pi_1
    (0, a) = 0$ and thus we have that $0 \in B$.
    
    \item If $n \in B$ then $\pi_1 ((\gamma)^{s (n)} (0, a)) = \pi_1 (\gamma
    ((\gamma)^n (0, a))) = \pi_1 (\gamma (\pi_1 ((\gamma)^n (0, a)), \pi_2
    ((\gamma)^n (0, a)))) \equallim_{n \in B} \pi_1 (\gamma (n, \pi_2
    ((\gamma)^n (0, a)))) = \pi_1 (s (n), g (n, \pi_2 ((\gamma)^n (0, a)))) =
    s (n)$ and thus $s (n) \in B$
  \end{enumerate}
  Using induction (see \ref{mathematical induction}) we have then that $B
  =\mathbbm{N}$ and thus proved our statement.
  
  Define now $f : \mathbbm{N} \rightarrow A$ by $n \rightarrow f (n) = \pi_2
  ((\gamma)^n (0, a))$ then we have
  \begin{enumerate}
    \item $f (0) = \pi_2 ((\gamma)^0 (0, a)) = \pi_2 (i_{N \times A} (0, a)) =
    \pi_2 (0, a) = a$
    
    \item If $n \in \mathbbm{N}$ then $f (s (n)) = \pi_2 ((\gamma)^{s (n)} (0,
    a)) = \pi_2 (\gamma ((\gamma)^n (0, a)) \nobracket = \pi_2 (\gamma (\pi_1
    ((\gamma)^n (0, a)), \pi_2 ((\gamma)^n (0, a)))) = \pi_2 (\gamma (n, \pi_2
    ((\gamma)^n (0, a)))) = \pi_2 (s (n), g (n, \pi_2 ((\gamma)^n (0, a)))) =
    g (n, \pi_2 ((\gamma)^n (0, a))) = g (n, f (n))$
  \end{enumerate}
  proving that $f$ is the function we search for . We are thus left with
  proving uniqueness. So assume that there exists a $f' : \mathbbm{N}
  \rightarrow A$ such that
  \begin{eqnarray*}
    f' (0) & = & a\\
    f' (s (n)) & = & g (n, f' (n))
  \end{eqnarray*}
  and define then $C = \{ n \in \mathbbm{N}|f (n) = f' (n) \}$ then we have
  \begin{enumerate}
    \item $f (0) = a = f' (0)$ and thus $0 \in C$
    
    \item If $n \in \mathbbm{N}$ then $f (s (n)) = g (n, f (n)) = g (n, f'
    (n)) = f' (s (n))$ and thus $s (n) \in C$
  \end{enumerate}
  Using mathematical induction (see \ref{mathematical induction}) we have that
  $C =\mathbbm{N}$ and thus $f = f'$
\end{proof}

\

\section{Arithmetic of the natural numbers}

\begin{definition}[Addition]
  \label{addition of natural number}{\index{addition of natural numbers}}Let
  $m, n \in \mathbbm{N}$ then we define $+ : \mathbbm{N} \times \mathbbm{N}
  \rightarrow \mathbbm{N}$ by $(n, m) \rightarrow n \upl m = (s)^m (n)$ (here
  $s$ is the successor function)
\end{definition}

\begin{example}
  If $n \in \mathbbm{N}$ then we have
  \begin{eqnarray*}
    n \upl 0 & = & (s)^0 (n) = i_{\mathbbm{N}} (n) = n\\
    n \upl 1 & = & (s)^1 (n) = (s)^{s (0)} (n) = (s \circ (s)^0) (n) = s
    (i_{\mathbbm{N}} (n)) = s (n)\\
    n \upl 2 & = & (s)^2 (n) = (s)^{s (1)} (n) = (s \circ (s)^1) (n) = s (s
    (n))\\
    & \ldots & 
  \end{eqnarray*}
\end{example}

Using the definition of $+$ we immediately have

\begin{theorem}
  \label{neutral element of naturals}{\index{neutral element of natural
  elements}}If $n \in \mathbbm{N}$ then $n \upl 0 = 0 \upl n = n$
\end{theorem}

\begin{proof}
  
  \begin{enumerate}
    \item $n \upl 0 = (s)^0 (n) = 1_{\mathbbm{N}} (n) = n$
    
    \item We prove this by mathematical induction, define $S = \{ n \in
    \mathbbm{N} | 0 \upl n = n \nobracket \} \subseteq \mathbbm{N}$ then we
    have
    \begin{enumerate}
      \item $0 \upl 0 \equallim_{(1)} 0 \Rightarrow 0 \in S$
      
      \item If $n \in S \Rightarrow 0 \upl n = n$ now for $s (n)$ we have $0
      \upl s (n) = (s)^{s (n)} (0) = (s \circ s^{(n)}) (0) = s (s^{(n)} (0)) =
      s (0 \upl n) = s (n) \Rightarrow s (n) \in S$
    \end{enumerate}
  \end{enumerate}
  Using mathematical induction \ref{mathematical induction} we have $S
  =\mathbbm{N}$ and thus if $n \in \mathbbm{N} \Rightarrow n \in S \Rightarrow
  0 \upl n = n$
\end{proof}

\begin{theorem}
  \label{successor function and addition}If $n \in \mathbbm{N}$ then $s (n) =
  n \upl 1 = 1 \upl n$
\end{theorem}

\begin{proof}
  
  \begin{enumerate}
    \item $n \upl 1 = (s)^1 (n) = (s)^{s (0)} (n) = (s \circ (s)^0) (n) = (s
    \circ i_{\mathbbm{N}}) (n) = s (n)$
    
    \item $1 \upl n = s (n)$ is proved by induction, so define $S = \{ n \in
    \mathbbm{N} | 1 \upl n = s (n) \nobracket \}$ then we have
    \begin{enumerate}
      \item $1 \upl 0 \equallim_{\text{\ref{neutral element of naturals}}} 1 =
      s (0) \Rightarrow 0 \in S$
      
      \item If $n \in S \Rightarrow 1 \upl n = s (n)$. Now $1 \upl s (n) =
      (s)^{s (n)} (1) = (s \circ (s)^n) (1) = s ((s)^n (1)) = s (1 \upl n) = s
      (s (n))$ and thus $s (n) \in S$
    \end{enumerate}
  \end{enumerate}
  from \ref{mathematical induction} we have then that $S =\mathbbm{N}
  \Rightarrow n \in \mathbbm{N} \Rightarrow n \in S \Rightarrow 1 \upl n = s
  (n)$
\end{proof}

\begin{lemma}
  \label{m+s(n)=s(m+n)}If $n, m \in \mathbbm{N}$ then $n \upl s (m) = s (n
  \upl m)$
\end{lemma}

\begin{proof}
  $n \upl s (m) = (s)^{s (m)} (n) = (s \circ (s)^m) (n) = s ((s)^m (n)) = s (n
  \upl m)$
\end{proof}

\begin{theorem}[Associativity]
  \label{addition of natural numbers is associative}If $n, m, k \in
  \mathbbm{N}$ then $(n \upl m) \upl k = n \upl (m \upl k)$
\end{theorem}

\begin{proof}
  The proof is by mathematical induction given $n, m \in \mathbbm{N}$ define
  $S_{n, m} = \{ k \in \mathbbm{N} | (n \upl m) \upl k = n \upl (m \upl k)
  \nobracket \}$ then we have
  \begin{enumerate}
    \item $(n \upl m) \upl 0 \equallim_{\text{\ref{neutral element of
    naturals}}} n \upl m \equallim_{\text{\ref{neutral element of naturals}}}
    n \upl (m \upl 0) \Rightarrow 0 \in S_{n, m}$
    
    \item If $k \in S_{n, m} \Rightarrow (n \upl m) \upl k = n \upl (m \upl
    k)$. We have then $(n \upl m) \upl s (k)
    \equallim_{\text{\ref{m+s(n)=s(m+n)}}} s ((n \upl m) \upl k) = s (n \upl
    (m \upl k)) \equallim_{\text{\ref{m+s(n)=s(m+n)}}} n \upl s (m \upl k)
    \equallim_{\text{\ref{m+s(n)=s(m+n)}}} n \upl (m \upl s (k))$ and thus $s
    (k) \in S_{n, m}$
  \end{enumerate}
  So if $n, m, k \in \mathbbm{N}$ then $k \in S_{n, m} \Rightarrow (n \upl m)
  \upl k = n \upl (m \upl k)$
\end{proof}

\begin{theorem}[Commutativity]
  \label{natural numbers are commutative}If $n, m \in \mathbbm{N}$ then $n
  \upl m = m \upl n$
\end{theorem}

\begin{proof}
  Again we prove this by induction. So let $n \in \mathbbm{N}$ define then
  $S_n = \{ m \in \mathbbm{N} | n \upl m = m \upl n \nobracket \}$ then we
  have
  \begin{enumerate}
    \item $n \upl 0 = n = 0 \upl n$ (see \ref{neutral element of naturals})
    and thus $0 \in S_n$
    
    \item If $m \in S_n \Rightarrow n \upl m = m \upl n$. We have then $n \upl
    s (m) \equallim_{\text{\ref{m+s(n)=s(m+n)}}} s (n \upl m) = s (m \upl n)
    \equallim_{\text{\ref{successor function and addition}}} 1 \upl (m \upl n)
    \equallim_{\text{\ref{addition of natural numbers is associative}}} (1
    \upl n) \upl m \equallim_{\text{\ref{successor function and addition}}} s
    (n) \upl m$ so we have that $s (m) \in S_n$
  \end{enumerate}
  Using \ref{mathematical induction} we have then that $S_n =\mathbbm{N}$. So
  if $n, m \in \mathbbm{N} \Rightarrow m \in S_n \Rightarrow n \upl m = m \upl
  n$
\end{proof}

\begin{corollary}
  \label{N,+ is a abelian semi-group}$\langle \mathbbm{N}, + \rangle$ forms a
  abelian semi-group
\end{corollary}

\begin{proof}
  This follows from \ref{addition of natural numbers is associative},
  \ref{neutral element of naturals} and \ref{natural numbers are commutative}
\end{proof}

\begin{definition}[Multiplication]
  \label{multiplaction of natural numbers}{\index{multiplication of natural
  numbers}}Given $n \in \mathbbm{N}$ define $\alpha_n : \mathbbm{N}
  \rightarrow \mathbbm{N}$ defined by $m \rightarrow \alpha_n (m) = n \upl m$.
  Then we define $\cdot : \mathbbm{N} \times \mathbbm{N} \rightarrow
  \mathbbm{N}$ by $(n, m) \rightarrow n \cdot m = (\alpha_n)^m (0)$
\end{definition}

\begin{example}
  We have the following examples to see how multiplication works by repeating
  summation
  \begin{eqnarray*}
    2 \cdot 0 & = & (\alpha_2)^0 (0) = i_{\mathbbm{N}} (0) = 0\\
    2 \cdot 1 & = & (\alpha_2)^1 (0) = (\alpha_2)^{s (0)} (0) = (\alpha_2
    \circ (\alpha_2)^0) (0) = \alpha_2 (0) = 2 \upl 0 = 2\\
    2 \cdot 2 & = & (\alpha_2)^2 (0) = (\alpha_2)^{s (1)} (0) = (\alpha_2
    ((\alpha_2)^1 (0))) = \alpha_2 (2) = 2 \upl 2 = 4
  \end{eqnarray*}
\end{example}

\begin{theorem}[Absorbing Element]
  \label{absorbing element of natural numbers}If $n \in \mathbbm{N}$ then $n
  \cdot 0 = 0 = 0 \cdot n$
\end{theorem}

\begin{proof}
  
  \begin{enumerate}
    \item $n \cdot 0 = (\alpha_n)^0 (0) = 1_{\mathbbm{N}} (0) = 0$
    
    \item We prove by induction that $0 \cdot n = 0$ so define $S = \{ n \in
    \mathbbm{N} | 0 \cdot n = 0 \nobracket \} \subseteq \mathbbm{N}$ then we
    have
    \begin{enumerate}
      \item $0 \cdot 0 \equallim_{(1)} 0 \Rightarrow 0 \in S$
      
      \item If $n \in S$ then $0 \cdot n = 0$ Now $0 \cdot s (n) =
      (\alpha_0)^{s (n)} (0) = (\alpha_0 \circ (\alpha_0)^n) (0) = \alpha_0
      ((\alpha_0)^n (0)) = \alpha_0 (0 \cdot n) = \alpha_0 (0) = 0 \upl 0
      \equallim_{\text{\ref{neutral element of naturals}}} 0$ and thus $s (n)
      \in S$
    \end{enumerate}
    proving by \ref{mathematical induction} that $S =\mathbbm{N}$
  \end{enumerate}
\end{proof}

\begin{theorem}[Neutral Element]
  \label{neutral element for mulitiplication of natural numbers}If $n \in
  \mathbbm{N}$ then $n \cdot 1 = n = 1 \cdot n$
\end{theorem}

\begin{proof}
  
  \begin{enumerate}
    \item $n \cdot 1 = (\alpha_n)^1 (0) = (\alpha_n^{})^{s (0)} (0) = \alpha_n
    ((\alpha_n)^0 (0)) = \alpha_n (0) = n \upl 0 \equallim_{\text{\ref{neutral
    element of naturals}}} n$
    
    \item We prove by induction that $1 \cdot n = n$ so define $S = \{ n \in
    \mathbbm{N} | 1 \cdot n = n \nobracket \}$ then we have
    \begin{enumerate}
      \item $1 \cdot 0 \equallim_{\text{\ref{absorbing element of natural
      numbers}}} 0 \Rightarrow 0 \in S$
      
      \item If $n \in S \Rightarrow 1 \cdot n = n$ Now $1 \cdot s (n) =
      (\alpha_1)^{s (n)} (0) = \alpha_1 ((\alpha_1)^n (0)) = \alpha_1 (1 \cdot
      n) = \alpha_1 (n) = 1 \upl n \equallim_{\text{\ref{successor function
      and addition}}} s (n)$ so $s (n) \in S$
    \end{enumerate}
    Using \ref{mathematical induction} we have then $S =\mathbbm{N}$
  \end{enumerate}
\end{proof}

\begin{theorem}
  \label{successor and multiplication}If $n, m \in \mathbbm{N}$ then $n \cdot
  s (m) = n \upl n \cdot m \equallim_{\text{\ref{natural numbers are
  commutative}}} n \cdot m \upl n$
\end{theorem}

\begin{proof}
  $n \cdot s (m) = (\alpha_n)^{s (m)} (0) = \alpha_n ((\alpha_n)^m (0)) =
  \alpha_n (n \cdot m) = n \upl n \cdot m$
\end{proof}

\begin{theorem}[Distributivity]
  \label{distributivity in natural numbers}$\forall n, m, k \in \mathbbm{N}$
  we have $(n \upl m) \cdot k = n \cdot k + m \cdot k$
\end{theorem}

\begin{proof}
  If $n, m \in \mathbbm{N}$ define $S_{n, m} = \{ k \in \mathbbm{N}| (n \upl
  m) \cdot k = n \cdot k \upl m \cdot k \}$ then we have
  \begin{enumerate}
    \item $(n \upl m) \cdot 0 \equallim_{\text{\ref{absorbing element of
    natural numbers}}} 0 \equallim_{\text{\ref{neutral element of naturals}}}
    0 \upl 0 \equallim_{\text{\ref{absorbing element of natural numbers}}} n
    \cdot 0 + m \cdot 0 \Rightarrow 0 \in S_{n, m}$
    
    \item Assume that $k \in S_{n, m}$ then $(n \upl m) \cdot k = n \cdot k
    \upl m \cdot k$. Now $(n \upl m) \cdot s (k)
    \equallim_{\text{\ref{successor and multiplication}}} (n \upl m) \cdot k
    \upl (n \upl m) = (n \cdot k + m \cdot k) \upl (n \upl m)
    \equallim_{\text{\ref{addition of natural numbers is associative}}} (n
    \cdot k + n) + (m \cdot k + m) \equallim_{\text{\ref{successor and
    multiplication}}} n \cdot s (k) + m \cdot s (k) \Rightarrow s (k) \in
    S_{n, m}$
  \end{enumerate}
  If now $n, m, k \in \mathbbm{N} \Rightarrow k \in S_{n, m} \Rightarrow (n
  \upl m) \cdot k = n \cdot k + m \cdot k$
\end{proof}

\begin{theorem}[Commutativity]
  \label{multiplication of natural numbers is commutative}If $n, m \in
  \mathbbm{N}$ then $n \cdot m = m \cdot n$
\end{theorem}

\begin{proof}
  We prove this by induction, let $S_n = \{ m \in \mathbbm{N} | n \cdot m = m
  \cdot n \nobracket \}$ then we have
  \begin{enumerate}
    \item Using \ref{absorbing element of natural numbers} we have $n \cdot 0
    = 0 = 0 \cdot n \Rightarrow 0 \in S_n$
    
    \item If $m \in S_n \Rightarrow n \cdot m = m \cdot n$. Now $n \cdot s (m)
    = n + n \cdot m \equallim_{m \in S_n} n \upl m \cdot n
    \equallim_{\text{\ref{neutral element for mulitiplication of natural
    numbers}}} 1 \cdot n \upl m \cdot n \equallim_{\text{\ref{distributivity
    in natural numbers}}} (1 \upl m) \cdot n \equallim_{\text{\ref{successor
    function and addition}}} s (m) \cdot n \Rightarrow s (m) \in S_n$
  \end{enumerate}
  This proves by \ref{mathematical induction} that $S_n =\mathbbm{N}$ and thus
  if $n, m \in \mathbbm{N} \Rightarrow m \in S_n \Rightarrow n \cdot m = m
  \cdot n$
\end{proof}

\begin{theorem}[Associativity]
  \label{multiplication of natural numbers is associative}If $n, m, k \in
  \mathbbm{N}$ then $(n \cdot m) \cdot k = n \cdot (m \cdot k)$
\end{theorem}

\begin{proof}
  Given $n, m \in \mathbbm{N}$ define $S_{n, m} = \{ k \in \mathbbm{N} | (n
  \cdot m) \cdot k = n \cdot (m \cdot k) \nobracket \}$ then we have
  \begin{enumerate}
    \item $(n \cdot m) \cdot 0 \equallim_{\text{\ref{absorbing element of
    natural numbers}}} 0 \equallim_{\text{\ref{absorbing element of natural
    numbers}}} n \cdot 0 \equallim_{\text{\ref{absorbing element of natural
    numbers}}} n \cdot (m \cdot 0) \Rightarrow 0 \in S_{n, m}$
    
    \item If $k \in S_{n, m}$ then $(n \cdot m) \cdot k = n \cdot (m \cdot
    k)$. Now $(n \cdot m) \cdot s (k) \equallim_{\text{\ref{successor and
    multiplication}}} (n \cdot m) \cdot k + n \cdot m = n \cdot (m \cdot k) +
    n \cdot m \equallim_{\text{\ref{multiplication of natural numbers is
    commutative}}} (m \cdot k) \cdot n + m \cdot n
    \equallim_{\text{\ref{distributivity in natural numbers}}} (m \cdot k + m)
    \cdot n \equallim_{\text{\ref{successor and multiplication}}} (m \cdot s
    (k)) \cdot n \equallim_{\text{\ref{multiplication of natural numbers is
    commutative}}} n \cdot (m \cdot s (k)) \Rightarrow s (k) \in S_{n, m}$
  \end{enumerate}
  Using mathematical induction \ref{mathematical induction} we have $S_{n, m}
  =\mathbbm{N}$ so if $n, m, k \in \mathbbm{N} \Rightarrow k \in S_{n, m}
  \Rightarrow (n \cdot m) \cdot k = n \cdot (m \cdot k)$
\end{proof}

\begin{corollary}
  \label{N,.is a semi-group}$\langle \mathbbm{N}, . \rangle$ is a abelian
  semi-group
\end{corollary}

\begin{proof}
  This follows from \ref{multiplication of natural numbers is commutative},
  \ref{multiplication of natural numbers is associative} and \ref{neutral
  element for mulitiplication of natural numbers}.
\end{proof}

\begin{theorem}
  \label{n+k=m+k=<gtr>n=m}If $n, m, k \in \mathbbm{N}$ is such that $n \upl k
  = m \upl k \Rightarrow n = m$, (and thus we have $n = m \Leftrightarrow n
  \upl k = m \upl k$)
\end{theorem}

\begin{proof}
  We prove this by induction so let $S_{n.m} = \{ k \in \mathbbm{N} | n \upl k
  = m \upl k \Rightarrow n = m \nobracket \}$ then we have
  \begin{enumerate}
    \item $n \upl 0 = m \upl 0 \Rightarrowlim_{\text{\ref{neutral element of
    naturals}}} n = m \Rightarrow 0 \in S_{n, m}$
    
    \item If $k \in S_{n, m} \Rightarrow n \upl k = m \upl k \Rightarrow n =
    m$. For $s (k)$ if we have $n \upl s (k) = m + s (k)
    \Rightarrowlim_{\text{\ref{m+s(n)=s(m+n)}}} s (n \upl k) = s (m \upl k)
    \Rightarrowlim_{\text{\ref{if successors are equal numbers are equal}}} n
    \upl k = m \upl k \Rightarrow n = m \Rightarrow s (k) \in S_{n, m}$
  \end{enumerate}
  So if $n, m, k \in \mathbbm{N} \Rightarrow k \in S_{n, m} \Rightarrow n \upl
  k = m \upl k \Rightarrow n = m$
\end{proof}

\section{Order relation on the natural numbers}

\begin{definition}
  Define the relation $\leqslant \in \mathbbm{N} \times \mathbbm{N}$ by
  $\leqslant = \{ (n, m) \in \mathbbm{N} \times \mathbbm{N} | n \in m \vee n =
  m \nobracket \}$ so we say that $n \leqslant m$ iff $n \in m \vee n = m$. If
  $n \leqslant m$ and $n \neq m$ then we say that $n < m$.
\end{definition}

\begin{theorem}
  $\label{natural numbers are partially ordered} \langle \mathbbm{N},
  \leqslant \rangle$ is a partially ordered set
\end{theorem}

\begin{proof}[reflectivity][anti-symmetry][transitivity]
  
  \begin{enumerate}
    \item If $m \in \mathbbm{N} \Rightarrow m = m \Rightarrow m \leqslant m$
    
    \item If $n, m \in \mathbbm{N}$ and $n \leqslant m \wedge m \leqslant n
    \Rightarrow (n \in m \vee n = m) \wedge (m \in n \vee n = m) \Rightarrow
    (n \in m \wedge m \in n) \vee (n = m)$ so we have the following cases
    \begin{enumerate}
      \item $n = m \Rightarrow$anti-symmetry is proved
      
      \item $n \in m \wedge m \in n$ then by \ref{every natural number is
      transitive} $n \subseteq m \wedge m \subseteq n \Rightarrow n = m$
      proving again anti-symmetry
    \end{enumerate}
    \item If $n \leqslant m \wedge m \leqslant k$ then we have the following
    possibilities
    \begin{enumerate}
      \item $n \in m \wedge m \in k$ then by \ref{every natural number is
      transitive} $n \subseteq m \wedge m \subseteq k \Rightarrow n \subseteq
      k \Rightarrow n \leqslant k$
      
      \item $n \in m \wedge m = k$ then by \ref{every natural number is
      transitive} $n \subseteq m \wedge m = k \Rightarrow n \subseteq k
      \Rightarrow n \leqslant k$
      
      \item $n = m \wedge m \in k$ then by \ref{every natural number is
      transitive} $n = m \wedge m \subseteq k \Rightarrow n \subseteq k
      \Rightarrow n \leqslant k$
      
      \item $n = m \wedge m = k$ then $n = k \Rightarrow n \leqslant k$
    \end{enumerate}
    So in all cases we have $n \leqslant k$ proving transitivity.
  \end{enumerate}
\end{proof}

\begin{theorem}
  \label{every natural number is bigger or equal to zero}If $n \in \mathbbm{N}
  \Rightarrow 0 \leqslant n$
\end{theorem}

\begin{proof}
  We prove this by mathematical induction, so let $S = \{ n \in \mathbbm{N} |
  0 \leqslant n \nobracket \} \subseteq \mathbbm{N}$ then we have
  \begin{enumerate}
    \item $0 = 0 \Rightarrow 0 \leqslant 0 \Rightarrow 0 \in S$
    
    \item Suppose $n \in S$ then $0 \leqslant n$ now by definition $s (n) = n
    \bigcup \{ n \} \Rightarrow n \in s (n) \Rightarrow n \leqslant s (n)
    \equallim_{\text{\ref{natural numbers are partially ordered}} \wedge 0
    \leqslant n} 0 \leqslant s (n) \Rightarrow s (n) \in S$
  \end{enumerate}
  Using \ref{mathematical induction} we have then $S =\mathbbm{N}$ proving our
  theorem.
\end{proof}

\begin{theorem}
  \label{n<less>s(n)}If $n \in \mathbbm{N} \Rightarrow n < s (n)$
\end{theorem}

\begin{proof}
  From $n \in n \bigcup \{ n \} = s (n)$ we have $n \leqslant s (n)
  \Rightarrowlim_{\text{\ref{a natural number is different from its
  successor}}} n < s (n)$
\end{proof}

\begin{theorem}
  If $n \in \mathbbm{N}$ then \ $k \in n \Leftrightarrow k < n$
\end{theorem}

\begin{proof}[$k \in n \Rightarrow k < n$][$k \in n$][$k = n$][$k < n
\Rightarrow k \in n$]
  
  \begin{enumerate}
    \item We prove this by induction so let $S = \{ n \in \mathbbm{N}|
    \tmop{if} k \in n \Rightarrow k < n \}$ then
    \begin{enumerate}
      \item If $n = 0$ then $n = \emptyset \Rightarrow k \in n$ is impossible
      and thus '$k \in 0 \Rightarrow k < n$' is always true giving that $0 \in
      S$
      
      \item If $n \in S$ then if $k \in s (n) = n \bigcup \{ n \}$ we have
      \begin{enumerate}
        \item then as $n \in S$ we have $k < n$ so $k \neq n$. From $k \in n
        \subseteq n \bigcup \{ n \} = s (n)$ we have $k \leqslant s (n)$ . Now
        if $k = s (n) = n \bigcup \{ n \}$ then as $n \in \{ n \} \Rightarrow
        n \in n \bigcup \{ n \} \Rightarrow n \in k \Rightarrow n \leqslant k
        \Rightarrowlim_{k < n \Rightarrow k \leqslant n \tmop{and}
        \text{\ref{natural numbers are partially ordered}}} n = k$
        contradicting $k \neq n$ so that we have $k \neq s (n)$ and thus $k <
        s (n)$
        
        \item then we have $k = n < s (n) \Rightarrow k < s (n)$
      \end{enumerate}
      from (i) and (ii) it follows then that $s (n) \in S$
    \end{enumerate}
    Using mathematical induction we have then $S =\mathbbm{N}$ proving that if
    $n \in \mathbbm{N}= S$ then if $k \in n \Rightarrow k < n$
    
    \item If $k < n$ then $k \leqslant n$ and $k \neq n$ so $k \in n \vee k =
    n$ and $k \neq n$ giving $k \in n$
  \end{enumerate}
\end{proof}

\begin{theorem}
  If $k, n, m \in \mathbbm{N}$ then we have
  \begin{enumerate}
    \item $k \leqslant n$ and $n < m$ then $k < m$
    
    \item $k < n$ and $n \leqslant m$ then $k < m$
  \end{enumerate}
\end{theorem}

\begin{proof}[$k < n$][$k = n$][$n < m$][$n = m$]
  We have
  \begin{enumerate}
    \item If $k \leqslant n$ and $n < m$ then we have as $n < m \Rightarrow n
    \leqslant m$ that $k \leqslant m$. Now for $k$ we have \ the following
    possibilities:
    \begin{enumerate}
      \item then using the previous theorem we have $k \in n$ and $n \in m$.
      If now $k = m$ then we have $n \in k$. From $n \in k$ and $k \in n$ it
      follows then that $k \leqslant n$ and $n \leqslant k$ giving $k = n$
      contradicting $k < n$. So we must have $k \neq m$ and thus $k < m$.
      
      \item if now $k = m$ then $m = n$ contradicting $n < m$ so we must have
      $k \neq m$ and thus we have $k < m$.
    \end{enumerate}
    \item If $k < n$ and $n \leqslant m$ then we have $k \leqslant m$. Now for
    $n$ we have the following possibilities
    \begin{enumerate}
      \item Using the previous theorem we have $n \in m$. If now $k = m$ then
      $n \in k \Rightarrow n \leqslant k$ this with $k < n \Rightarrow k
      \leqslant n$ means $k = n$ contradicting $k < n$. So we have $k \neq m$
      and thus $k < m$.
      
      \item then from $k < n \Rightarrowlim_{\tmop{previous} \tmop{theorem}} k
      \in n$ we have $k \in m \Rightarrowlim_{\tmop{previous} \tmop{theorem}}
      k < m$
    \end{enumerate}
  \end{enumerate}
\end{proof}

\begin{theorem}
  \label{n<less>m=<gtr>s(n)<less>=m}For $n, m \in \mathbbm{N}$ we have $n < m
  \Rightarrow s (n) \leqslant m$
\end{theorem}

\begin{proof}
  We prove this by mathematical induction, so let $S_n = \{ m \in \mathbbm{N}
  | n < m \Rightarrow s (n) \leqslant m \nobracket \}$ we have then
  \begin{enumerate}
    \item As $n < 0$ means that $n \in 0 = \emptyset$ which is false so $n < 0
    \Rightarrow s (n) \leqslant 0$ is true or $0 \in S_n$
    
    \item If $m \in S_n$ then $n < m \Rightarrow s (n) \leqslant m$, for $s
    (n) < m$ we have then $s (n) \in m$ we have now to prove that $s (m) \in
    S_n$ or
    \begin{eqnarray*}
      n < s (m) & \Rightarrow & s (n) \leqslant s (m)
    \end{eqnarray*}
    Now as $n < s (m) \Rightarrow n \in s (m)$ by \ref{n element of successor}
    then we have either
    \begin{enumerate}
      \item $n \in m \Rightarrow n < m \Rightarrow s (n) \leqslant m
      \Rightarrowlim_{\text{\ref{natural numbers are partially ordered} and $m
      \in s (m) = m \bigcup \{ m \} \Rightarrow m \leqslant s (m)$}} s (n)
      \leqslant s (m)$
      
      \item $n = m \Rightarrow s (n) = s (m) \Rightarrow s (n) \leqslant s
      (m)$
    \end{enumerate}
    we conclude thus that $s (m) \in S_n$
  \end{enumerate}
  Using \ref{mathematical induction} we have then that $S_n =\mathbbm{N}$. So
  if $n, m \in \mathbbm{N} \Rightarrow m \in S_n \Rightarrow n < m \Rightarrow
  s (n) \leqslant m$
\end{proof}

\

\begin{theorem}
  $\label{the natural numbers are well-ordered} \langle \mathbbm{N}, \leqslant
  \rangle$ is well-ordered
\end{theorem}

\begin{proof}
  We prove this by contradiction, so let $\emptyset \neq A \subseteq
  \mathbbm{N}$ be a nonempty without a least element. Define then $S_A = \{ n
  \in \mathbbm{N} | \forall m \in A \tmop{we} \tmop{have} n \leqslant m
  \nobracket \}$ then as $A$ has no least element we must have $S_A \bigcap A
  = \emptyset$ [Otherwise $x \in S_A \bigcap A$ would mean that $x$ is a least
  element of $A$].
  
  We prove now by induction that $S_A =\mathbbm{N}$
  \begin{enumerate}
    \item By \ref{every natural number is bigger or equal to zero} we have
    $\forall m \in A \subseteq \mathbbm{N}$ that $0 \leqslant m \Rightarrow 0
    \in S_A$
    
    \item If $n \in S_A$ then $\forall m \in A \tmop{we} \tmop{have} n
    \leqslant m$. Now if $n \in A$ then $n$ is a least element of $A$ (which
    we assumed to be not the case) so we must have $\forall m \in A \vdash n <
    m$, using \ref{n<less>m=<gtr>s(n)<less>=m} we have then that $\forall m
    \in A s (n) \leqslant m$ and thus that $s (n) \in S_A$. 
  \end{enumerate}
  By \ref{mathematical induction} we have thus that $S_A =\mathbbm{N}$ but
  this means that $S_A \bigcap A =\mathbbm{N} \bigcap A = A \neq \emptyset$
  contradicting that $S_A \bigcap A = \emptyset$. So every nonempty subset of
  $\mathbbm{N}$ must have a least element.
\end{proof}

\begin{theorem}
  \label{The natural numbers are fully ordered}$\langle \mathbbm{N}, \leqslant
  \rangle$ is fully-ordered
\end{theorem}

\begin{proof}
  This follows from the well-orderings of $\langle N, \leqslant \rangle$ and
  \ref{well ordering implies fully ordering}.
\end{proof}

\begin{theorem}
  \label{the set of natural numbers is conditional complete}{\index{$\langle
  \mathbbm{N}. \leqslant \rangle \tmop{is} \tmop{conditional}
  \tmop{complete}$}}$\langle \mathbbm{N}, \leqslant \rangle$ is conditional
  complete. 
\end{theorem}

\begin{proof}
  This follows from the fact that $\langle \mathbbm{N}, \leqslant \rangle$ is
  well-ordered and \ref{well ordering implies conditional completeness}
\end{proof}

\begin{theorem}
  \label{n<less>m=<gtr>s(n)<less>s(m) n,m natural numbers}$\forall n, m \in
  \mathbbm{N}$ then $n < m \Leftrightarrow s (n) < s (m)$
\end{theorem}

\begin{proof}[$n < m$][$s (n) < s (m)$]
  
  \begin{enumerate}
    \item From \ref{n<less>m=<gtr>s(n)<less>=m} we have then $s (n) \leqslant
    m$ and as $m < s (m)$ we have thus $s (n) < s (m)$
    
    \item Assume that $m \leqslant n$ then we have either $m = n$ which leads
    to the contradiction $s (m) = s (n)$, or $m < n \Rightarrowlim_{(1)} s (m)
    < s (n) \Rightarrow s (m) < s (m)$ again a contradiction. So we must
    conclude that $n < m$.
  \end{enumerate}
\end{proof}

\begin{theorem}
  \label{n<less>=m<less>=<gtr>n+k<less>=m+k (natural numbers)}If $n, m, k \in
  \mathbbm{N}$ then $n < m \Leftrightarrow n \upl k < m \upl k$. Notice that
  from $n = m \Leftrightarrow n \upl k = m \upl k$ we have also the
  equivalence $n \leqslant m \Leftrightarrow n \upl k \leqslant m \upl k$
\end{theorem}

\begin{proof}
  We prove this by induction on $k$ so let $B = \{ k \in \mathbbm{N}|
  \tmop{if} n, m \in \mathbbm{N} \tmop{then} n \leqslant m \Leftrightarrow n
  \upl k \Leftrightarrow m \upl k \}$ then we have
  \begin{enumerate}
    \item If $k = 0$ then clearly we have for $n, m \in \mathbbm{N}$ that $n
    \leqslant m \Leftrightarrow n \upl 0 \leqslant m \upl 0 \Leftrightarrow n
    \upl k \Leftrightarrow m \upl k \Rightarrow 0 \in B$
    
    \item If $k \in B$ then we have $n \leqslant m \Leftrightarrow n \upl k
    \Leftrightarrow m \upl k \Leftrightarrowlim_{\tmop{previous}
    \tmop{theorem}} s (n \upl k) = s (m \upl k) \Leftrightarrowlim_{s (n \upl
    k) = n \upl s (k), s (m \upl k) = m \upl s (k)} n \upl s (k) = m \upl s
    (k)$ so we have $s (k) \in B$
  \end{enumerate}
  using mathematical induction we have $B =\mathbbm{N}$ proving our theorem.
\end{proof}

\begin{theorem}
  \label{n<less>s(m)=<gtr>n<less>=m}If $n, m \in \mathbbm{N}$ with $n < s (m)$
  then $n \leqslant m$
\end{theorem}

\begin{proof}
  If we would assume the opposite then $m < n$ and using
  (\ref{n<less>m=<gtr>s(n)<less>=m}) we have then $s (m) \leqslant n < s (m)$
  a contradiction. So we must have $n \leqslant m$
\end{proof}

\begin{theorem}
  \label{n<less>m=<gtr>exists a k such that n+k=m n,m natural nubers}If $n, m
  \in \mathbbm{N}$ and $n < m$ then there exists a $k \in \mathbbm{N}, k \neq
  0$ such that $n \upl k = m$
\end{theorem}

\begin{proof}
  Let $S_n = \{ m \in \mathbbm{N} | n < m \Rightarrow \exists k \in
  \mathbbm{N} | k \neq 0 \vdash n \upl k = m \}$ then we have
  \begin{enumerate}
    \item $n < 0$ is false as from \ref{every natural number is bigger or
    equal to zero} $0 \leqslant n \Rightarrow 0 < 0 \Rightarrow 0 \neq 0$ a
    contradiction so we have that $n < 0 \Rightarrow \exists k \in
    \mathbbm{N}, k \neq 0 \vdash n + k = m$ is true, so $0 \in S_n$
    
    \item If $m \in S_n$ suppose now that $n < s (m)$ then as $\mathbbm{N}$ is
    fully ordered (see \ref{The natural numbers are fully ordered}) we have
    the following cases
    \begin{enumerate}
      \item $m \leqslant n$ here we have the following cases
      \begin{enumerate}
        \item $m < n \Rightarrowlim_{\text{\ref{n<less>m=<gtr>s(n)<less>=m}}}
        s (m) \leqslant n \Rightarrow s (m) < s (m)$ a contradiction so this
        case does not apply.
        
        \item $m = n \Rightarrow s (n) = s (m)$ contradiction $n < s (m)$ so
        this case does not apply.
      \end{enumerate}
      \item $n \leqslant m$ here we have the following possibilities
      \begin{enumerate}
        \item $n = m$ then $n \upl 1 \equallim_{\ref{successor function and
        addition}} s (n) = s (m) \Rightarrowlim_{1 \neq 0} s (m) \in S_n$
        
        \item $n < m \Rightarrowlim_{m \in S_s} \exists k \in \mathbbm{N}, k
        \neq 0 \vdash n \upl k = m \Rightarrow n \upl s (k)
        \equallim_{\text{\ref{m+s(n)=s(m+n)}}} s (n + k) = s (m)$. As we have
        $k \in s (k) = k \bigcup \{ k \} \Rightarrow k \leqslant s (k)$ and $0
        < k \Rightarrow 0 < s (k) \Rightarrow 0 \neq s (k)$ we conclude that
        $s (m) \in S_n$
      \end{enumerate}
    \end{enumerate}
  \end{enumerate}
  Using mathematical induction \ref{mathematical induction} we have then that
  $S_n =\mathbbm{N}$. So if $n, m \in \mathbbm{N} \Rightarrow m \in S_n
  \Rightarrow$if $n < m \Rightarrow \exists k \in \mathbbm{N}, k \neq 0 \vdash
  n + k = m$.
\end{proof}

\begin{lemma}
  \label{n<less>n+k}If $n \in \mathbbm{N}$, $k \in \mathbbm{N}_0 \Rightarrow n
  < n \upl k$
\end{lemma}

\begin{proof}
  We prove this by mathematical induction, so let $k \in \mathbbm{N}_0$ and
  $S_k = \{ n \in \mathbbm{N} | n < n \upl k \nobracket \}$ then we have
  \begin{enumerate}
    \item $0 \upl k \equallim_{\text{\ref{neutral element of naturals}}} k$
    and as $k \neq 0$ we have $0 < k = 0 \upl k \Rightarrow 0 \in S_k$
    
    \item If $n \in S_k$ then by \ref{n<less>s(n)} we have $n \upl k < s (n
    \upl k) \equallim_{\text{\ref{m+s(n)=s(m+n)}}} s (n) \upl k$ also as $n
    \in S_k$ we have $n < n \upl k \Rightarrow s (n) \leqslant n \upl k
    \Rightarrow s (n) < s (n) \upl k \Rightarrow s (n) \in S_k$
  \end{enumerate}
  By \ref{mathematical induction} we have $S_k =\mathbbm{N}$. So if $k \in
  \mathbbm{N}_0$ and $n \in \mathbbm{N} \Rightarrow n \in S_k$ and thus $n < n
  \upl k$
\end{proof}

\begin{theorem}
  \label{n+k=0=<gtr>n=k=0 if n,k are natural numbers}If $n, k \in \mathbbm{N}$
  then $n \upl k = 0 \Rightarrow n = k = 0$
\end{theorem}

\begin{proof}
  Suppose that $k \neq 0$ then as $0 \leqslant n
  \Rightarrowlim_{\text{\ref{n<less>n+k}}} 0 \leqslant n < n \upl k = 0
  \Rightarrow 0 < 0 \Rightarrow 0 \neq 0$ and thus we reach a contradiction.
  So $k = 0$ but then $n = n \upl 0 = n \upl k = 0$ and thus also $n = 0$
\end{proof}

\begin{theorem}
  \label{n<less>m=<gtr>n+k=m}If $n, m \in \mathbbm{N} \text{then } n < m
  \Leftrightarrow \exists !k \in \mathbbm{N}_0 \vdash n \upl k = m$.
\end{theorem}

\begin{proof}
  
  
  $\Rightarrow$
  
  We prove this by mathematical induction, so let $S_n = \{ m \in \mathbbm{N}
  | \tmop{if} n < m \Rightarrow \exists k \in \mathbbm{N}_0 \nobracket \vdash
  n \upl k = m \}$ then we have
  \begin{enumerate}
    \item We have by \ref{every natural number is bigger or equal to zero}
    that $0 \leqslant n$ so if $n < 0 \Rightarrow 0 < 0$ a contradiction. So
    we can never have $n < 0 \Rightarrow 0 \in S_n$
    
    \item If $m \in S_n$ suppose then that $n < s (m)$ then we have the
    following possible cases for $m$
    \begin{enumerate}
      \item $n = m \Rightarrow s (m) = s (n) \equallim_{\text{\ref{successor
      function and addition}}} n \upl 1 \Rightarrowlim_{1 \neq 0} s (m) \in
      S_n$
      
      \item $n < m \Rightarrowlim_{m \in S_n} \exists k \in \mathbbm{N}_0
      \vdash n \upl k = m \Rightarrowlim_{\text{\ref{m+s(n)=s(m+n)}}} n \upl s
      (k) = s (n \upl k) = s (m)$ and as from $0 \leqslant k \Rightarrowlim_{k
      \neq 0} 0 < k$ and by \ref{n<less>s(n)} $k < s (k) \Rightarrow 0 < s (k)
      \Rightarrow s (k) \neq 0$ and as we just have proved that $n \upl s (k)
      = s (m)$ we have $s (m) \in S_n$
      
      \item $m < n \Rightarrowlim_{\text{\ref{n<less>m=<gtr>s(n)<less>=m}}} s
      (m) \leqslant n \Rightarrowlim_{n < s (m)} s (m) < s (m)$ is a
      contradiction so this case does not apply
    \end{enumerate}
    In all the cases that applies we had $s (m) \in S_n$
  \end{enumerate}
  Using \ref{mathematical induction} we have then $S_n =\mathbbm{N}$ so if $m,
  n \in \mathbbm{N} \Rightarrow m \in S_n \Rightarrow$if $n < m \Rightarrow
  \exists k \in \mathbbm{N}_0 \vdash n \upl k = m$. Finally to prove
  uniqueness assume that $n \upl k = m = n \upl k'$ then by
  \ref{n+k=m+k=<gtr>n=m} we have $k = k'$.
  
  $\Leftarrow$
  
  Suppose $\exists k \in \mathbbm{N}_0 \vdash n \upl k = m$ then for $n, m$ we
  have by \ref{The natural numbers are fully ordered} that either $n = m$, $m
  < n$ or $n < m$. Now if $n = m$ then as $k \in \mathbbm{N}_0$ we have by
  \ref{n<less>n+k} that $n < n \upl k = m = n$ a contradiction. Also if $m <
  n$ then again from \ref{n<less>n+k} we have $n < n \upl k = m \Rightarrow n
  < n$ again a contradiction. So the only possibility left is $n < m$.
  
  \ 
\end{proof}

\begin{corollary}
  \label{n<less>=m<less>=<gtr>n+k=m}If $n, m \in \mathbbm{N}$ then $n
  \leqslant m \Leftrightarrow \exists !k \in \mathbbm{N} \vdash n \upl k = m$.
\end{corollary}

\begin{proof}
  
  
  $\Rightarrow$ if $n \leqslant m$ then we have either
  \begin{enumerate}
    \item $n < m \Rightarrowlim_{\text{previous theorem}} \exists !k \in
    \mathbbm{N}_0 \vdash n \upl k = m \Rightarrow \exists k \in \mathbbm{N}
    \vdash n \upl k = m$
    
    \item $n = m \Rightarrow m = n \upl 0 \Rightarrow \exists k \in
    \mathbbm{N} (k = 0) \vdash n \upl k = m$
  \end{enumerate}
  Finally to prove uniqueness assume that $n \upl k = m = n \upl k'$ then by
  \ref{n+k=m+k=<gtr>n=m} we have $k = k'$.
  
  \
  
  $\Leftarrow$ If $\exists k \in \mathbbm{N} \vdash n \upl k = m$ then we
  have for $k$ the following possibilities
  \begin{enumerate}
    \item $k = 0 \Rightarrow n = m \Rightarrow n \leqslant m$
    
    \item $k \neq 0 \Rightarrow k \in \mathbbm{N}_0$ and using the previous
    theorem we have then $n < m \Rightarrow n \leqslant m$
  \end{enumerate}
\end{proof}

\begin{definition}
  \label{m-n if n<less>=m, n,m natural numbers}If $n, m \in \mathbbm{N},$ $n
  \leqslant m$ then the unique $k$ that by the previous theorem exists so that
  $n \upl k = m$ is noted by $m \um n$ so $n \upl (m \um n) = m$.
  
  \begin{note}
    If $n \in \mathbbm{N} \Rightarrow n \um n = 0$ (this is trivial because of
    uniqueness and the fact that $n \upl 0 = n$
  \end{note}
\end{definition}

\begin{theorem}
  \label{n<less>=i<less>=m=<gtr>0<less>=i-n<less>=m-n}If $n, m \in
  \mathbbm{N}$ and $i \in \mathbbm{N}$ such that $n \leqslant i \leqslant m$
  then $0 \leqslant i \um n \leqslant m \um n$
\end{theorem}

\begin{proof}
  First if $m \um n < i \um n \Rightarrow m = (m \um n) \upl n < (i - n) + n =
  i \Rightarrow m < i \leqslant m \Rightarrow m < m$ a contradiction so we
  have that $i \um n \leqslant m \um n$. By applying the above on $n \leqslant
  i$ we find that $0 = n \um n \leqslant i \um n \Rightarrow 0 \leqslant i \um
  n$
\end{proof}

\begin{theorem}
  \label{n<less>s(n)=<gtr>no natural number between n and s(n)}If $n \in
  \mathbbm{N}$ then there does not exists a $k \in \mathbbm{N}$ such that $n <
  k < s (n)$
\end{theorem}

\begin{proof}
  We prove this by mathematical induction, so take $S = \{ n \in \mathbbm{N} |
  \tmop{there} \tmop{does} \tmop{not} \tmop{exists} k \in \mathbbm{N}
  \tmop{such} \tmop{that} n < k < s (n) \nobracket \}$. We have then
  \begin{enumerate}
    \item If $0 < k < 1
    \Rightarrowlim_{\text{\ref{n<less>m=<gtr>s(n)<less>=m}}\text{}} s (0) = 1
    \leqslant s (k) \leqslant 1 \Rightarrow s (k) = 1 = s (0)
    \Rightarrowlim_{\text{\ref{if successors are equal numbers are equal}}} k
    = 0$ contradicting $0 < k$. So we conclude that $0 \in S$
    
    \item Let $n \in S$. Assume now that $\exists k \in \mathbbm{N}$ such that
    $s (n) < k < s (s (n))$ then from \ref{n<less>m=<gtr>n+k=m} \ we have
    $\exists l \in \mathbbm{N}_0$ such that $k = s (n) \upl l = s (n \upl l)$.
    Take then $k' = n \upl l$ then as $l \in \mathbbm{N}_0$ we have by
    \ref{n<less>m=<gtr>n+k=m} again that $n < k'$. From $s (k') = k < s (s
    (n)) \Rightarrow s (k') < s (s (n))
    \Rightarrowlim_{\text{\ref{n<less>m=<gtr>s(n)<less>s(m) n,m natural
    numbers}}} k' < s (n)$ so we have $n < k' < s (n)$ contradicting $n \in
    S$. So the assumption turns out to be wrong and thus $s (n) \in
    \mathbbm{N}$
  \end{enumerate}
  So by mathematical induction we have $S =\mathbbm{N}$ proving our theorem.
\end{proof}

\begin{theorem}
  \label{sup(A) is element of A in N}If $\emptyset \neq A \subseteq
  \mathbbm{N}$ is a set such that $\sup (A)$ exists then $\sup (A) \in A$
\end{theorem}

\begin{proof}[$\sup (A) = 0$][$\sup (A) \neq 0$]
  Suppose that $A \neq \emptyset$ and that $\sup (A)$ exists. We have now the
  following cases for $\sup (A)$
  \begin{enumerate}
    \item As $A \neq \emptyset$ there exists a $x \in A$ and as a sup is a
    upper bound we have $x \leqslant 0$ and from \ref{every natural number is
    bigger or equal to zero} we have $0 \leqslant x \Rightarrow x = 0
    \Rightarrow \sup (A) \in A$
    
    \item Using \ref{non zero element is a successor} we have $\exists n \in
    \mathbbm{N}$ such that $\sup (A) = s (n)$. Then as $n < s (n) = \sup (A)$
    there must exists a $x \in A$ with $n < x \leqslant \sup (A)$ (otherwise
    $n$ would be the lowest upper bound). If $x \neq \sup (A)$ then we have $n
    < x < s (n)$ which is forbidden by \ref{n<less>s(n)=<gtr>no natural number
    between n and s(n)} so we must have $x = \sup (A) \Rightarrow \sup (A) \in
    A$.
  \end{enumerate}
\end{proof}

\begin{theorem}
  \label{n<less>m and r<less>s=<gtr>n+r<less>m+s}If $n, m, r, s \in
  \mathbbm{N}$ then if $n < m \wedge r < s \Rightarrow n \upl r < m \upl s$
\end{theorem}

\begin{proof}
  If $n < m$ and $r < s$ then there exists $k, l \in \mathbbm{N}_0$ such that
  $m = n \upl k$ and $s = r \upl l$ so $m \upl s = (n \upl k) \upl (r \upl l)
  = (n \upl r) \upl (k \upl l)$. Now as $0 < k$ we have by \ref{n<less>n+k}
  that $0 < k \upl l \Rightarrow k \upl l \in \mathbbm{N}_0$. Using
  \ref{n<less>m=<gtr>n+k=m} we have then that $n \upl r < m \upl s$.
\end{proof}

\begin{lemma}
  \label{product of non zero natural numbers is non zero}If $n, m \in
  \mathbbm{N}_0$ then $n \cdot m \in \mathbbm{N}_0$
\end{lemma}

\begin{proof}
  We prove this by mathematical induction. If $n \in \mathbbm{N}_0$ then let
  $S_n = \{ m \in \mathbbm{N} | m \neq 0 \Rightarrow n \cdot m \neq 0
  \nobracket \}$ then we have
  \begin{enumerate}
    \item $0 \in S$ [$0 \neq 0$ is false so [$0 \neq 0 \Rightarrow n \cdot m
    \neq 0$] is true]
    
    \item If $m \in S_n$ then $n \cdot s (m) \equallim_{\text{\ref{successor
    and multiplication}}} n \cdot m \upl n$ as $m \in S_n$ we have $n \cdot m
    \in S_n \Rightarrow 0 < n \cdot m \Rightarrowlim_{\text{\ref{n<less>n+k}}
    \tmop{and} n \in S_n} 0 < n \cdot m < n \cdot m + n \Rightarrow 0 < n
    \cdot m \upl n = n \cdot s (m) \Rightarrow n \cdot s (m) \neq 0$
  \end{enumerate}
  Using \ref{mathematical induction} we have then that $S_n =\mathbbm{N}$ So
  if $n, m \in \mathbbm{N}_0 \Rightarrow 0 \neq m \in S_n \Rightarrow n \cdot
  m \neq 0$
\end{proof}

\begin{theorem}
  \label{n<less>m=<gtr>nk<less>mk}If $n, m, k \in \mathbbm{N}$ then if $k \neq
  0$ and $n < m \Rightarrow k \cdot n < k \cdot m$
\end{theorem}

\begin{proof}
  As $n < m \Rightarrowlim_{\text{\ref{n<less>m=<gtr>n+k=m}}} \exists l \in
  \mathbbm{N}_0$ such that $n \upl l = m \Rightarrow k \cdot (n \upl l) = k
  \cdot m \Leftrightarrow k \cdot n \upl k \cdot l = k \cdot m$. As we have $0
  \leqslant n$ and $n < m \Rightarrow 0 < m \Rightarrow m \in \mathbbm{N}_0
  \Rightarrowlim_{\text{} \ref{product of non zero natural numbers is non
  zero}} k \cdot l \in \mathbbm{N}_0
  \Rightarrowlim_{\text{\ref{n<less>m=<gtr>n+k=m}}} k \cdot n < k \cdot m$
\end{proof}

\begin{corollary}
  \label{n<less>=m=<gtr>n.k<less>=m.k}If $n, m, k \in \mathbbm{N}$ then if $n
  \leqslant m \Rightarrow n \cdot k \leqslant m \cdot k$
\end{corollary}

\begin{proof}
  If $n \leqslant m$ then we have the following cases
  \begin{enumerate}
    \item $n = m \Rightarrow n \cdot k = m \cdot k \Rightarrow n \cdot k
    \leqslant m \cdot k$
    
    \item $n < m$ then we have for $k$ the following cases
    \begin{enumerate}
      \item $k = 0 \Rightarrowlim_{\text{\ref{absorbing element of natural
      numbers}}} n \cdot k = 0 = m \cdot k \Rightarrow n \cdot k \leqslant m
      \cdot k$
      
      \item $k \neq 0 \Rightarrowlim_{\tmop{previous} \tmop{theorem}} n \cdot
      k < m \cdot k \Rightarrow n \cdot k \leqslant m \cdot k$
    \end{enumerate}
  \end{enumerate}
\end{proof}

\begin{theorem}
  \label{elemination of non zero common factor in natural numbers}If $n, m \in
  \mathbbm{N}, k \in \mathbbm{N}_0$ and $n \cdot k = m \cdot k \Rightarrow n =
  m$
\end{theorem}

\begin{proof}
  We have by \ref{The natural numbers are fully ordered} that $n < m$, $m < n$
  or $n = m$. Now if $n < m
  \Rightarrowlim_{\text{\ref{n<less>m=<gtr>nk<less>mk}}} n \cdot k < m \cdot
  k$ a contradiction, also if $m < n \Rightarrow m \cdot k < n \cdot k$ a
  contradiction. So the only thing left is that $n = m$.
\end{proof}

\begin{theorem}[Archimedean property of $\mathbbm{N}$]
  \label{archimedean property of natural numbers}{\index{Archimedean property
  of $\mathbbm{N}$}}If $x, y \in \mathbbm{N}$ and $x > 0$ then there exists a
  $z \in \mathbbm{N}$ such that $z \cdot x > y$
\end{theorem}

\begin{proof}
  We have the following possibilities for $y$
  \begin{enumerate}
    \item $y \leqslant x$ then as $1 < 2 = s (1)$ (see \ref{n<less>s(n)}) we
    have by \ref{n<less>m=<gtr>nk<less>mk} that $y \leqslant x = x \cdot 1 < 2
    \cdot x$ so taking $z = 2$ we have $z \cdot z > y$
    
    \item $x < y$ then by \ref{n<less>m=<gtr>n+k=m} there exists a $k \in
    \mathbbm{N}_0$ such that $y = x \upl k$ now as $0 < x$ we have by
    \ref{n<less>m=<gtr>s(n)<less>=m} that $1 = s (0) \leqslant x \Rightarrow 1
    \leqslant x$, also we have that $x \upl k < s (x \upl k)$ \ This gives
    using \ref{n<less>=m=<gtr>n.k<less>=m.k} $s (x \upl k) \cdot 1 \leqslant s
    (x \upl k) \cdot x \Rightarrow y = x \upl k < s (x \upl k) \leqslant s (x
    \upl k) \cdot x \Rightarrow y < s (x \upl k) \cdot y$, so taking $z = s (x
    \upl k)$ we have $z \cdot x > y$
  \end{enumerate}
\end{proof}

\begin{theorem}[Division Algorithm]
  \label{division algorithm for natural numbers}If $m, n \in \mathbbm{N}$ and
  $n > 0$ then there exists a unique $r \in \mathbbm{N}$ with $0 \leqslant r <
  n$ and a unique $q \in \mathbbm{N}$ such that $m = n \cdot q \upl r$
\end{theorem}

\begin{proof}[$m = 0$][$0 < m$][$n = 1$][$1 < n$][$q < q''$][$q'' < q$]
  First we prove the existence of $r, q \in \mathbbm{N}$. For $m$ we have the
  following cases to consider
  \begin{enumerate}
    \item In this case we can take $r = 0$, $q = 0$ and then $0 = r < n$ and
    $n \cdot q \upl r = n \cdot 0 \upl 0 = 0 = m$
    
    \item We have then the following cases for $n$
    \begin{enumerate}
      \item Then we can take $q = m, r = 0$ and then $0 = r < n$ and $n \cdot
      q \upl r = 1 \cdot m \upl 0 = m$
      
      \item Then from \ref{n<less>m=<gtr>nk<less>mk} we have $m < n \cdot m$
      so $m \in A_{n, m} = \{ x \in \mathbbm{N}|m < n \cdot x \wedge x
      \leqslant m \} \Rightarrow A_{n, m} \neq \emptyset$. Using well-ordering
      (see \ref{the natural numbers are well-ordered}) there exists a $q' =
      \min (A_{n, m})$. We have $q' \neq 0$ [because if $q' = 0$ then from $q'
      \in A_{n, m}$ we have $0 < m < n \cdot 0 = 0 \Rightarrow 0 < 0$ a
      contradiction] and thus there exists a $q \in \mathbbm{N}$ such that $q'
      = s (q) = q \upl 1$. From $q < q' \leqslant m$ we have by the definition
      of the minimum that $q \nin A_{n, m}$ and thus $n \cdot q \leqslant m$
      so we can define $r = m \um n \cdot q$ where $r \in \mathbbm{N}$ and
      thus \tmtextbf{$\tmmathbf{m = n \cdot q \upl r} $}. If now $n \leqslant
      r \Rightarrow n \upl n \cdot q \leqslant r \upl n \cdot q = m
      \Rightarrow n \cdot (q \upl 1) \leqslant m \Rightarrow n \cdot q'
      \leqslant m$ and from $q' \in A_{n, m}$ we have $m < n \cdot q'$ so we
      reach the contradiction $n \cdot q' \leqslant m < n \cdot q' \Rightarrow
      n \cdot q < n \cdot q'$ and we conclude that \tmtextbf{$0 \leqslant r <
      n$}.
    \end{enumerate}
  \end{enumerate}
  Now to prove uniqueness assume that $m = n \cdot q \upl r = n \cdot q'' \upl
  r''$ then we have if $q \neq q''$ that either the following is true
  \begin{enumerate}
    \item then $q \upl 1 \leqslant q'' \Rightarrow n \cdot (q \upl 1)
    \leqslant n \cdot q'' \Rightarrow n \cdot q \upl n \leqslant n \cdot q''
    \Rightarrow n \cdot q \upl n \upl r \upl r'' \leqslant n \cdot q'' \upl r
    \upl r'' \Rightarrow m \upl n \upl r'' \leqslant m \upl r \Rightarrow n
    \upl r'' \leqslant r \Rightarrow n \leqslant r$ contradicting $r < n$
    
    \item then $q'' \upl 1 < q \Rightarrow n \cdot (q'' \upl 1) \leqslant n
    \cdot q \Rightarrow n \cdot q'' \upl n \leqslant n \cdot q \Rightarrow n
    \cdot q'' \upl n \upl r \upl r'' \leqslant n \cdot q \upl r \upl r''
    \Rightarrow m \upl n \upl r \leqslant m \upl r'' \Rightarrow n \upl r
    \leqslant r'' \Rightarrow n \leqslant r''$ contradicting $r'' < n$
  \end{enumerate}
  as neither of the above cases can be true we must have $q = q''$ but then $n
  \cdot q = n \cdot q'' \Rightarrowlim_{n \cdot q \upl r = n \cdot q'' \upl
  r''} r = r''$ proving uniqueness.
\end{proof}

\

\subsection{Other forms of Mathematical induction}

Once we have discussed the order relation on $\mathbbm{N}$ we can write down
some other forms of mathematical induction.

\begin{definition}
  {\index{$\{ n, \ldots \}$}}$\tmop{Given} n \in \mathbbm{N}$ we define $\{ n,
  \ldots \}$ to be equal to $\{ i \in \mathbbm{N}|n \leqslant i \}$
\end{definition}

\begin{definition}
  Given $n, m \in \mathbbm{N}$ with $n \leqslant m$ we define $\{ n, \ldots, m
  \} = \{ i \in \mathbbm{N}|n \leqslant i \wedge i \leqslant m \}$
\end{definition}

We use this in the following theorem

\begin{theorem}
  \label{mathematical induction form 2}If $n \in \mathbbm{N}$ and $X \subseteq
  \{ n, \ldots \}$ is such that
  \begin{enumerate}
    \item $n \in X$
    
    \item if $i \in X \Rightarrow i \upl 1 \in X$
  \end{enumerate}
  then $X = \{ n, \ldots \}$
\end{theorem}

\begin{proof}
  Take $B = \{ i \in \mathbbm{N}|i \upl n \in X \}$ then we have
  \begin{enumerate}
    \item As $n \in X$ we have $0 \upl n \in X \Rightarrow 0 \in B$
    
    \item If $i \in B$ then $i \upl n \in X \Rightarrow (i \upl 1) \upl n = (i
    \upl n) \upl 1 \in X \Rightarrow i \upl 1 \in B$
  \end{enumerate}
  by mathematical induction (see \ref{mathematical induction}) we have then
  $B =\mathbbm{N}$. If now $i \in \{ n, \ldots \}$ then $n \leqslant i
  \Rightarrow i \um n \in \mathbbm{N}= B \Rightarrow (i \um n) \upl n \in X
  \Rightarrow i \in X \Rightarrow \{ n, \ldots \} \subseteq X \subseteq \{ n,
  \ldots \} \Rightarrow X = \{ n, \ldots \}$
\end{proof}

\begin{theorem}
  \label{mathematical induction form 3}If $n \in \mathbbm{N}$ and $P (i)$ is a
  predicate defined $\forall i \in \{ n, \ldots \}$ depending on $i$ is such
  that
  \begin{enumerate}
    \item $P (n)$ is true
    
    \item If $n \leqslant i$ and $P (i)$ is true then $P (i \upl 1)$ is true
  \end{enumerate}
  then $\forall i \in \{ n, \ldots \}$ we have that $P (i)$ is true
\end{theorem}

\begin{proof}
  Define $X = \{ i \in \{ n, \ldots \} |P (i) \tmop{is} \tmop{true} \}$ then
  we have
  \begin{enumerate}
    \item $n \in X$
    
    \item if $i \in X \Rightarrow i \upl 1 \in X$
  \end{enumerate}
  thus $X = \{ n, \ldots \}$ and if $i \in \{ n, \ldots \} \Rightarrow i \in X
  \Rightarrow P (i)$ is true
\end{proof}

The last form is the form of induction used in most mathematical text. In this
text however we use mostly the form \ref{mathematical induction form 2} as I
think this forces you to express the induction hypotheses in a much clearer
form.

\chapter{Finite and Infinite Sets}

\section{Introduction}

\begin{definition}
  \label{equipotent}{\index{equipotent}}Two classes $A$ and $B$ are called
  equipotent if there exists a bijection $f : A \rightarrow B$, we note this
  as $A \approx B$
\end{definition}

\begin{definition}
  Given two classes $A$ and $B$ then $A \preccurlyeq B$ if there is a $C
  \subseteq B$ such that $A \approx C$
\end{definition}

\begin{definition}
  Given two classes $A$ and $B$ then $A \prec B$ if $A \preccurlyeq B$ and
  $\neg (A \approx B)$
\end{definition}

\begin{theorem}
  \label{AB and injection}Given two classes $A$ and $B$ then $A \preccurlyeq
  B$ if and only if there exists a injection $f : A \rightarrow C$ where $C
  \subseteq B$
\end{theorem}

\begin{proof}
  
  \begin{eqnarray*}
    A \preccurlyeq B & \Leftrightarrow & \tmop{there} \tmop{exists} a C
    \subseteq B \tmop{and} a \tmop{bijection} f : A \rightarrow C\\
    & \Rightarrow & f : A \rightarrow B \tmop{is} a \tmop{injection}\\
    f : A \rightarrow B \tmop{is} a \tmop{injection} & \Rightarrow & f : A
    \rightarrow f (A) \tmop{is} a \tmop{bijection}\\
    & \Rightarrowlim_{f (A) \subseteq B} & A \preccurlyeq B
  \end{eqnarray*}
  
\end{proof}

\begin{theorem}
  \label{there is no surjection between A and P(A)}If $A$ is a set then there
  exists no surjective function between $A$ and $\mathcal{P} (A)$
\end{theorem}

\begin{proof}
  We prove this by contradiction so assume that there is a surjective function
  $f : A \rightarrow \mathcal{P} (A)$ define then $B = \{ x \in A|x \nin f (x)
  \}$. As $B \subseteq A$ we have that $B \in \mathcal{P} (A)$ and by
  surjectivity of $f$ there exists a $y \in A$ such that $B = f (y)$. Now if
  $y \in B$ then $y \nin f (y) = B \Rightarrow y \nin B$ a contradiction. If
  $y \nin B$ then as $B = f (y)$ we have $y \nin f (y)$ and thus $y \in B$
  again a contradiction. So all the cases $y \in B$ or $y \nin B$ gives a
  contradiction and thus a surjective $f : A \rightarrow \mathcal{P} (A)$ can
  not exist.
\end{proof}

\begin{corollary}
  If $A$ is a set then no subset of $A$ can be equipotent with $\mathcal{P}
  (A)$ (or with $2^A$)
\end{corollary}

\begin{proof}[$B = A$][$B \subset A$]
  First we prove that no subset of $A$ can be equipotent with $\mathcal{P}
  (A)$. If $B \subseteq A$ then we have the following possible cases:
  \begin{enumerate}
    \item then by \ref{there is no surjection between A and P(A)} we can not
    have a bijection (which is surjective) between $A$ and $\mathcal{P} (A)$,
    so $B$ can not be equipotent with $\mathcal{P} (A)$
    
    \item in this case $A \backslash B$ is non-empty and $B \bigcap (A
    \backslash B) = \emptyset$. Assume now that $B$ is equipotent with
    $\mathcal{P} (A)$ then a bijection $f' : B \rightarrow \mathcal{P} (A)$
    exists. Form then the function (see \ref{union definition of functions})
    $f : A \rightarrow \mathcal{P} (A)$ defined by $x \rightarrow f (x) =
    \left\{ \begin{array}{l}
      \emptyset \tmop{if} x \in A \backslash B\\
      f' (x) \tmop{if} x \in B
    \end{array} \right.$. If now $y \in \mathcal{P} (A)$ there exists by
    bijectivity of $f'$ a $x \in B$ such that $f' (x) = y \Rightarrow f (x) =
    y \Rightarrow f$ is surjective which is impossible by \ref{there is no
    surjection between A and P(A)}. So we must have that $B$ is not equipotent
    with $\mathcal{P} (A)$
  \end{enumerate}
  Next if $B \subseteq A$ and $B$ is equipotent with $2^A$ then there exists a
  bijection $f : B \rightarrow 2^A$. By \ref{P(A) and 2^A are bijective} $2^A$
  and $\mathcal{P} (A)$ are bijective so there exists a bijection $g : 2^A
  \rightarrow \mathcal{P} (A)$, but this means that $g \circ f : B \rightarrow
  \mathcal{P} (A)$ is bijective which we have just proved is impossible. So
  $B$ and $\mathcal{P} (A)$ cannot be equipotent. 
\end{proof}

\begin{corollary}
  If $A$ is a set then $A$ cannot be equipotent with any $B \supseteq
  \mathcal{P} (A)$ or with any $B \supseteq 2^A$
\end{corollary}

\begin{proof}
  Suppose that $A$ is equipotent with $B \supseteq \mathcal{P} (A)$ then there
  exists a bijection $f : A \rightarrow B$, using \ref{image restricted
  function} we have that $_{B|} f : f^{- 1} (\mathcal{P} (A)) \rightarrow
  \mathcal{P} (A)$ is a bijection which by the previous corollary is
  impossible.
  
  Suppose that $A$ is equipotent with $B \supseteq 2^A$ then there exists a
  bijection $f : A \rightarrow B$, using \ref{image restricted function} we
  have that $_{B|} f : f^{- 1} (2^A) \rightarrow 2^A$ is a bijection which by
  the previous corollary is impossible. 
\end{proof}

\begin{theorem}
  If $A, B$ are classes then there exists a injection $f : A \rightarrow B$ if
  and only if there exists a surjection $g : B \rightarrow A$
\end{theorem}

\begin{proof}
  
  
  If there exists a injection $f : A \rightarrow B$ then by \ref{injective
  function implies function in other directory} there exists a function $g : B
  \rightarrow A$ such that $g \circ f = i_A$. Then if $y \in A$ we have $y =
  i_A (y) = g (f (y)) \Rightarrow g$ is surjective.
  
  If there exists a surjection $g : B \rightarrow A$ then by \ref{surjection
  implies function in other directory} there exists a $f : A \rightarrow B$
  such that $g \circ f = i_A$. Given $x, x' \in A$ if $f (x) = f (x')
  \Rightarrow g (f (x)) = g (f (x')) \Rightarrow (g \circ f) (x) = (g \circ f)
  (x') = i_A (x) = i_A (x') \Rightarrow x = x'$ and this we have that $f$ is
  injective.
\end{proof}

Using the above theorem and \ref{AB and injection} we have the following
corollary

\begin{corollary}
  \label{A<less>~B and surjectivity}If $A, B$ are classes then $A \preccurlyeq
  B$ if and only if there exists a surjection $f : B \rightarrow A$
\end{corollary}

\begin{theorem}
  If $A, B, C, D$ are classes where $A \bigcap C = \emptyset$ and $B \bigcap
  D$ are sets and $f : A \rightarrow B$ and $g : C \rightarrow D$ are
  bijections then $g \bigcup f : A \bigcup C \rightarrow B \bigcup D$ is a
  bijection
\end{theorem}

\begin{proof}[$x \in B$][$x \in D$][$y \in A$][$y \in C$]
  As $f : A \rightarrow B$ and $g : C \rightarrow D$ are functions then
  clearly $f : A \rightarrow B \bigcup D$ and $g : C \rightarrow B \bigcup D$
  are also functions and thus by \ref{union definition of functions} $f
  \bigcup g : A \bigcup C \rightarrow B \bigcup D$ is a function. Now as $f,
  g$ are bijective we have the existence of the bijections $f^{- 1} : B
  \rightarrow A$, $g^{- 1} : D \rightarrow C$ meaning that $f^{- 1} : B
  \rightarrow A \bigcup C$ and $g^{- 1} : D \rightarrow A \bigcup C$ are
  functions and by \ref{union definition of functions} we have $f^{- 1}
  \bigcup g^{- 1} : B \bigcup D \rightarrow A \bigcup C$ is also a function.
  Now $\forall x \in B \bigcup D$ we have either
  \begin{enumerate}
    \item then $\left( f^{- 1} \bigcup g^{- 1} \right) (x) = f^{- 1} (x) \in A
    \Rightarrow \left( f \bigcup g \right) \left( \left( f^{- 1} \bigcup g^{-
    1} \right) (x) \right) = \left( f \bigcup g \right) (f^{- 1} (x)) = f
    (f^{- 1} (x)) = i_B (x) = x = i_{B \bigcup D} (x)$
    
    \item then $\left( f^{- 1} \bigcup g^{- 1} \right) (x) = g^{- 1} (x) \in C
    \Rightarrow \left( f \bigcup g \right) \left( \left( f^{- 1} \bigcup g^{-
    1} \right) (x) \right) = \left( f \bigcup g \right) (g^{- 1} (x)) = g
    (g^{- 1} (x)) = i_D (x) = x = i_{B \bigcup D} (x)$
  \end{enumerate}
  So we have $\left( f \bigcup g \right) \circ \left( f^{- 1} \bigcup g^{- 1}
  \right) = i_{B \bigcup D}$. Also $\forall y \in A \bigcup C$ we have either
  \begin{enumerate}
    \item then $\left( f \bigcup g \right) (y) = f (y) \in B$ so that $\left(
    f^{- 1} \bigcup g^{- 1} \right) \left( \left( f \bigcup g \right) (y)
    \right) = \left( f^{- 1} \bigcup g^{- 1} \right) (f (y)) = f^{- 1} (f (y))
    = i_A (y) = y = i_{A \bigcup C} (y)$
    
    \item then $\left( f \bigcup g \right) (y) = g (y) \in D$ so that $\left(
    f^{- 1} \bigcup g^{- 1} \right) \left( \left( f \bigcup g \right) (y)
    \right) = \left( f^{- 1} \bigcup g^{- 1} \right) (g (y)) = g^{- 1} (g (y))
    = i_D (y) = y = i_{A \bigcup C} (y)$
  \end{enumerate}
  so we have then $\left( f^{- 1} \bigcup g^{- 1} \right) \circ \left( f
  \bigcup g \right) = i_{A \bigcup C}$. This proves that $f \bigcup g : A
  \bigcup C \rightarrow B \bigcup D$ is a bijective function with inverse
  function $f^{- 1} \bigcup g^{- 1} : B \bigcup D \rightarrow A \bigcup C$
\end{proof}

\begin{corollary}
  \label{union and equipotency}If $A, B, C, D$ are classes with $A \bigcap C =
  \emptyset$, $B \bigcap D = \emptyset$ and $A \approx B$ and $C \approx D$
  then $\left( A \bigcup C \right) \approx \left( B \bigcup D \right)$
\end{corollary}

\begin{proof}
  This is ease as $A \approx B$, $C \approx D$ implies that there exists
  bijections $f : A \rightarrow B$ and $g : C \rightarrow D$. Using the
  previous theorem we haven the bijection $f \bigcup g : A \bigcup C
  \rightarrow B \bigcup D$ and thus that $\left( A \bigcup C \right) \approx
  \left( B \bigcup D \right)$
\end{proof}

\begin{theorem}
  \label{product of classes and equipotence}If $A, B, C, D$ are classes and $A
  \approx C$ and $B \approx D$ then $(A \times B) \approx (C \times D)$
\end{theorem}

\begin{proof}
  As $A \approx C$ and $B \approx D$ we have the existence of the bijections
  $f : A \rightarrow C$ and $g : B \rightarrow D$. Define $h : A \times B
  \rightarrow C \times D$ by $(x, y) \rightarrow (f (x), g (y))$ then we have
  \begin{enumerate}
    \item If $(x, y), (x', y') \in A \times B$ such that $h ((x, y)) = h ((x',
    y'))$ then $(f (x), g (y)) = (f (x'), g (y')) \Rightarrow f (x) = f (x')
    \wedge g (y) = g (y') \Rightarrow x = x' \wedge y = y' \Rightarrow (x, y)
    = (x', y')$ proving injectivity.
    
    \item If $(u, v) \in C \times D$ then $u \in C \wedge v \in D
    \Rightarrowlim_{f, g \tmop{are} \tmop{surjective}} \exists x \in A, y \in
    B \vdash u = f (x) \wedge v = g (y) \Rightarrow (u, v) = (f (x), g (y)) =
    h ((x, y))$
  \end{enumerate}
  This proves that $h : A \times B \rightarrow C \times D$ is bijective and
  thus $(A \times B) \approx (C \times D)$
\end{proof}

\begin{theorem}
  If $A, B, C, D$ are sets such that $A \approx B$ and $C \approx D$ then $A^C
  \approx B^D$
\end{theorem}

\begin{proof}
  As $A \approx B$ and $C \approx D$ there exists bijections $f : A
  \rightarrow B$ and $g : D \rightarrow C$. Define now $h : A^C \rightarrow
  B^D$ by $s \in A^C \rightarrow h (s) = f \circ s \circ g \in B^D$ [as $s : C
  \rightarrow A$ we have that $f \circ s \circ g : D \rightarrow B \Rightarrow
  f \circ s \circ g \in B^D$]. We have now
  \begin{enumerate}
    \item If $s, t \in A^C$ is such that $h (s) = h (t)$ then $f \circ s \circ
    g = f \circ t \circ g \Rightarrow f^{- 1} \circ (f \circ s \circ g) = f^{-
    1} \circ (f \circ t \circ g) \Rightarrow s \circ g = t \circ g \Rightarrow
    (s \circ g) \circ g^{- 1} = (t \circ g) \circ g^{- 1} = s = t$ proving
    injectivity.
    
    \item If $u \in B^D$ then $u : D \rightarrow B$ and thus $f^{- 1} \circ u
    \circ g^{- 1} : C \rightarrow A \Rightarrow f^{- 1} \circ u \circ g^{- 1}
    \in A^C$ and $h (f^{- 1} \circ u \circ g^{- 1}) = f \circ (f^{- 1} \circ u
    \circ g^{- 1}) \circ g = u$ proving surjectivity.
  \end{enumerate}
\end{proof}

As we have that $\mathcal{P} (A) \approx 2^A$ and $\mathcal{P} (B) \approx
2^B$ and clearly $2 \approx 2$ we have the following corollary from the above
theorem

\begin{theorem}
  If $A, B$ are sets such that $A \approx B$ then $\mathcal{P} (A) \approx
  \mathcal{P} (B)$ and $2^A \approx 2^B$
\end{theorem}

\section{Infinite and finite sets}

\

First every natural number is in reality a set this is expressed in the
following definition and theorem.

\begin{definition}
  \label{Sn}{\index{$S_n$}}If $n \in \mathbbm{N}$ then $S_n = \{ m \in
  \mathbbm{N}|m < n \}$ (this is a initial segment of $\mathbbm{N}$ (see
  \ref{initial segment}))
\end{definition}

\begin{theorem}
  \label{n=S_n}If $n \in \mathbbm{N}$ then $n = S_n$
\end{theorem}

\begin{proof}[$m < n$][$n < m$][$n = m$]
  Define $A = \{ n \in \mathbbm{N}|n = S_n \}$ then we have
  \begin{enumerate}
    \item If $x \in S_0 \Rightarrow 0 \leqslant x < 0 \Rightarrow 0 < 0$ a
    contradiction so we have $S_0 = \emptyset = 0 \Rightarrow 0 \in A$
    
    \item If $n \in A$ then $n = S_n$ if we take then $s (n) = s (S_n) = S_n
    \bigcup \{ S_n \} = S_n \bigcup \{ n \}$. Now if $m \in s (n) \Rightarrow
    m \in S_n$ or $m = n \Rightarrow m < n$ or $m = n < s (n) \Rightarrow m
    \in S_{s (n)}$. If $m \in S_{s (n)}$ then $m < s (n)$ we have then either
    \begin{enumerate}
      \item $\Rightarrow m \in S_n \Rightarrow m \in s (n)$
      
      \item $\Rightarrow s (n) \leqslant m < s (n) \Rightarrow s (n < s (n))$
      a contradiction, so this case can not occur.
      
      \item $\Rightarrow m \in \{ n \} \Rightarrow m \in S_n \bigcup \{ n \}
      \Rightarrow m \in s (n)$
    \end{enumerate}
    So we have proved that $s (n) = S_{s (n)}$ and thus that $s (n) \in A$
  \end{enumerate}
  By the principle of induction (see \ref{mathematical induction}) we have
  that $A =\mathbbm{N}$ and thus $\forall n \in \mathbbm{N}$ that $n = S_n$.
\end{proof}

\begin{theorem}[Recursion on $S_{n \upl 1}$ - Step Form][$s (i) \nin S_{n \upl
1}$][$s (i) \in S_{n \upl 1}$]
  \label{recursion on S_n+1 - step form}{\index{recursion on $S_{n \upl 1}$ -
  step form}}Let $A$ be a set, $a \in A$ and $g : S_n \times A \rightarrow A$
  a function. Then there exists a unique function $f : S_{n \upl 1}
  \rightarrow A$ satisfying
  \begin{eqnarray*}
    f (0) & = & a\\
    \forall i \in S_n \tmop{we} \tmop{have} f (i \upl 1) & = & g (i, f (i))
  \end{eqnarray*}
  \begin{proof}
    Define $g' : \mathbbm{N} \times A \rightarrow A$ by $(i, x) \rightarrow g'
    (i, x) = \left\{ \begin{array}{l}
      g (i, x) \tmop{if} i < n\\
      x \tmop{if} n \leqslant i
    \end{array} \right.$ then by \ref{Recursion step form} there exists a
    function $f' : \mathbbm{N} \rightarrow A$ such that
    \begin{eqnarray*}
      f' (0) & = & a\\
      \forall i \in \mathbbm{N} \tmop{we} \tmop{have} f' (s (i)) = f' (i \upl
      1) & = & g' (i, f' (i))
    \end{eqnarray*}
    Define now $f : S_{n \upl 1} \rightarrow A$ by $f = f'_{|S_{n \upl 1}}$
    then we have
    \begin{enumerate}
      \item $f (0) \equallim_{0 \in S_{n \upl 1}  (\tmop{as} 0 < n \upl 1)} f'
      (0) = a$
      
      \item $\forall i \in S_n$ we have $i + 1 \in S_{n \upl 1}$ and thus $f
      (i \upl 1) = f' (i \upl 1) = g' (i, f' (i)) \equallim_{i < n} g (i, f'
      (i)) \equallim_{i \in S_n \subseteq S_{n \upl 1}} g (i, f (i))$
    \end{enumerate}
    so we proved that $f$ is the sought for function. Now to prove that it is
    unique let $h : S_{n \upl 1} \rightarrow A$ be such that :
    \begin{enumerate}
      \item $h (0) = a$
      
      \item $\forall i \in S_n$ we have $h (i \upl 1) = g (i, h (i))$
    \end{enumerate}
    Take now $B = \{ i \in \mathbbm{N}| i \nin S_{n \upl 1} \vee (i \in S_{n
    \upl 1} \wedge h (i) = f (i)) \}$ then we have
    \begin{enumerate}
      \item $h (0) = a = f (0)$ and thus $0 \in B$
      
      \item If $i \in B$ then we have the following cases
      \begin{enumerate}
        \item $\Rightarrow s (i) \in B$
        
        \item then $i \upl 1 < n \upl 1 \Rightarrow i < n \Rightarrow i \in
        S_n$ and $h (s (i)) = h (i \upl 1) = g (i, h (i)) \equallim_{i \in B
        \wedge i \in S_n \subseteq S_{n \upl 1} \Rightarrow h (i) = f (i)} g
        (i, f (i)) = f (i \upl 1) = f (s (i))$ so $s (i) \in B$
      \end{enumerate}
    \end{enumerate}
    Using mathematical induction (see \ref{mathematical induction}) we have
    that $B =\mathbbm{N}$. Now if $i \in S_{n \upl 1} \Rightarrow i \in
    \mathbbm{N}= B \Rightarrowlim_{i \in S_{n + 1}} h (i) = f (i)$ and thus $h
    = f$
  \end{proof}
\end{theorem}

We can create a second form of the above theorem as follows

\begin{theorem}[Recursion on $\mathbbm{N}$ - Step Form]
  \label{recursion on N step form}Let $A$ be a set $a \in A$ and $g :
  \mathbbm{N} \times A \rightarrow A$ a function then there exists a function
  $f : \mathbbm{N} \rightarrow A$ such that
  \begin{eqnarray*}
    f (0) & = & a\\
    \forall i \in \mathbbm{N} \tmop{we} \tmop{have} f (i \upl 1) & = & g (i, f
    (i))
  \end{eqnarray*}
\end{theorem}

\begin{proof}
  This is trivial by using \ref{Recursion step form} and the fact that $s (n)
  = n + 1$ 
\end{proof}

\begin{corollary}
  \label{recursion on N step form (general)}Let $A$ be a set, $n \in
  \mathbbm{N}$, $a \in A$ and $g : \{ n \ldots . \} \times A \rightarrow A$ a
  function then there exists a function $f : \{ n, \ldots \} \rightarrow A$
  such that
  \begin{eqnarray*}
    f (n) & = & a\\
    \forall i \in \{ n, \ldots \} \tmop{we} \tmop{have} f (i + 1) & = & g (i,
    f (i))
  \end{eqnarray*}
\end{corollary}

\begin{proof}
  Take $g' : \mathbbm{N} \times A \rightarrow A$ by $(i, a) \rightarrow g (i +
  n, a)$ then by the previous theorem there exists a $f' : \mathbbm{N}
  \rightarrow A$ such that
  \begin{eqnarray*}
    f' (0) & = & a\\
    \forall i \in \mathbbm{N} \tmop{we} \tmop{have} f' (i + 1) & = & g' (i, f'
    (i))
  \end{eqnarray*}
  Take now $f : \{ n, \ldots \} \rightarrow A$ by $i \rightarrow f' (i - n)$
  then we have
  \begin{eqnarray*}
    f (n) & = & f' (n - n) = f' (0) = a\\
    \forall i \in \{ n, \ldots \} \tmop{we} \tmop{have} i - n \in \mathbbm{N}
    \tmop{and} f (i + 1) & = & f' ((i - n) + 1) = g' ((i - n), f' (i - n)) = g
    (i, f (i))
  \end{eqnarray*}
\end{proof}

\begin{example}
  \label{faculty}{\index{$n!$}}Define $g : \mathbbm{N} \times \mathbbm{N}
  \rightarrow \mathbbm{N}$ by $g (i, n) = (i \upl 1) \cdot n$ and take $a = 1$
  then using the above theorem there exists a function $f : \mathbbm{N}
  \rightarrow \mathbbm{N}$ such that
  \begin{eqnarray*}
    f (0) & = & 1\\
    \forall i \in \mathbbm{N} \tmop{we} \tmop{have} f (i \upl 1) & = & (i \upl
    1) \cdot f (i)
  \end{eqnarray*}
  We note $f (i)$ as $i!$ called the faculty and as some values we have
  \begin{eqnarray*}
    0! & = & 1\\
    1! & = & (1 \upl 0) !\\
    &  & (1 \upl 0) \cdot 0! = 1\\
    2! & = & (1 + 1) !\\
    & = & 2 \cdot 1! = 2\\
    3! & = & 3 \cdot 2! = 6\\
    & \ldots & 
  \end{eqnarray*}
\end{example}

The step form of recursion leads to the following corollary using finite
sequences and sequences.

\begin{corollary}
  \label{recursion sequence form}Let $A$ be a set, $a \in A$, $\mathcal{M}
  \subseteq \{ \{ x_i \}_{i \in \{ 0, \ldots, n \}} |n \in \mathbbm{N} \wedge
  x_i \in A \}$ [the set of finite families of elements in $A$ (or the set of
  function from $S_n \rightarrow A$, $n \in \mathbbm{N}_0$] and $\rho :
  \mathcal{M} \rightarrow A$ a function then there exists a function $f :
  \mathbbm{N} \rightarrow A$ such that
  \begin{enumerate}
    \item $f (0) = a$
    
    \item $\forall n \in \mathbbm{N}$ we have $f (s (n)) = \rho (\{ f (i)
    \}_{i \in \{ 0, \ldots, n \}})$
  \end{enumerate}
\end{corollary}

\begin{proof}[$i = n + 1$][$i < n + 1$]
  Let $\{ c_i \}_{i \in \{ 0, \ldots, 0 \}}$ be defined by $c_0 = a$ and
  define now $g : \mathcal{M} \rightarrow \mathcal{M}$ by $\{ x_i \}_{i \in \{
  0, \ldots, n \}} \rightarrow g (\{ x_i \}_{i \in \{ 0, \ldots, n \}})$ where
  $\forall i \in \{ 0, \ldots, n + 1 \}$ we have $g (\{ x_i \}_{i \in \{ 0,
  \ldots, n \}}) = \left\{ \begin{array}{l}
    x_i \tmop{if} 0 \leqslant i \leqslant n\\
    \rho (\{ x_i \}_{i \in \{ 0, \ldots, n \}}) \tmop{if} i = n + 1
  \end{array} \right.$using recursion (see \ref{recursion}) we have then the
  existence of a function $h : \mathbbm{N} \rightarrow \mathcal{M}$ where
  \begin{enumerate}
    \item $h (0) = \{ c_i \}_{i \in \{ 0, \ldots, 0 \}}$
    
    \item $\forall n \in \mathbbm{N}$ we have $h (n + 1) = g (h (n))$
  \end{enumerate}
  Define then $S = \{ n \in \mathbbm{N}| \tmop{dom} (h (n)) = \{ 0, \ldots, n
  \} \}$ then we have
  \begin{enumerate}
    \item If $n = 0$ then $\tmop{dom} (h (0)) = \tmop{dom} (\{ c_i \}_{i \in
    \{ 0, \ldots, 0 \}}) = \{ 0, \ldots 0 \}$
    
    \item If $n \in S$ then $\tmop{dom} (h (n + 1)) = \tmop{dom} (g (h (n)))
    \equallim_{\tmop{definition} \tmop{of} h} \{ 0, \ldots, \tmop{dom} (h (n))
    + 1 \} \equallim_{n \in S} \{ 0, \ldots n + 1 \}$
  \end{enumerate}
  Using mathematical induction we have thus that $S =\mathbbm{N} \Rightarrow$
  $\forall n \in \mathbbm{N}$ that $\tmop{dom} (h (n)) = \{ 0, \ldots, n \}$.
  Define now $F = \{ n \in \mathbbm{N}| \tmop{if} i \leqslant n \wedge j
  \leqslant i \tmop{then} h (n)_i = h (i)_j \}$ then we have
  \begin{enumerate}
    \item If $n = 0$ then from $i \leqslant n \wedge j \leqslant i$ we have $i
    = j = 0$ then $h (n)_j = h (0)_0 = h (i)_j$ or $0 \in F$
    
    \item If $n \in F$ then we have that $\tmop{dom} (h (n + 1)) = \{ 0,
    \ldots, n + 1 \}$ and if $i \leqslant n + 1 \wedge j \leqslant i$ we have
    the following cases
    \begin{enumerate}
      \item then $h (n + 1)_j = h (i)_j$
      
      \item then $i \leqslant n$ and if $j \leqslant i$ we have $h (n + 1)_j =
      g (h (n))_j \equallim_{j < n + 1 + \tmop{definition} \tmop{of} g} h
      (n)_j \equallim_{n \in F} h (i)_j$
    \end{enumerate}
    So in both cases we have $h (n + 1)_j = h (i)_j$ proving that $n + 1 \in
    F$
  \end{enumerate}
  Using mathematical induction we have then that $F =\mathbbm{N}$ or $\forall
  n \in \mathbbm{N}$ we have then if $i \leqslant n$ and $j \leqslant i$ then
  $h (n)_j = h (i)_j$. Define now $f : \mathbbm{N} \rightarrow A$ by $f (n) =
  h (n)_n$ then we have
  \begin{enumerate}
    \item $f (0) = h (0)_0 = c_0 = a$
    
    \item $f (n + 1) = h (n + 1)_{n + 1} = g (h (n))_{n + 1} = \rho (h (n))$,
    where if $k \in \tmop{dom} (h (n)) = \{ 0, \ldots, n \}$ we have that $h
    (n)_k \equallim_{k \leqslant n, k \leqslant k} h (k)_k = f (k)$ so that $h
    (n) = \{ f (i) \}_{i \in \{ 0, \ldots, n \}}$ giving that $f (n + 1) =
    \rho (\{ f_i \}_{i \in \{ 1, \ldots, n \}})$
  \end{enumerate}
\end{proof}

\begin{lemma}
  \label{translation of families}Let $A$ be a set $\mathcal{M}= \{ \{ x_i
  \}_{i \in \{ 0, \ldots, n \}} |n \in \mathbbm{N} \wedge \forall i \in \{ 0,
  \ldots, n \} \tmop{we} \tmop{have} x_i \in A \}$, $m \in \mathbbm{N}$ and
  $\mathcal{M}_m = \{ \{ x_i \}_{i \in \{ m, \ldots, n \}} |n \in \{ m, \ldots
  \} \wedge \forall i \in \{ m, \ldots, n \} \tmop{we} \tmop{have} x_i \in A
  \}$. Define then $T_m : \mathcal{M} \rightarrow \mathcal{M}_m$ by $\{ x_i
  \}_{i \in \{ 0, \ldots, n \}} \rightarrow T_m (\{ x_i \}_{i \in \{ 0,
  \ldots, m \}}) = \{ (T_m (\{ x_i \}_{i \in \{ 0, \ldots, n \}})_{})_j \}_{j
  \in \{ m, \ldots, n + m \}}$ where for $j \in \{ m, \ldots, n + m \}$ we
  have $(T_m (\{ x_i \}_{i \in \{ 0, \ldots, \}}))_j = x_{j - m}$ then we have
  \begin{enumerate}
    \item $T_m$ is a bijection
    
    \item The inverse of $T_m$ is $T_{- m} : \mathcal{M}_m \rightarrow
    \mathcal{M}$ defined by $\{ x_i \}_{i \in \{ m, \ldots, n \}} \rightarrow
    \{ (T_{- m} (\{ x_i \}_{i \in \{ m, \ldots, n \}}))_j \}_{j \in \{ 0,
    \ldots, n - m \}}$ where if $j \in \{ 0, \ldots, n - m \}$ then $(T_{- m}
    (\{ x_i \}_{i \in \{ m, \ldots, n \}}))_j = x_{j + m}$
  \end{enumerate}
\end{lemma}

As a shorthand we note $\{ T_m (\{ x_i \}_{i \in \{ 0, \ldots, n \}}) \}_{j
\in \{ m, \ldots, n + m \}} \equallim_{\tmop{notation}} \{ x_{i - m} \}_{i \in
\{ m, \ldots, n + m \}}$ and $\{ T_m (\{ x_i \}_{i \in \{ m, \ldots, n \}})
\}_{j \in \{ 0, \ldots, n - m \}} \equallim_{\tmop{notation}} \{ x_{i + m}
\}_{i \in \{ 0, \ldots, n - m \}}$.

\begin{proof}[injectivity][surjectivity]
  
  \begin{enumerate}
    \item If $\{ x_i \}_{i \in \{ 0, \ldots, n_1 \}}, \{ y_i \}_{i \in \{ 0,
    \ldots, n_2 \}} \in \mathcal{M}$ then if $T_m (\{ x_i \}_{i \in \{ 0,
    \ldots, n_1 \}}) = T_m (\{ y_i \}_{i \in \{ 0, \ldots, n_2 \}})$ we must
    have that $n_1 + m = n_2 + m \Rightarrow n_1 = n_2 = n$ (consider families
    as functions which must be equal so their domain must be equal) and
    $\forall i \in \{ 0, \ldots, n \}$ we have $i + m \in \{ m, \ldots, n + m
    \}$ so that $x_i = x_{(i + m) - m} = (T_m (\{ x_i \}_{i \in \{ 0, \ldots,
    n \}}))_{i + m} = (T_m (\{ y_i \}_{i \in \{ 0, \ldots, n \}}))_{i + m} =
    y_{(i + m) - m} = y_i$ proving that $\{ x_i \}_{i \in \{ 0, \ldots, n_1
    \}} = \{ y_i \}_{i \in \{ 0, \ldots, n_2 \}}$
    
    \item  Let $\{ x_i \}_{i \in \{ m, \ldots, n \}} \in \mathcal{M}_m$ define
    then $\{ y_i \}_{i \in \{ 0, \ldots, n - m \}}$ by $\forall i \in \{ 0,
    \ldots, n - m \}$ we have $y_i = x_{i + m}$ then $\forall i \in \{ m,
    \ldots, n \}$ we have $(T_m (\{ y_i \}_{i \in 0, \ldots, n - m}))_i = y_{i
    - m} = x_{(i + m) - m} = x_i$ proving that $T_m (\{ y_i \}_{i \in 0,
    \ldots, n - m}) = \{ x_i \}_{i \in m, \ldots, n}$
  \end{enumerate}
  Finally to prove that $T_{- m}$ is the inverse of $T_m$, take $\{ x_i \}_{i
  \in \{ 0, \ldots ., n \}}$ then $T_{- m} (T_m (\{ x_i \}_{i \in \{ 0,
  \ldots, n \}})) = T_{- m} (\{ (T_m (x_i)_{i \in \{ 0, \ldots, n \}})_j \}_{j
  \in \{ m, \ldots, m + n \}}) = \{ (T_{- m} (\{ (T_m (x_i)_{i \in \{ 0,
  \ldots, n \}})_j \}_{j \in \{ m, \ldots, m + n \}}))_k \}_{k \in \{ 0,
  \ldots, (m + n) - m \}} = \{ (T_{- m} (\{ x_{j - m} \}_{j \in \{ m, \ldots,
  m + n \}}))_k \}_{k \in \{ 0, \ldots, n \}} = \{ x_{(k - m) + m} \}_{k \in
  \{ 0, \ldots, n \}} = \{ x_k \}_{k \in \{ 0, \ldots, k \}}$
\end{proof}

\begin{corollary}
  \label{recursion sequence form (general)}Let $A$ be a set, $a \in A$, $m \in
  \mathbbm{N},$ $\mathcal{M}_m = \{ \{ x_i \}_{i \in \{ m, \ldots, n \}} |n
  \in \{ m, \ldots, \} \wedge x_i \in A \}$ [the set of finite families of
  elements in $A$ and $\rho : \mathcal{M}_m \rightarrow A$ a function then
  there exists a function $f : \{ m, \ldots \} \rightarrow A$ such that
  \begin{enumerate}
    \item $f (m) = a$
    
    \item $\forall n \in \{ m, \ldots \}$ we have $f (n + 1) = f (s (n)) =
    \rho (\{ f (i) \}_{i \in \{ m, \ldots, n \}})$
  \end{enumerate}
\end{corollary}

\begin{proof}
  If we define $\mathcal{M}= \{ \{ x_i \}_{i \in \{ 0, \ldots, n \}} |n \in
  \mathbbm{N} \wedge \forall i \in \{ 0, \ldots, n \} \tmop{we} \tmop{have}
  x_i \in A \}$ then using the previous lemma (see \ref{translation of
  families}) we have the bijections $T_m : \mathcal{M} \rightarrow
  \mathcal{M}_m$ and $T_{- m} : \mathcal{M}_m \rightarrow \mathcal{M}$, define
  now $\rho' : \mathcal{M} \rightarrow A$ by $\rho' = \rho \circ T_m$ then
  $\rho' : \mathcal{M} \rightarrow A$ and by \ref{recursion sequence form}
  there exists a $f' : \mathbbm{N} \rightarrow A$ such that
  \begin{enumerate}
    \item $f' (0) = a$
    
    \item $f' (n + 1) = \rho' (\{ f (i) \}_{i \in \{ 0, \ldots, n \}})$
  \end{enumerate}
  Define now $f : \{ m, \ldots \} \rightarrow A$ by $f (i) = f' (i - m)$ so
  that if $j \in \mathbbm{N}$ then $f' (j) = f (j + m)$ then we have
  \begin{enumerate}
    \item $f (m) = f' (m - m) = f' (0) = a$
    
    \item $\forall n \in \{ m, \ldots \}$ we have $f (n + 1) = f' ((n + 1) -
    m) = f' ((n - m) + 1) = \rho' (\{ f' (i)_{i \in \{ 0, \ldots, n - m \}}
    \}) = \rho (T_m (\{ f' (i) \}_{i \in \{ 0, \ldots, n - m \}})) = \rho (T_m
    (\{ f (i + m) \}_{i \in \{ 0, \ldots, n - m \}})_{}) = \rho (T_m (T_{- m}
    (\{ f (i) \}_{i \in \{ m, \ldots, n \}}))) = \rho (\{ f (i) \}_{i \in \{
    m, \ldots, n \}})$
  \end{enumerate}
\end{proof}

\begin{corollary}
  \label{recursion restricted sequence}Let $A$ be a set, $a \in A$, $m \in
  \mathbbm{N}$, $\mathcal{M}= \{ \{ x_i \}_{i \in \{ m, \ldots, n \}} |n \in
  \{ m, \ldots \} \wedge \forall i \in \{ m, \ldots, n \} \tmop{we}
  \tmop{have} x_i \in A \}$ [the set of finite families of elements in $A$]
  and $\rho : \mathcal{M} \rightarrow A$ a function so that $\forall \{ x_i
  \}_{i \in \{ m, \ldots, n \}} \in \mathcal{M}$ we have that $P (\{ x_i \}_{i
  \in \{ m, \ldots, n \}}, \rho (\{ x_i \}_{i \in \{ m, \ldots, n \}}))$ is
  true then there exists a function $f : \{ m, \ldots \} \rightarrow A$ such
  that
  \begin{enumerate}
    \item $f (m) = a$
    
    \item $\forall n \in \mathbbm{N}$ we have $f (s (n)) = \rho (\{ f (i)
    \}_{i \in \{ m, \ldots, n \}})$
    
    \item $\forall n \in \mathbbm{N}$ we have that $P (\{ f (i) \}_{i \in \{
    m, \ldots, n \}}, f (n + 1))$ is true
  \end{enumerate}
\end{corollary}

\begin{proof}
  First we use the previous corollary to prove that there exists a $f : \{ m,
  \ldots \} \rightarrow A$ such that
  \begin{enumerate}
    \item $f (m) = a$
    
    \item $\forall n \in \{ m, \ldots \}$ we have $f (s (n)) = \rho (\{ f (i)
    \}_{i \in \{ m, \ldots, n \}})$
  \end{enumerate}
  We proceed now by induction so let $S = \{ n \in \{ m, \ldots \} |P (\{ f
  (i) \}_{i \in \{ m, \ldots, n \}}, f (n + 1)) \tmop{is} \tmop{true} \}$ then
  we have
  \begin{enumerate}
    \item if $n = m$ then as $P (\{ f (i) \}_{i \in \{ m, \ldots, m \}}, f (m
    + 1)) = P (\{ f (i)_{i \in \{ m, \ldots, m \}} \}, \rho (\{ f (i) \}_{i
    \in \{ m, \ldots, m \}}))$ is true by the hypothesis so that $m \in S$
    
    \item If $n \in S$ then by assumption we have that $P (\{ f (i) \}_{i \in
    \{ m, \ldots, n \}}, \rho (\{ f (i) \}_{i \in \{ m, \ldots n \}})) = P (\{
    f (i) \}_{i \in \{ m, \ldots, n \}}, f (n + 1))$ \ so we have that $n + 1
    \in S$
  \end{enumerate}
  Which by mathematical induction proves the theorem.
\end{proof}

Another variant of recursion and families is the following

\begin{theorem}
  \label{recursion on restricted sequences}Let $A$ be a set and $\mathcal{M}=
  \{ \{ x_i \}_{i \in \{ 0, \ldots, n \}} | \forall i \in \{ 0, \ldots, n \}
  \tmop{we} \tmop{have} x_i \in A \}$, $\mathcal{N} \subseteq \mathcal{M}$ and
  $\rho : \mathcal{N} \rightarrow A$ such that for $\rho (\{ x_i \}_{i \in \{
  0, \ldots, n \}})$ we have $\rho (\{ x_i \}_{i \in 0, \ldots, n}) \in A$ and
  $\{ x'_i \}_{i \in \{ 0, \ldots, n + 1 \}} \in \mathcal{N}$ (where $x'_i =
  \left\{ \begin{array}{l}
    x_i \tmop{is} i \in \{ 0, \ldots, n \}\\
    \rho (\{ x_i \}_{i \in \{ 0, \ldots, n \}}) \tmop{if} i = n + 1
  \end{array} \right.$. Then for $a \in A$ with $\{ a \}_{i \in \{ 0, \ldots,
  0 \}} \in \mathcal{N}$ there exists a function $f : \mathbbm{N} \rightarrow
  A$ such that
  \begin{enumerate}
    \item $f (0) = a$
    
    \item $\forall n \in \mathbbm{N}$ we have $f (n + 1) = \rho (\{ f (i)
    \}_{i \in \{ 0, \ldots, n \}})$
    
    \item $\forall n \in \mathbbm{N}$ we have $\{ f (i) \}_{i \in \{ 0,
    \ldots, n \}} \in \mathcal{N}$
  \end{enumerate}
\end{theorem}

\begin{proof}[$i = n + 1$][$i < n + 1$]
  Take $\{ a \}_{i \in \{ 0, \ldots, 0 \}} \in \mathcal{N}$ (by the
  hypothesis) and define $g : \mathcal{N} \rightarrow \mathcal{N}$ by $\{ x_i
  \}_{i \in \{ 0, \ldots, n \}} \rightarrow g (\{ x_i \}_{i \in \{ 0, \ldots,
  n \}})$ where $\forall i \in \{ 0, \ldots, n + 1 \}$ we have $g (\{ x_i
  \}_{i \in \{ 0, \ldots, n \}}) = \left\{ \begin{array}{l}
    x_i \tmop{if} 0 \leqslant i \leqslant n\\
    \rho (\{ x_i \}_{i \in \{ 0, \ldots, n \}}) \tmop{if} i = n + 1
  \end{array} \right. \in \mathcal{N}$(by the hypothesis), using recursion
  (see \ref{recursion}) we have then the existence of a function $h :
  \mathbbm{N} \rightarrow \mathcal{N}$ where
  \begin{enumerate}
    \item $h (0) = \{ a \}_{i \in \{ 0, \ldots, 0 \}}$
    
    \item $\forall n \in \mathbbm{N}$ we have $h (n + 1) = g (h (n))$
  \end{enumerate}
  Define then $S = \{ n \in \mathbbm{N}| \tmop{dom} (h (n)) = \{ 0, \ldots, n
  \} \}$ then we have
  \begin{enumerate}
    \item If $n = 0$ then $\tmop{dom} (h (0)) = \tmop{dom} (\{ a \}_{i \in \{
    0, \ldots, 0 \}}) = \{ 0, \ldots 0 \}$
    
    \item If $n \in S$ then $\tmop{dom} (h (n + 1)) = \tmop{dom} (g (h (n)))
    \equallim_{\tmop{definition} \tmop{of} h} \{ 0, \ldots, \tmop{dom} (h (n))
    + 1 \} \equallim_{n \in S} \{ 0, \ldots n + 1 \}$
  \end{enumerate}
  Using mathematical induction we have thus that $S =\mathbbm{N} \Rightarrow$
  $\forall n \in \mathbbm{N}$ that $\tmop{dom} (h (n)) = \{ 0, \ldots, n \}$.
  Define now $F = \{ n \in \mathbbm{N}| \tmop{if} i \leqslant n \wedge j
  \leqslant i \tmop{then} h (n)_i = h (i)_j \}$ then we have
  \begin{enumerate}
    \item If $n = 0$ then from $i \leqslant n \wedge j \leqslant i$ we have $i
    = j = 0$ then $h (n)_j = h (0)_0 = h (i)_j$ or $0 \in F$
    
    \item If $n \in F$ then we have that $\tmop{dom} (h (n + 1)) = \{ 0,
    \ldots, n + 1 \}$ and if $i \leqslant n + 1 \wedge j \leqslant i$ we have
    the following cases
    \begin{enumerate}
      \item then $h (n + 1)_j = h (i)_j$
      
      \item then $i \leqslant n$ and if $j \leqslant i$ we have $h (n + 1)_j =
      g (h (n))_j \equallim_{j < n + 1 + \tmop{definition} \tmop{of} g} h
      (n)_j \equallim_{n \in F} h (i)_j$
    \end{enumerate}
    So in both cases we have $h (n + 1)_j = h (i)_j$ proving that $n + 1 \in
    F$
  \end{enumerate}
  Using mathematical induction we have then that $F =\mathbbm{N}$ or $\forall
  n \in \mathbbm{N}$ we have then if $i \leqslant n$ and $j \leqslant i$ then
  $h (n)_j = h (i)_j$. Define now $f : \mathbbm{N} \rightarrow A$ by $f (n) =
  h (n)_n$ then we have
  \begin{enumerate}
    \item $f (0) = h (0)_0 = c_0 = a$
    
    \item $f (n + 1) = h (n + 1)_{n + 1} = g (h (n))_{n + 1} = \rho (h (n))$,
    where if $k \in \tmop{dom} (h (n)) = \{ 0, \ldots, n \}$ we have that $h
    (n)_k \equallim_{k \leqslant n, k \leqslant k} h (k)_k = f (k)$ so that $h
    (n) = \{ f (i) \}_{i \in \{ 0, \ldots, n \}}$ giving that $f (n + 1) =
    \rho (\{ f_i \}_{i \in \{ 1, \ldots, n \}})$
  \end{enumerate}
  Let $\mathcal{W}= \{ n \in \mathbbm{N}| \{ f (i)_{i \in \{ 0, \ldots, n \}}
  \} \in \mathcal{N} \}$ then we have:
  \begin{enumerate}
    \item $\{ f (i)_{i \in \{ 0, \ldots, 0 \}} \} = \{ f (0) \}_{i \in \{ 0,
    \ldots, 0 \}} = \{ a \}_{i \in \{ 0, \ldots, 0 \}} \in \mathcal{N}$
    
    \item If $n \in \mathcal{W}$ then $\{ f (i) \}_{i \in \{ 0, \ldots, n + 1
    \}} = \rho (\{ f (i)_{i \in \{ 0, \ldots, n + 1 \}} \}) \in \mathcal{N}$
    as $\{ f (i) \}_{i \in \{ 0, \ldots, n \}} \in \mathcal{N} [n \in
    \mathcal{W}] $ and the hypothesis about $\rho$. 
  \end{enumerate}
  Applying mathematical induction on $\mathcal{W}$ gives then
  $\mathcal{W}=\mathbbm{N}$ and thus (3) in our theorem.
\end{proof}

\begin{corollary}
  \label{recursion with restricted set (general)}Let $A$ be a set and
  $\mathcal{M}_m = \{ \{ x_i \}_{i \in \{ m, \ldots, n \}} |n \in \{ m, \ldots
  \} \wedge \forall i \in \{ 0, \ldots, n \} \tmop{we} \tmop{have} x_i \in A
  \}$, $\mathcal{N}_m \subseteq \mathcal{M}_m$ and $\rho : \mathcal{N}_m
  \rightarrow A$ such that for $\rho (\{ x_i \}_{i \in \{ m, \ldots, n \}})$
  we have $\rho (\{ x_i \}_{i \in m, \ldots, n}) \in A$ and $\{ x'_i \}_{i \in
  \{ m, \ldots, n + 1 \}} \in \mathcal{N}_m$ (where $x'_i = \left\{
  \begin{array}{l}
    x_i \tmop{is} i \in \{ m, \ldots, n \}\\
    \rho (\{ x_i \}_{i \in \{ m, \ldots, n \}}) \tmop{if} i = n + 1
  \end{array} \right.$. Then for $a \in A$ with $\{ a \}_{i \in \{ m, \ldots,
  m \}} \in \mathcal{N}_m$ there exists a function $f : \mathbbm{N}
  \rightarrow A$ such that
  \begin{enumerate}
    \item $f (m) = a$
    
    \item $\forall n \in \{ m, \ldots \}$ we have $f (n + 1) = \rho (\{ f (i)
    \}_{i \in \{ m, \ldots, n \}})$
    
    \item $\forall n \in \{ m, \ldots \}$ we have $\{ f (i) \}_{i \in \{ m,
    \ldots, n \}} \in \mathcal{N}$
  \end{enumerate}
\end{corollary}

\begin{proof}
  If we define $\mathcal{M}= \{ \{ x_i \}_{i \in \{ 0, \ldots, n \}} |n \in
  \mathbbm{N} \wedge \forall i \in \{ 0, \ldots, n \} \tmop{we} \tmop{have}
  x_i \in A \}$ then using \ref{translation of families} we have the
  bijections $T_m : \mathcal{M} \rightarrow \mathcal{M}_m$ and $T_{- m} :
  \mathcal{M}_m \rightarrow \mathcal{M}$. Define then $\mathcal{N}= T_{- m}
  (\mathcal{N}_m)$ so that we have the bijections $T_{- m|\mathcal{N}_m} :
  \mathcal{N}_m \rightarrow \mathcal{N}_{}$ and $T_{m|\mathcal{N}} :
  \mathcal{N} \rightarrow \mathcal{N}_m$. Define then $\rho' : \mathcal{N}
  \rightarrow A$ as $\rho' = \rho \circ T_{m|\mathcal{N}}$ then if $\{ x_i
  \}_{i \in \{ 0, \ldots, n \}} \in \mathcal{N}$ we have $\rho' (\{ x_i \}_{i
  \in \{ 0, \ldots, n \}}) = \rho (T_m (\{ x_i \}_{i \in \{ 0, \ldots, n \}}))
  = \rho (\{ x_{i - m} \}_{i \in \{ m, \ldots, n + m \}})$. If we then create
  $\{ x'_i \}_{i \in \{ 0, \ldots, n + 1 \}} \in \mathcal{N}$ by $x'_i =
  \left\{ \begin{array}{l}
    x_i \tmop{if} i \in \{ 0, \ldots, n \}\\
    \rho' (\{ x_i \}_{i = n + 1})
  \end{array} \right. = \left\{ \begin{array}{l}
    x_i \tmop{if} i \in \{ 0, \ldots, n \}\\
    \rho (T_m (\{ x_i \}_{i \in \{ 0, \ldots, n \}}))
  \end{array} \right.$ then we have for $T_{m|\mathcal{N}} (\{ x'_i \}_{i \in
  \{ 0, \ldots, n + 1 \}})$ that if $k \in \{ m, \ldots, n + m + 1 \}$ \
  $(T_{m|\mathcal{N}_m} (\{ x'_i \}_{i \in \{ 0, \ldots, n + 1 \}}))_k = x'_{j
  - m} = \left\{ \begin{array}{l}
    T_m (\{ x_i \}_{i \in \{ 0, \ldots, n \}})_k \tmop{if} k \in \{ m, \ldots,
    n + m \}\\
    \rho (T_m (\{ x_i \}_{i \in \{ 0, \ldots, n \}})) \tmop{if} k = n + m + 1
  \end{array} \right.$so by the properties of $\rho$ we have that $T_m (\{
  x'_i \}_{i \in \{ 0, \ldots, n + 1 \}}) \in \mathcal{N}_m$ so that $\{ x'_n
  \}_{i \in \{ 0, \ldots, n + 1 \}} \in \mathcal{N}$. Also using the fact that
  $\{ a \}_{i \in \{ m, \ldots, m \}} \in \mathcal{N}_m$ we have that $\{ a
  \}_{i \in \{ 0, \ldots, 0 \}} = T_{- m} (\{ a \}_{i \in \{ 0, \ldots, 0 \}})
  \in \mathcal{N}$ Using the above theorem (see \ref{recursion on restricted
  sequences}) there exists then a $f' : \mathbbm{N} \rightarrow A$ such that
  \begin{enumerate}
    \item $f' (0) = a$
    
    \item $\forall n \in \mathbbm{N}$ we have $f' (n + 1) = \rho' (\{ f' (i)
    \}_{i \in \{ 0, \ldots, n \}})$
    
    \item $\forall n \in \mathbbm{N}$ we have $\{ f' (i) \}_{i \in \{ 0,
    \ldots, n \}} \in \mathcal{N}$
  \end{enumerate}
  Define then $f : \{ m, \ldots \} \rightarrow A$ by $i \rightarrow f (i) = f'
  (i - m)$ or if $i \in \mathbbm{N}$ then $f' (i) = f (i + m)$ then we have
  \begin{enumerate}
    \item $f (m) = f' (m - m) = f' (0) = a$
    
    \item $\forall n \in \{ m, \ldots \}$ we have $f (n + 1) = f' ((n - m) +
    1) = \rho' (\{ f' (i) \}_{i \in \{ 0, \ldots, (n - m) \}}) = \rho' (\{ f
    (i + m) \}_{i \in \{ 0, \ldots, (n - m) \}}) = \rho' (T_{- m} (\{ f_i
    \}_{i \in \{ m, \ldots, n \}})) = \rho (T_m (T_{- m} (\{ f_i \}_{i \in \{
    m, \ldots, n \}}))) = \rho (\{ f (i) \}_{i \in \{ m, \ldots, n \}})$
    
    \item $\forall n \in \{ m, \ldots \}$ we have $\{ f' (i) \}_{i \in \{ 0,
    \ldots n - m \}} \in \mathcal{N}$ so that $T_m (\{ f' (i) \}_{i \in \{ 0,
    \ldots, n - m \}}) \in T_m (T_{- m} (\mathcal{N}_m)) =\mathcal{N}_m$, as
    $T_m (\{ f' (i) \}_{i \in \{ 0, \ldots, n - m \}}) = \{ f' (i - m) \}_{i
    \in \{ m, \ldots, n \}} = \{ f (i) \}_{i \in \{ m, \ldots, n \}}$ this
    means that $\{ f (i) \}_{i \in \{ m, \ldots, n \}} \in \mathcal{N}$
  \end{enumerate}
\end{proof}

\begin{definition}
  \label{finite and infinite sets}{\index{finite sets}}{\index{infinite
  sets}}A set $A$ is called finite if $\exists n \in \mathbbm{N}$ such that $n
  \approx A$. A set that is not finite is called a infinite set.
\end{definition}

\begin{note}
  $\emptyset : \emptyset \rightarrow 0 = \emptyset$ is a bijection as all the
  conditions of a bijection are vacuously true, so the empty set is finite.
\end{note}

\begin{notation}
  {\index{$\{ a_0, \ldots, a_n \}$}}{\index{$\{ a_1, \ldots, a_n \}$}}If $A
  \neq \emptyset$ is a finite set then there exists a bijection $a : S_{n \upl
  1} \rightarrow A$ and we note then $A = \{ a_0, \ldots, a_n \}$ where $a_i =
  a (i)$ (note that sometimes if $n \in \mathbbm{N}_0$ we write $A = \{ b_1,
  \ldots, b_n \}$ where $b_i = a (i + 1)$ and $a : S_n \rightarrow A$ is a
  bijection) 
\end{notation}

\begin{definition}
  \label{countable sets}{\index{infinite countable set}}{\index{enumerable
  set}}A set $A$ is called \tmtextbf{denumerable} or \tmtextbf{infinite
  countable} if $A \approx \mathbbm{N}$
\end{definition}

\begin{definition}
  {\index{countable set}}A set $A$ is called \tmtextbf{countable} if it is
  either \tmtextbf{finite} or \tmtextbf{infinite countable}
\end{definition}

\begin{lemma}
  \label{countable set minus one element is countable}If $A$ is a
  \tmtextbf{denumerable} set and $x \in A$ then $A \backslash \{ x \}$ is a
  \tmtextbf{denumerable} set.
\end{lemma}

\begin{proof}[injectivity][$m < n \wedge m' < n$][$m < n \wedge n \leqslant
m'$][$n \leqslant m \wedge m' < n$][$n \leqslant m \wedge n \leqslant
m'$][surjectivity][$i < n$][$n < i$]
  If $A$ is countable then $A \approx \mathbbm{N}$ and thus there exists a
  bijection $f : \mathbbm{N} \rightarrow A$. By surjectivity of $f$ and $x \in
  A$ there exists a $n \in \mathbbm{N}$ such that $f (n) = x$. \ Define now $g
  : \mathbbm{N} \rightarrow A\backslash \{ x \}$ by $g (m) = \left\{
  \begin{array}{l}
    f (m) \tmop{if} m < n\\
    f (m \upl 1) \tmop{if} n \leqslant m
  \end{array} \right.$. We prove now that $g$ is bijective:
  \begin{enumerate}
    \item If $g (m) = g (m')$ then we have the following possible cases
    \begin{enumerate}
      \item then we have $f (m) = g (m) = g (m') = f (m') \Rightarrow f (m) =
      f (m') \Rightarrowlim_{f \tmop{is} \tmop{injective}} m = m'$
      
      \item then we have $f (m) = g (m) = g (m') = f (m' \upl 1) \Rightarrow f
      (m) = f (m' \upl 1) \Rightarrow m = m' \upl 1$. Then as $n \leqslant m'
      \Rightarrow n < n \upl 1 \leqslant m' \upl 1 = m$ we have from $m < n$
      that $m < m$ a contradiction, so this case does not apply.
      
      \item then we have $f (m \upl 1) = g (m) = g (m') = f (m') \Rightarrow f
      (m \upl 1) = f (m') \Rightarrow m \upl 1 = m' \Rightarrowlim_{n
      \leqslant m} n \upl 1 \leqslant m \upl 1 = m' \Rightarrowlim_{m' < n} n
      \upl 1 < n < n \upl 1 \Rightarrow n \upl 1 < n \upl 1$ again a
      contradiction, so we can ignore this case as it never occurs.
      
      \item then $f (m \upl 1) = g (m) = g (m') = f (m' \upl 1) \Rightarrow f
      (m \upl 1) = f (m' \upl 1) \Rightarrow m \upl 1 = m' \upl 1 \Rightarrow
      m = m'$
    \end{enumerate}
    \item If $y \in A \backslash \{ x \}$ then there exists by surjectivity of
    $g$ a $i \in \mathbbm{N}$ such that $f (i) = y$. We can not have that $i =
    n$ as then $y = f (i) = x \nin A \backslash \{ x \}$. So the only cases
    for $i$ are
    \begin{enumerate}
      \item then $g (i) = f (i) = y \Rightarrow g (i) = y$
      
      \item then $n \upl 1 \leqslant i \Rightarrow n \leqslant i \um 1
      \Rightarrow g (i \um 1) = f ((i \um 1) + 1) = f (i) = y \Rightarrow g (i
      \um 1) = y$
    \end{enumerate}
    proving surjectivity.
  \end{enumerate}
\end{proof}

\begin{lemma}
  \label{a natural number does not comtains a countable subset}If $n \in
  \mathbbm{N}$ then $n$ has no \tmtextbf{denumerable} subset
\end{lemma}

\begin{proof}
  We prove this by induction on $n$, so let $S = \{ n \in \mathbbm{N}|n
  \tmop{does} \tmop{not} \tmop{contain} a \tmop{enumerable} \tmop{subset} \}$.
  \begin{enumerate}
    \item First as $0 = \emptyset$ we have if $A \subseteq 0 = \emptyset
    \Rightarrow \emptyset \subseteq A \subseteq \emptyset \Rightarrow A =
    \emptyset$. If now $\mathbbm{N} \approx A$ then there exists a bijection
    $f : \mathbbm{N} \rightarrow A$ and thus $f (0) \in A \Rightarrow A \neq
    \emptyset$ a contradiction. So $0$ does not contains a denumerable subset
    and we have $0 \in S$.
    
    \item Assume now that $n \in S$. We proceed now by contradiction, so
    assume that there exists a $A \subseteq n \upl 1 = s (n) = n \bigcup \{ n
    \}$ such that $A$ is denumerable. If $n \nin A \Rightarrow A \subseteq n$
    which is impossible as $n \in S \Rightarrow n \in A$. Now if $x \in A
    \backslash \{ n \} \subseteq A \subseteq n \bigcup \{ n \}$ then $x \in n
    \bigcup \{ n \}$ and from $x \neq n \text{we have that } x \in n$ so we
    have $A \backslash \{ n \} \subseteq n$ and as by the previous lemma we
    would have that $A \backslash \{ n \}$ is denumerable, we reach the
    contradiction that $n \nin S$. So we can not have a countable subset of $n
    \upl 1$ and thus $n \upl 1 \in S$
  \end{enumerate}
  Using \ref{mathematical induction} we conclude that $S =\mathbbm{N}$ proving
  our theorem.
\end{proof}

\begin{theorem}
  \label{infinite sets have countable subset}If $A$ is a infinite set if and
  only $A$ has a denumerable subset. So any denumerable set must be infinite.
\end{theorem}

\begin{proof}[$\langle \mathbbm{N}, \leqslant \rangle \tmop{is}
\tmop{isomorphic} \tmop{with} \langle A, \leqslant_A \rangle$][$\langle
\mathbbm{N}, \leqslant \rangle \tmop{is} \tmop{isomorphic} \tmop{with}
\tmop{an} \tmop{initial} \tmop{segment} \tmop{of} \langle A, \leqslant_A
\rangle$][$\langle A, \leqslant_A \rangle \tmop{is} \tmop{isomorphic}
\tmop{with} \tmop{an} \tmop{initial} \tmop{segment} \tmop{of} \langle
\mathbbm{N}, \leqslant \rangle$]
  First we prove that if $A$ is infinite then $A$ has a denumerable subset.
  Using \ the 'well ordering theorem' which is derived from the axiom of
  choice (see \ref{well-ordering theorem}) there exists a order $\leqslant_A$
  on $A$ such that $\langle A, \leqslant_A \rangle$ is a well ordered set.
  Using \ref{relation of well ordered classes} and the fact that $\langle
  \mathbbm{N}, \leqslant \rangle$ is also well ordered (see \ref{the natural
  numbers are well-ordered}) we have exactly one of the following:
  \begin{enumerate}
    \item but this would mean that $A \approx \mathbbm{N}$ and thus $A$ has a
    denumerable subset.
    
    \item but this would again mean that $A$ has a denumerable subset.
    
    \item But then there exists a $n \in \mathbbm{N}$ so that $A \approx S_n =
    n$ and thus $A$ is finite which contradicts the fact that $A$ is infinite,
    so this case does not apply.
  \end{enumerate}
  Second assume that $A$ has a denumerable subset $B$. If $A$ is finite then
  there exists a $n \in \mathbbm{N}$ such that $A \approx n$ and thus there
  exists a bijection $f : A \rightarrow n$. Then $f (B) \subseteq n$ and as
  $f_{|B} : B \rightarrow f (B)$ is a bijection and as $\mathbbm{N} \approx B$
  we have the existence of a bijection $g : \mathbbm{N} \rightarrow B$ so
  $f_{|B} \circ g : \mathbbm{N} \rightarrow f (B)$ is a bijection and thus $f
  (B)$ is a denumerable subset of $n$ which is impossible by the previous
  lemma. we must thus conclude that $A$ is not finite and thus is infinite.
\end{proof}

\begin{corollary}
  \label{every set with a infinite subset is finite}Every set with a infinite
  subset is infinite.
\end{corollary}

\begin{proof}
  If $A$ is a set such that there exists a infinite set $B$ with $B \subseteq
  A$ then as $B$ is infinite we have by \ref{infinite sets have countable
  subset} the existence of a denumerable set $C \subseteq B$ but then $C
  \subseteq A$ and thus $A$ has a denumerable subset and by \ref{infinite sets
  have countable subset} we have that $A$ is infinite.
\end{proof}

\begin{corollary}
  \label{subsets of finite sets are finite}Every subset of a finite set is
  finite.
\end{corollary}

\begin{proof}
  If a finite set would contain a infinite subset then by the previous theorem
  the finite set would be infinite.
\end{proof}

\

\begin{theorem}
  \label{union of two finite sets is finite}If $A, B$ are finite sets then $A
  \bigcup B$ is finite
\end{theorem}

\begin{proof}[injectivity][surjectivity]
  As $A$ is finite we have by \ref{subsets of finite sets are finite} that $A
  \backslash B$ is finite, this together with the finiteness of $B$ give the
  existence of two bijections $f : A \backslash B \rightarrow n = S_n$ and $g'
  : B \rightarrow m = S_m$ where $n, m \in \mathbbm{N}$. Define now $g : B
  \rightarrow \{ i \in \mathbbm{N}|n \leqslant i \wedge i < n \upl m \}$ by $g
  (x) = g' (x) \upl n$ then this function is bijective
  \begin{enumerate}
    \item If $g (x) = g (x')$ then $g' (x) \upl n = g' (x') \upl n
    \Rightarrowlim_{\text{\ref{n+k=m+k=<gtr>n=m}}} g' (x) = g' (x')
    \Rightarrow x = x'$
    
    \item If $i \in \{ i \in \mathbbm{N}|n \leqslant i \wedge i < n \upl m \}$
    then $n \leqslant i < n \upl m$ and thus using
    \ref{n<less>=m<less>=<gtr>n+k=m} there exists a $k \in \mathbbm{N}$ such
    that $n \upl k = i$. We must have $k < m$ [if $m \leqslant k \Rightarrow n
    \upl m \leqslant n \upl k = i < n \upl m \Rightarrow$we get the
    contradiction $n \upl m \neq n \upl m$] and thus $k \in S_m = m$. So using
    the fact that $g'$ is bijective and thus surjective we have $\exists x \in
    B$ such that $g' (x) = k \Rightarrow g (x) = g' (x) \upl n = k \upl n =
    i$.
  \end{enumerate}
  Further if $i \in n \bigcap \{ i \in \mathbbm{N} | n \leqslant i \wedge i <
  n \upl m | \} \Rightarrowlim_{n = S_n} i < n \wedge n \leqslant i
  \Rightarrow i \neq i$ a contradiction, so we have $n \bigcap \{ i \in
  \mathbbm{N}|n \leqslant i \wedge i < n \upl m \} = \emptyset$. So we can use
  \ref{union of bijections} to construct the bijection $f \bigcup g : A
  \bigcup B = (A \backslash B) \bigcup B \rightarrow n \bigcup \{ i \in
  \mathbbm{N} | n \leqslant i \wedge i < n \upl m | \} = S_{n \upl m} = n \upl
  m$ and thus we have proved that $A \bigcup B \approx n \upl m$.
\end{proof}

\begin{lemma}
  If $\{ A_i \}_{i \in S_n}$ is a finite family of finite sets then
  $\bigcup_{i \in S_n} A_i$ is finite.
\end{lemma}

\begin{proof}[$i = n$][$i \neq n$][$x \in A_n$][$x \nin A_n$]
  We prove this by induction on $n$. So define $S = \left\{ n \in \mathbbm{N}|
  \tmop{if} \{ A_i \}_{i \in S_n} \tmop{is} a \tmop{family} \tmop{of}
  \tmop{finite} \tmop{set} \tmop{then} \bigcup_{i \in S_n} A_i \tmop{is}
  \tmop{finite} \right\}$ then we have
  \begin{enumerate}
    \item If $n = 0$ take then $\{ A_i \}_{i \in S_0}$ \ then there exists a
    graph $A$ with $\tmop{dom} (A) = \emptyset$ so if $x \in \bigcup_{i \in
    S_0} A_i$ then there exists a $i \in S_0$ such that $x \in A_i$ which is a
    contradiction as $S_0 = \emptyset$ so we have that $\bigcup_{i \in S_0}
    A_i = \emptyset$ and thus finite. So we have $0 \in S$
    
    \item Assume that $n \in S$ and take $n \upl 1$ then we have for the
    family of finite sets $\{ A_i \}_{i \in S_{n \upl 1}}$ that there exists a
    graph $A$ with $\tmop{dom} (A) = S_{n \upl 1}$ . If we take the family$A'
    = \{ (x, y) \in A|x \in S_n \}$ [so $\tmop{dom} (A') = S_n $and $A'_i = \{
    (x, y) \in A' |x = i \}$] then $\bigcup_{i \in S_{n \upl 1}} A_i = A_n
    \bigcup \left( \bigcup_{i \in S_n} A'_i \right)$
    
    \begin{proof}
      
      \begin{enumerate}
        \item If $x \in \bigcup_{i \in S_{n \upl 1}} A_i$ then $\exists i \in
        \tmop{dom} (A) = S_{n \upl 1} \Rightarrow i < n \upl 1$ such that $x
        \in A_i$ then we have the following cases
        \begin{enumerate}
          \item then $x \in A_n \Rightarrow x \in A_n \bigcup \left(
          \bigcup_{i \in S_n} A'_i \right)$
          
          \item then $i < n$ [if $n \leqslant i \Rightarrowlim_{i \neq n} n <
          i < n \upl 1 \Rightarrow n \upl 1 \leqslant i < n + 1 \Rightarrow n
          \upl 1 \neq n \upl 1$ a contradiction] and thus $i \in S_n$. As $x
          \in A_i$ we have $(i, x) \in A \Rightarrowlim_{i \in S_n} (i, x) \in
          A' \Rightarrow x \in A'_i \Rightarrow x \in \bigcup_{i \in S_n} A'_i
          \Rightarrow x \in A_n \bigcup \left( \bigcup_{i \in S_n} A'_i
          \right)$
        \end{enumerate}
        in all cases we have $x \in A_n \bigcup \left( \bigcup_{i \in S_n}
        A'_i \right)$
        
        \item If $x \in A_n \bigcup \left( \bigcup_{i \in S_n} A'_i \right)$
        then we have the following cases
        \begin{enumerate}
          \item $\Rightarrow x \in \bigcup_{i \in S_{n \upl 1}} A_i$
          
          \item $\Rightarrow x \in \bigcup_{i \in S_n} A'_i \Rightarrow
          \exists i \in S_n \vdash x \in A'_i \Rightarrow (i, x) \in A'
          \subseteq A \Rightarrow x \in A_i \Rightarrow x \in \bigcup_{i \in
          S_{n \upl 1}} A_i$
        \end{enumerate}
        in all cases $x \in \bigcup_{i \in S_{n \upl 1}} A_i$
      \end{enumerate}
    \end{proof}
    
    As $n \in S$ we have $\bigcup_{i \in S_n} A'_i$ is finite and as by
    assumption $A_n$ is finite we have by \ref{union of two finite sets is
    finite} that $\bigcup_{i \in S_{n \upl 1}} A_i = A_n \bigcup \left(
    \bigcup_{i \in S_n} A'_i \right)$ is finite and thus $n + 1 \in S$. 
  \end{enumerate}
  By induction we have then $S =\mathbbm{N}$ proving our theorem.
\end{proof}

\begin{theorem}
  \label{finite union of finite sets is finite}If $\{ A_i \}_{i \in I}$ is a
  family of finite sets with $I$ finite then $\bigcup_{i \in I} A_i$ is finite
\end{theorem}

\begin{proof}
  As $I$ is finite we have the existence of a bijection $f : n = S_n
  \rightarrow I$ ($n \in \mathbbm{N}$) and thus using \ref{reindexing of a
  family} we have that $\bigcup_{i \in I} A_i = \bigcup_{j \in S_n} A_{f (j)}$
  \ which is finite by the previous lemma.
\end{proof}

\

\

\begin{theorem}
  \label{if a set is equipotent with a proper subset then it is infinite}A set
  $A$ is infinite if and only if $\exists B \subset A$ such that $B \approx A$
  (if $A$ is equipotent with a proper subset of itself)
\end{theorem}

\begin{proof}[$f (A) = A \backslash \{ b (0) \}$][$x \in A \backslash B$][$x
\in B$][$y \nin B$][$y \in B$][$f$ is injective][$x, y \in A \backslash B$][$x
\in A \backslash B \wedge x' \in B$][$x \in B \wedge x' \in A \backslash
B$][$x, x' \in B$]
  \
  
  First, suppose $A$ is infinite then by \ref{infinite sets have countable
  subset} there exists a denumerable subset $B \subseteq A$. So there exists a
  bijection $b : \mathbbm{N} \rightarrow B$ define then the function $f : A
  \rightarrow A$ by $f (x) = \left\{ \begin{array}{l}
    x \tmop{if} x \in A \backslash B\\
    b (b^{- 1} (x) \upl 1) \forall x \in B
  \end{array} \right.$ then we have that
  \begin{enumerate}
    \item
    
    \begin{proof}
      If $y \in f (A)$ then there exists a $x \in A$ such that $y = f (x)$ and
      then either
      \begin{enumerate}
        \item and thus $f (x) = x \in A \backslash B \Rightarrow f (x) = x
        \neq b (0) \in B \Rightarrow f (x) \in A \backslash \{ b (0) \}$
        
        \item and thus as $b^{- 1} (x) \in \mathbbm{N} \Rightarrow b^{- 1} (x)
        \geqslant 0 \Rightarrow b^{- 1} (x) \upl 1 > 0$, if we would have $b
        (b^{- 1} (x) \upl 1) = b (0)$ then as $b$ is a bijection we have $0 =
        b^{- 1} (x) \upl 1 > 0$ giving the contradiction $0 > 0$ so we must
        have that $f (x) = b (b^{- 1} (x) \upl 1) \neq b (0) \Rightarrow f (x)
        \in A \backslash \{ b (0) \}$ giving $f (A) \subseteq A \backslash \{
        b (0) \}$.
      \end{enumerate}
      Also if $y \in A \backslash \{ b (0) \}$ then we have either
      \begin{enumerate}
        \item $\Rightarrow y \in (A \backslash \{ b (0) \}) \backslash B$ then
        $f (y) = y \Rightarrow y \in f (A)$
        
        \item then as $b$ is a bijection $y = b (b^{- 1} (y))$ and as $y \neq
        b (0)$ we can not have $b^{- 1} (y) = 0$ [otherwise $y = b (b^{- 1}
        (y)) = b (0)$], so $0 < b^{- 1} (y) \Rightarrow 1 \leqslant b^{- 1}
        (y) \Rightarrowlim_{\text{\ref{n<less>=m<less>=<gtr>n+k=m}}} \exists m
        \in \mathbbm{N} \vdash m \upl 1 = b^{- 1} (y)$ and thus $b (m \upl 1)
        = b (b^{- 1} (y)) = y$. Take now $x = b (m) \in B \subseteq A$ then $f
        (x) = b (b^{- 1} (x) \upl 1) = b (m \upl 1) = y \Rightarrow y \in f
        (A)$.
      \end{enumerate}
      giving that $A \backslash \{ b (0) \} \subseteq f (A)$ and thus together
      with the already proven $f (A) \subseteq A \backslash \{ b (0) \}$ that
      $f (A) = A \backslash \{ b (0) \}$
    \end{proof}
    
    \item
    
    \begin{proof}
      If $f (x) = f (x')$ then we have the following possible cases to
      consider
      \begin{enumerate}
        \item then $x = f (x) = f (x') = x' \Rightarrow x = x'$
        
        \item then $f (x') = f (x) = x \in A \backslash B \Rightarrow f (x')
        \nin B$ contradicting the fact that $f (x') = b (b^{- 1} (x') \upl 1)
        \in B$, so this case does not apply.
        
        \item then $f (x) = f (x') = x' \in A \backslash B \Rightarrow f (x)
        \nin B$ contradicting the fact that $f (x) = b (b^{- 1} (x) + 1) \in
        B$ so this case does not apply.
        
        \item then we have $b (b^{- 1} (x) \upl 1) = f (x) = f (x') = b (b^{-
        1} (x') \upl 1) \Rightarrowlim_{b \tmop{is} \tmop{injective}} b^{- 1}
        (x) \upl 1 = b^{- 1} (x') \upl 1 \Rightarrow b^{- 1} (x) = b^{- 1}
        (x') \Rightarrow x = x'$
      \end{enumerate}
      so in all the cases that doe not lead to a contradiction we have $x =
      x'$ proving injectivity.
    \end{proof}
  \end{enumerate}
  From (1) and (2) we conclude that $f : A \rightarrow A \backslash \{ b (0)
  \}$ is a bijection and thus that $A$ is bijective with a proper subset of
  itself.
  
  Second assume that there exists a proper subset $B \subset A$ and a
  bijection $f : A \rightarrow B$, giving a injective function $f : A
  \rightarrow A$ with $f (A) = B$. As $B \subset A$ there exists a $c \in A
  \backslash B$ and thus $c \nin f (A)$. Using \ref{recursive injective
  function} there exists a injective function $\lambda : \mathbbm{N}
  \rightarrow A$ such that
  \begin{enumerate}
    \item $\lambda (0) = c$
    
    \item $\forall n \in \mathbbm{N} \vDash \lambda (s (n)) = f (\lambda (n))$
  \end{enumerate}
  So we found a bijective function $\lambda : \mathbbm{N} \rightarrow \lambda
  (\mathbbm{N}) \subseteq A$ giving a denumerable set $\lambda (\mathbbm{N})
  \subseteq A$ which by \ref{infinite sets have countable subset} means that
  $A$ is infinite. 
\end{proof}

\begin{theorem}
  \label{equipotent is the same as equal in the natural numbers}If $n, m \in
  \mathbbm{N}$ and $n \approx m$ then $n = m$
\end{theorem}

\begin{proof}
  Assume $n \approx m$ then
  \begin{enumerate}
    \item If $n < m \Rightarrowlim_{n = S_n, m = S_m} n \subset m$ and thus
    $m$ is bijective to a proper subset of itself, which means by the previous
    theorem that $m$ is infinite contradicting the finiteness of $m$ (as it is
    bijective with itself)
    
    \item If $m < n \Rightarrowlim_{n = S_n, m = S_m} m \subset n$ and thus
    $n$ is bijective to a proper subset of itself, which means by the previous
    theorem that $n$ is infinite contradicting the finiteness of $n$ (as it is
    bijective with itself) 
  \end{enumerate}
  as $\mathbbm{N}$ is totally ordered we must conclude that $n = m$.
\end{proof}

The previous theorem leads to the following observation: If $A$ is a finite
set then there exists a $n \in \mathbbm{N}$ such that $n \approx A$, if there
was also a $n' \in \mathbbm{N}$ such that $n' \approx A$ and thus $n \approx
n'$ and this leads to the following definition of the number of elements in a
set.

\begin{definition}
  \label{number of elements in a finite set}{\index{$\# (A)$}}If $A$ is a
  finite set then $\exists !n \in \mathbbm{N} \vDash n \approx A$ this number
  is called the number of elements in $A$ and is noted as $\# (A)$.
\end{definition}

\begin{theorem}
  \label{product of finite sets is finite}If $A, B$ are finite sets with $n
  =\# (A)$ and $m =\# (B)$ then $A \times B$ is finite and has size $n \cdot
  m$
\end{theorem}

\begin{proof}[injectivity][surjectivity]
  If $A = \emptyset$ (or $B = \emptyset$) then $\# (A) = 0$ (or $\# (B) = 0$)
  and then $A \times B = \emptyset$ and $\# (A \times B) = 0 =\# (A) \cdot \#
  (B)$. So assume that $A, B \neq \emptyset$ then there exists bijections $a :
  B \rightarrow S_n$, $b : B \rightarrow S_m$ where $n, m \in \mathbbm{N}_0$.
  Define now $f : A \times B \rightarrow S_{m \cdot n}$ by $(x, y) \rightarrow
  a (x) \cdot m + b (y)$ (as $\forall x \in A, \forall y \in B$ we have \ $0
  \leqslant a (x) < n$ and $0 \leqslant b (y) < m$ we have indeed that
  $\forall (x, y) \in A \times B$ that $0 \leqslant f (x, y) < n \cdot m \upl
  m$). We prove now that $f$ is a bijection:
  \begin{enumerate}
    \item If $f (x, y) = f (x', y')$ then $a (x) \cdot m \upl b (y) = a (x')
    \cdot m \upl b (y')$ and using the fact that $b (y), b (y') < m$ and
    \ref{division algorithm for natural numbers} we have that $a (x) = a (x')$
    and $b (y) = b (y') \Rightarrowlim_{a, b \tmop{are} \tmop{bijections}} x =
    x'$ and $y = y' \Rightarrow (x, y) = (x', y')$ proving injectivity.
    
    \item If $z \in S_{m \cdot n}$ then $0 \leqslant z < m \cdot n$, using the
    Division Algorithm (see \ref{division algorithm}) there exists a $q, r$ so
    that $z = q \cdot m \upl r$ and $0 \leqslant r < m \Rightarrow r \in S_m$,
    also $0 \leqslant q < n$ [if $n \leqslant q \Rightarrow n \cdot m
    \leqslant q \cdot m \Rightarrow n \cdot m \upl r \leqslant q \cdot m \upl
    r = z \Rightarrow n \cdot m \upl r \leqslant z \Rightarrow n \cdot m
    \leqslant z < n \cdot m$ a contradiction] and thus $q \in S_n$. Now as $a,
    b$ are bijections there exists $x \in A$ and $y \in B$ such that $a (x) =
    n$ and $b (x) = m$ and thus $f (x, y) = a (x) \cdot m \upl b (y) = q \cdot
    m \upl r = z$.
  \end{enumerate}
\end{proof}

\begin{theorem}
  \label{difference of finte sets}If $A$ is a finite set and $B \subseteq A$
  then $B, A \backslash B$ are finite and $\# (B) \leqslant \# (A)$ and $\# (A
  \backslash B) =\# (A) \um \# (B)$. Note that from the last equation it
  follows that $\# (A \backslash B) \upl \# (B) =\# (A)$ and thus that $\#
  (A\backslash B) \leqslant \# (B)$
\end{theorem}

\begin{proof}[$B = A$][$B = \emptyset$][$\emptyset \neq B \subset A$][$i =
0$][$i = 1$][$i = n$][$i \neq n$][$\emptyset \neq B = A \backslash \{ \alpha
\}$][$\emptyset \neq B \subset A \backslash \{ a
\}$][injectivity][surjectivity]
  As $A$ is finite with $n =\# (A)$ then there exists a bijection $a : n = S_n
  \rightarrow A$ we have now to consider the following cases:
  \begin{enumerate}
    \item  then obviously $B$ is finite and $A \backslash B = \emptyset$ and
    thus $A \backslash B$ is finite and $\# (A \backslash B) =\# (A) -\# (B)
    =\# (A) -\# (A) = 0$
    
    \item then obviously $B$ is finite and $A \backslash B = A$ and thus $A
    \backslash B$ is finite and $\# (A \backslash B) =\# (A) =\# (A) - 0 =\#
    (A) -\# (B)$
    
    \item Let now $X = \{ n \in \{ 2, \ldots, \} | \tmop{If} A \tmop{is}
    \tmop{such} \tmop{that} \# (A) = n \tmop{and} \emptyset \neq B \subset A
    \tmop{then} \tmop{there} \tmop{exists} a \tmop{bijection} f : S_n
    \rightarrow A \tmop{and} a m \in S_n \tmop{such} \tmop{that} f_{|S_m} :
    S_m \rightarrow B \tmop{is} a \tmop{bijection} \} $then we have
    \begin{enumerate}
      \item If $n = 2$ then if $\# (A) = 2$ then $2 = \{ 0, 1 \} = S_2$ and
      there exists a bijection$f' : \{ 0, 1 \} \rightarrow A$, now as
      $\emptyset \neq B \subset A$ there exists a $a \in A \backslash B$ and
      as $f'$ is a bijection then there exists a $i \in \{ 0, 1 \}$ such that
      $f' (i) = a$ we have now two cases:
      \begin{enumerate}
        \item then $B = \{ f' (1) \}$ so take the bijection $f : \{ 0, 1 \}
        \rightarrow A$ with $\left\{ \begin{array}{l}
          f (0) = f' (1)\\
          f (1) = f' (0)
        \end{array} \right.$then $f_{|S_1 = \{ 0 \}} : \{ 0 \} \rightarrow \{
        f' (1) \} = B$ is a bijection
        
        \item then $B - \{ f' (0) \}$ and take then $f = f'$ so that $f_{|S_1}
        = f'_{|S_1} : \{ 0 \} \rightarrow \{ f' () \} = B$ is a bijection
      \end{enumerate}
      we can thus conclude that $2 \in X$
      
      \item Assume now that $n \in X \nosymbol$. If now $\# (A) = n \upl 1$
      then there exists a bijection $f' : S_{n \upl 1} \rightarrow A$, now as
      $\emptyset \neq B \subset A$ there exists a $a \in A \backslash B$ and
      thus $\emptyset \neq B \subseteq A \backslash \{ a \}$. As $f'$ is a
      bijection there exists a $i \in S_{n \upl 1} = \{ 0, \ldots, n \}$ such
      that $f' (i) = a$. We have now two cases to consider:
      \begin{enumerate}
        \item take then $f = f'$
        
        \item take then $a' = f' (n)$ then $a \neq a'$[otherwise $i = n$].
        Then $f'_{|S_{n \upl 1} \backslash \{ i, n \}} : S_{n \upl 1}
        \backslash \{ i, n \upl 1 \} \rightarrow A \backslash \{ a, a' \}$ is
        a bijection (see \ref{bijections and exclusions}). Define further the
        bijection $f'' : \{ i, n \} \rightarrow \{ a, a' \}$ by $\left\{
        \begin{array}{l}
          f'' (i) = a'\\
          f'' (n) = a
        \end{array} \right.$ so that we have a bijection (see \ref{union of
        bijections}) $f = f'_{|S_{n \upl 1} \backslash \{ i, n \}} \bigcup f''
        : S_{n \upl 1} \rightarrow A$ with $(n) = f'' (n) = a$.
      \end{enumerate}
      So we have found a bijection $f : S_{n \upl 1} \rightarrow A$ such that
      $f (n) = a$. We have now the following cases to consider for $A
      \backslash \{ a \}$ (remember $B \subseteq A \backslash \{ a \}$):
      \begin{enumerate}
        \item then we take $m = n$ and $f_{|S_n} : S_n \rightarrow A
        \backslash \{ a \} = B$ is a bijection (as $f (\{ n \}) = \{ a \}$ and
        thus $n \upl 1 \in X$
        
        \item then as $n =\# (A \backslash \{ a \})$ and $n \in X$ there
        exists a bijection $f''' : S_n \rightarrow A \backslash \{ a \}$ and a
        $m \in S_n$ such that $f'''_{|S_m} : S_m \rightarrow B$ is a
        bijection, take then the bijection $f^{\tmop{iv}} : \{ n \}
        \rightarrow \{ a \}$ by $f^{\tmop{iv}} (n) = a$ and construct then the
        bijection $f^v = f''' \bigcup f^{\tmop{iv}} : S_{n \upl 1} \rightarrow
        A$. We have then that $f^v_{|S_m} = f''_{|S_m}$ is a bijection from
        $S_m$ to $B$ proving that $n \upl 1 \in X$
      \end{enumerate}
    \end{enumerate}
    Using mathematical induction (see \ref{mathematical induction form 2}) we
    have that $X = \{ 2, \ldots \}$, now as $\emptyset \neq B \subset A$ we
    must have $2 \leqslant \# (A) = n \Rightarrow n =\# (A) \in \{ 2, \ldots
    \} = X$ so there exists a bijection $f : S_n \rightarrow A$ and a $m \in
    S_n$ such that $f_{| S_m \nobracket} : S_m \rightarrow B$ is a bijection,
    this proves that $B$ is finite and $\# (B) = m$. Now if $i \in \{ m,
    \ldots, n \um 1 \}$ we have $f (i) \nin B$ [if $f (i) \in B$ then as $f_{|
    S_m \nobracket}$ is a bijection there exists a $j < m$ such that $f (j) =
    f_{|S_m} (j) = f (i) \Rightarrow i = j < m$ contradicting $i \in \{ m,
    \ldots, n \um 1 \}$] and thus $f (\{ m, \ldots, n \um 1 \}) \subseteq A
    \backslash B$. If $y \in A \backslash B$ then as $f$ is a bijection there
    exists a $i \in S_n$ such that $y = f (i)$ if now $i < m \Rightarrow i \in
    S_m \Rightarrow f (i) \in B$ a contradiction so we must have $i \in \{ m,
    \ldots, n \um 1 \} \Rightarrow A \backslash B \subseteq f (\{ m, \ldots, n
    \um 1 \})$. So $f (\{ m, \ldots, n - 1 \}) = A \backslash B$ and thus by
    \ref{bijections and exclusions} we have that $f_{| \{ m, \ldots, n \um 1
    \}} : \{ m, \ldots, n \um 1 \} \rightarrow A \backslash B$ is a bijection.
    Define now $g : S_{n \um m} \rightarrow \{ m, \ldots, n - 1 \}$ (note that
    from $m \in S_n \Rightarrow m < n \Rightarrow 0 < n \um m \in
    \mathbbm{N}$) by $i \rightarrow g (i) = i \upl m$ then $g$ is a bijection:
    \begin{enumerate}
      \item if $g (i) = g (j) \Rightarrow i \upl m = j \upl m \Rightarrow i =
      j$
      
      \item If $j \in \{ m, \ldots, n \um 1 \} \Rightarrow m \leqslant j < n
      \um 1 \Rightarrow 0 \leqslant j \um m < n \um m \um 1 \Rightarrow j \um
      m \in S_{n \um m}$ and $g (j \um m) = j$
    \end{enumerate}
    So we have then a bijection $f_{| \{ m, \ldots, n \um 1 \}} \circ g : S_{n
    \um m} \rightarrow A \backslash B$ proving that $A \backslash B$ is finite
    and $\# (A \backslash B) = n \um m =\# (A) -\# (B)$
  \end{enumerate}
\end{proof}

\begin{theorem}
  \label{number of elements of strict subsets of a finite set}If $A$ is a
  finite set then if $B \subset A$ we have that $\# (B) <\# (A)$
\end{theorem}

\begin{proof}
  Using the previous theorem we have that $B$ is finite and $\# (B) \leqslant
  \# (A)$, if now $\# (A) =\# (B) = n$ then $A \approx n \approx B$ and thus
  $A \approx B$ which by \ref{if a set is equipotent with a proper subset then
  it is infinite} would give that $A$ is infinite contradicting the finiteness
  of $A$. So we must have $\# (B) <\# (A)$
\end{proof}

\begin{theorem}
  \label{surjection f:Sn-<gtr>A implies A is finite}If A is a set, $n \in
  \mathbbm{N}$ and $f : S_n \rightarrow A$ a surjection then $A$ is finite and
  $\# (A) \leqslant n$. 
\end{theorem}

\begin{proof}[$\forall a \in A \vDash a = f (n)$][$\exists a \in A \vdash a
\neq f (n)$]
  If $n = 0$ then $S_0 = \emptyset$ and if $f : \emptyset \rightarrow A$ is a
  surjection then if $x \in A$ there exists a $i \in \emptyset$ such that $x =
  f (i)$ which would mean that $\emptyset \neq \emptyset$ a contradiction, so
  in this cases we must have $A = \emptyset \Rightarrow A$ is finite and $\#
  (A) = 0 \leqslant 0$ proving the theorem for the case $n = 0$. We prove the
  case $n \in \mathbbm{N}_0$ by mathematical induction so let $X = \{ n \in \{
  1, \ldots \} | \tmop{if} A \tmop{is} a \tmop{set} \tmop{such} \tmop{that}
  \tmop{there} \tmop{exists} a \tmop{surjection} f : S_n \rightarrow A
  \tmop{then} A \tmop{is} \tmop{finite} \tmop{and} \# (A) = n \}$ we have then
  that
  \begin{enumerate}
    \item If $n = 1$ then if $A$ is such that there exists a surjection $f :
    S_1 = \{ 0 \} \rightarrow A$ then $A = \{ f (0) \}$ which trivially means
    that $A$ is finite and that $\# (A) = 1 \leqslant 1 \Rightarrow 1 \in X$
    
    \item If now $n \in X$ then if $A$ is such that there exists a surjection
    $f : S_{n \upl 1} \rightarrow A$ . Then as $f (n) \in A$ we can not have
    $A = \emptyset$ and we are left with the following two cases
    \begin{enumerate}
      \item $\Rightarrow A = \{ f (n) \}$ and thus $A$ is finite with $\# (A)
      = 1 \leqslant n \upl 1$. This gives $n \upl 1 \in X$
      
      \item Define then $g : S_n \rightarrow A \backslash \{ f (n) \}$ by
      \begin{eqnarray*}
        g (i) & = f (i)  & \tmop{if} i \in S_n \backslash f^{- 1} (\{ f (n)
        \})  [\Rightarrow g (i) \in A \backslash \{ f (n) \}]\\
        & = a & \tmop{if} i \in S_n \bigcap f^{- 1} (\{ f (n) \}) 
        [\Rightarrow g (i) \in A \backslash \{ f (n) \}]
      \end{eqnarray*}
    \end{enumerate}
    If now $y \in A \backslash \{ f (n) \}$ then as $f$ is a surjection there
    exists a $i \in S_{n \upl 1}$ such that $f (i) = y$. We can not have $i =
    n$ [As then $f (i) = f (n) \Rightarrow y \in \{ f (n) \}$], also we can
    not have $i \in f^{- 1} (\{ f (n) \})$ [as then $f (i) \in \{ f (n) \}
    \Rightarrow y \in \{ f (n) \}$ so $i \in S_n \backslash f^{- 1} (\{ f (n)
    \})$ and thus $g (i) = f (i) = y$ . This proves that $g$ is a surjection.
    As $n \in X$ we have that $A \backslash \{ (f (n)) \}$ is finite and $m
    =\# (A \backslash \{ f (n) \}) \leqslant n$. So there exists a bijection
    $h : S_m \rightarrow A \backslash \{ f (n) \}$, combine this with the
    bijection $h' : \{ m \} \rightarrow \{ f (n) \}$ using \ref{union of
    bijections} to form the bijection $h \bigcup h' : S_{m \upl 1} \rightarrow
    A$. So $A$ is finite and $\# (A) = m \upl 1 \leqslant n + 1$ and this
    proves that $n \upl 1 \in X$
    
    
  \end{enumerate}
  Using mathematical induction (\ref{mathematical induction form 2}) we have
  that $X = \{ 1, \ldots \} =\mathbbm{N}_0$ proving the theorem for $n \in
  \mathbbm{N}_0$
\end{proof}

\begin{corollary}
  \label{surjection f:A-<gtr>B B is finite if A is finite}If $A, B$ are sets
  where $A$ is finite and $f : A \rightarrow B$ is a surjection then $B$ is
  finite and $\# (B) \leqslant \# (A)$
\end{corollary}

\begin{proof}
  If $A$ is finite and $\# (A) = n$ then there exists a bijection $b : n
  \rightarrow A$ so $f \circ g : S_n \rightarrow B$ is a surjection and by the
  previous theorem we have then $B$ is finite and $\# (B) \leqslant n =\# (A)$
\end{proof}

\begin{theorem}
  \label{injection from a infinite set}Let $A, B$ be sets, $A$ infinite and $f
  : A \rightarrow B$ a injection then $B$ is infinite
\end{theorem}

\begin{proof}
  Assume that $B$ is finite then $f (A) \subseteq B$ is finite and there is a
  bijection $f : \{ 1, \ldots, n \} \rightarrow f (A)$ so that as $f : A
  \rightarrow f (A)$ is a bijection we find that $f^{- 1} \circ h : \{ 1,
  \ldots, n \} \rightarrow A$ is a bijection making $A$ finite a
  contradiction.
\end{proof}

\begin{theorem}
  \label{finite ordered sets have a maximum}Let $\langle A, \leqslant \rangle$
  be a fully ordered non empty finite set then $\max (A)$ (or $\min (A)$)
  exists (see \ref{maximum and minimum})
\end{theorem}

\begin{proof}[$m' \leqslant b (n + 1)$][$b (n + 1) \leqslant m'$]
  We prove this by induction on $\# (A) = n$. So let $S = \{ n \in
  \mathbbm{N}_0 = \{ 1, \ldots \} | \tmop{if} \# (A) = n \tmop{then} \max (A)
  \tmop{exists} \}$ then we have :
  \begin{enumerate}
    \item If $n = 1$ then $A = \{ a \}$ with $\max (A)$
    
    \item If $n \in S$ let then $\# (A) = n + 1$ so there exists a bijection
    $b : \{ 1, \ldots, n + 1 \} \rightarrow A$, take then $A\backslash \{ b (n
    + 1) \}$ which is bijective to $\{ 1, \ldots, n \}$ so that $\#
    (A\backslash \{ b (n + 1) \}) = n$ and thus $m' = \max (A\backslash \{ b
    (n + 1) \})$ exist. We have now two cases to consider:
    \begin{enumerate}
      \item then $\max (A) = b (n + 1)$
      
      \item then $\max (A) = m'$
    \end{enumerate}
  \end{enumerate}
  The proof for the minimum is similar.
\end{proof}

\begin{theorem}
  If $A$ is finite and $f : \mathbbm{N}_0 \rightarrow A$ is a function then
  $\exists a \in A$ such that $\{ m \in \mathbbm{N}_0 |f (x) = a \} = f^{- 1}
  (\{ a \})$ is infinite.
\end{theorem}

\begin{proof}
  Assume that the theorem is not valid then $\forall a \in A$ we have that
  $f^{- 1} (\{ a \})$ is finite so that $\bigcup_{a \in A} f^{- 1} (\{ a \})$,
  being a finite union of finite sets, is finite. Now from $\mathbbm{N}_0 =
  f^{- 1} (A) = f^{- 1} \left( \bigcup_{a \in A} \{ a \} \right) = \bigcup_{a
  \in A} f^{- 1} (\{ a \})$ we would then conclude that $\mathbbm{N}_0$ is
  finite which is a contradiction.
\end{proof}

\begin{corollary}
  \label{mapping of N into a finite set}If $A$ is finite and $f :
  \mathbbm{N}_0 \rightarrow A$ is a function then $\exists a \in A$ so that
  $\forall n \in \mathbbm{N}_0$ there exists a $m \in \{ i \in \mathbbm{N}_0
  |i \geqslant n \}$ so that $f (m) = a$
\end{corollary}

\begin{proof}
  By the previous theorem $\exists a \in A$ such that $f^{- 1} (\{ a \}) = \{
  m \in \mathbbm{N}_0 |f (m) = a \}$ is infinite. We proceed non by
  contradiction. So assume that $\exists n \in \mathbbm{N}_0$ such that
  $\forall m \geqslant n$ we have $f (m) \neq a$. If then $m \in f^{- 1} (\{ m
  \}) \Rightarrow f (m) = a \Rightarrow m < n \Rightarrow f^{- 1} (\{ a \})
  \subseteq S_n$ meaning that $f^{- 1} (\{ a \})$ is finite contradicting the
  fact that $f^{- 1} (\{ a \})$ is infinite. 
\end{proof}

\begin{theorem}
  \label{ordering of a finite set}Let $\langle X, \leqslant \rangle$ be a
  fully ordered set and let $A \subseteq X$ be a non empty finite set with $\#
  (A) > 1$. Then there exists a bijection $i : \{ 1, \ldots, \# (A) \}
  \rightarrow A$ such that $\forall k \in \{ 1, \ldots, \# (A) - 1 \}$ we have
  $i (k) < i (k + 1)$. Second for such a bijection we must have $\forall k, l
  \in \{ 1, \ldots, \# (A) \}$ with $k < l$ that $i (k) < i (l)$. Third the
  above bijection is unique.
\end{theorem}

\begin{proof}
  We prove the theorem by induction so let $\mathcal{S}= \{ n \in \{ 2, \ldots
  \} | \tmop{if} \# (A) = n \tmop{then} \tmop{there} \tmop{exists} a
  \tmop{bijection} i : \{ 1, \ldots, n \} \rightarrow A \tmop{with} \forall k
  \in \{ 1, \ldots, n - 1 \} \tmop{we} \tmop{have} i (k) < i (k + 1) \}$. We
  have then:
  \begin{enumerate}
    \item If $n = 2$ then there exists a bijection $b : \{ 1, \ldots, 2 \} =
    \{ 1, 2 \} \rightarrow A$ so $A = \{ b (1), b (2) \}$. If $b (1) < b (2)$
    we have found our bijection. If $b (2) < b (1)$ construct then $i : \{ 1,
    2 \} \rightarrow A$ by $i (1) = b (2)$ and $i (2) = b (1)$ which is
    obviously a bijection with $i (1) < i (2)$. This proves that $2 \in
    \mathcal{S}$.
    
    \item Assume that $n \in \mathcal{S}$ and that we have a finite $A$ with
    $\# (A) = n + 1$. As $A$ is finite a maximum exists (see \ref{finite
    ordered sets have a maximum}) take then $A\backslash \{ \max (A) \}$.
    Using \ref{difference of finte sets} we have that $\# (A\backslash \{ \max
    (A) \}) =\# (A) -\# (\{ \max (A) \}) = (n + 1) - 1 = n$. As $n \in
    \mathcal{S}$ there exists a bijection $i' : \{ 1, \ldots, n \} \rightarrow
    A\backslash \{ \max (A) \}$ such that $\forall k \in \{ 1, \ldots, n - 1
    \}$ we have $' (k) < i' (k + 1)$. Define then the bijection $i : \{ 1,
    \ldots, n + 1 \} \rightarrow A$ by $k \rightarrow i (k) = \left\{
    \begin{array}{l}
      i' (k) \tmop{if} k \in \{ 1, \ldots, n \}\\
      \max (A) \tmop{if} k = n + 1
    \end{array} \right.$(see \ref{union of bijections}) then if $k \in \{ 1,
    \ldots, (n + 1) - 1 \} = \{ 1, \ldots, n \}$ we have either $k \in \{ 1,
    \ldots, n - 1 \}$ and then $i (k) = i' (k) < i' (k + 1) = i (k)$, if $k =
    n$ then $i (k) \leqslant \max (A) = i (k + 1)$. As $i (k) = i' (k) \in
    A\backslash \{ \max (A) \}$ we must have $i (k) \neq \max (A) = i (k + 1)$
    so that we have $i (k) < i (k + 1)$. This proves that $n + 1 \in
    \mathcal{S}$
  \end{enumerate}
  Using induction we have $\mathcal{S}= \{ 2, \ldots \}$ proving the
  existence. Next we prove by induction that if $i : \{ 1, \ldots, \# (A) \}
  \rightarrow A$ is a bijection such that $\forall k \in \{ 1, \ldots, \# (A)
  - 1 \}$ we have $i (k) < i (k + 1)$ then we have $\forall k, l \in \{ 1,
  \ldots, \# (A) \}$ that $i (k) < i (l)$. So let $k \in \{ 1, \ldots, \# (A)
  \}$ and take $\mathcal{P}_k = \{ m \in \mathbbm{N}_0 | \tmop{if} k + m
  \leqslant \# (A) \tmop{then} i (k) < i (k + m) \}$ then we have
  \begin{enumerate}
    \item If $m = 1$ then if $k + 1 <\# (A)$ we have $k \leqslant \# (A) - 1$
    so that $i (k) < i (k + 1) = i (k + m)$ proving that $1 \in \mathcal{P}_k$
    
    \item Let $m \in \mathcal{P}_k$ then if $k + (m + 1) \leqslant \# (A)$ we
    have $(k + m) \leqslant \# (A) - 1$ so that $i (k + m) < i ((k + m) + 1) =
    i (k + (m + 1))$ and as $m \in \mathcal{P}_k$ and $k + m \leqslant \# (A)$
    we have $i (k) < i (k + m)$ so that $i (k) < i (k + (m + 1))$ proving that
    $m + 1 \in \mathcal{P}_k$
  \end{enumerate}
  Using induction we have that $\mathcal{P}_k =\mathbbm{N}_0$ so if $k, l \in
  \{ 1, \ldots, \# (A) \}$ and $k < l$ then $m = l - k \in \mathbbm{N}_0
  =\mathcal{P}_k$ and as $k + m = l \leqslant \# (A)$ we have that $i (k) < i
  (k + m) = i (l)$.
  
  Finally to prove uniqueness suppose that $n =\# (A)$ and $i_1 : \{ 1,
  \ldots, n \} \rightarrow A$, $i_2 : \{ 1, \ldots, n \} \rightarrow A$ are
  bijections such that if $k \in \{ 1, \ldots, n - 1 \}$ then $i_1 (k) < i_1
  (k) \wedge i_2 (k) < i_2 (k + 1)$. Take now $\mathcal{Q}= \{ k \in
  \mathbbm{N}_0 | \tmop{if} k \leqslant n \tmop{then} \forall l \in \{ 1,
  \ldots, k \} \tmop{we} \tmop{have} i_1 (l) = i_2 (l) \}$ then we have:
  \begin{enumerate}
    \item If $k = 1$ then if $i_1 (1) \neq i_2 (1)$ we have either
    \begin{enumerate}
      \item $i_1 (1) < i_2 (1)$ then as $i_1 (1) \in A$ there exists a $l \in
      \{ 1, \ldots, n \}$ such that $i_2 (l) = i_1 (1)$. If now $l > 1$ we
      must have $i_2 (1) < i_2 (l) = i_1 (1) < i_2 (1)$ a contradiction, so we
      must have $l = 1$ but this means that $i_2 (1) = i_1 (1)$ contradicting
      $i_1 (1) < i_2 (1)$. So this case leads to a contradiction.
      
      \item $i_2 (1) < i_1 (1)$ then as $i_2 (1) \in A$ there exists a $l \in
      \{ 1, \ldots, n \}$ such that $i_1 (l) = i_2 (1)$. If now $l > 1$ we
      must have $i_1 (1) < i_1 (l) = i_2 (1) < i_1 (1)$ a contradiction, so we
      must have $l = 1$ but this means that $i_1 (1) = i_2 (1)$ contradicting
      $i_2 (1) < i_1 (1)$. So this case leads to a contradiction. 
    \end{enumerate}
    we must thus conclude that $i_1 (1) = i_2 (1)$ proving that $1 \in
    \mathcal{Q}$
    
    \item Let $k \in \mathcal{Q}$ then if $k + 1 \leqslant n$ we prove by
    contradiction that $i_1 (k + 1) = i_2 (k + 1)$. So assume that $i_1 (k +
    1) \neq i_2 (k + 1)$ then we have either
    \begin{enumerate}
      \item $i_1 (k + 1) < i_2 (k + 1)$ then as $i_1 (k + 1) \in A$ there
      exists a $l \in \{ 1, \ldots, n \}$ such that $i_2 (l) = i_1 (k + 1)$.
      If now $l = k + 1$ we would have the contradiction $i_1 (k + 1) = i_2
      (l) = i_2 (k + 1) > i_1 (k + 1)$, if $l > k + 1$ then $i_1 (k + 1) = i_2
      (l) > i_2 (k + 1) > i_1 (k + 1)_{}$ again a contradiction, if $l < k +
      1$ then $l \leqslant k$ so that $i_1 (l) = i_2 (l) = i_1 (k + 1) > i_1
      (l)$ again a contradiction. So all the cases leads to a contradiction.
      
      \item $i_2 (k + 1) < i_1 (k + 1)$ then as $i_2 (k + 1) \in A$ there
      exists a $l \in \{ 1, \ldots, n \}$ such that $i_1 (l) = i_2 (k + 1)$.
      If now $l = k + 1$ we would have the contradiction $i_2 (k + 1) = i_1
      (l) = i_1 (k + 1) > i_2 (k + 1)$, if $l > k + 1$ then $i_2 (k + 1) = i_1
      (l) > i_1 (k + 1) > i_2 (k + 1)_{}$ again a contradiction, if $l < k +
      1$ then $l \leqslant k$ so that $i_2 (l) = i_1 (l) = i_2 (k + 1) > i_2
      (l)$ again a contradiction. So all the cases leads to a contradiction.
    \end{enumerate}
    so we must have $i_1 (k + 1) = i_2 (k + 1)$ or $k + 1 \in \mathcal{Q}$
  \end{enumerate}
  By induction we have thus that $\mathcal{Q}=\mathbbm{N}_0$. So if $k \in \{
  1, \ldots, n \}$ then $k \in \mathcal{Q}$ and $k \leqslant n$ so that $i_1
  (k) = i_2 (k)$ proving that $i_1 = i_2$.
\end{proof}

\section{Properties of denumerable sets}

\begin{lemma}
  \label{every subset of N is finite or denumerable}Every subset of
  $\mathbbm{N}$ is finite or denumerable.
\end{lemma}

\begin{proof}
  As $\langle \mathbbm{N}, \leqslant \rangle$ is well-ordered (see \ref{the
  natural numbers are well-ordered}) we can use \ref{every subclass of a well
  ordered class is isomorphic to the class or a segment}. So if $A \subseteq
  \mathbbm{N}$ we have either that $A$ is isomorphic with $\mathbbm{N}$ (and
  thus denumerable as a isomorphism is a bijection) or $A$ is isomorphic with
  a segment $S_n = n$ and thus finite.
\end{proof}

\begin{theorem}
  \label{subsets of denumerable sets are finite or denumerable}Every subset of
  a denumerable set is finite or denumerable.
\end{theorem}

\begin{proof}[$f^{- 1} (B) \tmop{is} \tmop{enumerable}$][$f^{- 1} (B)
\tmop{is} \tmop{finite}$]
  Let $A$ be a denumerable set and take $B \subseteq A$. From the
  denumerability of $A$ there exists a bijective function $f : \mathbbm{N}
  \rightarrow A$. Then using the previous theorem we have that $f^{- 1} (B)
  \subseteq \mathbbm{N}$ is either denumerable or finite, so we have the
  following cases:
  \begin{enumerate}
    \item Then there exists a bijection $g : \mathbbm{N} \rightarrow f^{- 1}
    (B)$, together with the bijection $f_{|f^{- 1} (B)} : f^{- 1} (B)
    \rightarrow B$ this means that $B$ is bijective with $\mathbbm{N}$ and is
    thus denumerable.
    
    \item Then there exists a $n \in \mathbbm{N}$ such that there exists a
    bijection $g : n \rightarrow f^{- 1} (B)$, together with the bijection
    bijection $f_{|f^{- 1} (B)} : f^{- 1} (B) \rightarrow B$ this means that
    $B$ is bijective with $n$ and is thus finite.
  \end{enumerate}
  
\end{proof}

\begin{theorem}
  If $0 < m$ then there exists a number noted by $m \um 1$ such that
  \begin{enumerate}
    \item $(m \um 1) \upl 1 = m$
    
    \item $m \um 1 = m' \um 1 \Leftrightarrow m = m'$
    
    \item $(m \upl 1) \um 1 = m$
  \end{enumerate}
\end{theorem}

\begin{note}
  As $0 < m \Rightarrow 1 \leqslant m$ we have that this definition is
  equivalent with the definition \ref{m-n if n<less>=m, n,m natural numbers}.
\end{note}

\begin{proof}
  
  \begin{enumerate}
    \item If $0 < m$ then $1 \leqslant m$ and thus by
    \ref{n<less>=m<less>=<gtr>n+k=m} there exists a $k \in \mathbbm{N}$ such
    that $1 \upl k = k \upl 1 = m$ so we can take $m \um 1$ to be $k$.
    
    \item If $m \um 1 = m' \um 1 \Rightarrow m = (m \um 1) \upl 1 = (m' \um 1)
    \upl 1 = m' \Rightarrow m = m'$. If $m = m'$ then $(m \um 1) + 1 = m = m'
    = (m' \um 1) \upl 1 \Rightarrow s (m \um 1) = s (m' \um 1)
    \Rightarrowlim_{\text{\ref{if successors are equal numbers are equal}}} m
    \um 1 = m' \um 1$
    
    \item We have $((m \upl 1) - 1) \upl 1 = m \upl 1 \Rightarrow s ((m \upl
    1) - 1) = s (m) \Rightarrowlim_{\text{\ref{if successors are equal numbers
    are equal}}} (m \upl 1) \um 1 = m$
  \end{enumerate}
\end{proof}

\begin{theorem}
  $\mathbbm{N} \times \mathbbm{N} \approx \mathbbm{N}$
\end{theorem}

\begin{proof}[$m, m' = 0$][$m > 0, m' = 0$][$m = 0, m' > 0$][$m > 0, m' >
0$][$m = 0$][$m \neq 0$][$k = 0$][$k \neq 0$]
  First define the function $f : \mathbbm{N} \times \mathbbm{N} \rightarrow
  \mathbbm{N} \times \mathbbm{N}$ (using the previous theorem to define $m \um
  1$)
  \begin{eqnarray*}
    f (k, m) & = & \left\{ \begin{array}{l}
      (0, k \upl 1) \tmop{if} m = 0\\
      (k + 1, m \um 1) \tmop{if} m \neq 0 (\Rightarrow 0 < m)
    \end{array} \right.
  \end{eqnarray*}
  then if $f (k, m) = f (k', m')$ we have the following cases for $m, m'$
  \begin{enumerate}
    \item then $(0, k \upl 1) = f (k, m) = f (k', m') = (0, k' \upl 1)
    \Rightarrow k \upl 1 = k' \upl 1 \Rightarrow k = k' \Rightarrowlim_{m = 0
    = m'} (k, m) = (k', m')$
    
    \item then $(k \upl 1, m \um 1) = f (k, m) = f (k', m') = (0, k' \upl 1)
    \Rightarrow k \upl 1 = 0 \Rightarrowlim_{0 \leqslant k \Rightarrow 0 < k
    \upl 1} 0 < 0$ a contradiction, this case can not occur
    
    \item then $(0, k \upl 1) = f (k, m) = f (k', m') = (k' \upl 1, m' \um 1)
    \Rightarrow 0 = k' \upl 1 \Rightarrowlim_{0 \leqslant k' \Rightarrow 0 <
    k' \upl 1} 0 > 0$ a contradiction, this case can not occur
    
    \item then $(k \upl 1, m \um 1) = f (k, m) = f (k', m') = (k' \upl 1, m'
    \um 1) \Rightarrow k \upl 1 = k' \upl 1 \wedge m \um 1 = m' \um 1
    \Rightarrow k = k' \wedge m = m' \Rightarrow (k, m) = (k', m')$
  \end{enumerate}
  this proves that $f : \mathbbm{N} \times \mathbbm{N} \rightarrow \mathbbm{N}
  \times \mathbbm{N}$ is a injective function. If now $f (k, m) = (0, 0)$ then
  we have the following cases to consider for $m$
  \begin{enumerate}
    \item then $(0, 0) = f (k, m) = (0, k \upl 1) \Rightarrow 0 = k \upl 1
    \Rightarrowlim_{0 \leqslant k \Rightarrow 0 < k \upl 1} 0 < 0$ a
    contradiction.
    
    \item then $(0, 0) = f (k, m) = (k \upl 1, m \um 1) \Rightarrow 0 = k + 1
    \Rightarrowlim_{0 \leqslant k \Rightarrow 0 < k \upl 1} 0 < 0$ a
    contradiction. 
  \end{enumerate}
  so we have $f (k, m) \neq (0, 0)$ or $(0, 0) \nin f (\mathbbm{N} \times
  \mathbbm{N})$.
  
  Using \ref{recursive injective function} there exists a injective function
  $\lambda : \mathbbm{N} \rightarrow \mathbbm{N} \times \mathbbm{N}$ such that
  \begin{enumerate}
    \item $\lambda (0) = (0, 0)$
    
    \item $\forall n \in \mathbbm{N} \vDash \lambda (n \upl 1) = f (\lambda
    (n))$
  \end{enumerate}
  We prove now the following\tmtextbf{ property (a)} of $\lambda$, assume that
  there exists a $n, m \in \mathbbm{N}$ such that $\lambda (n) = (0, m)$ then
  $\forall k \in \mathbbm{N}$ with $\exists l \in \mathbbm{N}$ such that $k +
  l = m$ we have $\lambda (n \upl k) = (k, l)$. The proof is by induction on
  $k$ so let $S_{n, m} = \{ k \in \mathbbm{N}| \tmop{if} \exists l \in
  \mathbbm{N} \tmop{such} \tmop{that} k \upl l = m \tmop{then} \lambda (n \upl
  k) = (k, l) \}$ then:
  \begin{enumerate}
    \item If $k = 0$ then if $l = m$ we have $k \upl l = m$ and $\lambda (n
    \upl k) = \lambda (n) = (0, m) = (k, l)$ so we have $0 \in S_{n, m}$
    
    \item Assume $k \in S_{n, m}$ then consider $k \upl 1$ if there exists a
    $l \in \mathbbm{N}$ such that $(k \upl 1) \upl l = m$ then $k \upl (l \upl
    1) = m$ and as $k \in S_{n, m}$ we have $\lambda (n \upl k) = (k, l \upl
    1)$, now as $0 \leqslant l \Rightarrow 0 < l \upl 1$ we have $\lambda (n
    \upl (k \upl 1)) = \lambda ((n \upl k) \upl 1) = f (\lambda (n \upl k)) =
    f (k, l + 1) = (k \upl 1, l)$ so $k \upl 1 \in S_{n, m}$
  \end{enumerate}
  Thus if $k \upl l = m \Rightarrow k \in \mathbbm{N}= S_{n, m} \Rightarrow
  \lambda (n \upl k) = (k, l)$
  
  
  
  We shown now by induction that $\lambda$ is surjective, so let $S = \{ n
  \in \mathbbm{N}| \forall (k, m) \in \mathbbm{N} \times \mathbbm{N} \vdash k
  + m = n \vDash \exists n' \in \mathbbm{N} \vdash \lambda (n') = (k, m) \} $.
  \begin{enumerate}
    \item If $n = 0$ take $k = m \in \mathbbm{N}$ is such that $k \upl m = 0$
    then we must have $k = 0 = m$ [if $m > 0$ then $0 = k \upl m > 0
    \Rightarrow 0 \neq 0$ and if $k > 0$ again $0 = k \upl m > 0 \Rightarrow 0
    \neq 0$] so that $\lambda (0) = (0, 0) = (k, m)$ and thus $0 \in S$.
    
    \item Assume now that $n \in S$ and consider $n \upl 1$. If $(k, m) \in
    \mathbbm{N} \times \mathbbm{N}$ is such that $k \upl m = n \upl 1$. we can
    have then the following cases for $k$
    \begin{enumerate}
      \item then $k = k \upl m = n \upl 1$ and thus $(k, m) = (0, m) = (0, n
      \upl 1) = f (n, 0)$. As $n \upl 0 = n \in S$ there exists a $n'' \in
      \mathbbm{N}$ such that $(n, 0) = \lambda (n'')$ and thus $\lambda (n''
      \upl 1) = f (\lambda (n')) = f (n, 0) = (k, m)$ so we have found a $n' =
      n'' \upl 1$ such that $\lambda (n') = (k, m)$
      
      \item then $0 < k \Rightarrow k \um 1$ exists and as $0 \leqslant m
      \Rightarrow 0 < m \upl 1$ we have $f (k \um 1, m \upl 1) = ((k \um 1)
      \upl 1, (m \upl 1) \um 1) = (k, m)$. Now $((k \upl m) - 1) \upl 0 = (n
      \upl 1) - 1 = n \in S$ so that $\exists n'' \in \mathbbm{N}$ such that
      $\lambda (n'') = ((k \upl m) \um 1, 0)$. So $\lambda (n'' \upl 1) = f
      ((k \upl m) - 1, 0) = (0, k \upl m)$. Using \tmtextbf{property a} of
      $\lambda$ we have then that $\lambda ((n'' \upl 1) \upl k) = (k, m)$ so
      we have found a $n' = (n'' \upl 1) \upl k$ such that $\lambda (n') = (k,
      m)$
    \end{enumerate}
    
  \end{enumerate}
  If thus $(k, m) \in \mathbbm{N} \times \mathbbm{N}$ then $n = k \upl m \in
  \mathbbm{N}= S \Rightarrow \exists n' \in \mathbbm{N} \vdash \lambda (n') =
  (k, m)$ proving that $\lambda : \mathbbm{N} \times \mathbbm{N} \rightarrow
  \mathbbm{N}$ is surjective and as we have by construction of $\lambda$ it is
  injective we have that $\lambda : \mathbbm{N} \times \mathbbm{N} \rightarrow
  \mathbbm{N}$ is a bijection and thus $\mathbbm{N} \approx \mathbbm{N} \times
  \mathbbm{N}$
\end{proof}

\begin{corollary}
  \label{product of enumerable sets is enumerable}If $A, B$ are
  \tmtextbf{denumerable} then $A \times B$ is denumerable
\end{corollary}

\begin{proof}
  If $A, B$ are denumerable then $\mathbbm{N} \approx A$ and $\mathbbm{N}
  \approx B$ we have by \ref{product of classes and equipotence} that $A
  \times B \approx \mathbbm{N} \times \mathbbm{N} \approx \mathbbm{N}$ so $A
  \times B \approx \mathbbm{N}$
\end{proof}

\begin{theorem}
  If $\{ A_i \}_{i \in B}$ is a \tmtextbf{denumerable} family of
  \tmtextbf{denumerable} sets (so $B$ is denumerable and $\forall i \in B$ we
  have that $A_i$ is a denumerable set) then $\bigcup_{i \in B} A_i$ is
  \tmtextbf{denumerable}.
\end{theorem}

\begin{proof}
  As $\{ A_i \}_{i \in B}$ is a denumerable family of denumerable sets there
  exists a graph $A$ with $\tmop{dom} (G) = B$ where $\mathbbm{N} \approx B$,
  so there is a bijection $f : \mathbbm{N} \rightarrow B$, and $\forall i \in
  B$ we have $A_i \approx \mathbbm{N} \Rightarrow \forall i \in B$ there
  exists a bijection $f_i : \mathbbm{N} \rightarrow A_i$. So define $\forall i
  \in B$ the non empty set (see \ref{B^A is a set if B and A are sets})
  $\mathcal{A}_i = \{ f \in A_i^{\mathbbm{N}} : f \tmop{is} a \tmop{bijection}
  \}$, then $\mathcal{A}= \bigcup_{i \in B} \mathcal{A}_i$ is a set and
  $\forall i \in B$ we have $\mathcal{A}_i \subseteq \mathcal{A}$. Using the
  axiom of choice (see \ref{axiom of choice}) we have then the existence of a
  choice function $c : \mathcal{P}' (\mathcal{A}) \rightarrow \mathcal{A}$
  with $c (\mathcal{A}_i) \in \mathcal{A}_i \Rightarrow c (\mathcal{A}_i)$ is
  a bijection from $\mathbbm{N} \rightarrow A_i$. Define now $F : \mathbbm{N}
  \times \mathbbm{N} \rightarrow \bigcup_{i \in B} A_i$ by $F (n, m) = c
  (\mathcal{A}_{f (n)}) (m)$ then $F$ is surjective. For if $y \in \bigcup_{i
  \in B} A_i$ then there exists a $i \in B$ such that $y \in A_i$ and as $f$
  is bijective there exists a $n \in \mathbbm{N}$ such that $f (n) = i$, also
  as $c (A_{f (n)}) : \mathbbm{N} \rightarrow A_{f (n)} = A_i$ there exists a
  $m \in \mathbbm{N}$ such that $c (A_{f (n)}) (m) = y$ and thus $F (n, m) = c
  (\mathcal{A}_{f (n)}) (m)$. As $\Nu \times \mathbbm{N} \approx \mathbbm{N}$
  there exists a bijection $\varphi : \mathbbm{N} \rightarrow \mathbbm{N}
  \times \mathbbm{N}$ and thus $F \circ \varphi : \mathbbm{N} \rightarrow
  \bigcup_{i \in B} A_i$ is a surjective function. Using \ref{A<less>~B and
  surjectivity} we have then that $\bigcup_{i \in B} A_i \approx E \subseteq
  \mathbbm{N}$. Using \ref{subsets of denumerable sets are finite or
  denumerable} we have that $E$ is finite or denumerable hence $\bigcup_{i \in
  B} A_i$ is either finite or denumerable. As $A_{f (1)} \subseteq \bigcup_{i
  \in B} A_i$ and $A_{f (1)}$ is
  denumerable$\Rightarrowlim_{\text{\ref{infinite sets have countable
  subset}}} A_{f (1)}$ is infinite$\Rightarrowlim_{\text{\ref{every set with a
  infinite subset is finite}}} \bigcup_{i \in B} A_i$ is not finite so
  $\bigcup_{i \in B} A_i$ is a denumerable set.
\end{proof}

\

\begin{corollary}
  \label{the union of two denumerable sets is denumerable}The union of two
  \tmtextbf{denumerable} sets is \tmtextbf{denumerable}.
\end{corollary}

\begin{proof}
  If $A, B$ are denumerable sets then there exists bijections $f : \mathbbm{N}
  \rightarrow A$ and $g : \mathbbm{N} \rightarrow B$. Define now $F : \{ 1, 2
  \} \times \mathbbm{N} \rightarrow A \bigcup B$ by $F (1, n) = f (n)$ and $F
  (2, n) = g (n)$ then $F$ is a surjection as if $x \in A \bigcup B$ we have
  either $x \in A \Rightarrowlim_{f \tmop{is} \tmop{bijective}} \exists n \in
  \mathbbm{N} \vdash x = f (n) = F (1, n)$ or $x \in B \Rightarrowlim_{g
  \tmop{is} \tmop{bijective}} \exists n \in \mathbbm{N} \vdash x = g (n) F (2,
  n)$. Now $\{ 1.2 \} \times \mathbbm{N} \subseteq \mathbbm{N} \times
  \mathbbm{N}$ then by \ref{subsets of denumerable sets are finite or
  denumerable} and the fact that $\mathbbm{N} \approx \mathbbm{N} \times
  \mathbbm{N}$ we have that $\{ 1, 2 \} \times \mathbbm{N}$ is either finite
  or denumerable, as $\mathbbm{N} \approx \{ 1 \} \times \mathbbm{N} \subseteq
  \{ 1, 2 \} \times \mathbbm{N}$ we have that $\{ 1, 2 \} \times \mathbbm{N}$
  is infinite and thus is denumerable. So there exists a bijection $\varphi :
  \mathbbm{N} \rightarrow \{ 1, 2 \} \times \mathbbm{N}$ and thus a surjection
  $F \circ \varphi : \mathbbm{N} \rightarrow A \bigcup B$ which means by
  \ref{A<less>~B and surjectivity} that $A \bigcup B \approx E \subseteq
  \mathbbm{N}$. Using \ref{subsets of denumerable sets are finite or
  denumerable} we have that $E$ is finite or denumerable hence $A \bigcup B$
  is either finite or denumerable, as $A \subseteq A \bigcup B$ is a infinite
  subset we must have that $A \bigcup B$ is not finite and thus that $A
  \bigcup B$ is denumerable.
\end{proof}

As also the union of two finite sets is finite (see \ref{union of two finite
sets is finite}) we have the following corollary of the above corollary.

\begin{corollary}
  \label{the union of two countable sets is countable}The union of two
  \tmtextbf{countable} sets is \tmtextbf{countable}.
\end{corollary}

\section{Some properties of countable sets}

Using the definition of \tmtextbf{countability} and \ref{subsets of
denumerable sets are finite or denumerable} we have the following theorem

\begin{theorem}
  Every subset of a \tmtextbf{denumerable} set is \tmtextbf{countable}.
\end{theorem}

As the subsets of finite sets are finite we have that the following is true.

\begin{corollary}
  every subset of a \tmtextbf{countable} set is \tmtextbf{countable}.
\end{corollary}

\begin{theorem}
  \label{conditions for countability}If $X$ is a non-empty set then the
  following are equivalent
  \begin{enumerate}
    \item $X$ is countable
    
    \item There exists a surjective function $f : \mathbbm{N} \rightarrow X$
    
    \item There exists a injective function $f : X \rightarrow \mathbbm{N}$
  \end{enumerate}
\end{theorem}

\begin{proof}[$1 \Rightarrow 2$][$X$ is finite][$X \tmop{is}
\tmop{infinite}$][$2 \Rightarrow 3$][$3 \Rightarrow 1$]
  
  \begin{enumerate}
    \item If $X$ is countable then we have the following possibilities
    \begin{enumerate}
      \item Then $\exists n \in \mathbbm{N}$ such that $n \approx X$ so there
      exists a bijection $f' : n = S_n \rightarrow X$, also as $X \neq
      \emptyset$ there exists a $x \in X$. Define now $f : \mathbbm{N}
      \rightarrow X$ by
      \begin{eqnarray*}
        f (n) & = & \left\{ \begin{array}{l}
          f' (n) \tmop{if} i < n (\tmop{or} i \in S_n)\\
          x \tmop{if} n \leqslant i
        \end{array} \right.
      \end{eqnarray*}
      which is surjective because $f'$ is surjective.
      
      \item Then $\mathbbm{N} \approx X \Rightarrow$there exists a bijection
      (thus surjection) between $\mathbbm{N}$ and $X$
    \end{enumerate}
    \item If a surjective function $f : \mathbbm{N} \rightarrow X$ exists then
    by \ref{surjective function implies injective function in opposite
    direction} there exists a injective function $g : X \rightarrow
    \mathbbm{N}$
    
    \item If $X$ is finite then it is of course countable. So assume that $X$
    is infinite. Then from the existence of the injective function $f : X
    \rightarrow \mathbbm{N}$ we have that $f : X \rightarrow f (X)$ is a
    bijection and thus $f (X)$ is infinite. From \ref{every subset of N is
    finite or denumerable} we have then that $f (X)$ is
    denumerable$\Rightarrow X \approx f (X) \approx \mathbbm{N} \Rightarrow X
    \approx \mathbbm{N}$ and thus $X$ is countable.
  \end{enumerate}
\end{proof}

\begin{lemma}
  If $n \in \mathbbm{N}$ then $n \times \mathbbm{N}$ and $\mathbbm{N} \times
  n$ are \tmtextbf{countable}. If $n \neq 0$ then $n \times \mathbbm{N}$ and
  $\mathbbm{N} \times n$ are denumerable.
\end{lemma}

\begin{proof}[injectivity][surjectivity]
  First \ $f : n \times \mathbbm{N} \rightarrow \mathbbm{N} \times n$ define
  by $(i, j) \Rightarrow f (i, j) = (j, i)$ is bijective
  \begin{enumerate}
    \item If $f (i, j) = f (i', j') \Rightarrow (j, i) = (j', i') \Rightarrow
    i = i' \wedge j = j' \Rightarrow (i, j) = (i', j')$
    
    \item If $(i, j) \in \mathbbm{N} \times n \Rightarrow i \in \mathbbm{N}
    \wedge j \in n \Rightarrow (j, i) \in n \times N$ and $f (j, i) = (i, j)$
  \end{enumerate}
  so we have $n \times \mathbbm{N} \approx \mathbbm{N} \times n$. We must thus
  only prove that $n \times \mathbbm{N}$ is denumerable. First $n \times
  \mathbbm{N}= S_n \times \mathbbm{N} \subseteq \mathbbm{N} \times
  \mathbbm{N}$ so we have by \ref{subsets of denumerable sets are finite or
  denumerable} that $n \times \mathbbm{N}$ is either finite or denumerable and
  is thus countable. If $n \neq 0$ then $0 < n = S_n \Rightarrow 0 \in n
  \Rightarrow \{ 0 \} \subseteq \mathbbm{N} \Rightarrow \{ 0 \} \times
  \mathbbm{N} \subseteq n \times \mathbbm{N}$. And as it is trivial to prove
  that $f : \mathbbm{N} \rightarrow \{ 0 \} \times \mathbbm{N}: i \rightarrow
  (0, i)$ is a bijection we have that $\{ 0 \} \times \mathbbm{N}$ is
  denumerable and thus by \ref{infinite sets have countable subset} it is
  infinite, so by \ref{every set with a infinite subset is finite} $n \times
  \mathbbm{N}$ is not finite and must thus be denumerable.
\end{proof}

\begin{theorem}
  If $A, B$ are countable sets then $A \times B$ is countable
\end{theorem}

\begin{proof}[$A \tmop{finite} \wedge B \tmop{finite}$][$A \tmop{finte} \wedge
B \tmop{denumerable}$][$A \tmop{denumerable} \wedge B \tmop{finite}$][$A
\tmop{denumerable} \wedge B \tmop{denumerable}$]
  If $A, B$ are countable sets then we have the following cases to consider
  \begin{enumerate}
    \item then $A \times B$ is finite by \ref{product of finite sets is
    finite} and thus countable.
    
    \item Then $\exists n \in \mathbbm{N}$ such that $A \approx n$ and $B
    \approx \mathbbm{N} \Rightarrow A \times B \approx n \times \mathbbm{N}$
    which by the above lemma is countable so we have that $A \times B$ is
    countable.
    
    \item Then $\exists n \in \mathbbm{N}$ such that $A \approx \mathbbm{N}$
    and $B \approx n \Rightarrow A \times B \approx \mathbbm{N} \times n$
    which by the above lemma is countable;e so we have that $A \times B$ is
    countable.
    
    \item Then by \ref{product of enumerable sets is enumerable} we have that
    $A \times B$ is denumerable and thus countable.
  \end{enumerate}
\end{proof}

\begin{lemma}
  If $\{ A_i \}_{i \in S_n}$ is a family of countable sets then $\bigcup_{i
  \in S_n} A_i$ is countable.
\end{lemma}

\begin{proof}[$i = n$][$i \neq n$][$x \in A_n$][$x \nin A_n$]
  We prove this by induction on $n$. So define $S = \left\{ n \in \mathbbm{N}|
  \tmop{if} \{ A_i \}_{i \in S_n} \tmop{is} a \tmop{family} \tmop{of}
  \tmop{countable} \tmop{sets} \tmop{then} \bigcup_{i \in S_n} A_i \tmop{is}
  \tmop{countable} \right\}$ then we have
  \begin{enumerate}
    \item If $n = 0$ take then $\{ A_i \}_{i \in S_0}$ \ then there exists a
    graph $A$ with $\tmop{dom} (A) = \emptyset$ so if $x \in \bigcup_{i \in
    S_0} A_i$ then there exists a $i \in S_0$ such that $x \in A_i$ which is a
    contradiction as $S_0 = \emptyset$ so we have that $\bigcup_{i \in S_0}
    A_i = \emptyset$ and thus finite and countable. So we have $0 \in S$
    
    \item Assume that $n \in S$ and take $n \upl 1$ then we have for the
    family of countable sets $\{ A_i \}_{i \in S_{n \upl 1}}$ that there
    exists a graph $A$ with $\tmop{dom} (A) = S_{n \upl 1}$ . If we take $A' =
    \{ (x, y) \in A|x \in S_n \}$ (so $\tmop{dom} (A') = S_n $) then
    $\bigcup_{i \in S_{n \upl 1}} A_i = A_n \bigcup \left( \bigcup_{i \in S_n}
    A'_i \right)$
    
    \begin{proof}
      
      \begin{enumerate}
        \item If $x \in \bigcup_{i \in S_{n \upl 1}} A_i$ then $\exists i \in
        \tmop{dom} (A) = S_{n \upl 1} \Rightarrow i < n \upl 1$ such that $x
        \in A_i$ then we have the following cases
        \begin{enumerate}
          \item then $x \in A_n \Rightarrow x \in A_n \bigcup \left(
          \bigcup_{i \in S_n} A'_i \right)$
          
          \item then $i < n$ [if $n \leqslant i \Rightarrowlim_{i \neq n} n <
          i < n \upl 1 \Rightarrow n \upl 1 \leqslant i < n + 1 \Rightarrow n
          \upl 1 \neq n \upl 1$ a contradiction] and thus $i \in S_n$. As $x
          \in A_i$ we have $(i, x) \in A \Rightarrowlim_{i \in S_n} (i, x) \in
          A' \Rightarrow x \in \bigcup_{i \in S_n} A'_i \Rightarrow x \in A_n
          \bigcup \left( \bigcup_{i \in S_n} A'_i \right)$
        \end{enumerate}
        in all cases we have $x \in A_n \bigcup \left( \bigcup_{i \in S_n}
        A'_i \right)$
        
        \item If $x \in A_n \bigcup \left( \bigcup_{i \in S_n} A'_i \right)$
        then we have the following cases
        \begin{enumerate}
          \item $\Rightarrow x \in \bigcup_{i \in S_{n \upl 1}} A_i$
          
          \item $\Rightarrow x \in \bigcup_{i \in S_n} A'_i \Rightarrow
          \exists i \in S_n \vdash x \in A'_i \Rightarrow (i, x) \in A'
          \subseteq A \Rightarrow x \in A_i \Rightarrow x \in \bigcup_{i \in
          S_{n \upl 1}} A_i$
        \end{enumerate}
        in all cases $x \in \bigcup_{i \in S_{n \upl 1}} A_i$
      \end{enumerate}
    \end{proof}
    
    As $n \in S$ we have $\bigcup_{i \in S_n} A'_i$ is countable and as by
    assumption $A_n$ is finite we have by \ref{the union of two countable sets
    is countable} that $\bigcup_{i \in S_{n \upl 1}} A_i = A_n \bigcup \left(
    \bigcup_{i \in S_n} A'_i \right)$ is finite and thus $n + 1 \in S$. 
  \end{enumerate}
  By induction we have then $S =\mathbbm{N}$ proving our theorem.
\end{proof}

\begin{lemma}
  \label{finite union of countable sets is countable}If $\{ A_i \}_{i \in I}$
  is a \tmtextbf{finite} family ($I$ is finite) of \tmtextbf{countable} sets
  then $\bigcup_{i \in I} A_i$ is \tmtextbf{countable}.
\end{lemma}

\begin{proof}
  As $I$ is finite we have the existence of a bijection $f : n = S_n
  \rightarrow I$ ($n \in \mathbbm{N}$) and thus using \ref{reindexing of a
  family} we have that $\bigcup_{i \in I} A_i = \bigcup_{j \in S_n} A_{f (j)}$
  \ which is countable by the previous lemma.
\end{proof}

\begin{theorem}
  \label{a denumerable family of countable sets is countable}If $\{ A_i \}_{i
  \in B}$ is a \tmtextbf{denumerable} family of \tmtextbf{countable} sets (so
  $B$ is denumerable and $\forall i \in B$ we have that $A_i$ is a
  \tmtextbf{countable} set) then $\bigcup_{i \in B} A_i$ is
  \tmtextbf{countable}.
\end{theorem}

\begin{proof}[$\exists m \in \mathbbm{N} \vdash c (A_{f (n)}) : m \rightarrow
A_{f (n)}$ is a bijection][$c (A_{f (n)}) : \mathbbm{N} \rightarrow A_{f (n)}$
is a bijection]
  As $\{ A_i \}_{i \in B}$ is a denumerable family of countable sets, there
  exists a graph $A$ with $\tmop{dom} (G) = B$ where $\mathbbm{N} \approx B$,
  so there is a bijection $f : \mathbbm{N} \rightarrow B$, and $\forall i \in
  B$ we have that $A_i$ is countable. So $\forall i \in B$ there exists a
  bijection $g : N \rightarrow A_i$ where $N$ is either $\mathbbm{N}$ or $n
  \in \mathbbm{N}$. So if we define $\forall i \in B$ the non empty set (see
  \ref{B^A is a set if B and A are sets}) $\mathcal{A}_i = \{ f \in
  A_i^{\mathbbm{N}} |f \tmop{is} a \tmop{bijection} \} \bigcup \left(
  \bigcup_{n \in \mathbbm{N}} \{ f \in A_i^n |f \tmop{is} a \tmop{bijection}
  \} \right)$, then $\mathcal{A}= \bigcup_{i \in B} \mathcal{A}_i$ is a set
  and $\forall i \in B$ we have $\mathcal{A}_i \subseteq \mathcal{A}$. Using
  the axiom of choice (see \ref{axiom of choice}) we have then the existence
  of a choice function $c : \mathcal{P}' (\mathcal{A}) \rightarrow
  \mathcal{A}$ so that \ $c (\mathcal{A}_i) \in \mathcal{A}_i \Rightarrow c
  (\mathcal{A}_i)$ is a bijection from $\mathbbm{N} \rightarrow A_i$ or a
  bijection from $n \rightarrow A_i$. Define now $F : \mathbbm{N} \times
  \mathbbm{N} \rightarrow \bigcup_{i \in B} A_i$ by $F (n, m) = c
  (\mathcal{A}_{f (n)}) (m)$.
  
  We can then prove that $F$ is surjective. For if $y \in \bigcup_{i \in B}
  A_i \Rightarrow \exists i \in B \vdash y \in A_i \Rightarrowlim_{f :
  \mathbbm{N} \rightarrow B \tmop{is} \tmop{bijective}} \exists n \in
  \mathbbm{N} \vdash f (n) = i \Rightarrow y \in A_{f (n)} $. We have now the
  following cases for $c (A_i) = c (A_{f (n)})$:
  \begin{enumerate}
    \item $\Rightarrow \exists j \in m \vdash c (A_{f (n)}) (j) = y
    \Rightarrow F (n, j) = c (A_{f (n)}) (j) = y$
    
    \item $\Rightarrow \exists j \in \mathbbm{N} \vdash c (A_{f (n)}) (j) = y
    \Rightarrow F (n, j) = c (A_{f (n)}) (j) = y$
  \end{enumerate}
  proving that $F$ is surjective.
  
  As $\Nu \times \mathbbm{N} \approx \mathbbm{N}$ there exists a bijection
  $\varphi : \mathbbm{N} \rightarrow \mathbbm{N} \times \mathbbm{N}$ and thus
  $F \circ \varphi : \mathbbm{N} \rightarrow \bigcup_{i \in B} A_i$ is a
  surjective function. Using \ref{A<less>~B and surjectivity} we have then
  that $\bigcup_{i \in B} A_i \approx E \subseteq \mathbbm{N}$. Using
  \ref{subsets of denumerable sets are finite or denumerable} we have that $E$
  is finite or denumerable hence $\bigcup_{i \in B} A_i$ is either finite or
  denumerable and thus countable.
\end{proof}

Using the previous lemma and the above theorem we have then the following
theorem.

\begin{theorem}
  \label{the union of a countable family of countable sets is countable}If $\{
  A_i \}_{i \in B}$ is a \tmtextbf{countable} family of \tmtextbf{countable}
  sets then we have that $\bigcup_{i \in B} A_i$ is countable.
\end{theorem}

\begin{proof}[$B$ is finite][$B$ is denumerable]
  If $\{ A_i \}_{i \in B}$ is a countable family of countable sets then we
  have either:
  \begin{enumerate}
    \item then by \ref{finite union of countable sets is countable} we have
    that $\bigcup_{i \in B} A_i$ is countable.
    
    \item then by \ref{the union of a countable family of countable sets is
    countable} we have that $\bigcup_{i \in B} A_i$ is countable.
  \end{enumerate}
\end{proof}

\

\section{Sequences}

\begin{definition}
  \label{sequence}{\index{sequence}}If $X$ is a set then a family $\{ x_i
  \}_{i \in \mathbbm{N}_0}$ of elements in $X$ is called a sequence.
\end{definition}

\section{Finite cartesian product of sets and finite product of set}

First some notations about finite product of sets (see \ref{generalized
product of sets})

\begin{notation}
  Let $n \in \mathbbm{N}_0$, $\{ A_i \}_{i \in \{ 1, \ldots, n \}}$ be a
  finite family of sets then we note $x \in \prod_{i \in \{ 1, \ldots, n \}}
  A_i$ by $x = (x_1, \ldots, x_n)$ meaning $x = \{ (1, x_1), \ldots, (n, x_n)
  \} \subseteq \{ 1, \ldots, n \} \times \left( \bigcup_{i \in \{ 1, \ldots, n
  \}} A_i \right)$ such that $\langle x, \{ 1, \ldots, n \} \rangle$ is a
  pretuple and $\forall i \in \{ 1, \ldots, n \}$ we have $x_i \in A_i$ (see
  the definition of $\prod_{i \in \{ 1, \ldots, n \}} A_i$ at \ref{generalized
  product of sets})
\end{notation}

The following theorem will be used in recursive arguments to deal with
subtuples.

\begin{theorem}
  \label{subtuple of product of sets}Let $n \in \mathbbm{N}_0$, $\{ A_i \}_{i
  \in \{ 1, \ldots, n + 1 \}}$ be a finite family of sets and $x \in \prod_{i
  \in \{ 1, \ldots, n + 1 \}}$ then if we take \ $x_{| \{ 1, \ldots, n \}}
  \equallim_{\text{\ref{subtuple}}} \{ (i, y) \in x|i \in \{ 1, \ldots, n \}
  \}$ we have $x_{| \{ 1, \ldots, n \}} \in \prod_{i \in \{ 1, \ldots, n \}}
  A_i$
\end{theorem}

\begin{proof}
  Using \ref{subtuple} we have that $\langle x_{| \{ 1, \ldots ., n \}}, \{ 1,
  \ldots, n \} \rangle$ is a pretuple. Also if $(i, y) \in x_{| \{ 1, \ldots,
  n \}} \Rightarrow (i, y) \in x \wedge i \in \{ 1, \ldots, n \}
  \Rightarrowlim_{x \in \prod_{i \in \{ 1, \ldots, n \}} A_i} (i, y) \in x
  \wedge i \in \{ 1, \ldots, n \} \wedge y \in A_i \Rightarrow (i, y) \in \{
  1, \ldots, n \} \times \left( \bigcup_{i \in \{ 1, \ldots, n \}} A_i
  \right)$ proving $x_{| \{ 1, \ldots, n \}} \subseteq \{ 1, \ldots, n \}
  \times \left( \bigcup_{i \in \{ 1, \ldots, n \}} \right)$. As also $\forall
  i \in \{ 1, \ldots, n \}$ we have $x_{| \{ 1, \ldots, n \}} (i)
  \equallim_{(i, x_{| \{ 1, \ldots, n \}} (i)) \in x_{| \{ 1, \ldots, n \}}
  \Rightarrow (i, x_{| \{ 1, \ldots, n \}}) \in x} x (i) \in A_i$ proving that
  $x_{| \{ 1, \ldots, n \}} \in \prod_{i \in \{ 1, \ldots, n \}} A_i$
\end{proof}

The next theorem shows how to extend a tuple

\begin{theorem}
  \label{super tuple}Let $n \in \mathbbm{N}_0$, $\{ A_i \}_{i \in \{ 1,
  \ldots, n + 1 \}}$ be a finite family of sets and consider $\{ A_i \}_{i \in
  \{ 1, \ldots, n \}}$ the finite subfamily formed by restriction to $\{ 1,
  \ldots, n \}$ then if $x \in \prod_{i \in \{ 1, \ldots, n \}} A_i$, $y \in
  A_{n + 1}$ define $x_{+ y} = x \bigcup \{ (n + 1, y) \}$ then $x_{+ y} \in
  \prod_{i \in \{ 1, \ldots, n + 1 \}} A_i$ and $(x_+)_{| \{ 1, \ldots, n \}}
  = x$.
\end{theorem}

\begin{proof}
  As $x \subseteq \{ 1, \ldots, n \} \times \left( \bigcup_{i \in \{ 1,
  \ldots, n \}} A_i \right)$ we have then if $(i, z) \in x_+$ either
  \begin{description}
    \item[$(i, z) \in x$] so that $(i, z) \in \{ 1, \ldots, n \} \times \left(
    \bigcup_{i \in \{ 1, \ldots, n \}} A_i \right)
    \subseteq_{\text{\ref{product of subclasses}}} \{ 1, \ldots, n + 1 \}
    \times \left( \bigcup_{i \in \{ 1, \ldots, n \}} A_{n + 1} \right)$
    
    \item[$(i, z) \in \{ (n + 1, y) \}$] so that $i = n + 1 \wedge z = y
    \Rightarrowlim_{y \in A_{n + 1}} i \in \{ 1, \ldots, n + 1 \} \wedge y \in
    \bigcup_{i \in \{ 1, \ldots, n + 1 \}} A_i \Rightarrow (i, z) \in \{ 1,
    \ldots, n + 1 \} \times \left( \bigcup_{i \in \{ 1, \ldots, n + 1 \}} A_i
    \right)$
  \end{description}
  so we have in all cases $(i, y) \in \{ 1, \ldots, n + 1 \} \times \left(
  \bigcup_{i \in \{ 1, \ldots, n + 1 \}} A_i \right)$ proving that
  \begin{equation}
    \label{eq 5.1.4} x_{+ y} \subseteq \{ 1, \ldots, n + 1 \} \times \left(
    \bigcup_{i \in \{ 1, \ldots, n + 1 \}} A_i \right) .
  \end{equation}
  Also if $(i, z), (i, z') \in x_{+ y}$ we have the following cases to
  consider then
  \begin{description}
    \item[$i \in \{ 1, \ldots, n \}$] then $(i, z), (i, z') \in x
    \Rightarrowlim_{\langle x, \{ 1, \ldots, n \} \rangle \tmop{is} a
    \tmop{pretuple}} z = z'$
    
    \item[$i = n + 1$] then $z = y = z'$
  \end{description}
  proving that $x_{+ y}$ is a function graph. If $i \in \{ 1, \ldots, n + 1
  \}$ then if $i = n + 1$ we have $(n + 1, y) \in x_{+ y}$ and if $i \in \{ 1,
  \ldots, n \}$ there exists as $\tmop{dom} (x) = \{ 1, \ldots, n \}$ a $z$
  such that $(i, z) \in x \subseteq x_{+ y}$ proving that $\tmop{dom} (x_{+
  y}) = \{ 1, \ldots, n + 1 \}$. So we have proved that
  \begin{equation}
    \label{eq 5.2.4} \langle x_{+ y}, \{ 1, \ldots, n + 1 \} \rangle \tmop{is}
    a \tmop{pretuple}
  \end{equation}
  If $i \in \{ 1, \ldots, n + 1 \}$ then either $i \in \{ 1, \ldots, n \}
  \Rightarrow (i, x_{+ y} (i)) \in x \Rightarrow x_{+ y} (i) = x (i) \in A_i$
  or $i = n + 1$ then $x_{+ y} (i) = x_{+ y} (n + 1) = y \in A_{n + 1}$ so
  that
  \begin{equation}
    \label{eq 5.3.4} \forall i \in \{ 1, \ldots, n + 1 \} \tmop{we}
    \tmop{have} x_{+ y} (i) \in A_i
  \end{equation}
  Using \ref{eq 5.1.4}, \ref{eq 5.2.4} and \ref{eq 5.3.4} we have that $x_{+
  y} \in \prod_{i \in \{ 1, \ldots, n + 1 \}} A_i$ as also $(x_{+ y})_{| [1,
  \ldots, n]} = \{ (i, z) \in x_{+ y} |i \in \{ 1, \ldots, n \} \} = x$ we
  have proved the theorem.
\end{proof}

Here we define a recursive version of the product of a finite family of sets
and show its relation with the normal product of a family of sets (see
\ref{generalized product of sets}).

\begin{definition}[finite cartesian product of sets]
  \label{cartesion product of a family of sets}{\index{cartesian product of a
  family of sets}}Let $n \in \mathbbm{N}_0$ and $\{ A_i \}_{i \in \{ 1,
  \ldots, n \}}$ a finite family of sets then we define $\bigotimes_{i \in \{
  1, \ldots, n \}} A_i$ recursively as follows:
  \begin{description}
    \item[$n = 1$] $\bigotimes_{i \in \{ 1, \ldots, 1 \}} A_i = A_1$
    
    \item[$n > 1$] $\bigotimes_{i \in \{ 1, \ldots, n \}} A_i = \left(
    \bigotimes_{i \in \{ 1, \ldots, n - 1 \}} A_i \right) \times A_n$ (see
    \ref{cartesian product})
  \end{description}
\end{definition}

\begin{example}
  Let $\{ A_i \}_{i \in \{ 1, \ldots, 3 \}}$ be a finite family of sets then
  $\bigotimes_{i \in \{ 1, \ldots, 3 \}} A_i = \left( \bigotimes_{i \in \{ 1,
  \ldots, 2 \}} A_i \right) \times A_3 = \left( \left( \bigotimes_{i \in \{ 1
  \ldots, 1 \}} A_i \right) \times A_2 \right) \times A_3 = (A_1 \times A_2)
  \times A_3$ and $x \in \bigotimes_{i \in \{ 1, \ldots 3 \}} A_i
  \Leftrightarrow \exists x_1 \in A_1, x_2 \in A_2, x_3 \in A_3$ so that $x =
  ((x_1, x_2), x_3)$
\end{example}

\begin{definition}[product tuple]
  \label{product tuple}{\index{product tuple}}Let $\{ A_i \}_{i \in \{ 1,
  \ldots, n \}}$ be a finite family of sets and $\{ x_i \}_{i \in \{ 1,
  \ldots, n \}}$ be a finite family of elements with $\forall i \in \{ 1,
  \ldots, n \} \vDash x_i \in A_i$ then we define $[x_1, \ldots, x_n]$
  recusively by
  \begin{description}
    \item[$n = 1$] $[x_1, \ldots, x_1] = x_1$
    
    \item[$n > 1$] $[x_1, \ldots, x_n] = ([x_1, \ldots ., x_{n - 1}], x_n)$
  \end{description}
\end{definition}

\begin{example}
  Take $\{ A_i \}_{i \in \{ 1, \ldots, 3 \}}$ and $\{ x_i \}_{i \in \{ 1,
  \ldots, 3 \}}$ then $[x_1, \ldots, x_3] = ([x_1, \ldots, x_2], x_3) =
  (([x_1, \ldots, x_1], x_2), x_3) = ((x_1, x_2), x_3)$
\end{example}

Next we show that esssential $\bigotimes_{i \in \{ 1, \ldots, n \}} A_i$ is a
collection of $[x_1, \ldots, x_n]$

\begin{theorem}
  Let $\{ A_i \}_{i \in \{ 1, \ldots, n \}}$ be a finite family of sets and
  let $\{ x_i \}_{i \in \{ 1, \ldots, n \}}$ be a finite family of elements
  with $x_i \in A_i$ then $[x_1, \ldots, x_n] \in \bigotimes_{i \in \{ 1,
  \ldots, n \}} A_i$. Also if $x \in \bigotimes_{i \in \{ 1, \ldots, n \}}
  A_i$ then there exists a unique $\{ x_i \}_{i \in \{ 1, \ldots, n \}}$ such
  that $[x_1, \ldots, x_n] = x$.
\end{theorem}

\begin{proof}
  We prove both parts of the theorem with mathematical induction. First let $S
  = \{ n \in \mathbbm{N}_0 | \tmop{if} \{ A_i \}_{i \in \{ 1, \ldots, n \}}
  \tmop{is} a \tmop{finite} \tmop{family} \tmop{of} \tmop{sets} \tmop{and} \{
  x_i \}_{i \in \{ 1, \ldots, n \}} \tmop{is} \tmop{such} \tmop{that} x_i \in
  A_i  \}$ then we have
  \begin{description}
    \item[$1 \in S$] let then $[x_1, \ldots, x_1] = x_1 \in A_1 =
    \bigotimes_{i \in \{ 1, \ldots, 1 \}} A_i$ proving that $1 \in S$
    
    \item[$n \in S \Rightarrow n + 1 \in S$] take $n \in S$ and let now $\{
    A_i \}_{i \in \{ 1, \ldots, n + 1 \}}$ and $\{ x_i \}_{i \in \{ 1, \ldots,
    n + 1 \}}$ with $x_i \in A_i$ then as $n \in S$ we have that $[x_1,
    \ldots, x_n] \in \bigotimes_{i \in \{ 1, \ldots, n \}} A_i$ so that $[x_1,
    \ldots, x_{n + 1}] = ([x_1, \ldots, x_n], x_{n + 1}) \in \left(
    \bigotimes_{i \in \{ 1, \ldots, n \}} A_i \right) \times A_{n + 1} =
    \bigotimes_{i \in \{ 1, \ldots, n + 1 \}} A_i$ proving that $n + 1 \in S$
  \end{description}
  Mathematical induction proves then the first part of the theorem.
  
  Next let $T = \left\{ n \in \mathbbm{N}_0 | \tmop{for} \tmop{every} \{ A_i
  \}_{i \in \{ 1, \ldots, n \}} \tmop{we} \tmop{have} \tmop{that} \forall x
  \in \bigotimes_{i \in \{ 1, \ldots, n \}} A_i \tmop{there} \tmop{exists} a
  \tmop{unique} \{ x_i \}_{i \in \{ 1, \ldots, n \}} \tmop{such} \tmop{that} x
  = [x_1, \ldots, x_n] \right\}$ then we have
  \begin{description}
    \item[$1 \in T$] let $\{ A_i \}_{i \in \{ 1, \ldots, 1 \}}$ be a finite
    family take then $x \in \bigotimes_{i \in \{ 1, \ldots, 1 \}} A_i = A_1$
    define then $\{ x_i \}_{i \in \{ 1, \ldots, n \}}$ by $x_1 = x$ then
    $[x_1, \ldots, x_1] = x_1$, if $\{ y_i \}_{i \in \{ 1, \ldots, n \}}$ is
    such that $[y_1, \ldots, y_1] = x \Rightarrow y_1 = x = x_1$ proving that
    $\forall i \in \{ 1, \ldots, n \}$ we have $x_i = y_i$ or that $\{ x_i
    \}_{i \in \{ 1, \ldots, 1 \}} = \{ y_i \}_{i \in \{ 1, \ldots, 1 \}}$. So
    we have $1 \in T$
    
    \item[$n \in T \Rightarrow n + 1 \in T$] take $n \in T$ and let $\{ A_i
    \}_{i \in \{ 1, \ldots, n + 1 \}}$ be a finite family and le $x \in
    \bigotimes_{i \in \{ 1, \ldots, n + 1 \}} A_i = \left( \bigotimes_{i \in
    \{ 1, \ldots, n \}} A_i \right) \times A_{n + 1} \Rightarrow \exists y \in
    \bigotimes_{i \in \{ 1, \ldots, n \}} A_i$ and $\exists z \in A_{n + 1}$
    such that $(y, z) \in \left( \bigotimes_{i \in \{ 1, \ldots, n \}} A_i
    \right) \times A_{n + 1} = \bigotimes_{i \in \{ 1, \ldots, n + 1 \}} A_i$.
    As $n \in T$ there exists a $\{ y_i \}_{i \in \{ 1, \ldots, n \}}$ such
    that $[y_1, \ldots, y_n] = y$, take now $\{ x_i \}_{i \in \{ 1, \ldots, n
    + 1 \}}$ defined by $x_i = \left\{ \begin{array}{l}
      y_i \tmop{if} i \in \{ 1, \ldots, n \}\\
      z \tmop{if} i = n + 1
    \end{array} \right.$ then $[x_1, \ldots, x_{n + 1}] = ([x_1, \ldots, x_n],
    x_{n + 1}) = ([y_1, \ldots, y_n], z) = (y, z) = x$. Also if $\{ z_i \}_{i
    \in \{ 1, \ldots, n + 1 \}}$ is such that $[z_1, \ldots, z_{n + 1}] = x =
    [x_1, \ldots, x_{n + 1}] \Rightarrow ([z_1, \ldots, z_n], z_{n + 1}) =
    ([y_1, \ldots, y_n], y_{n + 1}) \Rightarrow z_{n + 1} = x_{n + 1} \wedge
    [z_1, \ldots, z_n] = [x_1, \ldots, x_n] \Rightarrowlim_{n \in T} z_{n + 1}
    = x_{n + 1} \wedge \{ z_i \}_{i \in \{ 1, \ldots, n \}} = \{ x_i \}_{i \in
    \{ 1, \ldots, n \}} \Rightarrow \{ z_i \}_{i \in \{ 1, \ldots, n + 1 \}} =
    \{ x_i \}_{i \in \{ 1, \ldots, n + 1 \}}$. So we conclude that $n + 1 \in
    T$
  \end{description}
  Mathematical induction proves then the last part of the theorem.
\end{proof}

The above theorem proves that very element of $\bigotimes_{i \in \{ 1, \ldots,
\}} A_i$ can be written as $[x_1, \ldots, x_n]$

We now proof that there is a bijection between the finite cartesian product of
sets and the product of sets.

\begin{theorem}
  \label{finite cartesian product of sets and product of sets are
  bijective}Let $n \in \mathbbm{N}_0$ and let $\{ A_i \}_{i \in \{ 1, \ldots,
  n \}}$ be a finite family of sets then there exists a bijection
  $\mathcal{P}: \bigotimes_{i \in \{ 1, \ldots, n \}} A_i \rightarrow \prod_{i
  \in \{ 1, \ldots, n \}} A_i$ so that
  \begin{enumerate}
    \item If $\{ B_i \}_{i \in \{ 1, \ldots, \}}$ is such that $B_i \subseteq
    A_i$ then $\mathcal{P} \left( \bigotimes_{i \in \{ 1, \ldots, n \}} B_i
    \right) = \prod_{i \in \{ 1, \ldots, n \}} B_i$
    
    \item $\forall [x_1, \ldots, x_n] \in \bigotimes_{i \in \{ 1, \ldots, n
    \}} A_i$ we have $\mathcal{P} ([x_1, \ldots, x_n]) = (x_1, \ldots, x_n)$
  \end{enumerate}
  \begin{enumerate}
    \ 
  \end{enumerate}
\end{theorem}

\begin{proof}
  We prove this by mathematical induction so let $S = \left\{ n \in
  \mathbbm{N}_0 | \tmop{if} \{ A_i \}_{i \in \{ 1, \ldots, n \}} \tmop{is} a
  \tmop{finite} \tmop{family} \tmop{of} \tmop{sets} \tmop{then} \tmop{there}
  \tmop{exists} a \tmop{bijection} \mathcal{P}: \bigotimes_{i \in \{ 1,
  \ldots, n \}} A_i \rightarrow \prod_{i \in \{ 1, \ldots, n \}} A_i
  \tmop{such} \tmop{that} (1) \tmop{and} (2) \tmop{of} \tmop{the}
  \tmop{theorem} \tmop{is} \tmop{satisfied} \right\}$ then we have
  \begin{description}
    \item[$1 \in S$] If $n = 1$ then $\{ A_i \}_{i \in \{ 1, \ldots, 1 \}}$
    has only one member $A_1$ and $\bigotimes_{i \in \{ 1, \ldots, 1 \}} A_i =
    A_1$ let then $x \in A_1 = \bigotimes_{i \in \{ 1, \ldots, 1 \}} A_i$ and
    define $\mathcal{P} (x)$ by $\mathcal{P} (x) = \{ (1, x) \}$ then we have
    that
    \begin{equation}
      \label{eq 5.1.3} \mathcal{P} (x) \subseteq \{ 1 \} \times A_1 = \{ 1 \}
      \times \left( \bigcup_{i \in \{ 1, \ldots, 1 \}} A_i \right)
    \end{equation}
    If $(y, z), (y, z') \in \mathcal{P} (x) \Rightarrow y = 1 \wedge z = z'
    \Rightarrow z = z'$ and also $\tmop{dom} (\mathcal{P} (x)) = \{ 1 \}$
    proving that
    \begin{equation}
      \label{eq 5.2.3} \langle \mathcal{P} (x), \{ 1, \ldots, 1 \} \rangle
      \tmop{is} a \tmop{pretuple}
    \end{equation}
    Finally if $i \in \{ 1, \ldots, 1 \} = \{ 1 \}$ then $\mathcal{P} (x) (1)
    = x \in A_1$ giving
    \begin{equation}
      \label{eq 5.3.3} \forall i \in \{ 1, \ldots, 1 \} \vDash \mathcal{P} (x)
      (1) \in A_i
    \end{equation}
    Using the definition of $\prod_{i \in \{ 1, \ldots, 1 \}} A_i$ (see
    \ref{generalized product of sets}) and \ref{eq 5.1.3}, \ref{eq 5.2.3},
    \ref{eq 5.3.3} we have then that
    \begin{equation}
      \label{eq 5.4.3} \mathcal{P} (x) \in \prod_{i \in \{ 1, \ldots, 1 \}}
      A_i
    \end{equation}
    So $\mathcal{P}: \bigotimes_{i \in \{ 1, \ldots, 1 \}} A_i \rightarrow
    \prod_{i \in \{ 1, \ldots, 1 \}} A_i$ defined by $x \rightarrow
    \mathcal{P} (x)$ is indeed a function, lets prove now that it is a
    bijection
    \begin{description}
      \item[injectivity] If $\mathcal{P} (x) =\mathcal{P} (y) \Rightarrow
      \mathcal{P} (x) (1) =\mathcal{P} (y) (1) \Rightarrow x = y$ proving
      injectivity.
      
      \item[surjectivity] Let $y \in \prod_{i \in \{ 1, \ldots, 1 \}} A_i$
      then $y \subseteq \{ 1 \} \times \left( \bigcup_{i \in \{ 1, \ldots, 1
      \}} A_i \right) = \{ 1 \} \times A_1$ with $\langle y, \{ 1 \} \rangle$
      a pretuple. Let $x = y (1)$ then $\mathcal{P} (x) (1) = x = y (1)$ so
      that $\mathcal{P} (x) = y$ proving surjectivity.
    \end{description}
    Let $\{ B_i \}_{i \in \{ 1, \ldots, 1 \}}$ be such that $B_1 \subseteq
    A_1$ then if $x \in \mathcal{P} \left( \bigotimes_{i \in \{ 1, \ldots, n
    \}} B_i \right) =\mathcal{P} (B_1)$ we have that $\mathcal{P} (x) (1) = x
    \in B_1$ and thus $\mathcal{P} (x) = \{ (1, x) \} \subseteq \{ 1 \} \times
    B_1 = \{ 1, \ldots, 1 \} \times \left( \bigcup_{i \in \{ 1, \ldots, 1 \}}
    B_i \right)$ which together with \ref{eq 5.2.3} gives that by definition
    $\mathcal{P} (x) \in \prod_{i \in \{ 1, \ldots, 1 \}} B_i$ proving that
    $\mathcal{P} \left( \bigotimes_{i \in \{ 1, \ldots, 1 \}} B_i \right)
    \subseteq \prod_{i \in \{ 1, \ldots, 1 \}} B_i$. If $y \in \prod_{i \in \{
    1, \ldots, 1 \}} A_i$ then $\mathcal{P} (y (1)) = y$ (see surjectivity)
    and as $y (1) \in B_1$ we have $y \in \mathcal{P} (B_1) =\mathcal{P}
    \left( \bigotimes_{i \in \{ 1, \ldots, 1 \}} B_i \right)$ proving that
    $\prod_{i \in \{ 1, \ldots, 1 \}} B_i \subseteq \mathcal{P} (\otimes_{i
    \in \{ 1, \ldots, 1 \}} B_i)$. So we have proved that $\prod_{i \in \{ 1,
    \ldots, 1 \}} B_i =\mathcal{P} \left( \bigotimes_{i \in \{ 1, \ldots, 1
    \}} B_i \right)$ proving (1) of the theorem. Next if $[x_1, \ldots, x_1]
    \in \bigotimes_{i \in \{ 1, \ldots, 1 \}} A_i$ then $[x_1, \ldots, x_1] =
    x_1 \in A_1$ and $\mathcal{P} ([x_1, \ldots, x_1]) =\mathcal{P} (x_1)
    \equallim_{\mathcal{P} (x_1) (1) = x_1} (x_1)$ proving (2) of the theorem.
    So we have proved that $1 \in \mathcal{S}$.
    
    \item[$n \in S \Rightarrow n + 1 \in S$] Let $n \in S$ and take $\{ A_i
    \}_{i \in \{ 1, \ldots, n + 1 \}}$ then as $n \in S$ there exists a
    bijection $\mathcal{Q}: \bigotimes_{i \in \{ 1, \ldots, n \}} A_i
    \rightarrow \prod_{i \in \{ 1, \ldots, n \}} A_i$ satisfying (1) and (2)
    of the theorem. Let now $x \in \bigotimes_{i \in \{ 1, \ldots, n + 1 \}}
    A_i = \left( \bigotimes_{i \in \{ 1, \ldots, n \}} A_i \right) \times A_{n
    + 1}$ then $x = (y, z)$ with $y \in \bigotimes_{i \in \{ 1, \ldots, n \}}
    A_i$ and $z \in A_{n + 1}$ define then $\mathcal{P} (x) =\mathcal{P} ((y,
    z)) =\mathcal{Q} (y)_{+ z} \in \prod_{i \in \{ 1, \ldots, n + 1 \}} A_i$
    (see \ref{super tuple}).
    
    So $\mathcal{P}: \bigotimes_{i \in \{ 1, \ldots, n + 1 \}} A_i \rightarrow
    \prod_{i \in \{ 1, \ldots, n + 1 \}} A_i$ defined by $x \rightarrow
    \mathcal{P} (x)$ is indeed a well defined function. We prove now that it
    is a bijection
    \begin{description}
      \item[injectivity] Let $x_1, x_2 \in \bigotimes_{i \in \{ 1, \ldots, n +
      1 \}} A_i = \left( \bigotimes_{i \in \{ 1, \ldots, n \}} \right) \times
      X_{n + 1}$ so that $x_1 = (y_1, z_1)$, $x_2 = (y_2, z_2)$ then if
      $\mathcal{P} (x_1) =\mathcal{P} (x_2) \Rightarrow \mathcal{P} ((y_1,
      z_1)) =\mathcal{P} ((y_2, z_2)) \Rightarrow \mathcal{Q} (y_1)_{+ z_1}
      =\mathcal{Q} (y_2)_{+ z_2} \Rightarrow (\mathcal{Q} (y_1)_{+ z_1})_{| \{
      1, \ldots, n \}} = (\mathcal{Q} (y_2))_{+ z_2}
      \Rightarrowlim_{\text{\ref{super tuple}}} \mathcal{Q} (y_1) =\mathcal{Q}
      (y_2) \Rightarrow y_1 = y_2$, also from $\mathcal{Q} (y_1)_{+ z_1}
      =\mathcal{Q} (y_2)_{+ z_2} \Rightarrowlim \mathcal{Q} (y_1)_{+ z_1} (n +
      1) =\mathcal{Q} (y_2)_{+ z_2} (n + 1) \Rightarrowlim_{\text{\ref{super
      tuple}}} z_1 = z_2$. So we have $x_1 = (y_1, z_1) = (y_2, z_2) = x_2$
      proving injectivity.
      
      \item[surjectivity] Let $y \in \prod_{i \in \{ 1, \ldots, n + 1 \}} A_i$
      take then $y_{| \{ 1, \ldots, n \}} = \{ (i, x) \in y|i \in \{ 1,
      \ldots, n \} \}$ we have by \ref{subtuple of product of sets} $y_{| \{
      1, \ldots, n \}} \in \prod_{i \in \{ 1, \ldots, n \}} A_i$. Using the
      fact that $\mathcal{Q}$ is a bijection there exists a $x \in
      \bigotimes_{i \in \{ 1, \ldots, n \}} A_i$ such that $\mathcal{Q} (x) =
      y_{| \{ 1, \ldots, n \}}$ take now $(x, y (n + 1))$ then $\forall i \in
      \{ 1, \ldots, n + 1 \}$ we have
      \begin{description}
        \item[$i \in \{ 1, \ldots, n \}$] then $\mathcal{P} ((x, y (n + 1)))
        =\mathcal{Q} (x)_{+ y (n + 1)} (i) =\mathcal{Q} (x) (i) = y (i)$
        
        \item[$i = n + 1$] then $\mathcal{P} ((x, y (n + 1))) =\mathcal{Q}
        (x)_{+ y (n + 1)} (n + 1) = y (n + 1) = y (i)$
      \end{description}
      proving that $\forall i \in \{ 1, \ldots, n + 1 \}$ we have $\mathcal{P}
      ((x, y (n + 1))) = y (i) \Rightarrow \mathcal{P} ((x, y (n + 1))) = y$
      proving surjectivity.
      
      
    \end{description}
    Let now $\{ B_i \}_{i \in \{ 1, \ldots, n + 1 \}}$ be such that $B_i
    \subseteq A_i$ and let $x \in \bigotimes_{i \in \{ 1, \ldots, n + 1 \}}
    B_i$ then $x = (y, z)$ where $y \in \bigotimes_{i \in \{ 1, \ldots, n \}}
    B_i$ and $z \in B_{n + 1}$ so that $\mathcal{P} (x) =\mathcal{P} ((x,
    y))$. If $i \in \{ 1, \ldots, n \}$ then $\mathcal{P} ((x, y)) (i)
    =\mathcal{Q} (x)_{+ y} (i) =\mathcal{Q} (x) (i) \in B_i$ [as $\mathcal{Q}
    (x) \in \prod_{i \in \{ 1, \ldots, n \}} B_i$ because $n \in S$ and (1)]
    and $\mathcal{P} ((x, y)) (n + 1) = (\mathcal{Q} (x)_{+ y}) (n + 1) = y
    \in B_{n + 1}$ so that as $\langle \mathcal{P} (x), \{ 1, \ldots, n + 1 \}
    \rangle$ is a pretuple we have \ $\mathcal{P} (x) \in \prod_{i \in \{ 1,
    \ldots, n + 1 \}} B_i$. This proves that $\mathcal{P} \left( \bigotimes_{i
    \in \{ 1, \ldots, n \}} B_i \right) \subseteq \prod_{i \in \{ 1, \ldots, n
    + 1 \}} B_i$. If $y \in \prod_{i \in \{ 1, \ldots, n + 1 \}} B_i$ then
    using \ref{subtuple of product of sets} we have that $y_{| \{ 1, \ldots, n
    \}} \in \prod_{i \in \{ 1, \ldots, n \}} B_i$ so that there exists a $x
    \in \bigotimes_{i \in \{ 1, \ldots, n \}} B_i$ with $\mathcal{Q} (x) =
    y_{| \{ 1, \ldots, n \}}$ [using $n \in S$ and $(1)$ for $Q$] then we have
    that $\mathcal{P} ((x, y (n + 1))) =\mathcal{Q} (x)_{+ y (n + 1)} = y$
    where $(x, y (n + 1)) \in_{y (n + 1) \in B_{n + 1}} \left( \bigotimes_{i
    \in \{ 1, \ldots, n \}} B_i \right) \times B_{n + 1} = \bigotimes_{i \in
    \{ 1, \ldots, n + 1 \}} B_i$ proving that $\prod_{i \in \{ 1, \ldots, n +
    1 \}} B_i \subseteq \mathcal{P} \left( \bigotimes_{i \in \{ 1, \ldots, n +
    1 \}} B_i \right)$. So $\mathcal{P} \left( \bigotimes_{i \in \{ 1, \ldots,
    n + 1 \}} B_i \right) = \prod_{i \in \{ 1, \ldots, n + 1 \}} B_i$ proving
    (1) of the theorem. Finally if $[x_1, \ldots, x_{n + 1}] \in \bigotimes_{i
    \in \{ 1, \ldots, n + 1 \}} A_i$ then $[x_1, \ldots, x_{n + 1}] = ([x_1,
    \ldots, x_n], x_{n + 1})$ and $\forall i \in \{ 1, \ldots, n \} \vDash
    \mathcal{P} (([x_1, \ldots, x_n], x_{n + 1})) (i) =\mathcal{Q} ([x_1,
    \ldots, x_n])_{+ x_{n + 1}} (i) =\mathcal{Q} ([x_1, \ldots, x_n]) (i)
    \equallim_{n \in S} x_i = x (i)_{}$ and $\mathcal{P} (([x_1, \ldots, x_n],
    x_{n + 1})) (n + 1) =\mathcal{Q} ([x_1, \ldots, x_n])_{+ x_{n + 1}} (n +
    1) = x_{n + 1} = x (n + 1)$. So $\forall i \in \{ 1, \ldots, n + 1 \}$ we
    have $\mathcal{P} ([x_1, \ldots, x_{n + 1}]) (i) = x (i)$ proving that
    $\mathcal{P} ([x_1, \ldots, x_{n + 1}]) = (x_1, \ldots, x_{n + 1})$ so
    that we have also that (2) of the theorem is satisfied. This finished the
    proof that $n + 1 \in S$
  \end{description}
  Using mathematical induction we have then proved the theorem.
\end{proof}

\

\

\

\

\chapter{The Integer numbers}

\section{Definition and arithmetic's}

The problem with the natural numbers is that we have no inverse number for
every natural number, meaning that $\mathbbm{N}$ can not be a group. In other
words given $n \in \mathbbm{N}$ with $n \neq 0$ there does not exists a $n'$
such that $n + n' = 0$. This is solved by introducing the whole numbers.

\begin{definition}
  We define the relation $\sim$ on $\mathbbm{N} \times \mathbbm{N}$ by $\sim =
  \{ ((n, m), (n', m')) \in (\mathbbm{N} \times \mathbbm{N}) \times
  (\mathbbm{N} \times \mathbbm{N}) | n \upl m' = m \upl n' \nobracket \}$
\end{definition}

\begin{theorem}
  $\sim$ is a equivalence relation on $\mathbbm{N}$
\end{theorem}

\begin{proof}[reflectivity][Symmetry][Transitivity]
  
  \begin{enumerate}
    \item If $(n, m) \in \mathbbm{N} \times \mathbbm{N}$ then $n \upl m
    \equallim_{\text{\ref{natural numbers are commutative}}} m \upl n
    \Rightarrow (n, m) \sim (n, m)$
    
    \item If $(n, m) \sim (n', m') \Rightarrow n \upl m' = m \upl n'
    \Rightarrowlim_{\text{\ref{natural numbers are commutative}}} m' \upl n =
    n' \upl m \Rightarrow n' \upl m = m' \upl n \Rightarrow (n', m') \sim (n,
    m)$
    
    \item If $(n, m) \sim (n', m')$ and $(n', m') \sim (n'', m'') \Rightarrow
    n \upl m' = m \upl n'$ and $n' \upl m'' = m' \upl n'' \Rightarrow (n \upl
    m') \upl (n' \upl m'') = (m \upl n') \upl (m' + n'')
    \Rightarrowlim_{\text{\ref{natural numbers are commutative}, \ref{addition
    of natural numbers is associative}}} (n \upl m'') + (m' + n') = (m \upl
    n'') \upl (m' \upl n') \Rightarrowlim_{\text{\ref{n+k=m+k=<gtr>n=m}}} n
    \upl m'' = m \upl n'' \Rightarrow (n, m) \sim (n'', m'')$
  \end{enumerate}
\end{proof}

\begin{definition}
  \label{integer numbers}{\index{$\mathbbm{Z}$}}We define the set of integer
  numbers $\mathbbm{Z}$ by $\mathbbm{Z}=\mathbbm{N}/ \sim$ (see \ref{A/R R is
  a equivalence relation})
\end{definition}

So we have that $\mathbbm{Z}= \{ \sim [(n, m)] | (n, m) \in \mathbbm{N} \times
\mathbbm{N} \nobracket \}$

\begin{theorem}
  \label{~[(n,m)]=~[(n+k,m+k)]}If $\sim [(n, m)] \in \mathbbm{Z}$ then if $k
  \in \mathbbm{N}$ we have $\sim [(n, m)] = \sim [(n \upl k, m \upl k)]$
\end{theorem}

\begin{proof}
  $n \upl (m \upl k)$=$(n \upl m) \upl k = (m \upl n) \upl k = m \upl (n \upl
  k) \Rightarrow (n, m) \sim (n \upl k, m \upl k) \Rightarrow \sim [(n, m)] =
  \sim [(n \upl k, m \upl k)]$
\end{proof}

\begin{theorem}
  $+ : \mathbbm{Z} \times \mathbbm{Z} \rightarrow \mathbbm{Z}$ defined by
  $(\sim [(n, m)], \sim [r, s]) \rightarrow \sim [(n, m)] + \sim [(r, r)] =
  \sim [(n \upl r, m \upl s)]$ is a function. Note that we have here two
  additions $\upl$ one defined in $\mathbbm{N}$ the other in $\mathbbm{Z}$.
\end{theorem}

\begin{proof}
  We must prove that if $\sim [(n, m)] = \sim [(n', m')]$ and $\sim [(r, s)] =
  \sim [(r', s')]$ then $\sim [(n \upl r), (m \upl s)] = \sim [(n' \upl r', m'
  \upl s')]$. Now from $\sim [(n, m)] = \sim [(n', m')]$ and $\sim [(r, s)] =
  \sim [(r', s')]$ we have by \ref{condition for R[x]=R[y]} that $(n, m) \sim
  (n', m')$ and $(r, s) \sim (r', s') \Rightarrow n \upl m' = m \upl n'$ and
  $r \upl s' = s \upl r'$. So $(n \upl r) \upl (m' \upl s') = (n \upl m') \upl
  (r \upl s') = (m \upl n') \upl (s \upl r') = (m \upl s) \upl (n' \upl r')
  \Rightarrow (n \upl r, m \upl s) \sim (n' \upl r', m' + s') \Rightarrow \sim
  [(n \upl r, m \upl s)] = \sim [(n' \upl r', m' + s')]$
\end{proof}

\begin{theorem}[$\langle \mathbbm{Z}, \upl \rangle$ is a
group][Associativity][Neutral element][Commutativity][Inverse Element]
  \label{the integer numbers forms a group}We have the following properties
  for $+ : \mathbbm{Z} \times \mathbbm{Z} \rightarrow \mathbbm{Z}$
  \begin{enumerate}
    \item $\forall n, m, k \in \mathbbm{Z}$ we have $(x \upl y) \upl z = x
    \upl (y \upl z)$
    
    \item $\forall n \in \mathbbm{Z}$ there exists a $0 \in \mathbbm{Z}$ with
    $n \upl 0 = 0 \upl n = n$ (here $0 = \sim [(1, 1)]$, note that $0$ here is
    not the same as the $0$ in $\mathbbm{N}$, although the $1 \in
    \mathbbm{N}$)
    
    \item $\forall n, m \in \mathbbm{Z}$ then $n \upl m = m \upl n$
    
    \item $\forall n \in \mathbbm{Z}$ there exists a \tmtextbf{unique} element
    $\um n$ such that $(\um n) + n = 0 = n \upl (\um n)$. If $n = \sim [(n_1,
    n_2)] \Rightarrow \um n = \sim [(n_2, n_1)]$ (note as inverse elements are
    always unique we don't have to prove that $- n$ is defined independent of
    its representation).
  \end{enumerate}
\end{theorem}

\begin{proof}
  
  \begin{enumerate}
    \item If $n = \sim [(n_1, n_2)], m = \sim [(m_1, m_2)]$ and $k = \sim
    [(k_1, k_2)]$ then $(n \upl m) \upl k = \sim [(n_1 \upl m_1, n_2 \upl
    m_2)] + k = \sim [((n_1 \upl m_1) \upl k_1, (n_2 \upl m_2) \upl k_2)]
    \equallim_{\text{\ref{addition of natural numbers is associative}}} \sim
    [(n_1 \upl (m_1 \upl k_1), n_2 \upl (m_2 \upl k_2))] = \sim [(n_1, n_2)] +
    \sim [(m_1 \upl k_1, m_2 \upl k_2)] = n \upl ([m_1, m_2] \upl \sim [(k_1,
    k_2)]) = n \upl (m \upl k)$
    
    \item If $n = \sim [(n_1, n_2)]$ then $\sim [(1, 1)] + \sim [(n_1, n_1)] =
    \sim [(1 + n_1, 1 + n_1)]$ now $(1 \upl n_1) \upl n_2
    \equallim_{\text{\ref{addition of natural numbers is associative},
    \ref{natural numbers are commutative}}} (1 \upl n_2) \upl n_1 \Rightarrow
    (1 \upl n_1, 1 \upl n_2) \sim (n_1, n_2) \Rightarrow \sim [(1, 1)] + \sim
    [(n_1, n_1)] = \sim [(n_1, n_2)]$. Also $\sim [(n_1, n_2)] + \sim [(1, 1)]
    = \sim [(n_1 \upl 1, n_2 \upl 1)] \equallim_{\text{\ref{natural numbers
    are commutative}}} \sim [(1 \upl n_1, 1 \upl n_2)]
    \equallim_{\tmop{already} \tmop{proved}} \sim [(n_1, n_2)]$
    
    \item If $n = \sim [(n_1, n_2)]$ and $m = \sim [(m_1, m_2)]$ then $n \upl
    m = \sim [(n_1 \upl m_1, n_2 \upl m_2)] \equallim_{\text{\ref{natural
    numbers are commutative}}} \sim [(m_1 \upl n_1, m_2 \upl n_2)] = m \upl n$
    
    \item If $n = \sim [(n_1, n_2)]$ define then $\um n = \sim [(n_2, n_1)]$
    then $n \upl (\um n) = \sim [(n_1 \upl n_2, n_2 \upl n_1)]
    \equallim_{\text{\ref{natural numbers are commutative}}} \sim [(n_1 \upl
    n_2, n_1 \upl n_2)]$ now $(n_1 \upl n_2) \upl 1 = (n_1 \upl n_2) \upl 1
    \Rightarrow (n_1 \upl n_2, n_1 \upl n_2) = \sim [(1, 1)] \Rightarrow n +
    (\um n) = 0$ and by $(3)$ we have $(\um n) \upl n = 0$
  \end{enumerate}
\end{proof}

\begin{theorem}
  \label{0=~[(n,n)]}If $n \in \mathbbm{N}$ then $0 = \sim [(n, n)]$
\end{theorem}

\begin{proof}
  $0 = \sim [(1, 1)]$ and $n \upl 1 = n \upl 1 \Rightarrow (1, 1) \sim (n, n)
  \Rightarrow \sim [(n, n)] = \sim [(1, 1)]$
\end{proof}

\begin{theorem}
  $\cdot : \mathbbm{Z} \times \mathbbm{Z} \rightarrow \mathbbm{Z}$ defined by
  $(\sim [(m, n)], \sim [(k, r)]) \rightarrow \sim [(m, n)] \cdot \sim [(k,
  r)] = \sim [(m \cdot k + n \cdot r, n \cdot k \upl m \cdot r)]$ is a
  function. Note that we have two multiplications $\cdot$ here one defined in
  $\mathbbm{N}$ the other in $\mathbbm{Z}$.
\end{theorem}

\begin{proof}
  If $(m, n) \sim (m', n')$ and $(k, r) \sim (k', r')$ then we have $m \upl n'
  = n \upl m' \wedge k \upl r' = r \upl k'$ \ so we have then
  \begin{eqnarray*}
    (m \upl n') \cdot [(r \upl k') \upl (r \upl k')] & = & (m \upl n') \cdot
    (r \upl k') \upl (m \upl n') \cdot (r \upl k')\\
    & = & [(m \upl n') \upl (m \upl n')] \cdot (r \upl k')\\
    & \Leftrightarrow & \\
    (m \upl n') \cdot [(r \upl k') \upl (k \upl r')] & = & [(m \upl n') \upl
    (n \upl m')] \cdot (r \upl k')\\
    & \Leftrightarrow & \\
    (m \upl n') \cdot [k \upl k' \upl r' + r] & = & [n \upl m \upl m' + n']
    \cdot (r \upl k')\\
    & \Leftrightarrow & \\
    (m \upl n') \cdot k \upl (m \upl n') \cdot k' \upl (m' \upl n) \cdot r'
    \upl (m' \upl n) \cdot r & = & n \cdot (r' \upl k) \upl m \cdot (r \upl
    k') \upl m' \cdot (r \upl k') \upl n' \cdot (r' \upl k)\\
    m \cdot k + n \cdot r + n' \cdot k' \upl m' \cdot r' \upl (n' \cdot k \upl
    m' \cdot r \upl n \cdot r' \upl m \cdot k') & = & n \cdot k \upl m \cdot r
    \upl m' \cdot k' \upl n' \cdot r' \upl (n' \cdot k \upl m' \cdot r \upl n
    \cdot r' \upl m \cdot k')\\
    & \Leftrightarrow & \\
    m \cdot k + n \cdot r + n' \cdot k' \upl m' \cdot r' & = & n \cdot k \upl
    m \cdot r \upl m' \cdot k' \upl n' \cdot r'\\
    & \Leftrightarrow & \\
    (m \cdot k \upl n \cdot r) \upl (n' \cdot k' \upl m' \cdot r') & = & (n
    \cdot k \upl m \cdot r) \upl (m' \cdot k' \upl n' \cdot r')\\
    & \Leftrightarrow & \\
    (m \cdot k \upl n \cdot r, n \cdot k \upl m \cdot r) & \sim & (m' \cdot k'
    \upl n' \cdot r', n' \cdot k' \upl m' \cdot r')\\
    \sim [(m \cdot k \upl n \cdot r, n \cdot k \upl m \cdot r)] & = & \sim
    [(m' \cdot k' \upl n' \cdot r', n' \cdot k' \upl m' \cdot r')]
  \end{eqnarray*}
  proving that our definition is independent of the representation and that
  $\cdot$ is a function.
  
  \ 
\end{proof}

\begin{theorem}[$\langle \mathbbm{Z}, . \rangle$ is a abelian
semi-group][Commutativity][Associativity][Neutral
Element][Distributivity][There does not exists a zero divisor]
  \label{properties of multiplication of integer numbers}If $n, m, k \in
  \mathbbm{Z}$ then we have
  \begin{enumerate}
    \item $n \cdot m = m \cdot n$
    
    \item $n \cdot (m \cdot k) = (n \cdot m) \cdot k$
    
    \item There exist a $1 \equallim \sim [(2, 1)]$ such that $n \cdot 1 = 1
    \cdot n = n$ (note that we have two $1$ here a $1 \in \mathbbm{Z}$ and a
    $1 \in \mathbbm{N}$)
    
    \item $n \cdot (m \upl k) = n \cdot m \upl n \cdot k$
    
    \item $n \cdot m = 0 \Rightarrow n = 0 \vee m = 0$
    
    \item $(- 1) \cdot (- 1) = 1$
  \end{enumerate}
  Note that the symbol $1$ is here used as the unit of $\mathbbm{N}$ and
  $\mathbbm{Z}$, of course these units are different objects, but context will
  always tell if $1 \in \mathbbm{N}$ or $1 = \sim [(2, 1)] \in \mathbbm{Z}$.
\end{theorem}

\begin{proof}
  Let $n = \sim [(n_1, n_2)]$, $m = \sim [(m_1, m_2)]$ and $k = \sim [(k_1,
  k_2)]$ then we have
  \begin{enumerate}
    \item $n \cdot m = \sim [(n_1, n_2)] \cdot \sim [(m_1, m_2)] = \sim [(n_1
    \cdot m_1 \upl n_2 \cdot m_2, n_2 \cdot m_1 \upl n_1 \cdot m_2)]
    \equallim_{\text{\ref{multiplication of natural numbers is commutative}}}
    \sim [(m_1 \cdot n_1 \upl m_2 \cdot n_2, m_1 \cdot n_2 \upl m_2 \cdot
    n_1)] =$
    
    \item $n \cdot (m \cdot k) = n \cdot \sim [(m_1 \cdot k_1 \upl m_2 \cdot
    k_2, m_2 \cdot k_1 \upl m_1 \cdot k_2)] = \sim \left[ \left( n_1 \cdot
    (m_1 \cdot k_1 \upl m_2 \cdot k_2) \upl n_2 \cdot (m_2 \cdot k_1 \upl m_1
    \cdot k_2), n_2 \cdot (m_1 \cdot k_1 \upl m_2 \cdot k_2) \upl n_1 \cdot
    (m_2 \cdot k_1 \upl m_1 \cdot k_2) \right) \right] = \sim \left[ \left(
    (n_1 \cdot m_1) \cdot k_1 \upl (n_1 \cdot m_2) \cdot k_2 \upl (n_2 \cdot
    m_2) \cdot k_1 \upl (n_2 \cdot m_1) \cdot k_2, (n_2 \cdot m_1) \cdot k_1
    \right) \upl (n_2 \cdot m_2) \cdot k_2 \upl (n_1 \cdot m_2) \noplus \cdot
    k_1 \upl (n_1 \cdot m_1) \cdot k_2 \right] = \sim \left[ \left( (n_1 \cdot
    m_1 + n_2 \cdot m_2) \cdot k_1 \upl (n_1 \cdot m_2 + n_2 \cdot m_1) \cdot
    k_2, (n_2 \cdot m_1 \upl n_1 \cdot m_2) \cdot k_1 \upl (n_2 \cdot m_2 \upl
    n_1 \cdot m_1) \cdot k_2 \right) \right] = \sim [(n_1 \cdot m_1 + n_2
    \cdot m_2, n_1 \cdot m_2 + n_2 \cdot m_1)] \cdot \sim [(k_1, k_2)] = (\sim
    [(n_1, n_2)] \cdot \sim [(m_1, m_2)]) \cdot \sim [(k_1, k_2)] = (n \cdot
    m) \cdot k$
    
    \item $1 \cdot n = \sim [2, 1] \cdot \sim [(n_1, n_2)] = \sim [(2 \cdot
    n_1 \upl 1 \cdot n_2, 1 \cdot n_1 \upl 2 \cdot n_2)] = \sim [((1 \upl 1)
    \cdot n_1 \upl n_2, n_1 \upl (1 \upl 1) \cdot n_2)] = \sim [(n_1 \upl n_1
    \upl n_2, n_1 \upl n_2 \upl n_2)] = \sim [(n_1, n_2)] = n$ as $n_1 \upl
    n_1 \upl n_2 \upl n_2 = n_1 \upl n_2 \upl n_2 \upl n_1$. The fact that $n
    \cdot 1 = n$ follows from (1).
    
    \item $n \cdot (m + k) = n \cdot \sim [(m_1 \upl k_1, m_2 \upl k_2)] =
    \sim [(n_1 \cdot (m_1 \upl k_1) \upl n_2 \cdot (m_2 \upl k_2), n_2 \cdot
    (m_1 \upl k_1) \upl n_1 \cdot (m_2 \upl k_2))] = \sim [(n_1 \cdot m_1 \upl
    n_1 \cdot k_1 \upl n_2 \cdot m_2 \upl n_2 \cdot k_2, n_2 \cdot m_1 \upl
    n_2 \cdot k_1 \upl n_1 \cdot m_2 \upl n_1 \cdot k_2)] = \sim [((n_1 \cdot
    m_1 \upl n_2 \cdot m_2 \upl n_1 \cdot k_1 \upl n_2 \cdot k_2, n_1 \cdot
    m_2 \upl n_2 \cdot m_1 \upl n_2 \cdot k_1 \upl n_1 \cdot k_2))] = \sim
    [(n_1 \cdot m_1 \upl n_2 \cdot m_2, n_1 \cdot m_2 \upl n_2 \cdot m_1)] +
    \sim [(n_1 \cdot k_1 \upl n_2 \cdot k_2, n_2 \cdot k_1 \upl n_1 \cdot
    k_2)] = \sim [(n_1, n_2)] \cdot \sim [(m_1, m_2)] + \sim [(n_1, n_2)]
    \cdot \sim [(k_1, k_2)] = n \cdot m + n \cdot k$
    
    \item If $n \cdot m = 0$ then $\sim [(n_1 \cdot m_1 \upl n_2 \cdot m_2,
    n_2 \cdot m_1 \upl n_1 \cdot m_2)] = \sim [(1, 1)] \Rightarrow$
    \begin{eqnarray*}
      n_1 \cdot m_1 \upl n_2 \cdot m_2 \upl 1 & = & n_2 \cdot m_1 \upl n_1
      \cdot m_2 \upl 1\\
      & \Rightarrowlim_{\text{\ref{n+k=m+k=<gtr>n=m}}} & \\
      n_1 \cdot m_1 \upl n_2 \cdot m_2 & = & n_2 \cdot m_1 \upl n_1 \cdot m_2
    \end{eqnarray*}
    Suppose now that $\sim [(n_1, n_2)] \neq \sim [(1, 1)] \Rightarrow n_1
    \upl 1 \neq n_2 \upl 1$ then if $n_1 = n_2 \Rightarrow n_1 \upl 1 = n_2
    \upl 1$ a contradiction so we have $n_1 \neq n_2$. So the following cases
    are possible
    \begin{enumerate}
      \item $n_1 < n_2 \Rightarrowlim_{\text{\ref{n<less>m=<gtr>n+k=m}}}
      \exists l \in \mathbbm{N}_0 \Rightarrow n_1 \upl l = n_2 \Rightarrow n_1
      \cdot m_1 \upl (n_1 \upl l) \cdot m_2 = (n_1 \upl l) \cdot m_1 \upl n_1
      \cdot m_2 \Rightarrow l \cdot m_2 + (n_1 \cdot m_1 + n_1 \cdot m_2) = l
      \cdot m_1 + (n_1 \cdot m_1 \upl n_1 \cdot m_2)
      \Rightarrowlim_{\text{\ref{n+k=m+k=<gtr>n=m}}} l \cdot m_1 = l \cdot m_2
      \Rightarrowlim_{\text{\ref{elemination of non zero common factor in
      natural numbers}}} m_1 = m_2 \Rightarrow m_1 \upl 1 = m_2 \upl 1
      \Rightarrow \sim [(m_1, m_2)] = \sim [(1, 1)] = 0$
      
      \item $n_2 < n_1 \Rightarrowlim_{\text{\ref{n<less>m=<gtr>n+k=m}}}
      \exists l \in \mathbbm{N}_0 \Rightarrow n_2 \upl l = n_1 \Rightarrow
      (n_2 \upl l) \cdot m_1 \upl n_2 \cdot m_2 = n_2 \cdot m_1 \upl (n_2 \upl
      l) \cdot m_2 \Rightarrow l \cdot m_1 + (n_2 \cdot m_1 + n_2 \cdot m_2) =
      l \cdot m_2 + (n_2 \cdot m_1 \upl n_2 \cdot m_2)
      \Rightarrowlim_{\text{\ref{n+k=m+k=<gtr>n=m}}} l \cdot m_2 = l \cdot m_1
      \Rightarrowlim_{\text{\ref{elemination of non zero common factor in
      natural numbers}}} m_2 = m_1 \Rightarrow m_2 \upl 1 = m_1 \upl 1
      \Rightarrow \sim [(m_1, m_2)] = \sim [(1, 1)] = 0$
    \end{enumerate}
    \item $(- 1) \cdot (- 1) = \sim [(1, 2)] \cdot \sim [(1, 2)] = \sim [(1
    \cdot 1 \upl 2 \cdot 2, 2 \cdot 1 \upl 1 \cdot 2)] = \sim [(5, 4)] = \sim
    [(2, 1)] = 1$
  \end{enumerate}
\end{proof}

From \ref{the integer numbers forms a group} and \ref{properties of
multiplication of integer numbers} we have that

\begin{theorem}
  \label{integer numbers form a ring}$\langle \mathbbm{Z}, +, \cdot \rangle$
  is a \tmtextbf{ring} even more $\langle \mathbbm{Z}, +, \cdot \rangle$ forms
  a {\tmstrong{integral domain}}
\end{theorem}

\begin{notation}
  Whenever we write $a \um b$ we actually mean $a \upl (\um b)$ (where $- b$
  is the inverse of $b$ in the group $\langle \mathbbm{Z}, \upl \rangle$.
\end{notation}

\begin{theorem}[Absorbing element]
  \label{absorbing element of the integers}{\index{absorbing element of
  integers}}If $n \in \mathbbm{Z}$ then $0 \cdot n = 0 = n \cdot 0$
\end{theorem}

\begin{proof}
  If $n = \sim [(n_1, n_2)]$ then we have
  \begin{eqnarray*}
    0 \cdot n & = & \sim [(1, 1)] \cdot \sim [(n_1, n_2)]\\
    & = & \sim [(1 \cdot n_1 \upl 1 \cdot n_2, 1 \cdot n_1 \noplus + 1 \cdot
    n_2)]\\
    & = & \sim [(n_1 \upl n_2, n_1 \upl n_2)]\\
    & \equallim_{\text{\ref{~[(n,m)]=~[(n+k,m+k)]}}} & \sim [(n_1, n_1)] =
    \sim [(0 \upl n_1, 0 \upl n_1)]\\
    & \equallim_{\text{\ref{~[(n,m)]=~[(n+k,m+k)]}}} & \sim [(0, 0)]\\
    & \equallim_{\text{\ref{~[(n,m)]=~[(n+k,m+k)]}}} & \sim [(0 \upl 1, 0
    \upl 1)]\\
    & \equallim_{\text{\ref{~[(n,m)]=~[(n+k,m+k)]}}} & \sim [(1, 1)]\\
    & = & 0
  \end{eqnarray*}
  The remaining is proved by commutativity of $\cdot$
  
  \ 
\end{proof}

\begin{lemma}
  \label{-n=(-1).n n is integer}If $n \in \mathbbm{Z}$ then $\um n = (\um 1)
  \cdot n$
\end{lemma}

\begin{proof}
  First $1 = \sim [(2, 1)] \Rightarrow \um 1 = \sim [(1, 2)]
  \equallim_{\text{\ref{~[(n,m)]=~[(n+k,m+k)]}}} \sim [(0, 1)]$. Now if $n \in
  \mathbbm{Z} \Rightarrow n = \sim [(n_1, n_2)]$ then we have
  \begin{eqnarray*}
    (\um 1) \cdot n & = & \sim [(0 \cdot n_1 \upl 1 \cdot n_2, 0 \cdot n_2
    \upl 1 \cdot n_1)]\\
    & = & \sim [(n_2, n_1)]\\
    & = & \um n
  \end{eqnarray*}
  
\end{proof}

\begin{theorem}
  \label{-(n.m)=(-n).m=n.(-m) n,m are integers}If $n, m \in \mathbbm{Z}$ then
  $\um (n \cdot m) = (\um n) \cdot m = n \cdot (\um m)$
\end{theorem}

\begin{proof}
  $\um (n \cdot m) = (\um 1) \cdot (n \cdot m) = ((- 1) \cdot n) \cdot m
  \equallim_{\text{\ref{-n=(-1).n n is integer}}} (\um n) \cdot m$, also $- (n
  \cdot m) = - (m \cdot n) = (\um m) \cdot n = n \cdot (\um m)$
\end{proof}

\begin{theorem}
  \label{n.r=k,r=<gtr>n=k and n,k,r integers}If $n, k, r \in \mathbbm{Z}$ and
  $r \neq 0$ then we have $n \cdot r = k \cdot r \Rightarrow n = k$
\end{theorem}

\begin{proof}
  From $n \cdot r = k \cdot r$ we get $n \cdot r \upl (\um (k \cdot r)) = k
  \cdot r \upl (\um (k \cdot r)) = 0
  \Rightarrowlim_{\text{\ref{-(n.m)=(-n).m=n.(-m) n,m are integers}}} n \cdot
  r \upl (\um k) \cdot r = 0 \Rightarrowlim_{\text{\ref{properties of
  multiplication of integer numbers}}} (n \upl (\um k)) \cdot r = 0
  \Rightarrowlim_{\text{\ref{properties of multiplication of integer numbers}
  and $r \neq 0$}} n \upl (\um k) = 0 \Rightarrow (n \upl (\um k)) \upl k = k
  \Rightarrow n \upl ((\um k) \upl k) = k \Rightarrow n \upl 0 = k \Rightarrow
  n = k$
\end{proof}

\subsection{Power in $\mathbbm{Z}$}

\begin{definition}
  As $\langle \mathbbm{Z}, \cdot \rangle$ is a abelian semi-group we have by
  \ref{iteration over a group} that given a $a \in \mathbbm{Z}$ and $n \in
  \mathbbm{N}$ that there exists a $a^n$ such that
  \begin{eqnarray*}
    a^0 & = & 1\\
    a^{n \upl 1} & = & a^n \cdot a \equallim_{\tmop{abelian}} a \cdot a^n
  \end{eqnarray*}
\end{definition}

\begin{theorem}
  If $n, n' \in \mathbbm{N}$ and $a \in \mathbbm{Z}$ then $a^{n' \upl n} =
  a^{n'} \cdot a^n$
\end{theorem}

\begin{proof}
  We prove this by induction on $n$. So let $X = \{ n \in \mathbbm{N}|a^{n'
  \upl n} = a^{n'} \cdot a^n \}$ then we have
  \begin{enumerate}
    \item If $n = 0$ then $a^{n' \upl n} = a^{n' \upl 0} = a^{n'} = a^{n'}
    \cdot 1 = a^{n'} \cdot a^0 \Rightarrow 0 \in X$
    
    \item If $n \in X$ then $a^{n' \upl (n \upl 1)} = a^{(n' \upl n) \upl 1} =
    a^{(n' \upl n)} \cdot a \equallim_{n \in X} (a^{n'} \cdot a^n) \cdot a =
    a^{n'} \cdot (a^n \cdot a) = a^{n'} \cdot a^{n \upl 1}$ and thus $n \upl 1
    \in X$
  \end{enumerate}
  Using mathematical induction (see \ref{mathematical induction}) we have $X
  =\mathbbm{N}$ proving the theorem
\end{proof}

\begin{theorem}
  \label{power in the whole numbers}In $\mathbbm{Z}$ we have
  \begin{eqnarray*}
    0^n & = & 0 (\tmop{if} n \neq 0)\\
    1^n & = & 1\\
    (- 1)^n & = & - 1 \tmop{or} 1\\
    (- 1)^{2 \cdot n} & = & 1\\
    (- 1)^{2 \cdot n \upl 1} & = & - 1
  \end{eqnarray*}
\end{theorem}

\begin{proof}
  
  \begin{enumerate}
    \item If $n \neq 0 \Rightarrow \exists m \in \mathbbm{N} \vdash n = m \upl
    1$ then $0^n = 0^{(m \upl 1)} = 0^m \cdot 0 = 0$
    
    \item $1^n = 1$ is proved by induction on $n$, let $X = \{ n \in
    \mathbbm{N}|1^n = 1 \}$ then
    \begin{enumerate}
      \item $1^0 = 1 \Rightarrow 0 \in X$
      
      \item If $n \in X \Rightarrow 1^{n \upl 1} = 1^n \cdot 1 \equallim_{n
      \in X} 1 \cdot 1 = 1 \Rightarrow n \upl 1 \in X$
    \end{enumerate}
    so $X =\mathbbm{N}$
    
    \item $(- 1)^n = \pm 1$ is proved by induction on $n$, let $X = \{ n \in
    \mathbbm{N}| (- 1)^n = - 1 \tmop{or} 1 \}$ then
    \begin{enumerate}
      \item $(- 1)^0 = 1 \Rightarrow 0 \in X$
      
      \item If $n \in X$ then $(- 1)^{n \upl 1} = (- 1)^n \cdot (- 1)
      \equallim_{n \in X} (- 1) \cdot (- 1) \vee 1 \cdot (- 1) = 1 \vee - 1
      \Rightarrow n \upl 1 \in X$
    \end{enumerate}
    so $X =\mathbbm{N}$
    
    \item $(- 1)^{2 \cdot n} = (- 1)^{(1 \upl 1) \cdot n} = (- 1)^{n \upl n} =
    (- 1)^n \cdot (- 1)^n \equallim_{(3)} (- 1) \cdot (- 1) \tmop{or} 1 \cdot
    1 = 1$
    
    \item $(- 1)^{2 \cdot n \upl 1} = (- 1)^{2 \cdot n} \cdot (- 1)
    \equallim_{(4)} 1 \cdot (- 1) = - 1$
  \end{enumerate}
\end{proof}

\

\section{Order relation of the Integers}

\begin{definition}
  {\index{$\mathbbm{N}_{\mathbbm{Z}}$}}{\index{positive whole
  numbers}}$\mathbbm{N}_{\mathbbm{Z}} = \{ \sim [(s (n), 1)] | n \in
  \mathbbm{N} \nobracket \} \subseteq \mathbbm{Z}$,
  $\mathbbm{N}_{\mathbbm{Z}}$ is called the collection of positive whole
  numbers.
\end{definition}

\begin{note}
  $0 = \sim [(1, 1)] \in \mathbbm{N}_{\mathbbm{Z}}, 1 = \sim [(2, 1)] \in
  \mathbbm{N}_{\mathbbm{Z}}$
\end{note}

\begin{theorem}
  \label{properties of positive numbers}We have the following
  \begin{enumerate}
    \item $\langle \mathbbm{N}_{\mathbbm{Z}}, + \rangle$ is a sub-semi-group
    of $\langle \mathbbm{Z}, + \rangle$
    
    \item $\langle \mathbbm{N}_{\mathbbm{Z}}, . \rangle$ is a sub-semi-group
    of $\langle \mathbbm{Z}, . \rangle$
    
    \item $i_{\mathbbm{N}} : \mathbbm{N} \rightarrow
    \mathbbm{N}_{\mathbbm{Z}}$ defined by $n \rightarrow \sim ([s (n), 1])$
    forms a
    \begin{enumerate}
      \item group isomorphism between $\langle \mathbbm{N}, + \rangle
      \rightarrow \langle \mathbbm{N}_{\mathbbm{Z}}, + \rangle$
      
      \item group isomorphism between $\langle \mathbbm{N}, \cdot \rangle
      \rightarrow \langle \mathbbm{N}_{\mathbbm{Z}}, \cdot \rangle$
    \end{enumerate}
    \item For every $z \in \mathbbm{Z}$ we have $\exists x, y \in
    \mathbbm{N}_z$ such that $z = x \upl (\um y)$
  \end{enumerate}
\end{theorem}

As $0, 1 \in \mathbbm{N}_{\mathbbm{Z}}$ we have by \ref{sub-semi-group is a
semi-group} that $\langle \mathbbm{N}_{\mathbbm{Z}}, + \rangle$ and $\langle
\mathbbm{N}_{\mathbbm{Z}}, \cdot \rangle$ are semi-groups

\begin{proof}[Injectivity][Surjectivity]
  
  \begin{enumerate}
    \item We have the following
    \begin{enumerate}
      \item $n, m \in \mathbbm{N}_{\mathbbm{Z}}$ then \ $\exists n', m' \in
      \mathbbm{N}$ such that $n = \sim [(s (n'), 1)]$, $m = \sim [(s (m'),
      1)]$ then $n \upl m = \sim [(s (n') \upl s (m'), 1 \upl 1)] = \sim [((n'
      \upl m' \upl 1) \upl 1, 1 \upl 1)]
      \equallim_{\text{\ref{~[(n,m)]=~[(n+k,m+k)]}}} \sim [(n' \upl m' \upl 1,
      1)] = \sim [(s (n' \upl m'), 1)] \in \mathbbm{N}_Z$
      
      \item $0 = \sim [(1, 1)] = \sim [(s (0), 1)] \in \mathbbm{N}_Z$
    \end{enumerate}
    using \ref{sub-semi-group} we have our proof.
    
    \item We have the following
    \begin{enumerate}
      \item $n, m \in \mathbbm{N}_{\mathbbm{Z}}$ then $\exists n', m' \in
      \mathbbm{N}$ such that $n = \sim [(s (n'), 1)]$, $m = \sim [(s (m'),
      1)]$. we have then that $n \cdot m = \sim [(s (n') \cdot s (m') \upl 1
      \cdot 1, 1 \cdot s (m') \upl 1 \cdot s (m'))] = \sim [s (n') \cdot s
      (m') \upl 1, s (m') \upl s (n')] \sim [(n' \upl 1) (m' \upl 1) \upl 1,
      n' \upl 1 \upl m' \upl 1] = \sim [(n' \cdot m' \upl n' \upl m' \upl 1,
      n' \upl m' \upl 1)] \equallim_{\text{\ref{~[(n,m)]=~[(n+k,m+k)]}}} \sim
      [(n' \cdot m' \upl 1, 1)] \in \mathbbm{N}_{\mathbbm{Z}}$
      
      \item $1 = \sim [(2, 1)] = \sim [(s (1), 1)] \in
      \mathbbm{N}_{\mathbbm{Z}}$
    \end{enumerate}
    so using \ref{sub-semi-group} we have our proof.
    
    \item First we prove that the function $i_{\mathbbm{N}} : \mathbbm{N}
    \rightarrow \mathbbm{N}_{\mathbbm{Z}}$ is a bijection
    \begin{enumerate}
      \item if $i_{\mathbbm{N}} (n) = i_{\mathbbm{N}} (m) \Rightarrow \sim [(s
      (n), 1)] = \sim [s (m), 1] \Rightarrow (s (n), 1) \sim (s (m), 1)
      \Rightarrow s (m) \upl 1 = s (n) \upl 1
      \Rightarrowlim_{\text{\ref{n+k=m+k=<gtr>n=m}}} s (m) = s (n)
      \Rightarrowlim_{\text{\ref{if successors are equal numbers are equal}}}
      m = n$
      
      \item If $n \in \mathbbm{N}_{\mathbbm{Z}}$ then there exists a $n' \in
      \mathbbm{N} \vdash n = \sim [(s (n'), 1)] = i_{\mathbbm{N}} (n')$
    \end{enumerate}
    Next we prove that $i_{\mathbbm{N}} (n + m) = i_{\mathbbm{N}} (n) +
    i_{\mathbbm{N}} (m)$ and $i_{\mathbbm{N}} (n \cdot m) = i_{\mathbbm{N}}
    (n) \cdot i_{\mathbbm{N}} (m)$
    \begin{enumerate}
      \item $i_{\mathbbm{N}} (n \upl m) = \sim [(s (n \upl m), 1)]
      \equallim_{\text{\ref{~[(n,m)]=~[(n+k,m+k)]}}} \sim [(s (n \upl m) \upl
      1, 1 \upl 1)] = \sim [(((n \upl m) \upl 1) \upl 1, 1 \upl 1)] = \sim
      [((n \upl 1) \upl (m \upl 1), 1 \upl 1)] = \sim [(n \upl 1, 1)] + \sim
      [(m \upl 1, 1)] = \sim [(s (n), 10)] \upl \sim [(s (m), 1)] =
      i_{\mathbbm{N}} (n) \upl i_{\mathbbm{N}} (m)$
      
      \item $i_{\mathbbm{N}} (n \cdot m) = \sim [(s (n \cdot m), 1)] = \sim
      [(n \cdot m \upl 1, 1)] \equallim_{\text{\ref{~[(n,m)]=~[(n+k,m+k)]}}}
      \sim [(n \cdot m, 0)] = \sim [(n, 0)] \cdot \sim [(m, 0)]
      \equallim_{\text{\ref{~[(n,m)]=~[(n+k,m+k)]}}} \sim [(n \upl 1, 1)] \upl
      \sim [(m \upl 1, 1)] = \sim [(s (n), 1)] \upl \sim [(s (m), 1)] =
      i_{\mathbbm{N}} (n) \cdot i_{\mathbbm{N}} (m)$
    \end{enumerate}
    \item If $z \in \mathbbm{Z}$ then $z = \sim [(n, m)] = \sim [(n, 0)] \upl
    \sim [(0, m)] = \sim [(n, 0)] \upl (\um (\sim [(m, 0)])]
    \equallim_{\text{\ref{~[(n,m)]=~[(n+k,m+k)]}}} \sim [(s (n), 1)] + (-
    (\sim [(s (m), 1)])) = x \upl (\um y)$ where $x = \sim [(s (n), 1)] \in
    \mathbbm{N}_{\mathbbm{Z}}$ and $y = \sim [(s (m), 1)] \in
    \mathbbm{N}_{\mathbbm{Z}}$
  \end{enumerate}
\end{proof}

\begin{definition}
  {\index{$-\mathbbm{N}_{\mathbbm{Z}}$}}{\index{negative whole numbers}}$\um
  \mathbbm{N}_{\mathbbm{Z}} = \{ \um n | n \in \mathbbm{N}_{\mathbbm{Z}}
  \nobracket \}$, $\um \mathbbm{N}_{\mathbbm{Z}}$ is the collection of
  negative numbers.
\end{definition}

\begin{lemma}
  \label{whole numbers is union of positive and negative
  numbers}$\mathbbm{Z}=\mathbbm{N}_{\mathbbm{Z}} \bigcup
  (-\mathbbm{N}_{\mathbbm{Z}})$, so the whole numbers is the union of the
  positive and negative numbers. 
\end{lemma}

\begin{proof}
  First as $\mathbbm{N}_{\mathbbm{Z}} \subseteq \mathbbm{Z}$ and thus
  $-\mathbbm{N}_{\mathbbm{Z}} \subseteq \mathbbm{Z}$ we have
  $\mathbbm{N}_{\mathbbm{Z}} \bigcup (-\mathbbm{N}_{\mathbbm{Z}}) \subseteq
  \mathbbm{Z}$. Next assume that $z \in \mathbbm{Z}$ then by \ref{properties
  of positive numbers} we have there exists a $x, y \in
  \mathbbm{N}_{\mathbbm{Z}}$ such that $z = x \upl (\um y)$. Using
  \ref{properties of positive numbers} again there exists $x', y' \in
  \mathbbm{N}$ with $x = i_{\mathbbm{N}} (x') \nocomma$ and $y =
  i_{\mathbbm{N}} (y')$ where $i_{\mathbbm{N}}$ is the isomorphism defined by
  $i_{\mathbbm{N}} : \mathbbm{N} \rightarrow \mathbbm{N}_{\mathbbm{Z}}$ $n
  \rightarrow f (n) = \sim [(s (n), 1)]$. We have then the following cases to
  consider ($\mathbbm{N}, \leqslant$ is fully ordered)
  \begin{enumerate}
    \item $x' = y'$ then $x = i_{\mathbbm{N}} (x') = i_{\mathbbm{N}} (y') = y
    \Rightarrow z = x + (\um x) = 0 = \sim [(1, 1)] = \sim [(s (0), 1)] \in
    \mathbbm{N}_{\mathbbm{Z}} \Rightarrow z \in \mathbbm{N}_{\mathbbm{Z}}
    \bigcup (-\mathbbm{N}_{\mathbbm{Z}})$
    
    \item $x' < y'$ then by \ref{n<less>m=<gtr>n+k=m} there exists a $k' \in
    \mathbbm{N}_0$ such that $y' = x' + k'$. Take then $k = i_{\mathbbm{N}}
    (k') \in \mathbbm{N}_{\mathbbm{Z}}$, by the linearity of $i_{\mathbbm{N}}$
    we have then $y = i_{\mathbbm{N}} (y') = i_{\mathbbm{N}} (x' \upl k') =
    i_{\mathbbm{N}} (x') \upl i_{\mathbbm{N}} (k') = x + k$ and thus $z = x
    \upl (- y) = x \upl (- 1) \cdot y = x \upl (\um 1) \cdot (x \upl k) = x
    \upl ((\um x) \upl (\um k)) = \um k \in \um \mathbbm{N}_{\mathbbm{Z}}
    \Rightarrow z \in \um \mathbbm{N}_{\mathbbm{Z}} \Rightarrow z \in
    \mathbbm{N}_{\mathbbm{Z}} \bigcup (\um \mathbbm{N}_{\mathbbm{Z}})$
    
    \item $y' < x'$ then by \ref{n<less>m=<gtr>n+k=m} there exists a $k' \in
    \mathbbm{N}_0$ such that $x' = y' \upl k'$, if $k = i_{\mathbbm{N}} (k')
    \in \mathbbm{N}_{\mathbbm{Z}}$ we have then $x = i_{\mathbbm{N}} (x') =
    i_{\mathbbm{N}} (y') + i_{\mathbbm{N}} (k') = y + k$ and thus $z = x \upl
    (- y) = y \upl k \upl (- y) = k \in \mathbbm{N}_{\mathbbm{Z}} \Rightarrow
    z \in \mathbbm{N}_{\mathbbm{Z}} \Rightarrow z \in
    \mathbbm{N}_{\mathbbm{Z}} \bigcup (\um \mathbbm{N}_{\mathbbm{Z}})$
  \end{enumerate}
\end{proof}

We can now define a order relation on $\mathbbm{Z}$.

\begin{definition}[Order relation in $\mathbbm{Z}$]
  \label{order relation on the integer numbers}{\index{order relation in
  $\mathbbm{Z}$}}We define $\leqslant \subseteq \mathbbm{Z} \times
  \mathbbm{Z}$ by $\leqslant = \{ (n, m) | m \upl (\um n) \in
  \mathbbm{N}_{\mathbbm{Z}} \nobracket \}$ in other words $n \leqslant m$ iff
  $m \upl (\um n) \in \mathbbm{N}_{\mathbbm{Z}}$
\end{definition}

\begin{theorem}
  \label{whole numbers are fully-ordered}$\langle \mathbbm{Z}, \leqslant
  \rangle$ is a partially ordered set that is fully-ordered
\end{theorem}

\begin{proof}[reflectivity][anti--symmetry][transitivity][fully ordering]
  \
  \begin{enumerate}
    \item If $n \in \mathbbm{Z}$ then $n \upl (\um n) = 0 \equallim \sim [(1,
    1)] = \sim [(s (0), 1)] \in \mathbbm{N}_{\mathbbm{Z}}$
    
    \item If $n, m \in \mathbbm{Z}$, $n \leqslant m$ and $m \leqslant n$ then
    $m \upl (\um n) \in \mathbbm{N}_{\mathbbm{Z}}$ and $n \upl (\um m) \in
    \mathbbm{N}_{\mathbbm{Z}}$. As $n, m \in \mathbbm{Z}$ we have $n = \sim
    [(n_1, n_2)]$ and $m = \sim [(m_1, m_2)]$ then
    \begin{eqnarray*}
      n \upl (\um m) & = & \sim [(n_1 \upl m_2, n_2 \upl m_1)] = \sim [(s (k),
      1)] \tmop{where} k \in \mathbbm{N}\\
      m \upl (\um n) & = & \sim [(m_1 \upl n_2, m_2 \upl n_1)] = \sim [(s (l),
      1)] \tmop{where} l \in \mathbbm{N}
    \end{eqnarray*}
    we have then
    \begin{eqnarray*}
      n_1 \upl m_2 \upl 1 & = & n_2 \upl m_1 \upl s (k)\\
      & = & n_2 \upl m_1 \upl k \upl 1\\
      & \Leftrightarrow & \\
      n_1 \upl m_2 & = & n_2 \upl m_1 \upl k\\
      m_1 \upl n_2 \upl 1 & = & m_2 \upl n_1 \upl s (l)\\
      & = & m_2 \upl n_1 \upl l \upl 1\\
      & \Leftrightarrow & \\
      m_1 \upl n_2 & = & m_2 \upl n_1 \upl l
    \end{eqnarray*}
    summing the two equalities we get
    \begin{eqnarray*}
      n_1 \upl m_2 \upl m_1 \upl n_2 & = & n_1 \upl m_2 \upl m_1 \upl n_2 \upl
      k \upl l\\
      & \Leftrightarrowlim_{\text{\ref{n+k=m+k=<gtr>n=m}}} & \\
      0 & = & k \upl l
    \end{eqnarray*}
    Using \ref{n+k=0=<gtr>n=k=0 if n,k are natural numbers} we have then that
    $k = l = 0$ but then we have from $n_1 \upl m_2 = n_2 \upl m_1 \upl k$
    that $n_1 \upl m_2 = n_2 \upl m_1$ or $(n_1, n_2) \sim (m_1, m_2)
    \Rightarrowlim \sim [(n_1, n_2)] = \sim [(m_1, m_2)] \Rightarrow n = m$
    
    \item If $n, m, k \in \mathbbm{Z}$ then if $n \leqslant m$ and $m
    \leqslant k$ we have $m \upl (\um n) \in \mathbbm{N}_{\mathbbm{Z}}$ and $k
    \upl (\um m) \in \mathbbm{N}_{\mathbbm{Z}}$ then as we have that
    $\mathbbm{N}_{\mathbbm{Z}}$ is a sub-semi-group we have that
    $\mathbbm{N}_{\mathbbm{Z}} \ni (k \upl (\um m)) \upl (m \upl (\um n)) = k
    \upl ((\um m) \upl m) \upl (\um n) = k \upl 0 \upl (\um n) = k \upl (\um
    n) \Rightarrow k \leqslant n$
    
    \item If $n, m \in \mathbbm{Z}$ then by \ref{whole numbers is union of
    positive and negative numbers} we have for $n + (\um m) \in \mathbbm{Z}$
    that either
    \begin{enumerate}
      \item $n \upl (\um m) \in \mathbbm{N}_{\mathbbm{Z}} \Rightarrow m
      \leqslant n$
      
      \item $n \upl (\um m) \in \um \mathbbm{N}_{\mathbbm{Z}} \Rightarrow - (n
      \upl (\um m)) \in \mathbbm{N}_{\mathbbm{Z}} \Rightarrow m \upl (\um n)
      \in \mathbbm{N}_{\mathbbm{Z}} \Rightarrow n \leqslant m$ proving fully
      ordering.
    \end{enumerate}
  \end{enumerate}
\end{proof}

\begin{theorem}
  \label{inverse and order in integers}If $n, m \in \mathbbm{Z}$ with $n
  \leqslant m$ (or $n < m$) then $\um m \leqslant \um n$ (or $- m < \um n$
  
  \begin{proof}
    
    \begin{enumerate}
      \item If $n \leqslant m$ then $m \upl (\um n) \in
      \mathbbm{N}_{\mathbbm{Z}} \Rightarrow (- n) \noplus \noplus \upl (- (-
      m)) = (- n) + m = m + (- n) \in \mathbbm{N}_{\mathbbm{Z}} \Rightarrow -
      m \leqslant - n$
      
      \item If $n < m \Rightarrow n \leqslant m \wedge n \neq m \Rightarrow -
      m \leqslant - n \wedge - n \neq - m \Rightarrow - m < - n$
    \end{enumerate}
  \end{proof}
\end{theorem}

\begin{theorem}
  \label{condition 1 for positive integers}If $n \in \mathbbm{Z}$ then we have
  $n \in \mathbbm{N}_{\mathbbm{Z}} \Leftrightarrow 0 \leqslant n$. Note that
  this is the motivation to call $\mathbbm{N}_{\mathbbm{Z}}$ the set of
  positive numbers.
\end{theorem}

\begin{proof}
  
  \begin{eqnarray*}
    0 \leqslant z & \Leftrightarrow & n \upl (\um 0) \in
    \mathbbm{N}_{\mathbbm{Z}}\\
    & \Leftrightarrowlim_{\text{\ref{-n=(-1).n n is integer}}} & n \upl (\um
    1) \cdot 0 \in \mathbbm{N}_{\mathbbm{Z}}\\
    & \Leftrightarrowlim_{\text{\ref{absorbing element of natural numbers}}}
    & n \upl 0 \in \mathbbm{N}_{\mathbbm{Z}}\\
    & \Leftrightarrow & n \in \mathbbm{N}_{\mathbbm{Z}}
  \end{eqnarray*}
  
\end{proof}

\begin{theorem}
  \label{condition 2 for positive integers}If $n \in \mathbbm{Z}$ so $n = \sim
  [(n_1, n_2)]$ then $0 < n$ iff $n_2 < n_1$
\end{theorem}

\begin{proof}
  
  
  $\Rightarrow$
  
  Assume that $0 < n$ then $n \neq 0$ and $0 \leqslant n$ so $0 \neq 0$ and
  $n \in \mathbbm{N}_{\mathbbm{Z}} \Rightarrow \exists m \in \mathbbm{N}
  \vdash n = \sim [(n_1, n_2)] \sim [(s (m), 1)] \Rightarrow n_1 \upl 1 = n_2
  \upl s (m) = n_2 \upl (m + 1) = (n_2 \upl m) \upl 1 \Rightarrow n_1 = n_2
  \upl m$. Now if $m = 0 \Rightarrow n_1 = n_2 \Rightarrow n = \sim [(n_1,
  n_1)] \equallim_{\text{\ref{0=~[(n,n)]}}} 0 \neq n$ a contradiction so we
  have $m \in \mathbbm{N}_0$ but then by \ref{n<less>m=<gtr>n+k=m} we have
  $n_2 < n_1$.
  
  $\Leftarrow$ If $n_2 < n_1 \Rightarrowlim_{\text{\ref{n<less>m=<gtr>n+k=m}}}
  \exists m \in \mathbbm{N}_0 \vdash n_1 = n_2 \upl m \Rightarrow n_1 \upl 1 =
  n_2 \upl s (m) \Rightarrow \sim [(n_1, n_2 \nobracket] = \sim [(s (m), 1)]
  \Rightarrow n \in \mathbbm{N}_{\mathbbm{Z}} \Rightarrow 0 \leqslant n,$ now
  if $n = 0 = \sim [(1, 1)]$ then $s (m) \upl 1 = 1 \upl 1 \Rightarrow m \upl
  (1 \upl 1) = 0 \upl (1 \upl 1) \Rightarrow m = 0$ a contradiction so $n \neq
  0$ and thus $0 < n$
\end{proof}

\begin{corollary}
  If $n \in \mathbbm{Z}$ so $n = \sim [(n_1, n_2)]$ then $0 \leqslant n$ iff
  $n_2 \leqslant n_1$
\end{corollary}

\begin{proof}
  
  
  $\Rightarrow$
  
  If $0 \leqslant n$ then we have either
  \begin{enumerate}
    \item $n = 0 \Rightarrowlim_{\text{\ref{0=~[(n,n)]}}} n_1 = n_2
    \Rightarrow n_2 \leqslant n_1$
    
    \item $0 < n \Rightarrowlim_{\text{previous theorem}} n_2 < n_1
    \Rightarrow n_2 \leqslant n_1$
  \end{enumerate}
  
  
  $\Leftarrow$
  
  If $n_2 \leqslant n_1$ then we have either
  \begin{enumerate}
    \item $n_2 = n_1 \Rightarrow n = \sim [(n_1, n_1)] = 0 \Rightarrow 0
    \leqslant n$
    
    \item $n_2 < n_1 \Rightarrow 0 < n \Rightarrow 0 \leqslant n$
  \end{enumerate}
  
\end{proof}

\begin{theorem}
  \label{i_N is order preserving}$i_{\mathbbm{N}} : \mathbbm{N} \rightarrow
  \mathbbm{N}_{\mathbbm{Z}}$ defined by $n \rightarrow i_{\mathbbm{N}} (n) =
  \sim [(s (n), 1)]$ is order preserving, so if $n, m \in \mathbbm{N}$ with $n
  \leqslant m$ then $i_{\mathbbm{N}} (n) \leqslant i_{\mathbbm{N}} (m)$
\end{theorem}

\begin{proof}
  If $n \leqslant m$ then we have the following cases
  \begin{enumerate}
    \item $n = m \Rightarrow i_{\mathbbm{N}} (n) = i_{\mathbbm{N}} (m)
    \Rightarrow \mathbbm{N}_{\mathbbm{Z}} \ni 0 = i_{\mathbbm{N}} (m) \upl
    (\um i_{\mathbbm{N}} (n)) \Rightarrow i_{\mathbbm{N}} (n) \leqslant
    i_{\mathbbm{N}} (m)$
    
    \item $n < m$ then $i_{\mathbbm{N}} (m) \upl (\um i_{\mathbbm{N}} (n)) =
    \sim [(s (m), 1)] \upl (\um (\sim [s (n), 1])) = \sim [(s (m), 1)] \upl
    \sim [(1, s (n))] = \sim [(s (m) \upl 1, s (n) \upl 1)] = \sim [(s (m), s
    (n))] = \sim [(m \upl 1, n \upl 1)] = \sim [(m, n)]$. By \ref{condition 2
    for positive integers} we have from $n < m$ that $0 < \sim [(m, n)] =
    i_{\mathbbm{N}} (m) \upl (\um i_{\mathbbm{N}} (n)) \Rightarrow 0 \leqslant
    i_{\mathbbm{N}} (m) \upl (\um i_{\mathbbm{N}} (n))
    \Rightarrowlim_{\text{\ref{condition 1 for positive integers}}}
    i_{\mathbbm{N}} (m) \upl (\um i_{\mathbbm{N}} (n)) \in
    \mathbbm{N}_{\mathbbm{Z}} \Rightarrow i_{\mathbbm{N}} (n) \leqslant
    i_{\mathbbm{N}} (m)$
  \end{enumerate}
\end{proof}

\begin{theorem}
  \label{n<less>m=<gtr>n+k<less>m+k for integers}If $n, m, k \in \mathbbm{Z}$
  then $n < m \Rightarrow n \upl k < m \upl k$
\end{theorem}

\begin{proof}
  From $n < m \Rightarrow n \neq m$ and $n \leqslant m$ we have $n \upl (\um
  m) \in \mathbbm{N}_{\mathbbm{Z}} \Rightarrow n \upl 0 \upl (- m) \in
  \mathbbm{N}_{\mathbbm{Z}} \Rightarrow n \upl k \upl (\um k) \upl (\um m) \in
  \mathbbm{N}_{\mathbbm{Z}} \Rightarrow (n \upl k) \upl (\um (m \upl k)) \in
  \mathbbm{N}_{\mathbbm{Z}} \Rightarrow n \upl k \leqslant m \upl k$. If now
  $n + k = m \upl k \Rightarrow n \upl k \upl (\um k) = m \upl k \upl (\um k)
  \Rightarrow n \upl 0 = m \upl 0 \Rightarrow n = m$ a contradiction with $n <
  m$ so we have $n \upl k < m \upl k$.
\end{proof}

\begin{corollary}
  If $n, m, k \in \mathbbm{Z}$ then $n \leqslant m \Rightarrow n \upl k
  \leqslant m \upl k$
\end{corollary}

\begin{proof}
  If $n \leqslant m$ then we have either
  \begin{enumerate}
    \item $n = m \Rightarrow n \upl k = m \upl k \Rightarrow n \upl k
    \leqslant m \upl k$
    
    \item $n < m \Rightarrowlim_{\text{previous theorem}} n \upl k < m \upl k
    \Rightarrow n \upl k \leqslant m \upl k$
  \end{enumerate}
\end{proof}

\begin{theorem}
  \label{n<less>m=<gtr>m-n<gtr>0 (integers)}If $n, m \in \mathbbm{Z}$ then $n
  < m \Leftrightarrow 0 < m \upl (- n)$
\end{theorem}

\begin{proof}
  
  
  $\Rightarrow$
  
  If $n < m \Rightarrow n \leqslant m$ and $n \neq m$ so $n \upl (\um m) \in
  \mathbbm{N}_{\mathbbm{Z}} \Rightarrowlim_{\text{\ref{condition 1 for
  positive integers}}} 0 \leqslant n \upl (\um m)$. Now if $0 = n \upl (\um m)
  \Rightarrow 0 \upl m = n \upl (\um m) \upl m \Rightarrow m = n \upl 0 = n$
  contradicting $n < m$.
  
  $\Leftarrow$
  
  If $0 < m \upl (\um n)$ then by the previous corollary we have $0 \upl n < m
  \upl (\um n) \upl n \Rightarrow n < m \upl 0 \Rightarrow n < m$
  
  
\end{proof}

\begin{corollary}
  If $n, m \in \mathbbm{Z}$ then $n \leqslant m \Leftrightarrow 0 \leqslant m
  \upl (\um n)$
\end{corollary}

\begin{proof}
  
  \begin{eqnarray*}
    n \leqslant m & \Leftrightarrow & n \upl (\um m) \in
    \mathbbm{N}_{\mathbbm{Z}}\\
    & \Leftrightarrow & 0 \leqslant n \upl (\um m)
  \end{eqnarray*}
\end{proof}

\begin{theorem}
  \label{0<less>n,0<less>m=<gtr>0<less>n.m n,m integers}If $n, m \in
  \mathbbm{Z}$ with $0 < n$ and $0 < m \Rightarrow 0 < n \cdot m$
\end{theorem}

\begin{proof}
  If $n = \sim [(n_1, n_2)], m = \sim [(m_1, m_2)]$ then from $0 < n$ and $0 <
  m$ we have by \ref{condition 2 for positive integers} $n_2 < n_1$ and $m_2 <
  m_1 \Rightarrowlim \exists k, l \in \mathbbm{N}_0 \vdash n_1 = n_2 \upl k$
  and $m_1 = m_2 \upl l$ this gives
  \begin{eqnarray*}
    n_1 \cdot m_1 \upl n_2 \cdot m_2 & = & (n_2 \upl k) \cdot (m_2 \upl l)
    \upl n_2 \cdot m_2\\
    & = & n_2 \cdot m_2 \upl n_2 \cdot m_2 \upl n_2 \cdot l + m_2 \cdot k + k
    \cdot l\\
    n_2 \cdot m_1 + n_1 \cdot m_2 & = & n_2 \cdot (m_2 + l) \upl (n_2 \upl k)
    \cdot m_2\\
    & = & n_2 \cdot m_2 + n_2 \cdot m_2 + n_2 \cdot l + m_2 \cdot k
  \end{eqnarray*}
  So using \ref{n<less>m=<gtr>n+k=m} and ($0 < k$ then by
  \ref{n<less>m=<gtr>nk<less>mk} and $0 < l$ we have $0 < k \cdot l$) we have
  that $n_2 \cdot m_1 \upl n_1 \cdot m_2 < n_1 \cdot m_1 + n_2 \cdot m_2$. Now
  $n \cdot m = \sim [(n_1 \cdot m_1 \upl n_2 \cdot m_2, n_2 \cdot m_1 \upl n_1
  \cdot m_2)]$ and thus we have $0 < n \cdot m$
\end{proof}

\begin{theorem}
  \label{n<less>m=<gtr>n.k<less>m.k if k<gtr>0,n,m integers}If $n, m \in
  \mathbbm{Z}$ and $k \in \mathbbm{Z} \vdash k > 0$ then $n < m \Rightarrow n
  \cdot k < m \cdot k$
\end{theorem}

\begin{proof}
  From $n < m$ we have by \ref{n<less>=m<less>=<gtr>n+k<less>=m+k (natural
  numbers)} that $0 < m \upl (- n)$ then by the above theorem we have $0 < (m
  \upl (\um n)) \cdot k = m \cdot k \upl (- 1) \cdot n \cdot k = m \cdot k +
  (- 1) (n \cdot k) = m \cdot k \upl (- (n \cdot k))
  \Rightarrowlim_{\text{\ref{n<less>m=<gtr>m-n<gtr>0 (integers)}}} n \cdot k <
  m \cdot k$
\end{proof}

\begin{corollary}
  \label{n<less>=m=<gtr>n.k<less>=m.k n,m,k integers k<gtr>=0}If $n, m \in
  \mathbbm{Z}$ and $k \in \mathbbm{Z} \vdash k \geqslant 0$ then $n \leqslant
  m \Rightarrow n \cdot k \leqslant m \cdot k$
\end{corollary}

\begin{proof}
  For $k$ we have the following possibilities
  \begin{enumerate}
    \item $k = 0 \Rightarrow n \cdot k = 0 = m \cdot k \Rightarrow n \cdot k
    \leqslant m \cdot k$
    
    \item $0 < k$ then for $n \leqslant m$ we have either
    \begin{enumerate}
      \item $n = m \Rightarrow n \cdot k = m \cdot k \Rightarrow n \cdot k
      \leqslant m \cdot k$
      
      \item $n < m \Rightarrowlim_{\text{previous theorem}} n \cdot k < m
      \cdot k = n \cdot k \leqslant m \cdot k$
    \end{enumerate}
  \end{enumerate}
\end{proof}

\begin{theorem}
  \label{0<less>=n.n if n is a integer}If $n \in \mathbbm{Z}$ then $0
  \leqslant n \cdot n$
\end{theorem}

\begin{proof}
  Let $n = \sim [(n_1, n_2)]$ then $n \cdot n = \sim [(n_1 \cdot n_1 \upl n_2
  \cdot n_2, n_2 \cdot n_1 \upl n_1 \cdot n_2)]$. From the totality of order
  in $\mathbbm{N}$ we have either
  \begin{enumerate}
    \item $n_1 = n_2$ then $n = \sim [(n_1, n_1)] = 0
    \Rightarrowlim_{\text{\ref{absorbing element of the integers}}} n \cdot n
    = 0 \cdot 0 = 0 \Rightarrow 0 \leqslant n \cdot n$
    
    \item $n_1 < n_2$ then we have $\exists k \in \mathbbm{N}_0$ with $n_2 =
    n_1 \upl k$
    \begin{eqnarray*}
      n_1 \cdot n_1 \upl n_2 \cdot n_2 & = & n_1 \cdot n_1 \upl (n_1 \upl k)
      \cdot (n_1 \upl k)\\
      & = & (n_1 \cdot n_1 \upl n_1 \cdot n_1 \upl n_1 \cdot k + n_1 \cdot k)
      \upl k \cdot k\\
      n_2 \cdot n_1 \upl n_1 \cdot n_2 & = & (n_1 \upl k) \cdot n_1 \upl n_1
      \cdot (n_1 \upl k)\\
      & = & (n_1 n_1 \upl n_1 \cdot n_1 + n_1 \cdot k + n_1 \cdot k)
    \end{eqnarray*}
    So as $0 < k \cdot k$ we have $n_2 \cdot n_1 \upl n_1 \cdot n_2 < n_1
    \cdot n_1 \upl n_2 \cdot n_2 \Rightarrow 0 < n \cdot n \Rightarrow 0
    \leqslant n \cdot n$
    
    \item $n_2 < n_1$ then we have \ $\exists k \in \mathbbm{N}_0$ with $n_1 =
    n_2 \upl k$
    \begin{eqnarray*}
      n_1 \cdot n_1 \upl n_2 \cdot n_2 & = & (n_2 \upl k) \cdot (n_2 \upl k)
      \upl n_2 \cdot n_2\\
      & = & (n_2 \cdot n_2 \upl n_2 \cdot n_2 \upl n_2 \cdot k + n_2 \cdot k)
      \upl k \cdot k\\
      n_2 \cdot n_1 \upl n_1 \cdot n_2 & = & n_2 \cdot (n_2 \upl k) + (n_2
      \upl k) \cdot n_2\\
      & = & (n_2 n_2 \upl n_2 \cdot n_2 + n_2 \cdot k + n_2 \cdot k)
    \end{eqnarray*}
    So as $0 < k \cdot k$ we have $n_2 \cdot n_1 \upl n_1 \cdot n_2 < n_1
    \cdot n_1 \upl n_2 \cdot n_2 \Rightarrow 0 < n \cdot n \Rightarrow 0
    \leqslant n \cdot n$
  \end{enumerate}
  
\end{proof}

\begin{theorem}[Archimedean property of $\mathbbm{Z}$]
  \label{archimedean property of whole numbers}{\index{Archimedean property of
  $\mathbbm{Z}$}}If $m, n \in \mathbbm{Z}$ with $m > 0$ then there exists a $k
  \in \mathbbm{N}_{\mathbbm{Z}}$ with $n < k \cdot m$
\end{theorem}

\begin{proof}
  We have the following cases for $n$
  \begin{enumerate}
    \item $n \leqslant 0$ then if we take $k = 1 \in
    \mathbbm{N}_{\mathbbm{Z}}$ then $n \leqslant 0 < m = 1 \cdot m = k \cdot m
    \Rightarrow n < k \cdot m$
    
    \item If $n > 0 \Rightarrow n \in \mathbbm{N}_{\mathbbm{Z}}$ as also $m >
    0$ we have $m \in \mathbbm{N}_{\mathbbm{Z}}$ and there exists a $n', m'
    \in \mathbbm{N}$ such that $n = i_{\mathbbm{Z}} (n')$, $m =
    i_{\mathbbm{Z}} (m')$. Assume now that $m' \leqslant 0 \Rightarrow
    i_{\mathbbm{Z}} (m') \leqslant i_{\mathbbm{Z}} (0) = \sim [(s (0), 1)] =
    \sim [(1, 1)] = 0$ a contradiction so we must have that $m' > 0$. Then by
    \ref{archimedean property of natural numbers} there exists a $k' \in
    \mathbbm{N}$ such that $n' < k' \cdot m' \Rightarrow n = i_{\mathbbm{Z}}
    (n') < i_{\mathbbm{Z}} (k' \cdot m') = i_{\mathbbm{Z}} (k') \cdot
    i_{\mathbbm{Z}} (m') = k \cdot m$ where $k = i_{\mathbbm{Z}} (k') \in
    \mathbbm{N}_{\mathbbm{Z}}$
  \end{enumerate}
\end{proof}

\begin{theorem}
  \label{naturals in whole numbers are well ordered}$\langle
  \mathbbm{N}_{\mathbbm{Z}}, \leqslant \rangle$ is well-ordered.
\end{theorem}

\begin{proof}
  Let $A \subseteq \mathbbm{N}_{\mathbbm{Z}}$ be a non empty subset of
  $\mathbbm{N}_{\mathbbm{A}}$ then as $i_{\mathbbm{N}} : \mathbbm{N}
  \rightarrow \mathbbm{N}_{\mathbbm{Z}}$ is a bijection we have that
  $i_{\mathbbm{N}}^{- 1} (A)$ is non empty and thus by the well-ordering of
  the natural numbers (see \ref{the natural numbers are well-ordered}) there
  exists a $m' = \min (i_{\mathbbm{N}}^{- 1} (A)) \in i^{- 1}_{\mathbbm{N}}
  (A)$. Take now $m = i_{\mathbbm{N}} (m')$ then we have
  \begin{enumerate}
    \item $m' \in i_{\mathbbm{N}}^{- 1} (A) \Rightarrow m = i_{\mathbbm{N}}
    (m') \in A$
    
    \item If $a \in A \Rightarrowlim_{i_{\mathbbm{N}} \tmop{is}
    \tmop{bijective}} \exists a' = i_{\mathbbm{N}}^{- 1} (a) \in
    i_{\mathbbm{N}}^{- 1} (A) \Rightarrowlim_{m' \tmop{is} \tmop{least}
    \tmop{element} \tmop{of} A} m' \leqslant a'
    \Rightarrowlim_{i_{\mathbbm{N}} \tmop{is} \tmop{order} \tmop{preserving}
    \left( \tmop{see} \text{\ref{i_N is order preserving}} \right)} m =
    i_{\mathbbm{N}} (m') \leqslant i_{\mathbbm{N}} (a') = a \Rightarrow m =
    \min (A)$
  \end{enumerate}
\end{proof}

\begin{corollary}
  \label{the naturals embedded in the integers are conditionally
  complete}$\langle \mathbbm{N}_{\mathbbm{Z}}, \leqslant \rangle$ is
  conditionally complete (any non empty subset in $\mathbbm{N}_{\mathbbm{Z}}$
  bounded above has a supremum (lowest upper bound). Furthermore if $\emptyset
  \neq A \subseteq \mathbbm{N}$ then if $\sup (A)$ exists we have $\sup (A)
  \in A$
\end{corollary}

\begin{proof}
  The fact that $\langle \mathbbm{N}_{\mathbbm{Z}}, \leqslant \rangle$ is
  conditionally complete follows from the fact that $\langle
  \mathbbm{N}_{\mathbbm{Z}}, \leqslant \rangle$ is well-ordered (see
  \ref{naturals in whole numbers are well ordered}) and \ref{well ordering
  implies conditional completeness}. Now if $\emptyset \neq A \subseteq
  \mathbbm{N}_{\mathbbm{Z}}$ then there exists a $a \in A \Rightarrow
  i_{\mathbbm{N}}^{- 1} (a) \in i_{\mathbbm{N}}^{- 1} (A) \Rightarrow
  \emptyset \neq i_{\mathbbm{N}}^{- 1} (A)$. Also if $M = \sup (A)$ take then
  $m = i_{\mathbbm{N}}^{- 1} (M)$. If now $a \in A \Rightarrow a \leqslant m$
  [For if $m < a \Rightarrowlim_{i_{\mathbbm{N}} \tmop{is} \tmop{order}
  \tmop{preserving} \tmop{and} \tmop{injective}} M = i_{\mathbbm{N}} (m) <
  i_{\mathbbm{N}} (a) \in A$ contradicting the fact that $M$ is a upper bound
  of $A$]. Furthermore if $m'$ is another upper bound of $i_{\mathbbm{N}}^{-
  1} (A)$ then if $a \in A$ we have $i_{\mathbbm{N}}^{- 1} (a) \in
  i_{\mathbbm{N}}^{- 1} (A) \Rightarrow i_{\mathbbm{N}}^{- 1} (a) \leqslant m'
  \Rightarrowlim_{i_{\mathbbm{N}}^{- 1} \tmop{is} \tmop{order}
  \tmop{preserving}} a = i_{\mathbbm{N}} (i_{\mathbbm{N}}^{- 1} (a)) \leqslant
  i_{\mathbbm{N}} (m') \Rightarrow i_{\mathbbm{N}} (m')$ is a upper bound of
  $A$ and as $M$ is the least upper bound we have $M \leqslant i_{\mathbbm{N}}
  (m')$ and then $i_{\mathbbm{N}}^{- 1} (M) \leqslant m'$ [otherwise is $m' <
  i_{\mathbbm{N}}^{- 1} (M) \Rightarrowlim_{i_{\mathbbm{N}} \tmop{is}
  \tmop{order} \tmop{preserving}} i_{\mathbbm{N}} (m') < i_{\mathbbm{N}}
  (i_{\mathbbm{N}}^{- 1} (M)) = M \leqslant i_{\mathbbm{N}} (m') \Rightarrow
  i_{\mathbbm{N}} (m') < i_{\mathbbm{N}} (m')$ a contradiction] so $m =
  i_{\mathbbm{N}}^{- 1} (M)$ is the lowest upper bound of $i_{\mathbbm{N}}^{-
  1} (A)$ or in other words $m = \sup (i_{\mathbbm{N}}^{- 1} (A))$, using
  \ref{sup(A) is element of A in N} we have then that $i_{\mathbbm{N}}^{- 1}
  (M) = m \in i_{\mathbbm{N}}^{- 1} (A) \Rightarrow M = i_{\mathbbm{N}}
  (i_{\mathbbm{N}}^{- 1} (M)) \in A \Rightarrow \sup (A) \in M$
\end{proof}

\begin{lemma}
  \label{n<less>m=m=n+k,k<less><gtr>0 in N embedded in integers}$\forall n, m
  \in \mathbbm{N}_Z$ then $n < m$ $\Leftrightarrow$ $\exists k \in
  \mathbbm{N}_{\mathbbm{Z}} \backslash \{ 0 \}$ such that $m = n \upl k$
\end{lemma}

\begin{proof}[$\Rightarrow$][$\Leftarrow$]
  \
  
  Take $n' = i_{\mathbbm{N}}^{- 1} (n)$ and $m' = i_{\mathbbm{N}}^{- 1} (m)$
  then $n' < m'$ [otherwise if $m' \leqslant n'$ we have by the order
  preserving properties of $i_{\mathbbm{N}}$ that $m = i_{\mathbbm{N}}
  (i_{\mathbbm{N}}^{- 1} (m)) \leqslant i_{\mathbbm{N}} (i_{\mathbbm{N}}^{- 1}
  (n)) = n \Rightarrow m \leqslant n < m \Rightarrow m < m$ a contradiction].
  Using \ref{n<less>m=<gtr>n+k=m} there exists a $k' \in \mathbbm{N}
  \backslash \{ 0 \}$ such that $m' = n' \upl k'$ then using the fact that
  $i_{\mathbbm{N}}$ is a isomorphism we have $m = i_{\mathbbm{N}}
  (i_{\mathbbm{N}}^{- 1} (m)) = i_{\mathbbm{N}} (m') = i_{\mathbbm{N}} (n'
  \upl k') = i_{\mathbbm{N}} (n') \upl i_{\mathbbm{N}} (k') = i_{\mathbbm{N}}
  (i_{\mathbbm{N}}^{- 1} (n)) \upl k = n \upl k$ where $k = i_{\mathbbm{N}}
  (k') \in \mathbbm{N}_{\mathbbm{Z}}$ and $k \neq 0$ [otherwise if $k = 0
  \Rightarrow i_{\mathbbm{N}} (0) = 0 = k = i_{\mathbbm{N}} (k')
  \Rightarrowlim_{i_{\mathbbm{N}} \tmop{is} a \tmop{bijection}} k' = 0$
  contradicting $k' \in \mathbbm{N} \backslash \{ 0 \}$]
  
  If $m = n \upl k$ where $k \in \mathbbm{N}_Z \backslash \{ 0 \} \Rightarrow
  0 < k = (n \upl k) \upl (- n) = m \upl (- n) \Rightarrow 0 < m \upl (- n)
  \Rightarrow n = 0 \upl n < m \upl (- n) \upl n = m \Rightarrow n < m$
\end{proof}

\begin{definition}
  \label{absolute value of integers}{\index{absolute value}}{\index{$|
  |$}}Given $z \in \mathbbm{Z}$ then $| z | \in \mathbbm{N}_{\mathbbm{Z}}$ is
  defined by $| z | = \left\{ \begin{array}{l}
    z \tmop{if} 0 \leqslant z\\
    - z \tmop{if} z < 0
  \end{array} \right.$. So if $z \geqslant 0 \Rightarrow z = | z |$
\end{definition}

\begin{note}
  Given $z, z' \in \mathbbm{Z}$ then its easy to prove that $| z \cdot z' | =
  | z | \cdot | z' |$
\end{note}

\begin{proof}[$0 \leqslant z \wedge 0 \leqslant z'$][$0 \leqslant z \wedge z'
< 0$][$z < 0 \wedge 0 \leqslant z'$][$z < 0 \wedge z' < 0$]
  We have the following cases for $z$ and $z'$ to consider
  \begin{enumerate}
    \item then $0 \leqslant z \cdot z'$ and $| z | \cdot | z' | = z \cdot z' =
    | z \cdot z' |$
    
    \item then $z \cdot z' \leqslant 0$ and $z \cdot z' = - | z \cdot z' |
    \Rightarrow - (z \cdot z') = | z \cdot z' |$, also $- (z \cdot z') = z
    \cdot (- z') = | z | \cdot | z' | \Rightarrow | z | \cdot | z' | = | z
    \cdot z' |$
    
    \item then $z \cdot z' \leqslant 0$ and $z \cdot z' = - | z \cdot z' |
    \Rightarrow - (z \cdot z') = | z \cdot z' |$, also $- (z \cdot z') = (\um
    z) \cdot z' = | z | \cdot | z' | \Rightarrow | z | \cdot | z' | = | z
    \cdot z' |$
    
    \item then $0 \leqslant - z \wedge 0 \leqslant - z' \Rightarrow 0
    \leqslant (- z) \cdot (- z') = z \cdot z' \Rightarrow | z \cdot z' | = z
    \cdot z' = (\um | z |) \cdot (\um | z' |) = | z | \cdot | z' |$
  \end{enumerate}
\end{proof}

\begin{lemma}
  \label{0<less>m=<gtr>1<less>=m if m is a integer}If $m \in \mathbbm{Z}$ is
  such that $0 < m \Rightarrow 1 \leqslant m$
\end{lemma}

\begin{proof}
  If $m \in \mathbbm{Z}$ is such that $0 < m$ then using \ref{condition 1 for
  positive integers} we have $m \in \mathbbm{N}_{\mathbbm{Z}}$. We can not
  have $i_{\mathbbm{N}}^{- 1} (m) = 0$ [for then $m = i_{\mathbbm{N}}
  (i_{\mathbbm{N}}^{- 1} (m)) = i_{\mathbbm{N}} (0) = \sim [(s (0), 1)] = \sim
  [(1, 1)] = 0$] and thus $i_{\mathbbm{N}}^{- 1} (m) \neq 0$. Using \ref{every
  natural number is bigger or equal to zero} we have then $0 <
  i_{\mathbbm{N}}^{- 1} (m)$ which by \ref{n<less>m=<gtr>s(n)<less>=m} means
  that $1 = s (0) \leqslant i_{\mathbbm{N}}^{- 1} (m) \Rightarrow 1 \leqslant
  i_{\mathbbm{N}}^{- 1} (m)$. From the order preserving of $i_{\mathbbm{N}}$
  we have then $i_{\mathbbm{N}} (1) \leqslant i_{\mathbbm{N}}
  (i_{\mathbbm{N}}^{- 1} (m)) = m \Rightarrow \sim [(s (1), 1)] \leqslant m
  \Rightarrow 1 = \sim [(2, 1)] \leqslant m \Rightarrow 1 \leqslant m$
\end{proof}

\begin{theorem}
  \label{n|m=<gtr>n<less>=|m|}If $n, m \in \mathbbm{Z}$, $m \neq 0$ and $n|m$
  then $n \leqslant | m |$
\end{theorem}

\begin{proof}[$0 < m \wedge n \leqslant 0$][$0 < m \wedge 0 < n$][$m < 0
\wedge n \leqslant 0$][$m < 0 \wedge 0 < n$]
  From $n|m$ there exists a $q$ such that $n \cdot q = m$ and as $m \neq 0$ we
  must have $q \neq 0$ (otherwise $m = n \cdot 0 = 0$). Then we have the
  following cases to consider
  \begin{enumerate}
    \item in this case we have trivially $n < m = | m | \Rightarrow n
    \leqslant | m |$
    
    \item we must have $0 < q$ [if $q \leqslant 0 \Rightarrowlim_{q \neq 0} q
    < 0 \Rightarrowlim_{0 < n} m = n \cdot q < n \cdot 0 = 0 \Rightarrow m < 0
    < m \Rightarrow m < m$ a contradiction] so using the previous lemma we
    have then $1 \leqslant q \Rightarrowlim_{0 < n} n = n \cdot 1 \leqslant n
    \cdot q = m = | m | \Rightarrow n \leqslant | m |$
    
    \item here $0 < \um m = | m | \Rightarrow n < | m | \Rightarrow n
    \leqslant | m |$
    
    \item then $q < 0$ \ \ [otherwise if $0 \leqslant q \Rightarrowlim_{q \neq
    0} 0 < q \Rightarrowlim_{0 < n} 0 < n \cdot q = m \Rightarrow 0 < 0$ a
    contradiction] and we have $0 < \um q \Rightarrowlim_{\tmop{previous}
    \tmop{lemma}} 1 \leqslant \um q \Rightarrowlim_{0 < n} n = n \cdot 1
    \leqslant n \cdot (- q) = - (n \cdot q) = \um m = | m | \Rightarrow n
    \leqslant | m |$
  \end{enumerate}
\end{proof}



\begin{theorem}[Division Algorithm]
  \label{division algorithm}{\index{division algorithm}}If $m, n \in
  \mathbbm{Z}$ and $n > 0$ then there exists a unique $r \in \mathbbm{N}$ with
  $0 \leqslant r < n$ and a unique $q \in \mathbbm{Z}$ such that $m = n \cdot
  q \upl r$
\end{theorem}

\begin{proof}[$r' = n$][$r' < n$][$n < r'$][$r < r''$][$r'' < r$]
  Let $A_{m, n} = \{ m \upl n \cdot q | q \in \mathbbm{Z} \wedge 0 \leqslant m
  \upl n \cdot q \nobracket \} \subseteq \mathbbm{N}_{\mathbbm{Z}}$. Using the
  Archimedean property of $\mathbbm{Z}$ (see \ref{archimedean property of
  whole numbers}) there exists a $k \in \mathbbm{N}_{\mathbbm{Z}}$ such that
  $k \cdot n > \um m \Rightarrow m \upl n \cdot k > 0 \Rightarrow m \upl n
  \cdot k \in A_{m, n} \Rightarrow A_{m, n} \neq \emptyset$. By the fact that
  $\langle \mathbbm{N}_{\mathbbm{Z}}, \leqslant \rangle$ is well-ordered we
  have then that $r' = \min (A_{m, n})$ exists. From $r' \in A_{m, n}$ we have
  then also $0 \leqslant r'$ and a $q' \in \mathbbm{N}$ such that $r' = m \upl
  n \cdot q'$. For the relation between $r'$ and $n$ we have then the
  following possibilities
  \begin{enumerate}
    \item in this case we have $m \upl n \cdot q' = r' = n \Rightarrow m = n
    \cdot (1 \um q')$, thus taking $q = (1 \um q')$ and $r = 0 \Rightarrow 0
    \leqslant r < n$ we have $m = n \cdot q \upl r$
    
    \item then $r' = m \upl n \cdot q' \Rightarrow m = n \cdot (- q') \upl r'$
    so if $r = r' \Rightarrow 0 \leqslant r < n$ and $q = - q'$ then $m = n
    \cdot q \upl r$
    
    \item using \ref{n<less>m=m=n+k,k<less><gtr>0 in N embedded in integers}
    there exists a $k \in \mathbbm{N}_{\mathbbm{Z}} \backslash \{ 0 \}
    \Rightarrow 0 < k$ such that $r' = n \upl k \Rightarrow n \upl k = m \upl
    n \cdot q' \Rightarrow 0 < k = m \upl n \cdot (q' - 1) \Rightarrow k \in
    A_{\tmop{mn}}$. From $r' = n \upl k \Rightarrowlim_{0 < n \tmop{and}
    \text{\ref{n<less>m=m=n+k,k<less><gtr>0 in N embedded in integers}}} k <
    r' \Rightarrow r' \neq \min (A_{m, n})$ a contradiction. So this case does
    not apply.
  \end{enumerate}
  Now to prove uniqueness assume that there exists another $q'', r'' \in
  \mathbbm{Z}$ with $0 \leqslant r'' < n$ and $m = n \cdot q'' \upl r''$. We
  have then that $n \cdot q'' \upl r'' = n \cdot q \upl r \Rightarrow n \cdot
  (q \um q'') = r'' \um r$. We prove now by contradiction that $r = r''$, so
  assume that $r \neq r''$ then we have either
  \begin{enumerate}
    \item then $0 < r'' \um r = n \cdot (q \um q'')$ which as $n > 0$ means
    that $0 < q \um q''$ [otherwise from $q \um q' \leqslant 0$ we would have
    $n \cdot (q \um q'') \leqslant 0$ contradicting $0 < n \cdot (q \um q'')$]
    so using \ref{0<less>m=<gtr>1<less>=m if m is a integer} we have $1
    \leqslant q \um q''$. Also from $0 \leqslant r, r'' < n$ we have $r'' \um
    r < n \um r \leqslant n \Rightarrow n \cdot (q \um q'') < n \Rightarrow q
    \um q'' < 1$ [if $1 \leqslant q \um q'' \Rightarrowlim_{n > 0} n = n \cdot
    1 \leqslant n \cdot (q - q'') < n \Rightarrow n < n$ a contradiction]. So
    we finally reach the conclusion that $1 \leqslant q \um q'' < 1
    \Rightarrow 1 < 1$ a contradiction.
    
    \item then $0 < r \um r'' = n \cdot (q'' \um q)$ which as $n > 0$ means
    that $0 < q'' \um q$ [otherwise from $q'' \um q \leqslant 0$ we would have
    $n \cdot (q'' \um q) \leqslant 0$ contradicting $0 < n \cdot (q'' \um q)$]
    so using \ref{0<less>m=<gtr>1<less>=m if m is a integer} we have $1
    \leqslant q'' \um q$. Also from $0 \leqslant r, r'' < n$ we have $r \um
    r'' < n \um r'' \leqslant n \Rightarrow n \cdot (q'' \um q) < n
    \Rightarrow q'' \um q < 1$ [if $1 \leqslant q'' \um q \Rightarrowlim_{n >
    0} n = n \cdot 1 \leqslant n \cdot (q'' - q) < n \Rightarrow n < n$ a
    contradiction]. So we finally reach the conclusion that $1 \leqslant q''
    \um q < 1 \Rightarrow 1 < 1$ a contradiction.
  \end{enumerate}
\end{proof}

\begin{definition}
  \label{divides}{\index{divides}}Given $n, m \in \mathbbm{Z}$ then we say
  that $n$ divides $m$, noted by $n|m$ if there exists a $q \in \mathbbm{Z}$
  such that $q \cdot n = m$
\end{definition}

\begin{note}
  If $n|m$ then we have also $(\um n) |m$ and this proves that then also $(| d
  |) |m$
\end{note}

\begin{proof}
  If $n|m$ then there exists a $q \in \mathbbm{Z}$ such that $m = n \cdot q =
  (\um n) \cdot (- q)$ where $- q \in \mathbbm{Z}$ so that $(\um n) |m$
\end{proof}

\begin{definition}
  If $n, m \in \mathbbm{Z}$ and $n|m$ then there exists a unique (see
  \ref{division algorithm}) $q \in \mathbbm{Z}$ such that $n \cdot q = m$.
  This number is \ called the quotient of $m$ and $n$ and noted by $m / n$. So
  if $n|m$ then there 
\end{definition}

\

\begin{definition}
  \label{common divisior and gcd}{\index{common divisor}}{\index{gcd}}Given
  $n, m \in \mathbbm{Z}$ then $d$ is a common divisor for $n$ and $m$ if $d|n$
  and $d|m$. The greatest common divisor of $m$ and $n$ noted by $\gcd (n, m)
  = \sup (D_{n, m}) \in \mathbbm{N}_{\mathbbm{Z}} \backslash \{ 0 \}$ if it
  exists, where $D_{n, m} = \{ d \in \mathbbm{N}_{\mathbbm{Z}} \backslash \{ 0
  \} | d|n \wedge d|m \}$. Note that $1 \in D_{n, m} \Rightarrow D_{n, m} \neq
  \emptyset$, note also that from \ref{the naturals embedded in the integers
  are conditionally complete} we have that if $\gcd (n, m) = \sup (D_{n, m})$
  exists then $\gcd (n, m) \in D_{n, m} \Rightarrow \gcd (n, m) |n$ and $\gcd
  (n, m) |m$ 
\end{definition}

\begin{theorem}
  Given $n, m \in \mathbbm{Z}$ then $\gcd (n, m)$ exists $\Leftrightarrow$ $n
  \neq 0$ or $m \neq 0$
\end{theorem}

\begin{proof}[$\Rightarrow$][$\Leftarrow$][$n \neq 0$][$m \neq 0$]
  Define $D_{n, m} = \{ d \in \mathbbm{Z}|d|n \wedge d|m \}$ so that $\gcd (n,
  m) = \sup (D_{n, m})$. The proof is delivered in two parts
  \begin{enumerate}
    \item So assume that $\gcd (n, m)$ exists and assume that $n = 0 = m$.
    Then if $q \in \mathbbm{Z}$ we have $q \cdot 0 = 0 = n = m \Rightarrow q|n
    \wedge q|m \Rightarrow \mathbbm{Z}= D_{n, m}$. Now from $0 < 1
    \Rightarrowlim_{\tmop{add} \gcd (n, m)} \gcd (n, m) < \gcd (n, m) \upl 1
    \in D_{n, m}$ contradicting the fact that $\gcd (n, m)$ is a upper bound
    of $D_{n, m}$.
    
    \item Suppose that $n \neq 0$ or $m \neq 0$ then we have the following
    cases
    \begin{enumerate}
      \item from \ref{n|m=<gtr>n<less>=|m|} we have then if $d \in D_{n, m}$
      then $d|n \Rightarrow d < | n |$ so $D_{n, m}$ has a upper bound and by
      conditional completeness (see \ref{the naturals embedded in the integers
      are conditionally complete}) we have that $\sup (D_{n, m})$ exists.
      
      \item from \ref{n|m=<gtr>n<less>=|m|} we have then if $d \in D_{n, m}$
      then $d|m \Rightarrow d < | m |$ so $D_{n, m}$ has a upper bound and by
      conditional completeness (see \ref{the naturals embedded in the integers
      are conditionally complete}) we have that $\sup (D_{n, m})$ exists. 
    \end{enumerate}
  \end{enumerate}
\end{proof}

\begin{theorem}
  \label{basic property of gcd(n,m)}If $n, m \in \mathbbm{Z}$ with $m \neq 0$
  then by the above theorem $\gcd (n, m)$ exists and from the fact that $\gcd
  (n, m) |m$, $\gcd (n, m) |n$ we can consider $n / \gcd (n, m)$ and $m / \gcd
  (n, m)$. We have now that if $d| (n / \gcd (n, m))$ and $d| (m / \gcd (n,
  m))$ then $d = 1$ or $d = - 1$ furthermore $\gcd (n / \gcd (n, m), m / \gcd
  (n, m)) = 1$
\end{theorem}

\begin{proof}
  Take $n' = n / \gcd (n, m)$ and $m' = n / \gcd (n, m)$ then we have $n'
  \cdot \gcd (n, m) = n$ and $m' \cdot \gcd (n, m) = m$. If $d|n'$ then there
  exists a $n''$ such that $n'' \cdot d = n' \Rightarrow n'' \cdot d \cdot
  \gcd (n, m) = n' \cdot \gcd (n, m) = n$ or $(d \cdot \gcd (n, m)) |n
  \Rightarrowlim_{0 < \gcd (n, m)} (| d | \cdot \gcd (n, m)) | n |$. Also if
  $d|m'$ then there exists a $m''$ such that $m'' \cdot d = m' \Rightarrow m''
  \cdot d \cdot \gcd (n, m) = m' \cdot \gcd (n, m) = m$ or $(d \cdot \gcd (n,
  m)) |m \Rightarrowlim_{0 < \gcd (n, m)} (| d | \cdot \gcd (n, m)) |m$. Now
  $0 \leqslant | d |$ and as $d \neq 0$ [otherwise $m = m'' \cdot d \cdot \gcd
  (n, m) = 0$ contradicting $m \neq 0$] we have $0 < | d |$ and as $0 < \gcd
  (n, m)$ we have $0 < | d | \cdot \gcd (n, m)$ and thus $| d | \cdot \gcd (n,
  m) \in D_{n, m} \Rightarrow | d | \cdot \gcd (n, m) \leqslant \gcd (n, m)$
  and thus $| d | \leqslant 1$ [otherwise $\gcd (n, m) < | d | \cdot \gcd (n,
  m)$]. As we have also by \ref{0<less>m=<gtr>1<less>=m if m is a integer} and
  $0 < | d |$ that $1 \leqslant | d |$ we must conclude that $| d | = 1$ and
  thus $d = 1$ or $d = - 1$. As $D_{n', m'} = \{ d \in
  \mathbbm{N}_{\mathbbm{Z}} \backslash \{ 0 \} |d|n' \wedge d|m' \} = \{ 1 \}$
  we have $\gcd (n', m') = 1$.
\end{proof}

\begin{definition}
  \label{even and odd}{\index{even numbers}}{\index{odd numbers}}Given $z \in
  \mathbbm{Z}$ then we say $z$ is even if $2| z$. If $z$ is not even then we
  call $Z$ odd.
\end{definition}

\begin{theorem}
  Given $z \in \mathbbm{Z}$ then we have
  \begin{enumerate}
    \item $z$ is even $\Leftrightarrow$ $\exists m \in \mathbbm{Z} \vdash z =
    2 \cdot m$
    
    \item $z$ is odd $\Leftrightarrow \exists m \in \mathbbm{Z} \vdash z = 2
    \cdot m \upl 1$
  \end{enumerate}
\end{theorem}

\begin{proof}[$z \tmop{is} \tmop{odd}$][$z = 2 \cdot m \upl 1$]
  
  \begin{enumerate}
    \item This follows directly from the definition of $2| z$
    
    \item Using the division algorithm (see \ref{division algorithm}) there
    exists unique $n, m \in \mathbbm{Z}$ with $z = 2 \cdot n \upl m$ where $0
    \leqslant m < 2$. If now $1 < m \Rightarrow 0 < m \um 1
    \Rightarrowlim_{\text{\ref{0<less>m=<gtr>1<less>=m if m is a integer}}} 1
    \leqslant m \um 1 \Rightarrow 2 \leqslant m < 2 \Rightarrow 2 < 2$ a
    contradiction so we have $0 \leqslant m \leqslant 1$. We have then
    \begin{enumerate}
      \item then $z$ is not $\tmop{even}$ so we must have $0 < m \Rightarrow 1
      \leqslant m \leqslant 1 \Rightarrow m = 1 \Rightarrow z = 2 \cdot m \upl
      1$
      
      \item from the uniqueness of $n, m$ we can not have $z = 2 \cdot m' \upl
      0$ because we would have then the contradiction $1 = 0$
    \end{enumerate}
  \end{enumerate}
\end{proof}

\begin{theorem}
  \label{m*m is even then m is even}Given $m \in \mathbbm{Z}$ such that $z
  \cdot z$ is even then we have that $z$ is even
\end{theorem}

\begin{proof}
  We prove this by contradiction, so assume that $z$ is odd then by the above
  theorem we have $z = 2 \cdot n \upl 1$ and thus $z \cdot z = (2 \cdot n \upl
  1) \cdot (2 \cdot n \upl 1) = 2 \cdot 2 \cdot n \cdot n + 2 \cdot n \cdot 1
  + 1 \cdot 2 \cdot n \upl 1 \cdot 1 = 2 \cdot (2 \cdot n) + 2 \cdot n + 2
  \cdot n \upl 1 = 2 \cdot (2 \cdot n \upl n \upl n) \upl 1$ meaning that $z
  \cdot z$ is odd contradicting the fact that $z \cdot z$ is even.
\end{proof}

\section{Denumerability of the integers}

We are going to prove now that the set of integers is \tmtextbf{denumerable}.

\begin{lemma}
  \label{naturals embedded in the integers are
  denumerable}$\mathbbm{N}_{\mathbbm{Z}}$ is \tmtextbf{denumerable}
\end{lemma}

\begin{proof}
  This follows directly from \ref{properties of positive numbers} (3) where it
  is proved that $i_{\mathbbm{Z}} : \mathbbm{N} \rightarrow
  \mathbbm{N}_{\mathbbm{Z}}$ is a bijection. 
\end{proof}

\begin{theorem}
  \label{the integer numbers are denumerable}$\mathbbm{Z}$ is
  \tmtextbf{denumerable}.
\end{theorem}

\begin{proof}
  Using \ref{whole numbers is union of positive and negative numbers} we have
  that $\mathbbm{Z}=\mathbbm{N}_{\mathbbm{Z}} \bigcup
  (-\mathbbm{N}_{\mathbbm{Z}})$ where $\mathbbm{N}_{\mathbbm{Z}} = \{ - n|n
  \in \mathbbm{N}_{\mathbbm{Z}} \}$. We can now construct the trivial
  bijection $f : \mathbbm{N}_{\mathbbm{Z}} \rightarrow
  (-\mathbbm{N}_{\mathbbm{Z}})$ defined by $n \rightarrow - n$. So we have
  $\mathbbm{N}_{\mathbbm{Z}} \approx (-\mathbbm{N}_{\mathbbm{Z}})$ and by the
  previous lemma that $\mathbbm{N}_{\mathbbm{Z}} \approx \mathbbm{N}$ and thus
  $(-\mathbbm{N}_{\mathbbm{Z}}) \approx \mathbbm{N}$. This means that
  $\mathbbm{Z}$ is the union of two denumerable sets and thus by \ref{the
  union of two denumerable sets is denumerable} that $\mathbbm{Z}$ is
  denumerable.
\end{proof}

\chapter{The rational numbers}

\section{Definition and arithmetic's}

Just like we defined integer numbers using a equivalence relation on
$\mathbbm{N}$ we define the rational numbers using a equivalence relation in
$\mathbbm{Z}$

\begin{definition}
  $\mathbbm{Z}_0 =\mathbbm{Z} \backslash \{ 0 \}$ (here $0 = \sim [(1, 1)]$)
\end{definition}

\begin{definition}
  $\simeq \subseteq (\mathbbm{Z} \times \mathbbm{Z}_0) \times (\mathbbm{Z}
  \times \mathbbm{Z}_0)$ is the relation defined on $\mathbbm{Z} \times
  \mathbbm{Z}_0$ by $\simeq = \{ ((n, m), (r, k)) \in \mathbbm{Z} \times
  \mathbbm{Z}_0 | n \cdot k = m \cdot \tmop{kr} \nobracket \}$. In other words
  if $(n, m), (r, k) \in \mathbbm{Z} \times \mathbbm{Z}_0$ then $(n, m) \simeq
  (r, k)$ iff $n \cdot k = m \cdot r$
\end{definition}

\begin{theorem}
  $\simeq$ is a equivalence relation on $\mathbbm{Z} \times \mathbbm{Z}_0$
\end{theorem}

\begin{proof}[reflexive][symmetry][transitive]
  
  \begin{enumerate}
    \item If $(n, m) \in \mathbbm{Z} \times \mathbbm{Z}_0$ then $n \cdot m
    \equallim_{\text{\ref{properties of multiplication of integer numbers}}} m
    \cdot n \Rightarrow (n, m) \simeq (n, m)$
    
    \item If $(n, m), (r, k) \in \mathbbm{Z} \times \mathbbm{Z}_0$ and $(n, m)
    \simeq (r, k)$ then $n \cdot k = m \cdot r
    \Rightarrowlim_{\text{\ref{properties of multiplication of integer
    numbers}}} r \cdot m = k \cdot n \Rightarrow (r, k) \simeq (n, m)$
    
    \item If $(i, j) \nocomma, (n, m), (r, k) \in \mathbbm{Z} \times
    \mathbbm{Z}_0$ and $(i, j) \simeq (n, m)$ and $(n, m) \simeq (r, k)$ then
    we have $i \cdot m = j \cdot n$ and $n \cdot k = m \cdot r$ then $(i \cdot
    m) \cdot k = (j \cdot n) \cdot k = j \cdot (n \cdot k) = j \cdot (m \cdot
    r) \Rightarrow (i \cdot k) \cdot m = (j \cdot r) \cdot m \Rightarrowlim_{m
    \neq 0, \text{\ref{n.r=k,r=<gtr>n=k and n,k,r integers}}} i \cdot k = j
    \cdot r \Rightarrow (i, j) \simeq (k, r)$
  \end{enumerate}
\end{proof}

\begin{definition}
  \label{n/k}{\index{$\frac{n}{k}$}}{\index{$\mathbbm{Q}$}}{\index{$\frac{n}{k}$}}We
  define the set of rationals $\mathbbm{Q}$ by $\mathbbm{Q}/ \simeq$ , we note
  $\simeq [(n, k)] \in \mathbbm{Q}/ \simeq$ as $\frac{n}{k}$, $n$ is called
  the denominator, $k$ is the nominator. We have then that $\frac{n}{k} =
  \frac{n'}{k'}$ iff $(n, k) \sim (n', k') \Leftrightarrow n \cdot k' = k
  \cdot n'$
\end{definition}

\begin{theorem}
  \label{addition of rational numbers}$\upl : \mathbbm{Q} \times \mathbbm{Q}
  \rightarrow \mathbbm{Q}$ defined by $\left( \frac{n}{m}, \frac{r}{k} \right)
  \rightarrow \frac{n}{m} \upl \frac{r}{k} = \frac{n \cdot k + m \cdot r}{m
  \cdot k}$ is a function
\end{theorem}

\begin{proof}
  First note that as $m, k \in \mathbbm{Z}_0$ then $m \cdot k \in
  \mathbbm{Z}_0$ [if $m \cdot k = 0 \Rightarrowlim_{\text{\ref{properties of
  multiplication of integer numbers}}} m = n = 0$ contradicting $m, k \in
  \mathbbm{Z}_0$] we have thus that $\frac{n \cdot k \upl m \cdot r}{m \cdot
  k} \in \mathbbm{Q}$
  
  Second, suppose that $\frac{n}{m} = \frac{n'}{m'}$ and $\frac{r}{k} =
  \frac{r'}{k'}$ \ then we have $n \cdot m' = m \cdot n'$ and $r \cdot k' = k
  \cdot r'$. Now
  \begin{eqnarray*}
    (n \cdot k \upl m \cdot r) \cdot (m' \cdot k') & = & (n \cdot k) \cdot (m'
    \cdot k') \upl (m \cdot r) \cdot (m' \cdot k')\\
    & = & (n \cdot m') \cdot (k \cdot k') \upl (r \cdot k') \cdot (m \cdot
    m')\\
    & = & (m \cdot n') \cdot (k \cdot k') \upl (k \cdot r') \cdot (m \cdot
    m')\\
    & = & (n' \cdot k') \cdot (m \cdot k) + (m' \cdot r') \cdot (m \cdot k)\\
    & = & (m \cdot k) \cdot (n' \cdot k' \upl m' \cdot r')\\
    & \Rightarrow & \\
    \frac{n \cdot k \upl m \cdot r}{m \cdot k} & = & \frac{n' \cdot k' \upl m'
    \cdot r'}{m' \cdot k'}
  \end{eqnarray*}
  meaning that the function is well defined.
\end{proof}

\begin{theorem}
  \label{a/b=a.k/b.k if k<less><gtr>0}If $k \in \mathbbm{Z}_0$ and
  $\frac{a}{b} \in \mathbbm{Q}$ then $\frac{a}{b} = \frac{a \cdot k}{b \cdot
  k}$
\end{theorem}

\begin{proof}
  $a \cdot (b \cdot k) = (b \cdot a) \cdot k = b \cdot (a \cdot k) \Rightarrow
  \frac{a}{b} = \frac{a \cdot k}{b \cdot k}$
\end{proof}

\begin{theorem}
  \label{the set of rational numbers forms a abelian group for
  addition}{\index{$\langle \mathbbm{Q}, \upl \rangle$}}$\langle \mathbbm{Q},
  + \rangle$ forms a \tmtextbf{abelian group} with neutral element $0 =
  \frac{0}{1}$. Note that the same symbol $0$ is used for neutral elements in
  $\mathbbm{Z}$ and $\mathbbm{Q}$, context tell's us always what we mean by
  $0$. 
\end{theorem}

\begin{proof}[associative][neutral element][inverse][commutative]
  
  \begin{enumerate}
    \item If $\frac{a}{b}, \frac{c}{d}, \frac{e}{f} \in \mathbbm{Q}$ then we
    have
    \begin{eqnarray*}
      \frac{a}{b} \upl \left( \frac{c}{d} \upl \frac{e}{f} \right) & = &
      \frac{a}{b} \upl \frac{c \cdot f \upl d \cdot e}{d \cdot f}\\
      & = & \frac{a \cdot (d \cdot f) \upl b \cdot (c \cdot f \upl d \cdot
      e)}{b \cdot (d \cdot f)}\\
      & = & \frac{(a \cdot d) \cdot f \upl (b \cdot c) \cdot f + (b \cdot d)
      \cdot e}{(b \cdot d) \cdot f}\\
      & = & \frac{(a \cdot d + b \cdot c) \cdot f + (b \cdot d) \cdot e}{(b
      \cdot d) \cdot f}\\
      & = & \frac{a \cdot d + b \cdot c}{b \cdot d} \upl \frac{e}{f}\\
      & = & \left( \frac{a}{b} \upl \frac{c}{d} \right) \upl \frac{e}{f}
    \end{eqnarray*}
    \item We define the neutral element in $\mathbbm{Q}$ to be $0 =
    \frac{0}{1}$ (note that one $0 \in \mathbbm{Z}$ and the other is in
    $\mathbbm{Q}$ context will always tell us which is which). If $\frac{a}{b}
    \in \mathbbm{Q}$ then we have
    \begin{eqnarray*}
      \frac{a}{b} \upl 0 & = & \frac{a}{b} \upl \frac{0}{1}\\
      & = & \frac{a \cdot 1 + b \cdot 0}{b \cdot 1}\\
      & = & \frac{a \upl 0}{b}\\
      & = & \frac{a}{b}\\
      0 \upl \frac{a}{b} & = & \frac{0}{1} \upl \frac{a}{b}\\
      & = & \frac{0 \cdot b + 1 \cdot a}{1 \cdot b}\\
      & = & \frac{a}{b}
    \end{eqnarray*}
    \item If $\frac{a}{b} \in \mathbbm{Q}$ then we prove that the inverse is
    $\frac{\um a}{b}$.
    
    \begin{proof}
      We have
      \begin{eqnarray*}
        \frac{a}{b} \upl \frac{\um a}{b} & = & \frac{a \cdot b \upl b \cdot
        (- a)}{b \cdot b}\\
        & = & \frac{b \cdot (a \upl (\um a))}{b \cdot b}\\
        & = & \frac{b \cdot 0}{b \cdot b}\\
        & \equallim_{\text{\ref{a/b=a.k/b.k if k<less><gtr>0}}} &
        \frac{0}{b}\\
        & = & \frac{b \cdot 0}{b \cdot 1}\\
        & = & \frac{0}{1} = 0\\
        \frac{\um a}{b} \upl \frac{a}{b} & = & \frac{(\um a) \cdot b \upl b
        \cdot a}{b \cdot b}\\
        & = & \frac{b \cdot (\um a \upl a)}{b \cdot b}\\
        & = & \frac{b \cdot 0}{b \cdot b}\\
        & = & 0
      \end{eqnarray*}
    \end{proof}
    
    \item If $\frac{a}{b}, \frac{c}{d} \in \mathbbm{Q}$ then
    \begin{eqnarray*}
      \frac{a}{b} \upl \frac{c}{d} & = & \frac{a \cdot d \upl b \cdot c}{b
      \cdot d}\\
      & = & \frac{c \cdot b \upl d \cdot a}{d \cdot b}\\
      & = & \frac{c}{d} \upl \frac{a}{b}
    \end{eqnarray*}
  \end{enumerate}
  
\end{proof}

\begin{notation}
  Whenever we write $a - b$ we mean $a \upl (\um b)$
\end{notation}

\begin{theorem}
  \label{multiplication of rational numbers}$\cdot : \mathbbm{Q} \times
  \mathbbm{Q} \rightarrow \mathbbm{Q}$ defined by $\left( \frac{n}{m},
  \frac{r}{k} \right) \rightarrow \frac{n}{m} \cdot \frac{r}{k} = \frac{n
  \cdot r}{m \cdot k}$ is a function.
\end{theorem}

\begin{proof}
  First as $m, k \in \mathbbm{Z}_0$ we have that $m \cdot k \in \mathbbm{Z}_0$
  [if $m \cdot k = 0 \Rightarrow m = k = 0$ by \ref{properties of
  multiplication of integer numbers}] so we have that $\frac{n \cdot r}{m
  \cdot k} \in \mathbbm{Q}$. Secondly if $\frac{n}{m} = \frac{n'}{m'}$ and
  $\frac{r}{k} = \frac{r'}{k'}$ then we have $n \cdot m' = m \cdot n'$ and $r
  \cdot k' = k \cdot r'$. We have then
  \begin{eqnarray*}
    (n \cdot r) \cdot (m' \cdot k') & = & (n \cdot m') \cdot (r \cdot k')\\
    & = & (m \cdot n') \cdot (k \cdot r')\\
    & = & (m \cdot k) \cdot (n' \cdot r')\\
    & \Rightarrow & \\
    \frac{n \cdot r}{m \cdot k} & = & \frac{n' \cdot r'}{m' \cdot k'}
  \end{eqnarray*}
  proving that the function is well-defined.
\end{proof}

\begin{theorem}
  \label{the rational numbers form a field}$\langle \mathbbm{Q}, +, \cdot
  \rangle$ forms a \tmtextbf{field}, the neutral element is $1 = \frac{1}{1}$
  and if $\frac{a}{b} \in \mathbbm{Q} \backslash \{ 0 \}$ the inverse for
  multiplication is $\frac{b}{a}$ (so $\left( \frac{a}{b} \right)^{\um 1} =
  \frac{b}{a}$). Note that $1$ stands here for the unit in $\mathbbm{Z}$ and
  $\mathbbm{Q}$, context tell's us which is which.
\end{theorem}

\begin{proof}[Distributive][Neutral][Commutative][Associative][Inverse for non
zero element]
  By \ref{the set of rational numbers forms a abelian group for addition} we
  have that $\langle \mathbbm{Q}, + \rangle$ is a abelian group, we must thus
  only prove the additional properties (see \ref{field}) of the multiplication
  in a field.
  \begin{enumerate}
    \item If $\frac{a}{b}, \frac{c}{d}, \frac{e}{f} \in \mathbbm{Q}$ then we
    have
    \begin{eqnarray*}
      \frac{a}{b} \cdot \left( \frac{c}{d} \upl \frac{e}{f} \right) & = &
      \frac{a}{b} \cdot \left( \frac{c \cdot f \upl e \cdot d}{d \cdot f}
      \right)\\
      & = & \frac{a \cdot (c \cdot f \upl e \cdot d)}{b \cdot (d \cdot f)}\\
      & \equallim_{\text{\ref{a/b=a.k/b.k if k<less><gtr>0}}} & \frac{b \cdot
      (a \cdot (c \cdot f \upl e \cdot d))}{b \cdot (b \cdot (d \cdot f))}\\
      & = & \frac{(a \cdot c) \cdot (b \cdot f) \upl (a \cdot e) \cdot (b
      \cdot d)}{(b \cdot d) \cdot (b \cdot f)}\\
      & = & \frac{a \cdot c}{b \cdot d} \upl \frac{a \cdot e}{b \cdot f}\\
      & = & \left( \frac{a}{b} \cdot \frac{c}{d} \right) \upl \left(
      \frac{a}{b} \cdot \frac{e}{f} \right)
    \end{eqnarray*}
    \item The neutral element $1 = \frac{1}{1}$ (note that we have two 1's
    here context will always tell if $1 \in \mathbbm{Z}$ or $1 \in
    \mathbbm{Q}$), we have then if $\frac{a}{b} \in \mathbbm{Q}$ that
    \begin{eqnarray*}
      1 \cdot \frac{a}{b} & = & \frac{1}{1} \cdot \frac{a}{b}\\
      & = & \frac{1 \cdot a}{1 \cdot b}\\
      & = & \frac{a}{b}\\
      & = & \frac{a \cdot 1}{b \cdot 1}\\
      & = & \frac{a}{b} \cdot \frac{1}{1}\\
      & = & \frac{a}{b} \cdot 1
    \end{eqnarray*}
    \item If $\frac{a}{b}, \frac{c}{d} \in \mathbbm{Q}$ then we have
    \begin{eqnarray*}
      \frac{a}{b} \cdot \frac{c}{d} & = & \frac{a \cdot c}{b \cdot d}\\
      & = & \frac{c \cdot a}{d \cdot b}\\
      & = & \frac{c}{d} \cdot \frac{a}{b}
    \end{eqnarray*}
    \item If $\frac{a}{b}, \frac{c}{d}, \frac{e}{f} \in \mathbbm{Q}$ then we
    have
    \begin{eqnarray*}
      \frac{a}{b} \cdot \left( \frac{c}{d} \cdot \frac{e}{f} \right) = &
      \frac{a}{b} \cdot \frac{c \cdot e}{d \cdot f} & \\
      = & \frac{a \cdot (c \cdot e)}{b \cdot (d \cdot f)} & \\
      = & \frac{(a \cdot c) \cdot e}{(b \cdot d) \cdot f} & \\
      = & \frac{a \cdot c}{b \cdot d} \cdot \frac{e}{f} & \\
      = & \left( \frac{a}{b} \cdot \frac{c}{d} \right) \cdot \frac{e}{f} & \\
      &  & 
    \end{eqnarray*}
    \item If $\frac{a}{b} \in \mathbbm{Q} \backslash 0$ then $\frac{a}{b} \neq
    \frac{0}{1}$. Now if $a = 0$ then $a \cdot 1 = 0 = b \cdot 0 \Rightarrow
    \frac{a}{b} = \frac{0}{1}$ a contradiction, so we must have $a \neq 0$ and
    thus $\frac{b}{a} \in \mathbbm{Q}$. we prove now that $\left( \frac{a}{b}
    \right)^{- 1} = \frac{b}{a}$
    \begin{eqnarray*}
      \frac{b}{a} \cdot \frac{a}{b} & = & \frac{b \cdot a}{a \cdot b}\\
      & = & \frac{a \cdot b}{a \cdot b}\\
      & = & \frac{(a \cdot b) \cdot 1}{(a \cdot b) \cdot 1}\\
      & \equallim_{\text{\ref{a/b=a.k/b.k if k<less><gtr>0}}} & \frac{1}{1}\\
      & = & 1
    \end{eqnarray*}
  \end{enumerate}
\end{proof}

\section{Order relation}

\begin{theorem}
  $\tmop{sign} : \mathbbm{Q} \rightarrow \{ 1, \um 1 \} \subseteq \mathbbm{N}$
  defined by
  \begin{eqnarray*}
    \tmop{sign} \left( \frac{a}{b} \right) & = & \left\{ \begin{array}{l}
      1 \tmop{if} 0 \leqslant a \cdot b\\
      - 1 \tmop{if} \neg (0 \leqslant a \cdot b)
    \end{array} \right.
  \end{eqnarray*}
\end{theorem}

is a well defined function.

\begin{proof}
  If $\frac{a}{b} = \frac{a'}{b'}$ then we must prove that $\tmop{sign} \left(
  \frac{a}{b} \right) = \tmop{sign} \left( \frac{a'}{b'} \right)$. As
  $\frac{a}{b} = \frac{a'}{b'}$ then $a \cdot b' = b \cdot a'$. We have now
  the following cases
  \begin{enumerate}
    \item $\tmop{sign} \left( \frac{a}{b} \right) = 1 \Rightarrow 0 \leqslant
    a \cdot b$ then by \ref{0<less>=n.n if n is a integer} we have $0
    \leqslant b' \cdot b'$ giving by \ref{n<less>=m=<gtr>n.k<less>=m.k n,m,k
    integers k<gtr>=0} that $0 \cdot (b' \cdot b') \leqslant (a \cdot b) \cdot
    (b' b') \Rightarrow 0 \leqslant (a \cdot b') \cdot (b \cdot b')
    \Rightarrow 0 \leqslant (b \cdot a') \cdot (b \cdot b') \Rightarrow 0
    \leqslant (a' \cdot b') \cdot (b \cdot b)$. We proceed now by
    contradiction that $0 \leqslant a' \cdot b'$, so assume that $a' \cdot b'
    < 0 \Rightarrowlim_{\text{\ref{0<less>=n.n if n is a integer},
    \ref{n<less>m=<gtr>n.k<less>m.k if k<gtr>0,n,m integers}}} (a' \cdot b')
    \cdot (b \cdot b) < 0$ contradicting $0 \leqslant (a' \cdot b') \cdot (b
    \cdot b)$ so $0 \leqslant a' \cdot b' \Rightarrow \tmop{sign} \left(
    \frac{a'}{b'} \right) = 1$
    
    \item $\tmop{sign} \left( \frac{a}{b} \right) = \um 1 \Rightarrow \neg (0
    \leqslant a \cdot b) \Rightarrowlim_{\text{\ref{linear ordered class},
    \ref{whole numbers are fully-ordered}}} a \cdot b < 0
    \Rightarrowlim_{\text{\ref{0<less>=n.n if n is a integer},
    \ref{n<less>m=<gtr>n.k<less>m.k if k<gtr>0,n,m integers}}} (a \cdot b)
    \cdot (b' \cdot b') < 0 \Rightarrow (a \cdot b') \cdot (b \cdot b') < 0
    \Rightarrow (a' \cdot b) \cdot (b \cdot b') < 0 \Rightarrow (a' \cdot b')
    \cdot (b \cdot b) < 0$. Suppose now that $0 \leqslant (a' \cdot b')$ then
    by \ref{n<less>=m=<gtr>n.k<less>=m.k n,m,k integers k<gtr>=0},
    \ref{0<less>=n.n if n is a integer} we have $0 \leqslant (a' \cdot b')
    \cdot (b \cdot b) < 0 \Rightarrow 0 < 0$ a contradiction, so we must have
    $(a' \cdot b') < 0$ or $\tmop{sign} \left( \frac{a'}{b'} \right) = - 1$
  \end{enumerate}
\end{proof}

\begin{lemma}
  \label{sign(q)=-1=<gtr>sign(-q)=1}If $\tmop{sign} (q) = \um 1$ then
  $\tmop{sign} (\um q) = 1$
\end{lemma}

\begin{proof}
  If $q = \frac{a}{b}$ then from $\tmop{sign} (q) = 1$ we have $a \cdot b < 0$
  and thus by \ref{inverse and order in integers} we have $\um 0 < - (a \cdot
  b) \Rightarrow 0 < (\um a) \cdot b \Rightarrow 0 \leqslant (- a) \cdot b
  \Rightarrow \tmop{sign} \left( \frac{- a}{b} \right) = 1 \Rightarrow
  \tmop{sign} (- q) = 1$
\end{proof}

\begin{definition}[Order Relation $\mathbbm{Q}$]
  \label{order relation on rational numbers}{\index{order relation on rational
  numbers}}$\leqslant \subseteq \mathbbm{Q} \times \mathbbm{Q}$ is defined by
  $\leqslant = \{ (q, r) \in \mathbbm{Q} | \tmop{sign} (r \upl (\um q)) = 1
  \nobracket \}$ (note that this definition is independent of the
  representation of a rational number as $\tmop{sign}$ is a well-defined
  function). So we have $q \leqslant r \Leftrightarrow \tmop{sign} (r \upl
  (\um q)) = 1$
\end{definition}

\begin{lemma}
  \label{0<less>=q<less>=0=<gtr>q=0 q rational}If $q \in \mathbbm{Q}$ with $0
  \leqslant q$ and $q \leqslant 0$ then $q = 0$
\end{lemma}

\begin{proof}
  Let $q = \frac{a}{b}$ then $- q = \frac{- a}{b}$. From $0 \leqslant q \wedge
  q \leqslant 0$ we have $1 = \tmop{sign} (q \upl (- 0)) = \tmop{sign} (q)$
  and $1 = \tmop{sign} (0 \upl (- q)) = \tmop{sign} (- q)$ and thus
  $\tmop{sign} \left( \frac{a}{b} \right) = 1 \wedge \tmop{sign} \left(
  \frac{- a}{b} \right) = 1$. This gives $0 \leqslant a \cdot b \wedge 0
  \leqslant (- a) \cdot b \Rightarrowlim_{\ref{inverse and order in integers}
  \text{}} 0 \leqslant a \cdot b \wedge (- ((- a) \cdot b)) \leqslant - 0
  \Rightarrow 0 \leqslant a \cdot b \leqslant 0 \Rightarrow a \cdot b = 0
  \Rightarrowlim_{b \neq 0 \wedge \ref{properties of multiplication of integer
  numbers} \text{}} a = 0 \Rightarrow q = \frac{0}{b} = \frac{0}{1} = 0$
\end{proof}

\begin{lemma}
  \label{q<less>=r=<gtr>q+s<less>=r+s rationals}If $q, r, s \in \mathbbm{Q}$
  with $q \leqslant r \Rightarrow q \upl s \leqslant r \upl s$
\end{lemma}

\begin{proof}
  $q \leqslant r \Rightarrow 1 = \tmop{sign} (r + (- q)) = \tmop{sign} (r + (s
  + (- s)) + (- q)) = \tmop{sign} ((r + s) \upl (- (q \upl s))) \Rightarrow q
  \upl s \leqslant r \upl s$
\end{proof}

\begin{lemma}
  \label{0<less>=q and 0<less>=r=<gtr>0<less>=q+r}If $q, r \in \mathbbm{Q}$
  with $0 \leqslant q$ and $0 \leqslant r$ then $0 \leqslant q + r$
\end{lemma}

\begin{proof}
  Let $q = \frac{a}{b}$ and $r = \frac{a'}{b'}$. From $0 \leqslant q$ we have
  $1 = \tmop{sign} (q \upl (- 0)) = \tmop{sign} (q)$ and from $0 \leqslant r$
  we have $1 = \tmop{sign} (r \upl (- 0)) = \tmop{sign} (r)$. This gives us $0
  \leqslant a \cdot b$ and $0 \leqslant a' \cdot b'$. Now $\frac{a}{b} +
  \frac{a'}{b'} = \frac{a \cdot b' + b \cdot a'}{b \cdot b'}$ and \ 
\end{proof}

\begin{theorem}
  \label{rational numbers are fully ordered}$\langle \mathbbm{Q}, \leqslant
  \rangle$ forms a partially ordered set that is fully-ordered
\end{theorem}

\begin{proof}[Reflexivity][Anti-symmetry][Transitivity][Fully-ordered]
  
  \begin{enumerate}
    \item If $q \in \mathbbm{Q} \Rightarrow q \um q = 0 = \frac{0}{1}
    \Rightarrow \tmop{sign} (q \um q) = \tmop{sign} \left( \frac{0}{1} \right)
    \equallim_{0 \leqslant 0 = 0 \cdot 1} 1 \Rightarrow q \leqslant q$
    
    \item If $q \leqslant r$ and $r \leqslant q$ then we have using
    \ref{q<less>=r=<gtr>q+s<less>=r+s rationals} that $q \upl (- q) \leqslant
    r \upl (- q)$ and $r \upl (- r) \leqslant q \upl (- r) \Rightarrow 0
    \leqslant r \upl (- q)$ and $0 \leqslant q + (- r) = - (r \upl (- q))
    \Rightarrow 0 \leqslant r + (- q) \leqslant 0
    \Rightarrowlim_{\text{\ref{0<less>=q<less>=0=<gtr>q=0 q rational}}} r \upl
    (- q) = 0 \Rightarrow r = q$
    
    \item If $q \leqslant r$ and $r \leqslant s$ then using
    \ref{q<less>=r=<gtr>q+s<less>=r+s rationals} we have $0 = q \upl (\um q)
    \leqslant r \upl (- q)$ and $0 = r \upl (- r) \leqslant s \upl (- r)$,
    using \ref{0<less>=q and 0<less>=r=<gtr>0<less>=q+r} we have then that $0
    \leqslant (r \upl (- q)) + (s + (- r)) = s \upl (- q)$ which using
    \ref{q<less>=r=<gtr>q+s<less>=r+s rationals} again gives $q = 0 \upl q
    \leqslant (s \upl (- q)) \upl q \Rightarrow q \leqslant s$ proving
    transitivity.
    
    \item Let $q, r \in \mathbbm{Q}$ then for $q \upl (- r)$ we have either
    \begin{enumerate}
      \item $\tmop{sign} (q \upl (\um r)) = 1 \Rightarrow r \leqslant q$
      
      \item $\tmop{sign} (q \upl (- r)) = \um 1$ then by
      \ref{sign(q)=-1=<gtr>sign(-q)=1} we have $\tmop{sign} (- (q + (- r))) =
      1 \Rightarrow \tmop{sign} (r \upl (- q)) = 1 \Rightarrow q \leqslant r$
    \end{enumerate}
  \end{enumerate}
\end{proof}

\begin{theorem}
  \label{q<less>=r=<gtr>-r<less>=-q for rational numbers}If $q, r \in
  \mathbbm{Q}$ with $q \leqslant r$ then we have $- r \leqslant - q$
\end{theorem}

\begin{proof}
  If $q \leqslant r$ then $\tmop{sign} (r \um q) = 1$. Now $(- q) - (- r) = (-
  q) \upl (- (- r)) = (- q) + r = r \upl (- q) = r \um q$ and this
  $\tmop{sign} ((\um q) \um (\um r)) = \tmop{sign} (r \um q) = 1 \Rightarrow -
  r \leqslant - q$
\end{proof}

\begin{theorem}
  \label{q-1<less>q}If $q \in \mathbbm{Q} \Rightarrow q < q \upl 1$ and $q \um
  1 < q$
\end{theorem}

\begin{proof}
  First in $\mathbbm{N}$ we have $0 < 1$ so that using the injectivity and
  order preserving off $i_{\mathbbm{N}}$ we have $0 = \sim (s (0), 1) =
  i_{\mathbbm{N}} (0) < i_{\mathbbm{N}} (1) = \sim (s (1), 1) = 1$ so that $0
  < 1$ in $\mathbbm{Z}$. In the same way we have that $0 = \frac{0}{1} =
  i_{\mathbbm{Z}} (0) < i_{\mathbbm{Z}} (1) = \frac{1}{1} = 1$ so that $0 < 1$
  in $\mathbbm{Q}$. Now we have $0 < 1 \Rightarrow 0 \leqslant 1
  \xRightarrow[\text{\ref{q<less>=r=<gtr>q+s<less>=r+s rationals}}]{} 0 \upl q
  \leqslant 1 \upl q \Rightarrow q \leqslant q \upl 1$ if now $q = q + 1$ then
  $0 = q \upl (\um q) = q \upl (- q) + 1 = 1$ contradicting $0 < 1$ so we have
  that $q < q \upl 1$. Finally $q < q \upl 1 \Rightarrow q \leqslant q \upl 1
  \Rightarrow q \upl (- 1) \leqslant q + 1 \upl (- 1) \Rightarrow q - 1
  \leqslant q$, if now $q - 1 = q$ then $- 1 = 0 \Rightarrow 0 = 1$ a
  contradiction, so we have $q \um 1 < q$.
\end{proof}

\begin{definition}
  {\index{$\mathbbm{Z}_{\mathbbm{Q}}$}}$\mathbbm{Z}_{\mathbbm{Q}} = \left\{
  \frac{a}{1} | a \in \mathbbm{Z} \nobracket \right\}$
\end{definition}

\begin{theorem}
  \label{embedding of the whole numbers in the rationals}$\langle
  \mathbbm{Z}_{\mathbbm{Q}}, +, \cdot \rangle$ is a sub-ring of $\langle
  \mathbbm{Q}, +, \cdot \rangle$ and $i_{\mathbbm{Z}} : \mathbbm{Z}
  \rightarrow \mathbbm{Z}_{\mathbbm{Q}}$ defined by $n \rightarrow
  \frac{n}{1}$ is a ring-isomorphism that is order preserving
\end{theorem}

\begin{proof}[Injectivity][Surjectivity]
  First we prove that $\langle \mathbbm{Z}_{\mathbbm{Q}}, \upl, \cdot \rangle$
  is a sub ring of $\mathbbm{Q}$. If $m, n \in \mathbbm{Z}_{\mathbbm{Q}}$ then
  there exists a $m', n' \in \mathbbm{Z}$ then $m = \frac{m'}{1}, n =
  \frac{n'}{1}$ we have then:
  \begin{enumerate}
    \item $m \upl n = \frac{m'}{1} \upl \frac{n'}{1} = \frac{m' \cdot 1 + n'
    \cdot 1}{1 \cdot 1} = \frac{m' \upl n'}{1} \in \mathbbm{Z}_{\mathbbm{Q}}$
    
    \item $- \frac{n}{1} = \frac{- n}{1} \in \mathbbm{Z}_{\mathbbm{Q}}$
    
    \item $m \cdot n = \frac{m' \cdot n'}{1 \cdot 1} = \frac{m' \cdot n'}{1}
    \in \mathbbm{Z}_{\mathbbm{Q}}$
    
    \item $1 = \frac{1}{1} \in \mathbbm{Z}_{\mathbbm{Q}}$
    
    \item $0 = \frac{0}{1} \in \mathbbm{Z}_{\mathbbm{Q}}$
  \end{enumerate}
  proving that $\langle \mathbbm{Z}_{\mathbbm{Q}}, +, \cdot \rangle$ is a
  sub-ring of $\langle \mathbbm{Q}, +, \cdot \rangle$. Next prove that
  $i_{\mathbbm{Z}}$ is a ring isomorphism
  \begin{enumerate}
    \item If $m, n \in \mathbbm{Z}$ then $i_{\mathbbm{Z}} (m \upl n) = \frac{m
    \upl n}{1} = \frac{m \cdot 1 + n \cdot 1}{1 \cdot 1} = \frac{m}{1} \upl
    \frac{n}{1} = i_{\mathbbm{Z}} (m) + i_{\mathbbm{Z}} (n)$
    
    \item If $m, n \in \mathbbm{Z}$ then $i_{\mathbbm{Z}} (m \cdot n) =
    \frac{m \cdot n}{1} = \frac{m \cdot n}{1 \cdot 1} = \frac{m}{1} \cdot
    \frac{n}{1} = i_{\mathbbm{Z}} (m) \cdot i_{\mathbbm{Z}} (n)$
    
    \item $i_{\mathbbm{Z}} (1) = \frac{1}{1} = 1$ (note that $1$ stands for
    the unit in $\mathbbm{Z}$ and $\mathbbm{Q}$)
  \end{enumerate}
  Secondly we have to prove that $i_{\mathbbm{Z}}$ is a bijection.
  \begin{enumerate}
    \item If $i_z (m) = i_z (n) \Rightarrow \frac{m}{1} = \frac{n}{1}
    \Rightarrow m \cdot 1 = n \cdot 1 \Rightarrow m = n$
    
    \item If $a \in \mathbbm{Z}$ then $\exists a' \in \mathbbm{Q}$ with $a =
    \frac{a'}{1} = i_{\mathbbm{Z}} (a')$
  \end{enumerate}
  Finally to prove that $i_{\mathbbm{Z}}$ is order preserving, assume that $m,
  n \in \mathbbm{Z}$ with $n \leqslant m$ then we have $i_{\mathbbm{Z}} (m)
  \upl (\um i_{\mathbbm{Z}} (n)) = \frac{m}{1} \upl \frac{- n}{1} = \frac{m
  \cdot 1 + (\um n) \cdot 1}{1 \cdot 1} = \frac{m - n}{1}$, now from $n
  \leqslant m$ we have $0 \leqslant m - n$ so that $(m - n) \cdot 1 = m - n
  \geqslant 0 \Rightarrow \tmop{sign} (i_{\mathbbm{Z}} (m) \upl (-
  i_{\mathbbm{Z}} (n))) = \tmop{sign} \left( \frac{m - n}{1} \right) = 1
  \Rightarrow i_{\mathbbm{Z}} (n) \leqslant i_{\mathbbm{Z}} (m)$.
\end{proof}

\begin{lemma}
  \label{lemma for archimedean of rationals}If $q \in \mathbbm{Q}$ with $q >
  0$ then there exists $a, b \in \mathbbm{N}_{\mathbbm{Z}} \backslash \{ 0 \}$
  such that $q = \frac{a}{b}$
\end{lemma}

\begin{proof}
  If $q \in \mathbbm{Q}$ then there exist a $a', b' \in \mathbbm{Z}, b \neq 0$
  such that $q = \frac{a'}{b'}$ and as $q > 0$ $a' \cdot b' \geqslant 0$. We
  have the following possibilities (as $a' = 0$ would mean that $q = 0$ a
  contradiction).
  \begin{enumerate}
    \item $a' < 0 \wedge b' > 0$ then using \ref{n<less>m=<gtr>n.k<less>m.k if
    k<gtr>0,n,m integers} $a' \cdot b' < 0 \cdot b' = 0$ contradicting $a'
    \cdot b' \geqslant 0$, so this case does not apply.
    
    \item $a' > 0 \wedge b' < 0$ then using \ref{n<less>m=<gtr>n.k<less>m.k if
    k<gtr>0,n,m integers} $a' \cdot b' < 0 \cdot a' = 0$ contradicting $a'
    \cdot b' \geqslant 0$, so this case does not apply.
    
    \item $a' < 0 \wedge b' < 0$ then using \ref{inverse and order in
    integers} we have $a = \um a' > 0$, $b = - b' > 0$ and $\frac{a}{b} =
    \frac{(\um 1) \cdot a'}{(\um 1) \cdot b'} = \frac{a'}{b'} = q$
    
    \item $a' > 0 \wedge b' > 0$ then we take $a = a', b = b'$ and have $q =
    \frac{a}{b}$
  \end{enumerate}
\end{proof}

\begin{lemma}
  \label{0<less>r,0<less>=r=<gtr>0<less>r.s,0<less>=r.s 0<less>s r,s
  rational}If $r \in \mathbbm{Q}$ with $0 < r$ (or $0 \leqslant r$) then for
  $s \in \mathbbm{Q}, s > 0$ we have $0 < s \cdot r$ \ (or $0 \leqslant s
  \cdot r$)
\end{lemma}

\begin{proof}[$0 < r$][$0 \leqslant r$]
  Using the previous lemma (see \ref{lemma for archimedean of rationals}) we
  have that $s = \frac{a}{b}$ where $a, b \in \mathbbm{N}_{\mathbbm{Z}}
  \backslash \{ 0 \} \Rightarrow a, b > 0$. Then
  \begin{enumerate}
    \item Again by the previous lemma we have $r = \frac{c}{d}$ where $c, d >
    0$. Now $s \cdot r = \frac{a}{b} \cdot \frac{c}{d} = \frac{a \cdot c}{b
    \cdot d}$. Using \ref{0<less>n,0<less>m=<gtr>0<less>n.m n,m integers} we
    have $0 < a \cdot c, 0 < b \cdot d$ and using
    \ref{0<less>n,0<less>m=<gtr>0<less>n.m n,m integers} again we have $0 < (a
    \cdot c) \cdot (b \cdot d)$ so that $\tmop{sign} (s \cdot r - 0) =
    \tmop{sign} (s \cdot r) = 1 \Rightarrow 0 \leqslant s \cdot r$. If $s
    \cdot r = 0 \Rightarrow \frac{a \cdot c}{b \cdot d} = \frac{0}{1}
    \Rightarrow a \cdot c = 0$ contradicting $0 < a \cdot c \Rightarrow s
    \cdot r \neq 0 \Rightarrow 0 < s \cdot r$
    
    \item The we have the following cases
    \begin{enumerate}
      \item $r = 0 \Rightarrow r \cdot s = 0 \Rightarrow 0 \leqslant r \cdot
      s$
      
      \item $0 < r \Rightarrowlim_{(1)} 0 < s \cdot r = < 0 \leqslant r \cdot
      s$
    \end{enumerate}
  \end{enumerate}
  
\end{proof}

\begin{theorem}
  \label{n<less>m,0<less>s=<gtr>n.s<less>m.s rationals}If $r, q \in
  \mathbbm{Q}$ with $r < q$ (or $r \leqslant q$) and $s \in \mathbbm{Q}, 0 <
  s$ then $s \cdot r < s \cdot q$ (or $s \cdot r \leqslant s \cdot q$)
\end{theorem}

\begin{proof}[$r < q$][$r \leqslant q$]
  
  \begin{enumerate}
    \item In this case we have $0 < (q - r)$ and by the previous lemma we have
    then $0 < s \cdot (q - r) = s \cdot q - s \cdot r \Rightarrow s \cdot r <
    s \cdot q$
    
    \item In this case $0 \leqslant (q - r)$ and by the previous lemma we have
    then $0 \leqslant s \cdot (q - r) = s \cdot q - s \cdot r \Rightarrow s
    \cdot r \leqslant s \cdot q$
  \end{enumerate}
\end{proof}

\begin{definition}
  \label{naturals embedded in the
  rationals}{\index{$\mathbbm{N}_{\mathbbm{Q}}$}}$\mathbbm{N}_{\mathbbm{Q}} =
  \left\{ \frac{a}{1} | a \in \mathbbm{N}_{\mathbbm{Z}} \nobracket \right\}$
\end{definition}

\begin{theorem}
  \label{N_Q is semi group}$\langle \mathbbm{N}_{\mathbbm{Q}}, + \rangle$ is a
  sub-semi-group of $\langle \mathbbm{Q}, + \rangle$, furthermore if $a, b \in
  \mathbbm{N}_{\mathbbm{Q}}$ and $a \leqslant b$ then $b - a \in
  \mathbbm{N}_{\mathbbm{Q}}$
\end{theorem}

\begin{proof}
  Let's first prove that $\langle \mathbbm{N}_{\mathbbm{Q}}, + \rangle$ is a
  sub-semi-group of $\langle \mathbbm{Q}, + \rangle$.
  \begin{enumerate}
    \item If $a, b \in \mathbbm{N}_{\mathbbm{Q}}$ then there exists $a', b'
    \in \mathbbm{N}_{\mathbbm{Z}}$ such that $a = \frac{a'}{1}, b =
    \frac{b'}{1}$ and then $a \upl b = \frac{a'}{1} + \frac{b'}{1} = \frac{a'
    \cdot 1 + b' \cdot 1}{1 \cdot 1} = \frac{a' + b'}{1}$. As
    $\mathbbm{N}_{\mathbbm{Z}}$ is a semi-group (being a non empty
    semi-sub-group of $\mathbbm{Z}$, we have $a' \upl b' \in
    \mathbbm{N}_{\mathbbm{Z}} \Rightarrow a + b \in \mathbbm{N}_{\mathbbm{Q}}$
    
    \item $1 = \frac{1}{1} \in \mathbbm{N}_{\mathbbm{Q}}$
  \end{enumerate}
  Next if $a \leqslant b \Rightarrow \tmop{sign} (b - a) = a$ then $b \um a =
  b \upl (- a) = \frac{b'}{1} + \frac{- a'}{1} = \frac{b' - a'}{1}$ and as
  $\tmop{sign} (b - a) = 1$ we have then $(b' - a') \cdot 1 \geqslant 0
  \Rightarrow b' - a' \geqslant 0 \Rightarrow b' - a' \in
  \mathbbm{N}_{\mathbbm{Z}} \Rightarrow b - a \in \mathbbm{N}_{\mathbbm{Q}}$.
\end{proof}

\begin{theorem}
  \label{natural numbers in rational numbers are well-ordered}$\langle
  \mathbbm{N}_{\mathbbm{Q}}, \leqslant \rangle$ is well-ordered
\end{theorem}

\begin{proof}
  Let $A \subset \mathbbm{N}_{\mathbbm{Q}}$ be a non-empty subset of
  $\mathbbm{Q}$ then $i_{\mathbbm{Z}}^{- 1} (A)$ is non-empty as
  $i_{\mathbbm{Z}} : \mathbbm{Z} \rightarrow \mathbbm{Z}_{\mathbbm{Q}}$ is a
  bijection. Also if $a \in i_{\mathbbm{Z}}^{- 1} (A)$ then $\frac{a}{1} =
  i_{\mathbbm{Z}} (a) \in A \subseteq \mathbbm{N}_{\mathbbm{Q}} \Rightarrow
  \frac{a}{1} = \frac{a'}{1}$ where $a' \in \mathbbm{N}_{\mathbbm{Z}}
  \Rightarrow a \cdot 1 = a' \cdot 1 \Rightarrow a = a' \Rightarrowlim a \in
  \mathbbm{N}_{\mathbbm{Z}}$ so we have that $i_{\mathbbm{Z}}^{- 1} (A)
  \subseteq \mathbbm{N}_{\mathbbm{Z}}$. By well-ordering of
  $\mathbbm{N}_{\mathbbm{Z}}$ (see \ref{naturals in whole numbers are well
  ordered}) there exists a $m' = \min (i_{\mathbbm{Z}}^{- 1} (A))$. Take now
  $m = i_{\mathbbm{Z}} (m')$ then we have
  \begin{enumerate}
    \item $m' \in i_{\mathbbm{Z}}^{- 1} (A) \Rightarrow m = i_{\mathbbm{Z}}
    (m') \in A$
    
    \item $\forall a \in A$ we have by bijectivity of $i_{\mathbbm{Z}}$ $a' =
    i_{\mathbbm{Z}}^{- 1} (a) \in i_{\mathbbm{Z}}^{- 1} (A) \Rightarrow m'
    \leqslant a' \Rightarrow m = i_{\mathbbm{Z}} (m') \leqslant
    i_{\mathbbm{Z}} (a') = a \Rightarrow m = \min (A)$
  \end{enumerate}
\end{proof}

\begin{theorem}
  \label{N_Q forms positive integers}If $a \in \mathbbm{N}_{\mathbbm{Q}}$ then
  $0 \leqslant a$ and if $a \neq 0$ then $1 \leqslant a$
\end{theorem}

\begin{proof}
  If $a \in \mathbbm{N}_{\mathbbm{Q}}$ then there exists a $a' \in
  \mathbbm{N}_{\mathbbm{Z}}$ (so $0 \leqslant a'$ (see \ref{condition 1 for
  positive integers})) with $a = \frac{a'}{1}$, but then $0 \leqslant a' = a'
  \cdot 1 \Rightarrow \tmop{sign} (a - 0) = \tmop{sign} (a) = 1 \Rightarrow 0
  \leqslant a$. If $a \neq 0$ then $a' \neq 0$ and as $i_{\mathbbm{N}} :
  \mathbbm{N} \rightarrow \mathbbm{N}_{\mathbbm{Z}}$ is a bijection, there
  exists a $a'' \in \mathbbm{N}$ with $a' = i_{\mathbbm{N}} (a'')$, as $a'
  \neq 0$ we must have that $a'' \neq 0$ (otherwise $i_{\mathbbm{N}} (0) =
  \sim [(s (0), 1)] = \sim [(1, 1)] = 0$) and thus $0 < a''$ (by \ref{every
  natural number is bigger or equal to zero}). Using
  \ref{n<less>m=<gtr>s(n)<less>=m} we have then that $1 = s (0) \leqslant a''$
  and using the order preserving of $i_{\mathbbm{N}}$ we have then $1 = \sim
  [(s (1), 1)] = i_{\mathbbm{N}} (1) \leqslant i_{\mathbbm{N}} (a'') = a'
  \Rightarrow 1 \leqslant a'$. As $a - 1 = \frac{a'}{1} - \frac{1}{1} =
  \frac{a' - 1}{1}$ and $(a' - 1) \cdot 1 = a' \upl (- 1) \geqslant 0$ (as $a'
  \geqslant 1$) we have $\tmop{sign} (a - 1) = 1$ or $1 \leqslant a$.
\end{proof}

\

\begin{theorem}[Archimedean Property of $\mathbbm{Q}$]
  \label{archimedean property of the rationals}{\index{Archimedean property of
  $\mathbbm{Q}$}}If $x, y \in \mathbbm{Q}$ with $x > 0$ then there exists a $z
  \in \mathbbm{N}_{\mathbbm{Q}}$ such that $z \cdot x > y$
\end{theorem}

\begin{proof}
  For $y \in \mathbbm{Q}$ we have the following possibilities
  \begin{enumerate}
    \item $y \leqslant 0$ take then $z = 1 \in \mathbbm{Q}$ then as the unit
    $1$ in $\mathbbm{Z}$ is in $\mathbbm{N}_{\mathbbm{Z}}$ we have that $1 =
    \frac{1}{1} \in \mathbbm{N}_{\mathbbm{Q}}$ and $z \cdot x = 1 \cdot x = x
    > 0 \geqslant y \Rightarrow z \cdot x > y$.
    
    \item $0 < y$. As also $x > 0$ we have by the previous lemma (see
    \ref{lemma for archimedean of rationals}) that there exists $p, q, r, s
    \in \mathbbm{N}_{\mathbbm{Z}} \backslash \{ 0 \}$ such that $x =
    \frac{p}{q}$, $y = \frac{r}{s}$ as $p > 0, s > 0 \Rightarrow p \cdot s >
    0$ then by \ref{archimedean property of whole numbers} there exists a $z'
    \in \mathbbm{N}_{\mathbbm{Z}}$ such that $z' \cdot p \cdot s > q \cdot r
    \Rightarrow z' \cdot p \cdot s > q \cdot r$. Define \ $z = \frac{z'}{1}
    \in \mathbbm{N}_{\mathbbm{Q}}$ then $z \cdot x - y = \frac{z' \cdot p}{1
    \cdot q} - \frac{r}{s} = \frac{z' \cdot p \cdot s + (\um (q \cdot r))}{p
    \cdot q}$, as $p > 0, q > 0 \Rightarrow p \cdot q > 0 \Rightarrow (p \cdot
    q) \cdot (z' \cdot p \cdot s + (\um (q \cdot r))) > 0 \Rightarrow
    \tmop{sign} (z \cdot x - y) = 1 \Rightarrow z \cdot x > y$. \ 
  \end{enumerate}
\end{proof}

\begin{lemma}
  \label{0<less>s=<gtr>0<less>s^-1}If $s \in \mathbbm{Q}$ then $0 < s$ iff $0
  < s^{- 1}$
\end{lemma}

\begin{proof}
  If $0 < s$ then by \ref{lemma for archimedean of rationals} we have $s =
  \frac{a}{b}$ where $a, b > 0$ as $s^{- 1} = \frac{b}{a}$ we have $b \cdot a
  > 0 \Rightarrow \tmop{sign} (s^{- 1} \um 0) = \tmop{sign} (s^{- 1}) = 1
  \Rightarrow 0 < s^{- 1}$. If $0 < s^{- 1}$ then if $s \leqslant 0$ we have
  by \ref{0<less>r,0<less>=r=<gtr>0<less>r.s,0<less>=r.s 0<less>s r,s
  rational} $s \cdot s^{- 1} \leqslant s^{- 1} \cdot 0 \Rightarrow 1 \leqslant
  0$ a contradiction, so we must have $0 < s$.
\end{proof}

\begin{lemma}
  \label{some properties of rational numbers}If $r, q \in \mathbbm{Q}$ then we
  have
  \begin{enumerate}
    \item $0 < r < 1 \Rightarrow 1 < r^{- 1}$
    
    \item $1 < r \Rightarrow r^{- 1} < 1$
    
    \item $1 < r$ and $0 < s \Rightarrow s < r \cdot s$
    
    \item $r < 1$ and $0 < s \Rightarrow s \cdot r < s$
    
    \item $0 < r < s \Rightarrow s^{- 1} < r^{- 1}$
    
    \item $0 < r^{- 1} < s^{- 1} \Rightarrow s < r$
    
    \item If $r \neq 0$ then $- (r^{- 1}) = (- r)^{- 1}$
  \end{enumerate}
\end{lemma}

\begin{proof}
  
  \begin{enumerate}
    \item If $0 < r < 1
    \Rightarrowlim_{\text{\ref{0<less>s=<gtr>0<less>s^-1}}} 0 < r^{- 1}
    \Rightarrowlim_{\text{\ref{n<less>m,0<less>s=<gtr>n.s<less>m.s
    rationals}}} r \cdot r^{- 1} < 1 \cdot r^{- 1} \Rightarrow 1 < r^{- 1}$
    
    \item If $1 < r \Rightarrowlim_{0 < 1} 0 < r
    \Rightarrowlim_{\text{\ref{0<less>s=<gtr>0<less>s^-1}}} 0 < r^{- 1}
    \Rightarrowlim_{\text{\ref{n<less>m,0<less>s=<gtr>n.s<less>m.s
    rationals}}} 1 \cdot r^{- 1} < r \cdot r^{- 1} \Rightarrow r^{- 1} < 1$
    
    \item If $0 < s$ and $1 < r$ then by
    \ref{n<less>m,0<less>s=<gtr>n.s<less>m.s rationals} we have $1 \cdot s < r
    \cdot s = r \cdot s \Rightarrow s < r \cdot s$
    
    \item If $0 < s$ and $r < 1$ then by
    \ref{n<less>m,0<less>s=<gtr>n.s<less>m.s rationals} we have $r \cdot s < 1
    \cdot s \Rightarrow s \cdot r < s$
    
    \item If $0 < r < s$ then by the previous theorem we have $0 < r^{- 1},
    s^{- 1}$ and given $r < s
    \Rightarrowlim_{\text{\ref{n<less>m,0<less>s=<gtr>n.s<less>m.s
    rationals}}} r \cdot r^{- 1} < s \cdot r^{- 1} \Rightarrow 1 < s \cdot
    r^{- 1} \Rightarrowlim_{(3)} s^{- 1} < s^{- 1} (s \cdot r^{- 1}) = r^{- 1}
    \Rightarrow s^{- 1} < r^{- 1}$
    
    \item If $0 < r^{- 1} < s^{- 1}$ then using the previous lemma we have $0
    < r, 0 < s$ so using $r^{- 1} < s^{- 1}
    \Rightarrowlim_{\text{\ref{n<less>m,0<less>s=<gtr>n.s<less>m.s
    rationals}}} r^{- 1} \cdot r < s^{\um 1} \cdot r \Rightarrow 1 < s^{- 1}
    \cdot r \Rightarrowlim_{(3)} 1 \cdot s < s \cdot (s^{- 1} \cdot r) = s <
    r$
    
    \item If $r \neq 0$ then $- r \neq 0$ so $r^{- 1}$ and $(- r)^{- 1}$
    exists. Now if $r \neq 0$ then $r = \frac{a}{b}, a, b \neq 0$ and $- (r^{-
    1}) = - \left( \frac{b}{a} \right) = \left( \frac{- b}{a} \right) = \left(
    \frac{a}{- b} \right)^{- 1} = \left( \frac{- 1}{- 1} \cdot \frac{a}{- b}
    \right)^{- 1} = \left( \frac{- a}{b} \right)^{- 1} = (- r)^{- 1}$
  \end{enumerate}
\end{proof}

\begin{theorem}[$\mathbbm{Q}$ is dense]
  \label{rational numbers are dense}If $x, y \in \mathbbm{Q}$ with $x < y$
  then $\exists z \in \mathbbm{Q} \vdash x < z < y$
\end{theorem}

\begin{proof}
  Take $z = \frac{1}{2} \cdot (x \upl y) \in \mathbbm{Q}$, we have then $x < y
  \Rightarrowlim_{\text{\ref{n<less>m=<gtr>n+k<less>m+k for integers}}} x \upl
  x < x \upl y$ and $x \upl y < y \upl y$ so $\frac{2}{1} \cdot x = (1 + 1)
  \cdot x = x \upl x < x \upl y$ and $x \upl y < \frac{2}{1} \cdot y$. Using
  the fact that $0 < \frac{1}{2}$ and \ref{n<less>m=<gtr>n.k<less>m.k if
  k<gtr>0,n,m integers} we have then $x = \frac{1}{2} \cdot \frac{2}{1} \cdot
  x < \frac{1}{2} \cdot (x \upl y)$ and $\frac{1}{2} \cdot (x \upl y) <
  \frac{1}{2} \cdot \frac{2}{1} \cdot y = y$ and thus $x < \frac{1}{2} \cdot
  (x \upl y) < y$
\end{proof}

The following theorem shows that $\langle \mathbbm{Q}, \leqslant \rangle$ is
not well-ordered and this is the reason that we extend the rational numbers to
the set of real numbers.

\begin{lemma}
  $\forall r \in \mathbbm{Q}$ we have $r \cdot r \neq 2 = \frac{2}{1}$
\end{lemma}

\begin{proof}
  We prove this by contradiction so assume that $\exists r' \in \mathbbm{Q}$
  such that $r' \cdot r' = 2$. Of course $r' \neq 0$ [otherwise $0 = r' \cdot
  r' = 2 \Rightarrow 0 = 2 \neq 0$ a contradiction]. If now $r' < 0$ take then
  $r = - r' \Rightarrow r \cdot r = (\um r') \cdot (- r') = r' \cdot r' = 2$
  otherwise take $r = r'$. So we have that $\exists r \in \mathbbm{Q}$ with $0
  < r$ and $r \cdot r = 2$. Then there exists a $m, n \in \mathbbm{Z}$ such
  that $r = \frac{n}{m}$ and $m \neq 0$. Take now $n' = n / \gcd (n, m)$ and
  $m' = m / \gcd (n, m)$ then using \ref{basic property of gcd(n,m)} we have
  $d|n'$ and $d|m' \Rightarrow d = 1$ or $d = \um 1$. As $n = n' \cdot \gcd
  (n, m)$ and $m = m' \cdot \gcd (n, m)$ we have $r = \frac{n' \cdot \gcd (n,
  m)}{m' \cdot \gcd (n, m)} = \frac{n'}{m'}$. Now if $r \cdot r = 2
  \Rightarrow \frac{n' \cdot n'}{m' \cdot m'} = \frac{2}{1} \Rightarrow n'
  \cdot n' = 2 \cdot m' \cdot m'$ then we have that $n' \cdot n'$ is even and
  using \ref{m*m is even then m is even} we have that $n'$ is even and thus
  there exists a $k \in \mathbbm{Z}$ such that $n' = 2 \cdot k$. From this it
  follows that $2 \cdot m' \cdot m' = n' \cdot n' = 2 \cdot k \cdot 2 \cdot k
  \Rightarrow m' \cdot m' = 2 \cdot (k \cdot k)$ proving that $m' \cdot m'$ is
  even and by \ref{m*m is even then m is even} that $m'$ is even and thus the
  existence of a $l \in \mathbbm{Z}$ such that $m' = 2 \cdot l$. We have thus
  that $2| m'$ and $2| n'$ but then we must have either $2 = - 1$ or $2 = 1$
  both of which are impossible so we reach a contradiction. 
\end{proof}

\begin{theorem}
  \label{the set of rationals is not conditional complete}$\langle
  \mathbbm{Q}, \leqslant \rangle$ is not conditional complete, so there exists
  a non-empty set which is bounded above that does not have a least upper
  bound.
\end{theorem}

\begin{proof}[$u \cdot u = \frac{2}{1}$][$u \cdot u <
\frac{2}{1}$][$\frac{2}{1} < u \cdot u$]
  Consider the set $A = \left\{ r \in \mathbbm{Q} \left| r > 0 \wedge r \cdot
  r < \frac{2}{1} \right. \right\}$, then as $0 < \frac{4}{3}$ and
  $\frac{2}{1} - \frac{4}{3} \cdot \frac{4}{3} = \frac{18 - 16}{9} =
  \frac{2}{9} > 0 \Rightarrow \frac{4}{3} \cdot \frac{4}{3} < 2$ we have
  $\frac{4}{3} \in A$ so that $A \neq \emptyset$. Also if $x \in A$ then $x
  \leqslant \frac{2}{1}$ [if $\frac{2}{1} < x \Rightarrowlim_{x \in A
  \Rightarrow 0 < x} \frac{2}{1} \cdot x < x \cdot x$ and $\frac{2}{1} \cdot
  \frac{2}{1} < \frac{2}{1} \cdot x \Rightarrow \frac{4}{1} < x \cdot x
  \Rightarrow \frac{2}{1} < x \cdot x$ contradicting $x \in A \Rightarrow x
  \cdot x < \frac{2}{1}$] so $\frac{2}{1}$ is a upper bound of $A$. We prove
  now by contradiction that $u = \sup (A)$ does not exists . So assume that
  $\sup (A)$ exists. As $\frac{4}{2} \in A$ and $\frac{4}{3} - 1 = \frac{4 -
  3}{3} = \frac{1}{3} > 0 \Rightarrow 1 < \frac{4}{3} \leqslant u \Rightarrow
  0 < 1 < u$ we can have now only the following cases for $u \cdot u$
  ($\langle \mathbbm{Q}, \leqslant \rangle$ is fully-ordered)
  \begin{enumerate}
    \item Using the previous lemma we have shown that this is impossible.
    
    \item Then $u \in A$. Now given a $n \in \mathbbm{N}_{\mathbbm{Z}}
    \backslash \{ 0 \}$ we have
    \begin{eqnarray*}
      \left( u \upl \frac{1}{n} \right) \left( u \upl \frac{1}{n} \right) & =
      & u \cdot u + u \cdot \frac{1}{n} \upl u \cdot \frac{1}{n} \upl
      \frac{1}{n} \cdot \frac{1}{n}\\
      & = & u \cdot u \upl 2 \cdot u \cdot \frac{1}{n} \upl \frac{1}{n} \cdot
      \frac{1}{n}
    \end{eqnarray*}
    and thus
    \begin{eqnarray*}
      \left( u \upl \frac{1}{n} \right) \cdot \left( u \upl \frac{1}{n}
      \right) < \frac{2}{1} & \Leftrightarrow & u \cdot u \upl 2 \cdot u \cdot
      \frac{1}{n} \upl \frac{1}{n} \cdot \frac{1}{n} < \frac{2}{1}\\
      & \Leftrightarrow & 2 \cdot u \cdot \frac{1}{n} \upl \frac{1}{n} \cdot
      \frac{1}{n} < \frac{2}{1} \um u \cdot u
    \end{eqnarray*}
    Since $\frac{2}{1}$ is a upper bound of $A$ we have $u \leqslant
    \frac{2}{1}$. Now if $n \in \mathbbm{N}_{\mathbbm{Z}} \backslash \{ 0 \}$
    then $0 < n$ and using \ref{0<less>m=<gtr>1<less>=m if m is a integer} we
    have then $1 \leqslant n \Rightarrow 0 \leqslant n \um 1$ and as
    $\frac{1}{n} - \frac{1}{n} \cdot \frac{1}{n} = \frac{1}{n} - \frac{1}{n
    \cdot n} = \frac{n \um 1}{n \cdot n} \geqslant 0 \Rightarrow \frac{1}{n}
    \cdot \frac{1}{n} \leqslant \frac{1}{n}$ and so we have
    \begin{eqnarray*}
      2 \cdot u \cdot \frac{1}{n} \upl \frac{1}{n} \cdot \frac{1}{n} &
      \leqslant & \frac{2}{1} \cdot \frac{2}{1} \cdot \frac{1}{n} \upl
      \frac{1}{n}\\
      & \leqslant & \frac{5}{n}
    \end{eqnarray*}
    Since $0 < 5$ and by assumption $0 < 2 \um u \cdot u$ we have by the
    Archimedean property (see \ref{archimedean property of the rationals}) the
    existence of a $n_0' \in \mathbbm{N}_{\mathbbm{Q}}$ (thus $n_0' =
    \frac{n_0}{1}, n_0 \in \mathbbm{N}_{\mathbbm{Z}}$) such that
    \begin{eqnarray*}
      5 & < & n'_0 \cdot (2 \um u \cdot u) = \frac{n_0}{1} \cdot (2 - u \cdot
      u)
    \end{eqnarray*}
    as $n_0 \neq 0$ (otherwise $5 = 0$) we have then by multiplying both sides
    of the above by $\frac{1}{n_0}$ that
    \begin{eqnarray*}
      \frac{5}{n_0} & < & 2 - u \cdot u
    \end{eqnarray*}
    and this gives that $2 \cdot u \cdot \frac{1}{n_0} \upl \frac{1}{n_0}
    \cdot \frac{1}{n_0} \leqslant \frac{5}{n_0} < \frac{2}{1} - u \cdot u$ and
    thus that $\left( u \upl \frac{1}{n_0} \right) \cdot \left( u +
    \frac{1}{n_0} \right) < \frac{2}{1}$ which as $0 < u < u + \frac{1}{n_0}$
    means that $u + \frac{1}{n_0} \in A$ contradicting then fact that $u$ is a
    upper-bound of $A$. Hence this case is impossible.
    
    
    
    \item If $n \in \mathbbm{N}_{\mathbbm{Z}} \backslash \{ 0 \}$ then $u -
    \frac{1}{n}$ can not be a upper bound of $A$ as $u - \frac{1}{n} < u$
    (because $0 < \frac{1}{n}$) and $u$ is the least upper bound. So there
    exists a $r \in A$ such that $u - \frac{1}{n} < r$. Now as $0 < n$ we have
    by \ref{0<less>m=<gtr>1<less>=m if m is a integer} $1 \leqslant n
    \Rightarrow 0 \leqslant n - 1$ and thus $1 - \frac{1}{n} = \frac{n - 1}{n}
    \geqslant 0 \Rightarrow \frac{1}{n} \leqslant 1 < u \Rightarrow 0 < u -
    \frac{1}{n}$ and thus from $u - \frac{1}{n} < r$ we have $\left( u -
    \frac{1}{n} \right) \cdot \left( u - \frac{1}{n} \right) < \left( u -
    \frac{1}{n} \right) \cdot r$ and $\left( u \um \frac{1}{n} \right) \cdot r
    < r \cdot r \Rightarrow \left( u - \frac{1}{n} \right) \cdot \left( u -
    \frac{1}{n} \right) < r \cdot r < \frac{2}{1} \Rightarrow \left( u -
    \frac{1}{n} \right) \cdot \left( u - \frac{1}{n} \right) < \frac{2}{1}
    \Rightarrow u \cdot u - \frac{2}{1} \cdot u \cdot \frac{1}{n} +
    \frac{1}{n} \cdot \frac{1}{n} < \frac{2}{1}$ and thus we have $u \cdot u -
    \frac{2}{1} \cdot u \cdot \frac{1}{n} < u \cdot u - \frac{2}{1} \cdot u
    \cdot \frac{1}{n} + \frac{1}{n} \cdot \frac{1}{n} < \frac{2}{1}$. Now $0 <
    \frac{2}{1} \cdot u$ and $0 < u \cdot u - \frac{2}{1}$ so using the
    Archimedean property there exists a $n_0 \in \mathbbm{N}_{\mathbbm{Z}}$
    such that $\frac{2}{1} \cdot u < n_0 \cdot \left( u \cdot u - \frac{2}{1}
    \right)$ and as $n_0 \neq 0$ [otherwise $2 \cdot u < 0$] we have $n \in
    \mathbbm{N}_{\mathbbm{Z}} \backslash \{ 0 \}$ but also by multiplying the
    last expression by $\frac{1}{n_0}$ that $\frac{2}{n_0} \cdot u < u \cdot u
    - \frac{2}{1} \Rightarrow \frac{2}{1} < u \cdot u - \frac{2}{n_0} \cdot u
    = u \cdot u - \frac{2}{1} \cdot u \cdot \frac{1}{n_0} < 2 \Rightarrow 2 <
    2$ a contradiction. So this case is also impossible. 
  \end{enumerate}
  As (1),(2) and (3) can never occur we have finally reach a contradiction. So
  $\sup (A)$ does not exists.
\end{proof}

\section{Denumerability of the rationals}

\begin{lemma}
  \label{natural and integer numbers embedded in the rationals are
  denumerable}The sets $\mathbbm{N}_{\mathbbm{Q}}$ and
  $\mathbbm{Z}_{\mathbbm{Q}}$ are denumerable 
\end{lemma}

\begin{proof}[injective][surjective]
  The proof that $\mathbbm{Z}_{\mathbbm{Q}}$ is denumerable is simple, using
  \ref{embedding of the whole numbers in the rationals} we have that
  $\mathbbm{Z}_{\mathbbm{Q}} \approx \mathbbm{Z}$ and as we know by \ref{the
  integer numbers are denumerable} that $Z \approx \mathbbm{N}$ we have that
  $\mathbbm{Z}_{\mathbbm{Q}} \approx \mathbbm{N}$, so we have that
  $\mathbbm{Z}_{\mathbbm{Q}}$ is enumerable. To prove that
  $\mathbbm{N}_{\mathbbm{Q}}$ is denumerable first note that by definition
  (see \ref{naturals embedded in the rationals}) we have
  $\mathbbm{N}_{\mathbbm{Q}} = \left\{ \frac{a}{1} |a \in
  \mathbbm{N}_{\mathbbm{Z}} \right\}$. Define now $f :
  \mathbbm{N}_{\mathbbm{Z}} \rightarrow \mathbbm{N}_{\mathbbm{Q}}$, we prove
  then that $f$ is bijective:
  \begin{enumerate}
    \item If $f (a) = f (a') \Rightarrow \frac{a}{1} = \frac{a'}{1}
    \Rightarrow a \cdot 1 = a' \cdot 1 \Rightarrow a = a'$
    
    \item If $y \in \mathbbm{N}_{\mathbbm{Q}} \Rightarrow \exists a \in
    \mathbbm{N}_{\mathbbm{Z}}$ such that $y = \frac{a}{1} = f (a)$
  \end{enumerate}
  So we have that $\mathbbm{N}_{\mathbbm{Z}} \approx
  \mathbbm{N}_{\mathbbm{Q}}$, as by \ref{naturals embedded in the integers are
  denumerable} we have $\mathbbm{N}_{\mathbbm{Z}} \approx \mathbbm{N}$ we must
  have $\mathbbm{N}_{\mathbbm{Q}} \approx \mathbbm{N}$ and thus that
  $\mathbbm{N}_{\mathbbm{Q}}$ is denumerable.
\end{proof}

\begin{theorem}
  \label{The rational numbers are denumerable}The set $\mathbbm{Q}$ is
  denumerable
\end{theorem}

\begin{proof}
  Define the mapping $f : \mathbbm{Z} \times \mathbbm{Z} \rightarrow
  \mathbbm{Q}$ by
  \begin{eqnarray*}
    f (x, y) & = & \left\{ \begin{array}{l}
      \frac{x}{y} \tmop{if} (x, y) \in \mathbbm{Z} \times \mathbbm{Z}_0
      =\mathbbm{Z} \times (\mathbbm{Z} \backslash \{ 0 \})\\
      0 \tmop{if} (x, y) \in \mathbbm{Z} \times \{ 0 \}
    \end{array} \right.
  \end{eqnarray*}
  then we prove that $f$ is a surjection. If $y \in \mathbbm{Q}$ then there
  exists a $a \in \mathbbm{Z}, b \in \mathbbm{Z}_0$ such that $y = \frac{a}{b}
  = f (a, b) \Rightarrow y = f (a, b)$. Now by \ref{the integer numbers are
  denumerable} we have that $\mathbbm{Z}$ is denumerable and thus by
  \ref{product of enumerable sets is enumerable} we have that $\mathbbm{Z}
  \times \mathbbm{Z}$ is denumerable so there exists a bijection $g :
  \mathbbm{N} \rightarrow \mathbbm{Z} \times \mathbbm{Z}$ and thus a
  surjection $f \circ g : \mathbbm{N} \rightarrow \mathbbm{Q}$. Using
  \ref{conditions for countability} we have then that $\mathbbm{Q}$ is
  countable and thus finite or denumerable. As $\mathbbm{Q}$ contains the
  denumerable and thus infinite subset we have by \ref{subsets of finite sets
  are finite} that $\mathbbm{Q}$ is infinite and thus denumerable.
\end{proof}

\

\

\chapter{The real numbers}

\section{Definition}

\

\

\begin{definition}[Dedekind's Cut]
  \label{dedekind's cut}{\index{Dedekind's Cut}}A subset $\alpha \subseteq
  \mathbbm{Q}$ is a Dedekind's cut if the following properties are true
  \begin{enumerate}
    \item $\alpha \neq \emptyset$
    
    \item $\alpha \neq \mathbbm{Q}$
    
    \item $\forall r \in \alpha \wedge \forall s \in \mathbbm{Q} \backslash
    \alpha$ we have $r < s$
    
    \item $\alpha$ does not have a greatest element
  \end{enumerate}
\end{definition}

\begin{definition}[$\mathbbm{R}$]
  \label{the real numbers}{\index{the real numbers}}{\index{$\mathbbm{R}$}}We
  define the set of real numbers to be the set of Dedekind's cuts.
  $\mathbbm{R}= \{ \alpha \subseteq \mathbbm{Q} | \alpha \tmop{is} a
  \tmop{Dedekind}' s \tmop{cut} \nobracket \}$
\end{definition}

\begin{lemma}
  \label{property to determine membership of a cut}$\forall \alpha \in
  \mathbbm{R}$, $\forall r \in \alpha$ and $\forall s \in \mathbbm{Q} \vdash s
  \leqslant r$ we have $s \in \alpha$
\end{lemma}

\begin{proof}
  We prove this by contradiction, so assume that there exists a $\alpha \in
  \mathbbm{R}$ and a $s \in \mathbbm{Q}$ with $s \leqslant r$ and $s \nin
  \alpha \Rightarrow s \in \mathbbm{Q} \backslash \alpha
  \Rightarrowlim_{\text{\ref{dedekind's cut} , 3}} r < s \leqslant r
  \Rightarrow r < r$ a contradiction. 
\end{proof}

\begin{theorem}[Rational cuts]
  \label{rational cuts}{\index{rational cuts}}If $r \in \mathbbm{Q}$ then
  $\alpha_r = \{ x \in \mathbbm{Q} | x < r \nobracket \}$ is a cut, called a
  rational cut. Furthermore we have
  \begin{enumerate}
    \item $\alpha_r = \alpha_s$ iff $r = s$
    
    \item $\alpha$ is a rational cut iff $r = \min (\mathbbm{Q} \backslash
    \alpha)$ exists and in that case $\alpha = \alpha_r$
  \end{enumerate}
\end{theorem}

\begin{proof}
  
  
  First we prove that $\alpha_r$ is a cut.
  \begin{enumerate}
    \item If $r \in \mathbbm{Q}$ then there exists $a, b \in \mathbbm{Z}$ with
    $b \neq 0$ such that $r = \frac{a}{b}$, now if $b < 0$ then we can always
    take $a' = - a, b' = - a$ such that $\frac{a'}{b'} = \frac{- a}{- b} =
    \frac{(- 1) \cdot a}{(- 1) \cdot b} = \frac{a}{b} = q$. So we can always
    assume that $b > 0$. Taken now $q = \frac{a \upl (- 1)}{b}$ then $r \um q
    = \frac{a}{b} \upl \frac{(a \upl (- 1))}{b} = \frac{\um 1}{b}$ and as $0 <
    b$ we have $- b < 0 \Rightarrow (- 1) \cdot b < 0 \Rightarrow q < r$ ($q =
    r$ is impossible as $b \cdot (a \upl (\um 1)) \cdot b = b \cdot a \upl
    (\um b) \neq b \cdot a$). This proves that $\alpha_r \neq \emptyset$.
    
    \item As we don't have $r < r$ we have that $r \nin \alpha_r \Rightarrow
    \alpha_r \neq \mathbbm{Q}$
    
    \item If $w \in \alpha_r$ then $w < r$ and if $u \in \mathbbm{Q}
    \backslash \alpha_r$ then $\neg (u < r) \Rightarrow r \leqslant u$ and
    thus $w < u$
    
    \item If $m = \max (\alpha_r)$ is the greatest element of $\alpha_r$ then
    as $m \in \alpha_r$ we have $m < r$ then by the dense theorem of rational
    numbers (see \ref{rational numbers are dense}) there exists a $q \in
    \mathbbm{Q}$ such that $m < q < r \Rightarrow q \in \alpha_r$
    contradiction the definition of a greatest element.
  \end{enumerate}
  Next we prove that $\alpha_r = \alpha_s \Leftrightarrow r = s$
  \begin{enumerate}
    \item $\Rightarrow$ If $r \neq s$ then we have either
    \begin{enumerate}
      \item $r < s \Rightarrow r \in \alpha_s \backslash \alpha_r \Rightarrow
      \alpha_r \neq \alpha_s$ a contradiction
      
      \item $s < r \Rightarrow s \in \alpha_r \backslash \alpha_s \Rightarrow
      \alpha_s \neq \alpha_r$ a contradiction
    \end{enumerate}
    So we conclude that $r = s$
    
    \item $\Leftarrow$ If $r = s$ then clearly $\alpha_r = \alpha_s$
  \end{enumerate}
  Finally we prove the equivalence of '$\alpha$ is a rational cut iff the
  least element $r$ of $\mathbbm{Q} \backslash \alpha$ exists and in that case
  $\alpha = \alpha_r$'.
  \begin{enumerate}
    \item If $\alpha$ is a rational cut then $\exists r \in \mathbbm{Q} \vdash
    \alpha = \alpha_r$. Then as $r \in \alpha_r = \alpha$ we have that $r \in
    \mathbbm{Q} \backslash \alpha$. If $s \in \mathbbm{Q} \backslash \alpha$
    then $s \nin \alpha = \alpha_r \Rightarrow \neg (s < r)$ which by the fact
    that $\langle \mathbbm{Q}, \leqslant \rangle$ is fully-ordered means that
    $r \leqslant s$ and thus that $r$ is the least element of $\mathbbm{Q}
    \backslash \alpha$ or $r = \min (\mathbbm{Q} \backslash \alpha)$.
    
    \item Assume that $\alpha$ is a cut and that $r = \min (\mathbbm{Q}
    \backslash \alpha)$ exists then we claim that $\alpha = \alpha_r$.
    \begin{enumerate}
      \item If $s \in \alpha$ then as $r \in \mathbbm{Q} \backslash \alpha$ we
      have by the definition of a cut (see \ref{dedekind's cut}-3) that $s < r
      \Rightarrow s \in \alpha_r$. So we have $\alpha \subseteq \alpha_r$
      
      \item If $s \in \alpha_r \subseteq \mathbbm{Q}$ then we have $s < r$ and
      thus $s \nin \mathbbm{Q} \backslash \alpha$ (otherwise $r$ would not be
      the least element). From this it follows that $s \in \alpha \Rightarrow
      \alpha_r \subseteq \alpha$.
    \end{enumerate}
    (a) and (b) gives finally $\alpha = \alpha_r$
  \end{enumerate}
\end{proof}

A consequence of the above and the fact that $\mathbbm{Q}$ is not empty proves
the following corollary.

\begin{corollary}
  $\mathbbm{R} \neq \emptyset$ The set of real numbers is not empty.
\end{corollary}

\begin{definition}
  \label{set of rational cuts}{\index{$\mathbbm{Q}_{\mathbbm{R}}$}}We define
  the set $\mathbbm{Q}_{\mathbbm{R}}$ as follows $\mathbbm{Q}_{\mathbbm{R}} =
  \{ \alpha_r | r \in \mathbbm{Q} \nobracket \}$, so $\mathbbm{Q}_r$ is the
  set of all the rational cuts.
\end{definition}

\begin{theorem}
  \label{gap theorem}If $\alpha \in \mathbbm{R}$ then $\forall \varepsilon \in
  \mathbbm{Q} \backslash \{ 0 \}$ there $\exists r \in \alpha$ and $\exists s
  \nin \alpha$ such that $s - r < \varepsilon$
\end{theorem}

\begin{proof}
  Take $\alpha \in \mathbbm{R}$ and $\varepsilon \in \mathbbm{Q}$ by the
  definition of a cut (see \ref{dedekind's cut} 1,2) there exists a $r' \in
  \alpha$ and a $s' \nin \alpha$. Using the definition again (see
  \ref{dedekind's cut} 3) we have $r' < s'$. Using the Archimedean property of
  $\mathbbm{Q}$ (see \ref{archimedean property of the rationals}) there exists
  a $k \in \mathbbm{N}_{\mathbbm{Q}}$ such that $s' \um r' < k \cdot
  \varepsilon$. Now as $r' < s' \Rightarrow 0 < s' - r'$ we can't have that $k
  = 0 \Rightarrow 0 < k$ so $n^{\um 1}$ exists, also if $k^{\um 1} \leqslant 0
  \Rightarrowlim_{0 < k} k \cdot k^{\um 1} \leqslant 0 \Rightarrow 1 \leqslant
  0 < 1$ a contradiction, so we must conclude that $0 < k^{- 1}$. From $s' -
  r' < k \cdot \varepsilon$ it follows then that $k^{- 1} \cdot (s' \um r') <
  k^{- 1} \cdot k \cdot \varepsilon \Rightarrow k^{- 1} \cdot (s' \um r') <
  \varepsilon$. Define now $A = \{ n \in \mathbbm{N}_{\mathbbm{Q}} | r' \upl
  (n \cdot k^{- 1}) \cdot (s' - r') \nin \alpha \nobracket \} \subseteq
  \mathbbm{N}_{\mathbbm{Q}}$. As we have $r' + (k \cdot k^{- 1}) \cdot (s' -
  r') = r' \upl (s' \um r') = s' \nin \alpha$ we have that $k \in A
  \Rightarrow A$ is non empty. Using the fact that $\mathbbm{N}_{\mathbbm{Q}}$
  is well-ordered (see \ref{natural numbers in rational numbers are
  well-ordered}) there exists a $k' = \min (A)$. Now if $k' = 0$ then we would
  have that $r' \upl 0 \cdot (s' \um r') = r' \nin \alpha$ contradicting that
  $r' \in \alpha$ so we can't have $k' = 0$, using \ref{N_Q forms positive
  integers} we must then have $k' \geqslant 1$. Using \ref{N_Q is semi group}
  we have then that $k' \um 1 \in \mathbbm{N}_{\mathbbm{Q}}$. By the
  definition of a minimum and $k' \um 1 < k'$ (see \ref{q-1<less>q}) we have
  also $k' - 1 \in A \Rightarrowlim_{k' - 1 \in \mathbbm{N}_{\mathbbm{Q}}} r'
  \upl (k' \um 1) \cdot k^{- 1} \cdot (s' - r') \in \alpha$. If we now define
  \begin{eqnarray*}
    r & = & r' \upl ((k' \um 1) \cdot k^{- 1}) \cdot (s' - r')\\
    s & = & r' + (k' \cdot k^{- 1}) \cdot (s' - r')
  \end{eqnarray*}
  then we have that $r \in a$ and $s \nin \alpha$ (as $k' \in A$). Now $s - r
  = r' + (k' \cdot k^{- 1}) \cdot (s' - r') - (r' \upl ((k' \um 1) \cdot k^{-
  1}) \cdot (s' - r')) = (s' - r') \cdot k^{- 1} < \varepsilon$.
\end{proof}

\begin{theorem}[Negative cut]
  \label{negative cut}{\index{negative cut}}If $\alpha \in \mathbbm{R}$ then
  $- \alpha = \{ - s | s \nin \alpha \wedge s \neq \min (\mathbbm{Q}
  \backslash \alpha) \nobracket \}$ is a Dedekind's cut called the negative
  cut of $\alpha$. Note that $s \neq \min (\mathbbm{Q} \backslash \alpha)$
  does not mean that $\min (\mathbbm{Q} \backslash \alpha)$ must exists, but
  that $\alpha$ may not be the minimum element of $\mathbbm{Q} \backslash
  \alpha$ (in other words if the minimum exists it may not be $\alpha$).
\end{theorem}

\begin{proof}
  
  \begin{enumerate}
    \item Since $\alpha \neq \mathbbm{Q}$ there exists a $\exists s \in
    \mathbbm{Q} \vdash s \nin \alpha$ then $s \upl 1 \nin \alpha$ [if $s \upl
    1 \in \alpha$ then as $s < s \upl 1$ (see \ref{q-1<less>q}) and $s \in
    \mathbbm{Q} \backslash \alpha$ we have by \ref{dedekind's cut} that $s
    \upl 1 < s < s \upl 1 \Rightarrow s + 1 < s + 1$ a contradiction]. As $s,
    s + 1 \in \mathbbm{Q} \backslash \alpha, s < s + 1$ we can't have $s \upl
    1 = \min (\mathbbm{Q} \backslash \alpha) \Rightarrow s + 1 \in - \alpha
    \Rightarrow \um \alpha \neq \emptyset$
    
    \item Since $\alpha \neq \emptyset$ there exists a $r \in \alpha \subseteq
    \mathbbm{Q}$ then $- r \nin \alpha \Rightarrow \alpha \neq \mathbbm{Q}$
    
    \item Let now $r \in - \alpha$ and $s \in \mathbbm{Q} \backslash (\um a)$
    suppose now that $s \leqslant r
    \Rightarrowlim_{\text{\ref{q<less>=r=<gtr>-r<less>=-q for rational
    numbers}}} \um r \leqslant \um s$. From $r \in \um \alpha$ we have $\um r
    \nin \alpha$ as $s \in \mathbbm{Q} \backslash (- \alpha)$ we have $s \nin
    \um \alpha$ and thus $\neg (- s \in \alpha \wedge - s \neq \min
    (\mathbbm{Q} \backslash \alpha)) \Rightarrow - s \nin \alpha \vee - s =
    \min (\mathbbm{Q} \backslash \alpha)$ so we have the following cases to
    consider
    \begin{enumerate}
      \item $- s \in \alpha$. Now $- r \leqslant - s \Rightarrow \neg (- s < -
      r)$ \ together with $- s \in \alpha$ and $- r \in \mathbbm{Q} \backslash
      \alpha$ contradicts the definition of $\alpha$ as a cut (see
      \ref{dedekind's cut} , 3). So this cases is impossible.
      
      \item $- s = \min (\mathbbm{Q} \backslash \alpha)$ As $- s \leqslant -
      r$ we must have $- s = - r \Rightarrow s = r \Rightarrow s \in (\um
      \alpha) \bigcap (\mathbbm{Q} \backslash (\um \alpha)) = \emptyset$ a
      contradiction. So this case is also impossible. 
    \end{enumerate}
    As (a) and (b) are impossible we must conclude that $r < s$
    
    \item We prove by contradiction that $\max (\um \alpha)$ does not exists.
    So assume that $\max (\um \alpha)$ exists, then as $\max (- \alpha) \in
    \um \alpha$ we must have that $\um \max (\um \alpha) \nin \alpha
    \Rightarrow - \max (\um \alpha) \in \mathbbm{Q} \backslash \alpha$ and $-
    \max (\um \alpha) \neq \min (\mathbbm{Q} \backslash \alpha)$. We consider
    now the two only cases for $\min (\mathbbm{Q} \backslash \alpha)$
    \begin{enumerate}
      \item $\min (\mathbbm{Q} \backslash \alpha)$ does not exists. If now
      $\forall s \in \mathbbm{Q} \backslash \alpha$ we have $- \max (\um
      \alpha) \leqslant s$ we would have as $- \max (- \alpha) \in \mathbbm{Q}
      \backslash \alpha$ that $\um \max (\um \alpha) = \min (\mathbbm{Q}
      \backslash \alpha)$ contradicting the fact that we assumed that $\min
      (\mathbbm{Q} \backslash \alpha)$ does not exists. So we must have that
      $\exists s \in \mathbbm{Q} \backslash \alpha$ so that $s < \um \max (\um
      \alpha) \Rightarrow \max (- \alpha) < \um s$, as $s \in \mathbbm{Q}
      \backslash \alpha$ and $\min (\mathbbm{Q} \backslash \alpha)$ does not
      exists so $- s \neq \min (\mathbbm{Q} \backslash \alpha)$ we must have
      that $- s \in - \alpha \Rightarrow - s \leqslant \max (\um \alpha)$ but
      this contradicts $\max (\um \alpha) < \um s$. So we reach a final
      contradiction.
      
      \item $\min (\mathbbm{Q} \backslash \alpha)$ exists. As $- \max (\um
      \alpha) \in \mathbbm{Q} \backslash \alpha$ and $- \max (\um \alpha) \neq
      \min (\mathbbm{Q} \backslash \alpha)$ we have $\min (\mathbbm{Q}
      \backslash \alpha) < \max (\um \alpha)$. Using the density theorem (see
      \ref{rational numbers are dense}) there exists a $s \in \mathbbm{Q}$
      such that $\min (\mathbbm{Q} \backslash \alpha) < s < - \max (-
      \alpha)$. As $\min (\mathbbm{Q} \backslash \alpha) \in \mathbbm{Q}
      \backslash \alpha$ we would have by the fact that $\alpha$ is a cut (see
      \ref{dedekind's cut} 3) that we can not have $s \in \alpha$ [as this
      would mean $[\min (\mathbbm{Q} \backslash \alpha) \in \mathbbm{Q}
      \backslash \alpha]$ $s < \min (\mathbbm{Q} \backslash \alpha) < s$ a
      contradiction]. So $s \in \mathbbm{Q} \backslash \alpha$ and $\min
      (\mathbbm{Q} \backslash \alpha) \neq s \Rightarrow - s \in - \alpha$,
      from $s < - \max (\um \alpha)
      \Rightarrowlim_{\text{\ref{q<less>=r=<gtr>-r<less>=-q for rational
      numbers}}} - s \leqslant \max (\um \alpha) < \um s \Rightarrow - s < -
      s$ which is a contradiction. 
    \end{enumerate}
    As all the possible cases gives a contradiction we must conclude that
    $\max (- \alpha)$ does not exists. 
  \end{enumerate}
  (1),(2),(3) and (4) proves that indeed $- \alpha$ is a cut.
\end{proof}

For rational cuts there is a easy way to construct its negative cut as is
expressed in the following theorem.

\begin{theorem}
  \label{negative of rational cut}If $r \in \mathbbm{Q}$ then $- \alpha_r =
  \alpha_{\um r}$
\end{theorem}

\begin{proof}
  
  \begin{eqnarray*}
    x \in \um \alpha_r & \Leftrightarrow & - x \nin \alpha_r \wedge - x \neq
    \min (\mathbbm{Q} \backslash \alpha_r)\\
    & \Leftrightarrowlim_{\text{\ref{rational cuts}} \Rightarrow r = \min
    (\mathbbm{Q} \backslash \alpha_r)} & - x \nin \alpha_r \wedge - x \neq r\\
    & \Leftrightarrow & r \leqslant - x \wedge - x \neq r\\
    & \Leftrightarrow & - x > r\\
    & \Leftrightarrow & x < \um r\\
    & \Leftrightarrow & x \in \alpha_{\um r}
  \end{eqnarray*}
\end{proof}

\section{Arithmetic's on $\mathbbm{R}$}

\subsection{Addition in $\mathbbm{R}$}

\begin{definition}
  If $\alpha, \beta \in \mathbbm{R}$ then we define $\alpha \upl \beta = \{ r
  \upl s | r \in \alpha, s \in \beta \nobracket \}$
\end{definition}

\begin{lemma}
  \label{lemma for sum of reals}$\forall \alpha \in \mathbbm{R}$ and
  $\varepsilon \in \mathbbm{Q} \backslash \{ 0 \}$ there exists a $r \in
  \alpha$ such that $r \upl \varepsilon \nin \alpha$
\end{lemma}

\begin{proof}
  Using \ref{gap theorem} we have there exists a $r \in \alpha$ and a $s \nin
  \alpha$ such that $s \um r < \varepsilon \Rightarrow s < r \upl
  \varepsilon$. Assume now that $r \upl \varepsilon \in \alpha$ then as $s
  \nin \alpha \Rightarrow s \in \mathbbm{Q} \backslash \alpha$ we have by
  \ref{dedekind's cut}, 3 that $r \upl \varepsilon < s$ contradicting $s < r
  \upl \varepsilon$ so we must have that $r \upl \varepsilon \nin \alpha$
\end{proof}

\begin{theorem}
  $\forall \alpha, \beta \in \mathbbm{R}$ then $\alpha \upl \beta \in
  \mathbbm{R}$
\end{theorem}

\begin{proof}[$\alpha \upl \beta \neq \emptyset$][$\alpha \upl \beta \neq
\mathbbm{Q}$][$\forall w \in \alpha \upl \beta \wedge \forall v \in
\mathbbm{Q} \backslash (\alpha \upl \beta)$ we have $w < v$][$\alpha \upl
\beta \tmop{does} \tmop{not} \tmop{have} a \tmop{maximum}$]
  Given $\alpha, \beta \in \mathbbm{R}$ we must prove that $\alpha \upl \beta$
  is a Dedekind's cut.
  \begin{enumerate}
    \item Since $\alpha \neq \emptyset \wedge \beta \neq \emptyset \Rightarrow
    \exists r \in \alpha \wedge \exists s \in \beta \Rightarrow r \upl s \in
    \alpha \upl \beta \Rightarrow \alpha \upl \beta \neq \emptyset$
    
    \item Given $\varepsilon = \frac{1}{2} \in \mathbbm{Q}$ there exists by
    \ref{lemma for sum of reals} a $r' \in \alpha$ and a $s' \in \beta$ such
    that $r' \upl \frac{1}{2} \nin \alpha$ and $s' \upl \frac{1}{2} \in
    \beta$. Now $\forall r \in \alpha$ and $\forall s \in \beta$ we have by
    \ref{dedekind's cut} , 3 that $r < r' \upl \frac{1}{2}$ and $s < s' \upl
    \frac{1}{2} \Rightarrow r \upl s < r' \upl s' \upl \frac{1}{1 + 1} +
    \frac{1}{1 + 1} = r' \upl s' \upl 1 \Rightarrow r \upl s < r' \upl s' \upl
    1$. Now if $r' \upl s' + 1 \in \alpha \upl \beta$ then there exists a $r
    \in \alpha$ and a $s \in \beta$ such that $r' \upl s' \upl 1 = r + s$ and
    by the above we would have the contradiction $r' \upl s' \upl 1 = r \upl s
    < r' \upl s' \upl 1$ so we conclude that $r' \upl s' \upl 1 \nin \alpha
    \upl \beta \Rightarrow \alpha + \beta \neq \mathbbm{Q}$
    
    \item Let $w \in \alpha \upl \beta$ then there exists a $r \in \alpha$ and
    a $s \in \beta$ \ such that $w = r \upl s$. Assume now that there exists a
    $v \in \mathbbm{Q} \backslash (\alpha \upl \beta)$ such that $v \leqslant
    w$. We have then $v \leqslant r \upl s \Rightarrow v \um s \leqslant r$,
    using \ref{property to determine membership of a cut} we have then $v \um
    s \in \alpha$ and thus $v = (v \um s) + s \in \alpha \upl \beta$
    contradicting $v \in \mathbbm{Q} \backslash (\alpha + \beta)$ and thus our
    assumption. So we must have $\forall v \in \mathbbm{Q} \backslash (\alpha
    \upl \beta)$ that $w < v$.
    
    \item Suppose that $m = \max (\alpha \upl \beta)$ then as $m \in \alpha
    \upl \beta$ there exists a $r \in \alpha$ and a $\beta \in \beta$ such
    that $m = r \upl s$. As $\max (\alpha)$ and $\max (\beta)$ does not exists
    there exists a $r' \in \alpha$ such that $r < r' \Rightarrow r \upl s < r'
    \upl s$ which as $r' \upl s \in \alpha \upl \beta$ would give $r' \upl s
    \leqslant m = r \upl s < r' \upl s \Rightarrow r' \upl s < r' \upl s$ a
    contradiction. The only conclusion left is that $\alpha + \beta$ does not
    have a maximum
  \end{enumerate}
  .
\end{proof}

\begin{theorem}
  \label{real numbers form a additive group}$\langle \mathbbm{R}, + \rangle$
  forms a abelian group, the identity element is $\alpha_0$ which we will note
  as usually as $0$ (so $0 = \alpha_0$ where the $0$ in the lower index is the
  identity in $\langle \mathbbm{Q}, + \rangle$). If $\alpha \in \mathbbm{R}$
  then its inverse is $\um \alpha$.
\end{theorem}

\begin{proof}[associative][commutative][neutral element][inverse][$\min
(\mathbbm{Q} \backslash \alpha) \tmop{does} \tmop{not} \tmop{exists}$][$\min
(\mathbbm{Q} \backslash \alpha) \tmop{exists}$]
  We rely here heavily on the fact that $\langle \mathbbm{Q}, + \rangle$ forms
  a abelian group (see \ref{the set of rational numbers forms a abelian group
  for addition})
  \begin{enumerate}
    \item If $\alpha, \beta, \gamma \in \mathbbm{R}$ then we have
    \begin{eqnarray*}
      z \in (\alpha \upl \beta) \upl \gamma & \Leftrightarrow & z = r \upl s
      \wedge r \in (\alpha \upl \beta) \wedge s \in \gamma\\
      & \Leftrightarrow & z = (u \upl t) \upl s \wedge u \in \alpha \wedge t
      \in \beta \wedge s \in \gamma\\
      & \Leftrightarrowlim_{\text{\ref{the set of rational numbers forms a
      abelian group for addition}}} & z = u \upl (t \upl s) \wedge u \in
      \alpha \wedge t \in \beta \wedge s \in \gamma\\
      & \Leftrightarrow & z = u \upl v \wedge u \in \alpha \wedge v \in \beta
      \upl \gamma\\
      & \Leftrightarrow & z \in \alpha \upl (\beta + \gamma)
    \end{eqnarray*}
    So we have $(\alpha \upl \beta) \upl \gamma = \alpha \upl (\beta \upl
    \gamma)$
    
    \item If $\alpha, \beta \in \mathbbm{R}$ then we have
    \begin{eqnarray*}
      z \in \alpha \upl \beta & \Leftrightarrow & z = r \upl s \wedge r \in
      \alpha \wedge s \in \beta\\
      & \Leftrightarrowlim_{\text{\ref{the set of rational numbers forms a
      abelian group for addition}}} & z = s \upl r \wedge r \in \alpha \wedge
      s \in \beta\\
      & \Leftrightarrow & z \in \beta \upl \alpha
    \end{eqnarray*}
    giving $\alpha \upl \beta = \beta \upl \alpha$
    
    \item Let $\alpha \in \mathbbm{R}$. If $z \in \alpha + \alpha_0$ then $z =
    r \upl s$ where $r \in \alpha$ and $s < 0 \Rightarrow r \upl s < r
    \Rightarrow r \upl s \leqslant r \Rightarrowlim_{\text{\ref{property to
    determine membership of a cut}}} (r \upl s) \in \alpha \Rightarrow z \in
    \alpha \Rightarrow (\alpha \upl \alpha_0) \subseteq \alpha$. If $z \in
    \alpha$ then as $\alpha$ has no maximum (\ref{dedekind's cut} 4), there
    exists a $z' \in \alpha$ with $z < z'$, then we have $z \um z' < 0
    \Rightarrow z \um z' \in \alpha_0 \Rightarrow z = z' \upl (z - z') \in
    \alpha \upl \alpha_0 \Rightarrow \alpha \subseteq (\alpha \upl \alpha_0)$.
    This gives then $\alpha = \alpha \upl \alpha_0$ and by commutativity we
    have also $\alpha = \alpha_0 \upl \alpha$.
    
    \item Let $\alpha \in \mathbbm{R}$ take then $\um \alpha$ and consider the
    following possible cases for $\min (\mathbbm{Q} \backslash \alpha)$
    \begin{enumerate}
      \item If $x \in \alpha_0$ then $x < 0$ and thus $0 < \um x$ using
      \ref{lemma for sum of reals} there exists a $r \in \alpha$ such that $r
      \um x = r \upl (- x) \nin \alpha$, as $\min (\mathbbm{Q} \backslash
      \alpha)$ does not exists, we have $r \um x \neq \min (\mathbbm{Q}
      \backslash \alpha)$. Thus by definition of $\um \alpha$ we have $- (r
      \um x) \in - \alpha \Rightarrow x \um r \in - \alpha \Rightarrow x = r
      \upl (x - r) \in \alpha \upl (\um \alpha) \Rightarrow \alpha_0 \subseteq
      \alpha \upl (\um \alpha)$. Also if $x \in \alpha \upl (\um \alpha)$ then
      $x = r \upl s$ where $r \in \alpha$ and $s \in \um \alpha \Rightarrow -
      s \nin \alpha$, using \ref{dedekind's cut} 3 we have then that $r < - s
      \Rightarrow r + s < 0 \Rightarrow x = r + s \in \alpha_0 \Rightarrow
      \alpha \upl (- \alpha) \subseteq \alpha_0$. Combining our two results
      gives $\alpha_0 = \alpha \upl (\um \alpha)$.
      
      \item If $r = \min (\mathbbm{Q} \backslash \alpha)$ exist then by
      \ref{rational cuts} we have that $\alpha = \alpha_r$, by \ref{negative
      of rational cut} we have $\um \alpha = \um \alpha_r = \alpha_{\um r}$.
      So we have $z \in \alpha \upl (\um \alpha) = \alpha_r \upl \alpha_{\um
      r}$ then there exists a $s < r$ and a $t < - r$ such that $z = s \upl
      t$, this gives $s \upl t < r \upl t$ and $t \upl r < r \upl (- r) = 0
      \Rightarrow s \upl t < 0 \Rightarrow z < 0 \Rightarrow z \in \alpha_0
      \Rightarrow \alpha \upl (\um \alpha) \subseteq \alpha_0$. If $z \in
      \alpha_0 \Rightarrow z < 0 \Rightarrow 0 < \um z$ and as $0 <
      \frac{1}{2}$ we have by \ref{n<less>m,0<less>s=<gtr>n.s<less>m.s
      rationals} that $0 < (\um z) \cdot \frac{1}{2} = - \left( z \cdot
      \frac{1}{2} \right) \Rightarrow \left( z \cdot \frac{1}{2} \right) < 0$.
      From this it follows that $r \upl \left( z \cdot \frac{1}{2} \right) < r
      \Rightarrow \left( r \upl \left( z \cdot \frac{1}{2} \right) \right) \in
      \alpha_r$ and $- r \upl \left( z \cdot \frac{1}{2} \right) < \um r
      \Rightarrow \left( - r + \left( z \cdot \frac{1}{2} \right) \right) \in
      \alpha_{- r}$. Finally $z = \left( \frac{1}{2} \cdot z \right) \upl
      \left( \frac{1}{2} \cdot z \right) = \left( r + \left( z \cdot
      \frac{1}{2} \right) \right) + \left( - r \upl \left( z \cdot \frac{1}{2}
      \right) \right) \in \alpha_r \upl \alpha_{- r} = \alpha + (- \alpha)
      \Rightarrow \alpha_0 \subseteq \alpha \upl (\um \alpha)$. Combining our
      two results gives $\alpha_0 = \alpha \upl (\um \alpha)$
    \end{enumerate}
    Using commutativity we have the also $\alpha_0 = (\um \alpha) \upl \alpha$
  \end{enumerate}
\end{proof}

Next we define the set of positive and negative numbers

\subsection{Multiplication in $\mathbbm{R}$}

\begin{definition}
  {\index{$\mathbbm{R}_+$}}{\index{$\mathbbm{R}_-$}}The set of positive real
  numbers $\mathbbm{R}_+$ is defined by $\mathbbm{R}_+ = \{ \alpha \in
  \mathbbm{R} | 0 < \alpha \nobracket \} \subseteq \mathbbm{R}$, the set of
  negative real numbers $\mathbbm{R}_-$ is defined by $\mathbbm{R}_- = \{
  \alpha | - \alpha \in \mathbbm{R}_+ \nobracket \}$
\end{definition}

\begin{theorem}
  \label{disjoint union of reals}$\mathbbm{R}=\mathbbm{R}_+ \bigcup
  \mathbbm{R}_- \bigcup \{ 0 \}$ and $\mathbbm{R}_+ \bigcap \{ 0 \} =
  \emptyset =\mathbbm{R}_- \bigcap \{ 0 \}$ and $\mathbbm{R}_+ \bigcap
  \mathbbm{R}_- = 0$. In other words if $\alpha \in \mathbbm{R}$ then we have
  either only one of $\alpha \in \mathbbm{R}_+$ or $\alpha \in \mathbbm{R}_-$
  or $\alpha = 0$.
\end{theorem}

\begin{proof}[$0 \in \alpha$][$0 \nin \alpha$][$0 = \min (\mathbbm{Q}
\backslash \alpha)$][$0 \neq \min (\mathbbm{Q} \backslash \alpha)$]
  First we prove that $\mathbbm{R}=\mathbbm{R}_+ \bigcup \mathbbm{R}_- \bigcup
  \{ 0 \}$, as $\mathbbm{R}_+, \mathbbm{R}_-, \{ 0 \} \subseteq \mathbbm{R}$
  we have $\mathbbm{R}_+ \bigcup \mathbbm{R}_- \bigcup \{ 0 \} \subseteq
  \mathbbm{R}$. Now if $\alpha \in \mathbbm{R}$ then we have the following
  possible cases
  \begin{enumerate}
    \item $\Leftrightarrow \alpha \in \mathbbm{R}_+$
    
    \item here we have again two possible cases
    \begin{enumerate}
      \item $\Leftrightarrowlim_{\text{\ref{rational cuts}}} \alpha = \alpha_0
      = 0 \Leftrightarrow \alpha \in \{ 0 \}$
      
      \item $\Leftrightarrowlim_{0 \in \alpha \tmop{and} \tmop{definition}
      \tmop{of} \um \alpha \left( \tmop{see} \text{\ref{negative cut}}
      \right)} 0 = - 0 \in - \alpha \Leftrightarrow - \alpha \in \mathbbm{R}_+
      \Leftrightarrow \alpha \in \mathbbm{R}_-$
    \end{enumerate}
  \end{enumerate}
  from (1),(2.a),(2.b) we conclude that $\mathbbm{R} \subseteq \mathbbm{R}_+
  \bigcup \mathbbm{R}_- \bigcup \{ 0 \}$ and thus we have
  $\mathbbm{R}=\mathbbm{R}_+ \bigcup \mathbbm{R}_- \bigcup \{ 0 \}$.
  
  Now if $0 \in \mathbbm{R}_+$ then $0 = \alpha_0 \in \mathbbm{R}_+
  \Rightarrow 0 \in \alpha_0 \Rightarrow 0 < 0$ a contradiction, so we have
  $\{ 0 \} \bigcap \mathbbm{R}_+ = \emptyset$.
  
  If $0 \in \mathbbm{R}_-$ then $- 0 \in \mathbbm{R}_+ \Rightarrowlim
  \Rightarrowlim_{- 0 \equallim_{\tmop{neutral} \tmop{element}} - 0 + 0
  \equallim_{\tmop{inverse} \tmop{element}} 0} 0 \in \mathbbm{R}_+$ which we
  have already prove to be impossible.
  
  Finally if $\alpha \in \mathbbm{R}_+ \bigcap \mathbbm{R}_-$ then $\alpha
  \in \mathbbm{R}_+ \wedge \alpha \in \mathbbm{R}_- \Rightarrow \alpha \in
  \mathbbm{R}_+ \wedge - \alpha \in \mathbbm{R}_+ \Rightarrow 0 \in \alpha
  \wedge 0 \in - \alpha \Rightarrow 0 \in \alpha \wedge - 0 \in - \alpha$
  which is a contradiction by the definition of the negative (see
  \ref{negative cut}). 
\end{proof}

Let's now proceed to definition of multiplication in $\mathbbm{R}$, first we
define multiplication on the set of positive reals

\begin{definition}
  $\forall \alpha, \beta \in \mathbbm{R}_+$ then $\alpha \odot \beta = \{ r
  \in \mathbbm{Q} | r \leqslant 0 \nobracket \} \bigcup \{ s \cdot t | s \in
  \alpha \wedge t \in \beta, s > 0 \wedge t > 0 \nobracket \}$
\end{definition}

We have then the following theorem

\begin{theorem}
  $\forall \alpha, \beta \in \mathbbm{R}_+$ we have that $\alpha \odot \beta
  \in \mathbbm{R}_+$
\end{theorem}

\begin{proof}[$\alpha \odot \beta \neq \emptyset$][$\mathbbm{Q} \backslash
(\alpha \odot \beta) \neq \emptyset$][$r \in \{ s \in \mathbbm{Q} | s
\leqslant 0 \nobracket \}$][$r \in \{ s \in \mathbbm{Q} | s \leqslant 0 |
\}$][$m \in \{ s \in \mathbbm{Q} | s \leqslant 0 \nobracket \}$][$m \in \{ s
\in \mathbbm{Q} | s \leqslant 0 \nobracket \}$]
  Let's first prove that $\alpha \odot \beta \in \mathbbm{R}$
  \begin{enumerate}
    \item As $0 \in \{ r \in \mathbbm{Q} | r \leqslant 0 \nobracket \}
    \Rightarrow 0 \in \alpha \odot \beta \Rightarrow \alpha \odot \beta \neq
    \emptyset$
    
    \item As $\alpha, \beta \in \mathbbm{R}_+$ we have $0 \in \alpha$ and $0
    \in \beta$, from the fact that $\alpha, \beta$ do not have a maximum we
    must then have the existence of a $s_1 \in \alpha$ and a $t_1 \in \beta$
    such that $0 < s_1, 0 < t_1$. As $1 = \frac{1}{1} \in \mathbbm{Q}
    \backslash \{ 0 \}$ there exists by \ref{lemma for sum of reals} \ a $s_2
    \in \alpha$ and a $t_2 \in \beta$ such that $s_2 \upl 1 \nin \alpha$ and
    $t_2 \upl 1 \in \beta$. Take $s = \max (s_1, s_2), t = \max (t_1, t_2)$
    then $s \in \alpha, t \in \beta$ [if $s \nin \alpha
    \Rightarrowlim_{\text{\ref{dedekind's cut} 3}} s_1 < s, s_2 < s
    \Rightarrow s \nin \{ s_1, s_2 \}$ a contradiction, if $t \nin \beta
    \Rightarrowlim_{\text{\ref{dedekind's cut} 3}} t_1 < t, t_2 < t
    \Rightarrow t \nin \{ t_1, t_2 \}$ a contradiction] and $0 < s, 0 < t$ and
    $s_2 \upl 1 \leqslant s \upl 1, t_2 \upl 1 \leqslant t \upl 1$ [as $s_2
    \leqslant \max (s_1, s_2) = s, t_2 \leqslant \max (t_1, t_2) = t$]. We
    have then also $s \upl 1 \nin \alpha, t \upl 1 \nin \beta$ [if $s \upl 1
    \in \alpha$ then by \ref{dedekind's cut} 3 and $s_2 \upl 1 \nin \alpha$ we
    have the contradiction $s \upl 1 < s_2 \upl 1 \leqslant s + 1 \Rightarrow
    s \upl 1 < s \upl 1$, if $t \upl 1 \in \beta$ then by \ref{dedekind's cut}
    3 and $t_2 \upl 1 \nin \alpha$ we have the contradiction $t \upl 1 < t_2
    \upl 1 \leqslant t + 1 \Rightarrow t \upl 1 < t \upl 1$]. We claim now
    that $s \cdot t \upl s \upl t \upl 1 \nin \alpha \odot \beta$ by
    contradiction, so let $s \cdot t \upl s \upl t \in \alpha \odot \beta$.
    First $0 < s, 0 < t
    \Rightarrowlim_{\text{\ref{n<less>m,0<less>s=<gtr>n.s<less>m.s
    rationals}}} 0 < s \cdot t$ and thus $0 < s \cdot t \upl s \Rightarrow 0 <
    s \cdot t \upl s \upl t \Rightarrow 0 < s \cdot t \upl s \upl t \upl 1
    \Rightarrow s \cdot t \upl s \upl t \upl 1 \nin \{ r \in \mathbbm{Q} | r
    \leqslant 0 \nobracket \}$, from this it follows that then we must have $s
    \cdot t \upl s \upl t \upl 1 \in \{ s \cdot t | s \in \alpha \wedge t \in
    \beta \wedge 0 < s \wedge 0 < t \nobracket \}$ so there exists a $s' \in
    \alpha, 0 < s'$ and a $t' \in \beta, 0 < t'$ such that $s \cdot t \upl s
    \upl t \upl 1 = s' \cdot t'$. Using \ref{dedekind's cut} 3 and $s \upl 1
    \nin \alpha, t \upl 1 \nin \beta$ we have that $s' < s \upl 1, t' < t \upl
    1 \Rightarrowlim_{0 < s', 0 < t' < t \upl 1} s' \cdot t' < (s \upl 1)
    \cdot t', (s \upl 1) \cdot t = t' \cdot (s \upl 1) < (s \upl 1) \cdot (t
    \upl 1) \Rightarrow s' \cdot t' < (s \upl 1) \cdot (t \upl 1) = s \cdot t
    \upl s \upl t \upl 1$ and this gives then the contradiction $s \cdot t
    \upl s \upl t \upl 1 < s \cdot t \upl s \upl t \upl 1$. So we conclude
    that $s \cdot t \upl s \upl t \upl 1 \nin \alpha \odot \beta \Rightarrow s
    \cdot t \upl s \upl t \upl 1 \in \mathbbm{Q} \backslash (\alpha \odot
    \beta) \Rightarrow \mathbbm{Q} \backslash (\alpha \odot \beta) \neq
    \emptyset$
    
    \item If $r \in \alpha \odot \beta$ and $s \in \mathbbm{Q} \backslash
    (\alpha \odot \beta)$then we have either
    \begin{enumerate}
      \item $\Rightarrow r \leqslant 0$ now for $s \in \mathbbm{Q} \backslash
      (\alpha \odot \beta)$ we have $s \nin \{ t \in \mathbbm{Q} | t \leqslant
      0 \nobracket \} \Rightarrow 0 < s \Rightarrow r < s$
      
      \item $\Rightarrow r \in \{ s \cdot t | s \in \alpha \wedge t \in \beta
      \wedge 0 < s \wedge 0 < t \nobracket \}$ and $0 \leqslant r$ so $r = s'
      \cdot t'$ where $s' \in \alpha \wedge t' \in \beta \wedge s' > 0 \wedge
      t' > 0$ so we have by \ref{n<less>m,0<less>s=<gtr>n.s<less>m.s
      rationals} that $0 < s' \cdot t' = r$. Suppose now that $s \in
      \mathbbm{Q} \backslash (\alpha \odot \beta)$ and that $s \leqslant r$,
      we will derive a contradiction from this. As $s \nin \alpha \odot \beta$
      we have $s \neq r$ and $s \nin \{ s \in \mathbbm{Q} | s \leqslant 0
      \nobracket \} \Rightarrow 0 < s$ and thus $0 < s < r$ using \ref{the
      rational numbers form a field} we have that $s^{- 1}$ exists and by
      \ref{0<less>s=<gtr>0<less>s^-1} $0 < s^{- 1}$. Take now $t = s^{- 1}
      \cdot r \Rightarrow s \cdot t = r$ then we have $0 < t$, using
      \ref{n<less>m,0<less>s=<gtr>n.s<less>m.s rationals} and $s < r$ this
      gives $s^{- 1} s^{} < s^{- 1} \cdot r = t \Rightarrow 1 < t$. As $r = s
      \cdot t = s' \cdot t'$ and $0 < 1 < t$ we have $s = t^{- 1} \cdot (s'
      \cdot t')$. From $1 < t$ we have by \ref{some properties of rational
      numbers} that $t^{- 1} < 1$ and by \ref{some properties of rational
      numbers} and $0 < s'$ we get $t^{- 1} \cdot s' < s'$ and thus $t^{- 1}
      \cdot s' \in \alpha$ [if $t^{- 1} \cdot s' \in \alpha
      \Rightarrowlim_{\ref{dedekind's cut}, 3} s' \leqslant t^{- 1} \cdot s' <
      s' \Rightarrow s' < s'$ a contradiction]. From $t^{- 1} \cdot s' \in
      \alpha \tmop{and} t' \in \beta$ we have $s = (t^{- 1} \cdot s') \cdot t'
      \in \alpha \odot \beta$ contradiction $s \in \mathbbm{Q} \backslash
      (\alpha \odot \beta)$ so we must have $r < s$.
    \end{enumerate}
    \item We prove now by contradiction that $\max (\alpha \odot \beta)$ does
    not exists. So assume that $m = \max (\alpha \odot \beta)$ exists then as
    $m \in \alpha \odot \beta$ we have to consider the following two cases
    \begin{enumerate}
      \item $\Rightarrow m \leqslant 0$, now $\alpha, \beta \in \mathbbm{R}_+
      \Rightarrow 0 \in \alpha, 0 \in \beta$ and as $\alpha, \beta$ don't have
      a maximum there exists a $s' \in \alpha, 0 < s'$ and a $t' \in \beta, 0
      < \beta \Rightarrow 0 < s' \cdot t' \in \alpha \odot \beta$ and thus $m
      < s' \cdot t'$ contradicting the fact that $m = \max (\alpha \odot
      \beta)$, so we reach a contradiction.
      
      \item $\Rightarrow 0 < m$ and $m \in \{ s \cdot t | s \in \alpha \wedge
      t \in \beta \wedge 0 < s \wedge 0 < t \nobracket \} \Rightarrow \exists
      s \in \alpha, 0 < s, \exists t \in \beta, 0 < t$ with $m = s \cdot t$.
      As $\max (\alpha)$ and $\max (\beta)$ does not exists we have that
      $\exists s' \in \alpha \vdash 0 < s < s', \exists t' \in \beta \vdash 0
      < t < t' \Rightarrow m = s \cdot t < s' \cdot t' \in \alpha \odot \beta$
      contradicting the maximality of $m$.
    \end{enumerate}
  \end{enumerate}
\end{proof}

Using the above definition we can define multiplication on $\mathbbm{R}$,
first we prove the following lemma.

\begin{lemma}
  $\mathbbm{R} \times \mathbbm{R}$ is the disjoint union of $\mathbbm{R}_+
  \times \mathbbm{R}_+, \mathbbm{R}_- \times \mathbbm{R}_-, \mathbbm{R}_+
  \times \mathbbm{R}_-, \mathbbm{R}_- \times \mathbbm{R}_+, (\{ 0 \} \times
  \mathbbm{R}) \bigcup (\mathbbm{R} \times \{ 0 \})$
\end{lemma}

\begin{proof}
  First as $\mathbbm{R}_+, \mathbbm{R}_-, \{ 0 \} \subseteq \mathbbm{R}$ and
  $(\{ 0 \} \times \mathbbm{R}) \bigcup (\mathbbm{R} \times \{ 0 \}) \subseteq
  \mathbbm{R} \times \mathbbm{R}$ we have $(\mathbbm{R}_+ \times
  \mathbbm{R}_+) \bigcup (\mathbbm{R}_- \times \mathbbm{R}_-) \bigcup
  (\mathbbm{R}_+ \times \mathbbm{R}_-) \bigcup (\mathbbm{R}_- \times
  \mathbbm{R}_+) \bigcup \left( (\{ 0 \} \times \mathbbm{R}) \bigcup
  (\mathbbm{R} \times \{ 0 \}) \right) \subseteq \mathbbm{R} \times
  \mathbbm{R}$. Second if $(x, y) \in \mathbbm{R} \times \mathbbm{R}$ then
  using \ref{disjoint union of reals} we have $(x \in \mathbbm{R}_+ \vee x \in
  \mathbbm{R}_- \vee x = 0) \wedge (y \in \mathbbm{R}_+ \vee y \in
  \mathbbm{R}_- \vee y = 0) \Rightarrow (x \in \mathbbm{R}_+ \wedge y \in
  \mathbbm{R}_+) \vee (x \in \mathbbm{R}_+ \wedge y \in \mathbbm{R}_-) \vee (x
  \in \mathbbm{R}_+ \vee y = 0) \vee (x \in \mathbbm{R}_- \wedge y \in
  \mathbbm{R}_+) \vee (x \in \mathbbm{R}_- \wedge y \in \mathbbm{R}_-) \vee (x
  \in \mathbbm{R}_- \vee y = 0) \vee (x \in 0 \wedge y \in \mathbbm{R}_+) \vee
  (x \in 0 \wedge y \in \mathbbm{R}_-) \vee (x \in 0 \vee y = 0) \Rightarrow
  (x, y) \in (\mathbbm{R}_+ \times \mathbbm{R}_+) \bigcup (\mathbbm{R}_-
  \times \mathbbm{R}_-) \bigcup (\mathbbm{R}_+ \times \mathbbm{R}_-) \bigcup
  (\mathbbm{R}_- \times \mathbbm{R}_+) \bigcup \left( (\{ 0 \} \times
  \mathbbm{R}) \bigcup (\mathbbm{R} \times \{ 0 \}) \right) \Rightarrow
  \mathbbm{R} \times \mathbbm{R} \subseteq (\mathbbm{R}_+ \times
  \mathbbm{R}_+) \bigcup (\mathbbm{R}_- \times \mathbbm{R}_-) \bigcup
  (\mathbbm{R}_+ \times \mathbbm{R}_-) \bigcup (\mathbbm{R}_- \times
  \mathbbm{R}_+) \bigcup \left( (\{ 0 \} \times \mathbbm{R}) \bigcup
  (\mathbbm{R} \times \{ 0 \}) \right)$. This proves that $(\mathbbm{R}_+
  \times \mathbbm{R}_+) \bigcup (\mathbbm{R}_- \times \mathbbm{R}_-) \bigcup
  (\mathbbm{R}_+ \times \mathbbm{R}_-) \bigcup (\mathbbm{R}_- \times
  \mathbbm{R}_+) \bigcup \left( (\{ 0 \} \times \mathbbm{R}) \bigcup
  (\mathbbm{R} \times \{ 0 \}) \right) =\mathbbm{R} \times \mathbbm{R}$. Next
  to prove that the union is a disjoint union. From the fact that
  $\mathbbm{R}$ is the disjoint union of $\mathbbm{R}_+, \mathbbm{R}_-
  \tmop{and} \{ 0 \}$ we have the following
  \begin{enumerate}
    \item If $(x, y) \in (\{ 0 \} \times \mathbbm{R}) \bigcup (\mathbbm{R}
    \times \{ 0 \})$ then $x = 0 \vee y = 0$ so $(x, y) \nin \mathbbm{R}_+
    \times \mathbbm{R}_+, \mathbbm{R}_+ \times \mathbbm{R}_-, \mathbbm{R}_-
    \times \mathbbm{R}_+, \mathbbm{R}_- \times \mathbbm{R}_-$ and thus $\left(
    (\{ 0 \} \times \mathbbm{R}) \bigcup (\mathbbm{R} \times \{ 0 \}) \right)
    \bigcap (\mathbbm{R}_+ \times \mathbbm{R}_+) = \emptyset, \left( (\{ 0 \}
    \times \mathbbm{R}) \bigcup (\mathbbm{R} \times \{ 0 \}) \right) \bigcap
    (\mathbbm{R}_+ \times \mathbbm{R}_-) = \emptyset, \left( (\{ 0 \} \times
    \mathbbm{R}) \bigcup (\mathbbm{R} \times \{ 0 \}) \right) \bigcap
    (\mathbbm{R}_- \times \mathbbm{R}_+) = \emptyset, \left( (\{ 0 \} \times
    \mathbbm{R}) \bigcup (\mathbbm{R} \times \{ 0 \}) \right) \bigcap
    (\mathbbm{R}_- \times \mathbbm{R}_-) = \emptyset$
    
    \item If $(x, y) \in \mathbbm{R}_+ \times \mathbbm{R}_+$ then $x \neq 0, x
    \nin \mathbbm{R}_-, y \neq 0, y \nin \mathbbm{R}_-$ so $(x, y) \nin (\{ 0
    \} \times \mathbbm{R}) \bigcup (\mathbbm{R} \times \{ 0 \}), \mathbbm{R}_+
    \times \mathbbm{R}_-, \mathbbm{R}_- \times \mathbbm{R}_+, \mathbbm{R}_-
    \times \mathbbm{R}_-$ and thus $(\mathbbm{R}_+ \times \mathbbm{R}_+)
    \bigcap (\mathbbm{R}_+ \times \mathbbm{R}_-) = \emptyset, (\mathbbm{R}_+
    \times \mathbbm{R}_+) \bigcap (\mathbbm{R}_- \times \mathbbm{R}_+) =
    \emptyset, (\mathbbm{R}_+ \times \mathbbm{R}_+) \bigcap (\mathbbm{R}_-
    \times \mathbbm{R}_-) = \emptyset, (\mathbbm{R}_+ \times \mathbbm{R}_+)
    \bigcap \left( (\{ 0 \} \times \mathbbm{R}) \bigcup (\{ \mathbbm{R} \times
    \{ 0 \} \}) \right) = \emptyset$
    
    \item If $(x, y) \in \mathbbm{R}_+ \times \mathbbm{R}_-$ then $x \neq 0, x
    \nin \mathbbm{R}_-, y \neq 0, y \nin \mathbbm{R}_+$ so $(x, y) \nin (\{ 0
    \} \times \mathbbm{R}) \bigcup (\mathbbm{R} \times \{ 0 \}), \mathbbm{R}_+
    \times \mathbbm{R}_+, \mathbbm{R}_- \times \mathbbm{R}_+, \mathbbm{R}_-
    \times \mathbbm{R}_-$ and thus $(\mathbbm{R}_+ \times \mathbbm{R}_-)
    \bigcap (\mathbbm{R}_+ \times \mathbbm{R}_+) = \emptyset, (\mathbbm{R}_+
    \times \mathbbm{R}_-) \bigcap (\mathbbm{R}_- \times \mathbbm{R}_+) =
    \emptyset, (\mathbbm{R}_+ \times \mathbbm{R}_-) \bigcap (\mathbbm{R}_-
    \times \mathbbm{R}_-) = \emptyset, (\mathbbm{R}_+ \times \mathbbm{R}_+)
    \bigcap \left( (\{ 0 \} \times \mathbbm{R}) \bigcup (\{ \mathbbm{R} \times
    \{ 0 \} \}) \right) = \emptyset$
    
    \item If $(x, y) \in \mathbbm{R}_- \times \mathbbm{R}_+$ then $x \neq 0, x
    \nin \mathbbm{R}_+, y \neq 0, y \nin \mathbbm{R}_-$ so $(x, y) \nin (\{ 0
    \} \times \mathbbm{R}) \bigcup (\mathbbm{R} \times \{ 0 \}), \mathbbm{R}_+
    \times \mathbbm{R}_-, \mathbbm{R}_+ \times \mathbbm{R}_+, \mathbbm{R}_-
    \times \mathbbm{R}_-$ and thus $(\mathbbm{R}_- \times \mathbbm{R}_+)
    \bigcap (\mathbbm{R}_+ \times \mathbbm{R}_-) = \emptyset, (\mathbbm{R}_-
    \times \mathbbm{R}_+) \bigcap (\mathbbm{R}_+ \times \mathbbm{R}_-) =
    \emptyset, (\mathbbm{R}_+ \times \mathbbm{R}_-) \bigcap (\mathbbm{R}_-
    \times \mathbbm{R}_-) = \emptyset, (\mathbbm{R}_+ \times \mathbbm{R}_-)
    \bigcap \left( (\{ 0 \} \times \mathbbm{R}) \bigcup (\{ \mathbbm{R} \times
    \{ 0 \} \}) \right) = \emptyset$
    
    \item If $(x, y) \in \mathbbm{R}_- \times \mathbbm{R}_-$ then $x \neq 0, x
    \nin \mathbbm{R}_+, y \neq 0, y \nin \mathbbm{R}_+$ so $(x, y) \nin (\{ 0
    \} \times \mathbbm{R}) \bigcup (\mathbbm{R} \times \{ 0 \}), \mathbbm{R}_+
    \times \mathbbm{R}_+, \mathbbm{R}_+ \times \mathbbm{R}_-, \mathbbm{R}_-
    \times \mathbbm{R}_+$ and thus $(\mathbbm{R}_- \times \mathbbm{R}_-)
    \bigcap (\mathbbm{R}_+ \times \mathbbm{R}_+) = \emptyset, (\mathbbm{R}_-
    \times \mathbbm{R}_-) \bigcap (\mathbbm{R}_+ \times \mathbbm{R}_-) =
    \emptyset, (\mathbbm{R}_- \times \mathbbm{R}_-) \bigcap (\mathbbm{R}_-
    \times \mathbbm{R}_+) = \emptyset, (\mathbbm{R}_- \times \mathbbm{R}_-)
    \bigcap \left( (\{ 0 \} \times \mathbbm{R}) \bigcup (\{ \mathbbm{R} \times
    \{ 0 \} \}) \right) = \emptyset$
  \end{enumerate}
\end{proof}

Using the previous lemma and \ref{union definition of functions} we can define
$\cdot : \mathbbm{R} \times \mathbbm{R} \rightarrow \mathbbm{R}$

\begin{definition}
  We define $\cdot : \mathbbm{R} \times \mathbbm{R} \rightarrow \mathbbm{R}$
  $(\alpha, \beta) \rightarrow \alpha \cdot \beta$ as follows
  \begin{eqnarray*}
    \alpha \cdot \beta = \left\{ \begin{array}{l}
      \alpha \odot \beta \tmop{if} (\alpha, \beta) \in \mathbbm{R}_+ \times
      \mathbbm{R}_+\\
      - ((- \alpha) \odot \beta) \tmop{if} (\alpha, \beta) \in \mathbbm{R}_-
      \times \mathbbm{R}_+\\
      - (\alpha \times (- \beta)) \tmop{if} (\alpha, \beta) \in \mathbbm{R}_+
      \times \mathbbm{R}_-\\
      (\um \alpha) \odot (\um \beta) \tmop{if} (\alpha, \beta) \in
      \mathbbm{R}_+ \times \mathbbm{R}_+\\
      0 \tmop{if} (\alpha, \beta) \in (\{ 0 \} \times \mathbbm{R}) \bigcup
      (\mathbbm{R} \times \{ 0 \})
    \end{array} \right. &  & 
  \end{eqnarray*}
  
\end{definition}

If we want to prove something about multiplication we have to consider each
time 5 different cases, luckily the following lemma will allow us to bring
down the different cases.

\begin{lemma}
  \label{lemma to help prove that the reals forms a field}$\forall (\alpha,
  \beta) \in \mathbbm{R} \times \mathbbm{R}$ we have $- (\alpha \cdot \beta) =
  (\um \alpha) \cdot \beta = \alpha \cdot (- \beta)$
\end{lemma}

\begin{proof}[$(\alpha, \beta) \in \mathbbm{R}_+ \times
\mathbbm{R}_+$][$(\alpha, \beta) \in \mathbbm{R}_+ \times
\mathbbm{R}_-$][$(\alpha, \beta) \in \mathbbm{R}_- \times
\mathbbm{R}_+$][$(\alpha, \beta) \in \mathbbm{R}_- \times
\mathbbm{R}_-$][$(\alpha, \beta) \in (\{ 0 \} \times \mathbbm{R}) \bigcup
(\mathbbm{R} \times \{ 0 \})$]
  We have to consider the following 5 cases
  \begin{enumerate}
    \item We have then
    \begin{enumerate}
      \item $\um \alpha \in \mathbbm{R}_-$ and (see \ref{inverse of inverse})
      $\alpha = - (- \alpha)$ then we have
      \begin{eqnarray*}
        - (\alpha \cdot \beta) & = & - (\alpha \odot \beta)\\
        & = & - (- (- \alpha) \odot \beta)\\
        & = & - (- (- \alpha \cdot \beta))\\
        & = & - (- \alpha \cdot \beta)
      \end{eqnarray*}
      \item $- \beta \in \mathbbm{R}_-$ and (see \ref{inverse of inverse})
      $\beta = \um (\um \beta)$ then we have
      \begin{eqnarray*}
        - (\alpha \cdot \beta) & = & - (\alpha \odot \beta)\\
        & = & - (\alpha \odot (\um (\um \beta)))\\
        & = & - (- (\alpha \cdot (- \beta)))\\
        & = & (\alpha \cdot (- \beta))
      \end{eqnarray*}
    \end{enumerate}
    \item We have then
    \begin{enumerate}
      \item $- \alpha \in \mathbbm{R}_-$ and $\alpha = \um (\um \alpha)$ then
      we have
      \begin{eqnarray*}
        - (\alpha \cdot \beta) & = & - (- (\alpha \odot (- \beta)))\\
        & = & (- (- \alpha) \odot (- \beta))\\
        & = & ((\um \alpha) \cdot \beta)
      \end{eqnarray*}
      \item $- \beta \in \mathbbm{R}_+$ and $\beta = - (- \beta)$ then we have
      \begin{eqnarray*}
        - (\alpha \cdot \beta) & = & - (- (\alpha \odot (- \beta)))\\
        & = & (\alpha \odot (- \beta))\\
        & = & (\alpha \cdot (- \beta))
      \end{eqnarray*}
    \end{enumerate}
    \item We have then
    \begin{enumerate}
      \item $- \alpha \in \mathbbm{R}_+$ and $\alpha = - (- \alpha)$ then we
      have
      \begin{eqnarray*}
        - (\alpha \cdot \beta) & = & - (- (- \alpha \odot \beta))\\
        & = & (- \alpha \cdot \beta)
      \end{eqnarray*}
      \item $- \beta \in \mathbbm{R}_-$ and $\beta = - (- \beta)$ then we have
      \begin{eqnarray*}
        - (\alpha \cdot \beta) & = & - (- (- \alpha \odot \beta))\\
        & = & (- \alpha \odot \beta)\\
        & = & (- \alpha \odot (- (- \beta)))\\
        & = & (\alpha \cdot (- \beta))
      \end{eqnarray*}
    \end{enumerate}
    \item We have then
    \begin{enumerate}
      \item $- \alpha \in \mathbbm{R}_+$ and $\alpha = - (- \alpha)$ then we
      have
      \begin{eqnarray*}
        - (\alpha \cdot \beta) & = & - (- \alpha \odot (- \beta))\\
        & = & - (- (- \alpha \cdot \beta))\\
        & = & (- \alpha \cdot \beta)
      \end{eqnarray*}
      \item $- \beta \in \mathbbm{R}_+$ and $\beta = - (- \beta)$ then we have
      \begin{eqnarray*}
        - (\alpha \cdot \beta) & = & - (- \alpha \odot - \beta)\\
        & = & - (- (\alpha \cdot (- \beta)))\\
        & = & (\alpha \cdot (- \beta))
      \end{eqnarray*}
    \end{enumerate}
    \item then we have the following (non exclusive cases)
    \begin{enumerate}
      \item $\alpha = 0$ We have then
      \begin{eqnarray*}
        - (\alpha \cdot \beta) & = & \um (0 \cdot \beta)\\
        & = & 0\\
        & = & (0 \cdot \beta)\\
        & = & (- 0 \cdot \beta)\\
        & = & (\um \alpha \cdot \beta)\\
        - (\alpha \cdot \beta) & = & - (0 \cdot \beta)\\
        & = & 0\\
        & = & (0 \cdot (- \beta))\\
        & = & (\alpha \cdot (- \beta))
      \end{eqnarray*}
      \item $\beta = 0$ We have then
      \begin{eqnarray*}
        - (\alpha \cdot \beta) & = & - (\alpha \cdot 0)\\
        & = & 0\\
        & = & (- \alpha, 0)\\
        & = & (- \alpha, \beta)\\
        - (\alpha \cdot \beta) & = & - (\alpha \cdot 0)\\
        & = & 0\\
        & = & (\alpha \cdot (- 0))\\
        & = & (\alpha \cdot (- \beta))
      \end{eqnarray*}
    \end{enumerate}
  \end{enumerate}
\end{proof}

\begin{theorem}
  \label{reciprocal cut}{\index{reciprocal cut}}$\forall \alpha \in
  \mathbbm{R}_+$ we have that $\tmop{rep} (\alpha) = \{ r \in \mathbbm{Q} | r
  \leqslant 0 \nobracket \} \bigcup \{ s^{- 1} | s \in \mathbbm{Q} \backslash
  \alpha \nobracket \wedge s > 0 \wedge s \neq \min (\mathbbm{Q} \backslash
  \alpha) \}$ is a cut. This is called the reciprocal cut of $\alpha$.
\end{theorem}

\begin{proof}[$\tmop{rep} (\alpha) \neq \emptyset$][$\tmop{rep} (\alpha) \neq
\mathbbm{Q}$][$\forall r \in \tmop{rep} (\alpha) \wedge \forall s \in
\mathbbm{Q} \backslash \tmop{rep} (\alpha)$ we have $r < s$][$r \leqslant
0$][$0 < r$][$s^{- 1} = \min (\mathbbm{Q} \backslash \alpha)$][$s^{- 1} \neq
\min (\mathbbm{Q} \backslash \alpha)$][$\tmop{rep} (\alpha)$ does not have a
greatest element][$m \leqslant 0$][$0 < m$]
  We have
  \begin{enumerate}
    \item As $0 \in \{ r \in \mathbbm{Q} | r \leqslant 0 \nobracket \}$ we
    have $0 \in \tmop{rep} (\alpha) \Rightarrow \tmop{rep} (\alpha) \neq
    \emptyset$
    
    \item As $0 \in \alpha$ then from the fact that there does not exists a
    maximum for $\alpha$ there exists a \ $s \in \alpha$ with $s > 0$. Using
    \ref{0<less>s=<gtr>0<less>s^-1} we have $0 < s^{- 1} \Rightarrow s^{- 1}
    \nin \{ r \in \mathbbm{Q} | r \leqslant 0 \nobracket \}$ and as $s \in
    \alpha$ we have $s^{- 1} \nin \{ s^{- 1} | s \in \mathbbm{Q} \backslash
    \alpha \nobracket \wedge s > 0 \wedge s \neq \min (\mathbbm{Q} \backslash
    \alpha) \}$ so $s^{- 1} \in \mathbbm{Q} \backslash \tmop{rep} (\alpha)$
    
    \item Let $r \in \tmop{rep} (\alpha)$ and $s \in \mathbbm{Q} \backslash
    \tmop{rep} (\alpha) \Rightarrow s \nin \tmop{rep} (\alpha)$. We have for
    $r$ the following exclusive cases
    \begin{enumerate}
      \item then as $s \nin \tmop{rep} (\alpha)$ we must have $s \in \{ r \in
      \mathbbm{Q} | r \leqslant 0 \nobracket \} \Rightarrow 0 < s \Rightarrow
      r < s$
      
      \item then $0 < r^{- 1}$, $r^{- 1} \in \mathbbm{Q} \backslash \alpha
      \Rightarrow r^{- 1} \nin \alpha$ and $r^{- 1} \neq \min (\mathbbm{Q}
      \backslash \alpha)$. As $s \in \mathbbm{Q} \backslash \tmop{rep}
      (\alpha)$ we have $0 < s \Rightarrow 0 < s^{- 1}$. Consider now the
      following possible cases for $s^{- 1}$ then
      \begin{enumerate}
        \item then as $r^{- 1} \nin \alpha \Rightarrow r^{- 1} \nin
        \mathbbm{Q} \backslash \alpha \Rightarrowlim_{s^{- 1} = \min
        (\mathbbm{Q} \backslash \alpha)} s^{- 1} \leqslant r^{- 1}
        \Rightarrowlim_{r^{- 1} \neq \min (\mathbbm{Q} \backslash \alpha)}
        s^{- 1} < r^{- 1} \Rightarrowlim_{\text{\ref{some properties of
        rational numbers}}} r < s$
        
        \item As $s \nin \tmop{rep} (\alpha)$ we have $0 < s$ [$s \nin \{ r
        \in \mathbbm{Q} | r \leqslant 0 \nobracket \}$] and thus by \ref{some
        properties of rational numbers} we have $0 < s^{- 1}$. We must then
        have $s^{- 1} \in \alpha$ [otherwise $s^{- 1} \in \{ s^{- 1} | s \in
        \mathbbm{Q} \backslash \alpha \nobracket \wedge s > 0 \wedge s \neq
        \min (\mathbbm{Q} \backslash \alpha) \} \subseteq \tmop{rep}
        (\alpha)$] and using \ref{dedekind's cut} , 3 we have then $s^{- 1} <
        r^{- 1} \Rightarrowlim_{\text{\ref{some properties of rational
        numbers}}} r < s$
      \end{enumerate}
    \end{enumerate}
    \item We prove this by contradiction. So assume that $m = \max (\tmop{rep}
    (\alpha)) \in \tmop{rep} (\alpha)$ exists, then we we derive a
    contradiction from this. We can now have for $m \in \tmop{rep} (\alpha)$
    the following possibilities
    \begin{enumerate}
      \item Using \ref{dedekind's cut} , 2 there exists a $s \in \mathbbm{Q}
      \backslash \alpha \Rightarrow s \nin \alpha$, as $\alpha \in
      \mathbbm{R}_+ \Rightarrow 0 \in \alpha
      \Rightarrowlim_{\text{\ref{dedekind's cut} , 3}} 0 < s \Rightarrow 0 < s
      < s \upl 1$. From $0 < s < s \upl 1$ we have $s \upl 1 \nin \alpha$ [for
      otherwise a s $s \nin \alpha$ we have by \ref{dedekind's cut} , 3 that
      $s + 1 < s$ contradicting $s < s \upl 1$], as $s < s + 1$ we have $s
      \upl 1 \neq \min (\mathbbm{Q} \backslash \alpha)$ and from $0 < s + 1$
      we have $0 < (s \upl 1)^{- 1}$ and thus we have $(s \upl 1)^{- 1} \in
      \tmop{rep} (\alpha)$. Then $0 \leqslant m < (s \upl 1)^{- 1} \leqslant m
      \Rightarrow m < m$ and we reach a contradiction.
      
      \item then as $m \in \tmop{rep} (\alpha)$ we must have $0 < m^{- 1},
      m^{- 1} \nin \alpha$ and $m^{- 1} \neq \min (\mathbbm{Q} \backslash
      \alpha)$. From the last inequality we have the existence of a $t \in
      \mathbbm{Q} \backslash \alpha$ such that $t < m^{- 1}$. Using the fact
      that $\mathbbm{Q}$ is dense (see \ref{rational numbers are dense}) there
      exists a $s \in \mathbbm{Q}$ such that $t < s < m^{- 1}$ then we must
      have $s \nin \alpha$ [for otherwise by \ref{dedekind's cut} , 3 I would
      have $s < t$ contradicting $t < s$]. As $\alpha \in \mathbbm{R}_+
      \Rightarrow 0 \in \alpha \Rightarrow 0 < t < s < m^{- 1}$ then as $t
      \nin \alpha$ we must have $s \nin \alpha$ [otherwise we must have $s <
      t$ contradicting $t < s$]. Since $t < s$ and $t, s \in \mathbbm{Q}
      \backslash \alpha$ we have $s \neq \min (\mathbbm{Q} \backslash \alpha)
      \nocomma$. We must then from $0 < s \Rightarrow 0 < s^{- 1}, s \neq \min
      (\mathbbm{Q} \backslash \alpha)$, $s \in \mathbbm{Q} \backslash \alpha$
      conclude that $s^{- 1} \in \tmop{rep} (\alpha)$. From $0 < s < m^{- 1}$
      we have $m = (m^{- 1})^{- 1} < s^{- 1} \Rightarrow m < s^{- 1}$
      contradicting the maximality of $m$.
    \end{enumerate}
  \end{enumerate}
\end{proof}

\begin{definition}
  $\mathbbm{R}_0 =\mathbbm{R} \backslash \{ 0 \}$. Note that by \ref{disjoint
  union of reals} we have $\mathbbm{R}_0 =\mathbbm{R}_+ \bigcup \mathbbm{R}_-$
\end{definition}

We are now ready to define the multiplicative inverse of non zero real
numbers.

\begin{definition}
  \label{inverse for multiplication of reals}If $r \in \mathbbm{R}_0$ the we
  define $r^{- 1}$ as follows $r^{- 1} = \left\{ \begin{array}{l}
    \tmop{rep} (r) \tmop{if} r \in \mathbbm{R}_+\\
    - \tmop{rep} (- r) \tmop{if} r \in \mathbbm{R}_+
  \end{array} \right.$
\end{definition}

Actual the reciprocal cut of a cut takes a rather simple form in the case of
positive rational cuts.

\begin{lemma}
  \label{inverse of element in Q in R}If $r \in \mathbbm{Q}$ with $\alpha_r
  \in \mathbbm{R}_0$ then $r^{- 1} = \alpha_{r^{- 1}}$ (note that there are
  two different inverses here one in $\mathbbm{Q}$ and one in $\mathbbm{R}_0$)
\end{lemma}

\begin{proof}[$x \leqslant 0$][$0 < x$][$x \leqslant 0$][$0 < x$]
  First we take the case of $a_r \in \mathbbm{R}_+$ then we have as $0 \in
  \alpha_r \Rightarrow 0 < r \Rightarrow 0 < r^{- 1}$. For $x \in
  (\alpha_r)^{- 1}$ the following possible cases
  \begin{enumerate}
    \item then $x \leqslant 0 < r^{- 1} \Rightarrow x \in \alpha_{r^{- 1}}$
    
    \item then we must have $0 < x^{- 1}, x^{- 1} \in \mathbbm{Q} \backslash
    \alpha_r, x^{- 1} \neq \min (\mathbbm{Q} \backslash \alpha_r)
    \equallim_{\text{\ref{rational cuts}}} r \Rightarrow r < x^{- 1}
    \Rightarrowlim_{\text{\ref{some properties of rational numbers}}} (x^{-
    1})^{- 1} < r^{- 1} \Rightarrow x < r^{- 1} \Rightarrow x \in \alpha_{r^{-
    1}}$
  \end{enumerate}
  so we have $(\alpha_r)^{- 1} \subseteq \alpha_{r^{- 1}}$
  
  If $x \in \alpha_{r^{- 1}}$ then we have the following possible cases for
  $x$
  \begin{enumerate}
    \item Then $x \in (\alpha_r)^{- 1}$ trivially.
    
    \item Then as $x \in \alpha_{r^{- 1}} \Rightarrow x < r^{- 1}
    \Rightarrowlim_{\text{\ref{some properties of rational numbers}}} r < x^{-
    1} \Rightarrow r = \min (\mathbbm{Q} \backslash \alpha_r) \neq x^{- 1}, 0
    < x^{- 1}$ and $x^{- 1} \in \alpha_r \Rightarrow x^{- 1} \in \mathbbm{Q}
    \backslash \alpha_r \Rightarrow x \in (\alpha_r)^{- 1}$
  \end{enumerate}
  so we have $\alpha_{r^{- 1}} \in (\alpha_r)^{- 1}$. And we conclude thus
  that $\alpha_{r^{- 1}} = (\alpha_r)^{- 1}$.
  
  Second if $\alpha_r \in \mathbbm{R}_-$ then $(\alpha_r)^{- 1} = -
  (\tmop{rep} (\um \alpha)) \equallim_{- \alpha \in \mathbbm{R}_+ \tmop{and}
  \tmop{previous}} - (\alpha_{(- r)^{- 1}}) \equallim_{\text{\ref{some
  properties of rational numbers}}} - (\alpha_{- (r^{- 1})})
  \equallim_{\text{\ref{negative of rational cut}}} \alpha_{(- (- (r^{- 1})))}
  = \alpha_{r^{- 1}}$
\end{proof}

\begin{theorem}[$\mathbbm{R} \tmop{forms} a \tmop{field}$]
  \label{the real numbers forms a field}{\index{$\langle \mathbbm{R}, +, \cdot
  \rangle \tmop{is} a \tmop{field}$}}$\langle \mathbbm{R}, +, \cdot \rangle$
  forms a field. The neutral element (identity) from multiplication is
  $\alpha_1$, to simplify our notations in the future we note $\alpha_1$ as
  $1$ (again $1$ means different things in different fields). The
  multiplicative inverse of $\alpha$ is $\alpha^{- 1}$
\end{theorem}

\begin{proof}[Commutativity][$(\alpha, \beta) \in \mathbbm{R}_+ \times
\mathbbm{R}_+$][$(\alpha, \beta) \in \mathbbm{R}_+ \times
\mathbbm{R}_-$][$(\alpha, \beta) \in \mathbbm{R}_- \times
\mathbbm{R}_+$][$(\alpha, \beta) \in \mathbbm{R}_- \times
\mathbbm{R}_-$][$(\alpha, \beta) \in \left( (\{ 0 \} \times \mathbbm{R})
\bigcup (\mathbbm{R} \times \{ 0 \}) \right)$][Neutral element][$\alpha \in
\mathbbm{R}_+$][$x \leqslant 0$][$0 < x$][$x \leqslant 0$][$0 < x$][$\alpha
\in \mathbbm{R}_-$][$\alpha = 0$][Inverse for non zero element][$\alpha \in
\mathbbm{R}_+$][$x \leqslant 0$][$0 < x$][$t \leqslant 0$][$0 < t$][$x
\leqslant 0$][$0 < x$][$\alpha \in \mathbbm{R}_-$][Distributive][$x \leqslant
0$][$0 < x$][$x \leqslant 0$][$0 < x$][$r \leqslant 0 \wedge t \leqslant
0$][$r \leqslant 0 \wedge 0 < t$][$0 < r \wedge t \leqslant 0$][$0 < r \wedge
0 < t$][$u = u'$][$u < u'$][$u' < u$][$\beta \upl \gamma \in
\mathbbm{R}_+$][$\beta \upl \gamma \in \mathbbm{R}_-$][Associative][$x
\leqslant 0$][$0 < x$]
  First we have already proved that $\langle \mathbbm{R}, + \rangle$ is a
  abelian group (see \ref{real numbers form a additive group}) we prove then
  the following for multiplication. In the proof we make heavily use of the
  lemma \ref{lemma to help prove that the reals forms a field}.
  \begin{enumerate}
    \item we have to consider the following five cases
    \begin{enumerate}
      \item Here we have
      \begin{eqnarray*}
        \alpha \cdot \beta & = & \alpha \odot \beta\\
        & = & \{ r \in \mathbbm{Q} | r \leqslant 0 \nobracket \} \bigcup \{ s
        \cdot t | s \in \alpha, t \in \beta, s > 0, t > 0 \nobracket \}\\
        & \equallim_{\text{\ref{the rational numbers form a field}}} & \{ r
        \in \mathbbm{Q} | r \leqslant 0 \nobracket \} \bigcup \{ t \cdot s | s
        \in \alpha, t \in \beta, s > 0, t > 0 \nobracket \}\\
        & = & \beta \odot \alpha\\
        & = & \beta \cdot \alpha
      \end{eqnarray*}
      \item Here we have
      \begin{eqnarray*}
        \alpha \cdot \beta & = & - (\alpha \cdot (- \beta))\\
        & \equallim_{1. a} & - (\beta \cdot (\um \alpha))\\
        & = & \beta \cdot \alpha
      \end{eqnarray*}
      \item Here we have
      \begin{eqnarray*}
        \alpha \cdot \beta & = & - (- \alpha \cdot \beta)\\
        & \equallim_{1. a} & - (\beta \cdot (- \alpha))\\
        & = & \beta \cdot \alpha
      \end{eqnarray*}
      \item Here we have
      \begin{eqnarray*}
        \alpha \cdot \beta & = & - (- (- \alpha \cdot (- \beta)))\\
        & = & ((- \alpha) \cdot (- \beta))\\
        & \equallim_{1. a} & ((- \beta) \cdot (- \alpha))\\
        & = & \beta \cdot \alpha
      \end{eqnarray*}
      \item Here we have
      \begin{eqnarray*}
        \alpha \cdot \beta & = & 0\\
        & = & \beta \cdot \alpha
      \end{eqnarray*}
    \end{enumerate}
    \item 
    
    First note that $0 < 1$ so $0 \in \alpha_1$ we have the to consider the
    following cases
    \begin{enumerate}
      \item If $x \in \alpha \odot \alpha_1$ then we have either
      \begin{enumerate}
        \item As $\alpha \in \mathbbm{R}_+ \Rightarrow 0 \in \alpha$ then
        using \ref{property to determine membership of a cut} we have $x \in
        \alpha$
        
        \item So there exists a $s \in \alpha, t \in \alpha_1 \Rightarrow t <
        1$ with $0 < s, t$ and $x = s \cdot t$, from $t < 1$ we have $x = s
        \cdot t < s \Rightarrow x \leqslant s$, as $s \in \alpha$ we have by
        \ref{property to determine membership of a cut} that $x \in \alpha$.
      \end{enumerate}
      So we conclude that $\alpha \odot \alpha_1 \subseteq \alpha$. If $x \in
      \alpha$ then we have the following possible cases
      \begin{enumerate}
        \item then $x \in \alpha \odot \alpha_1$
        
        \item then as $\max (\alpha)$ does not exists there exists a $s \in
        \alpha$ such that $0 < x < s$ then $0 < x \cdot s^{- 1} < 1
        \Rightarrow 0 < x \cdot s^{- 1} \in \alpha_1$ and from $x = s \cdot (x
        \cdot s^{- 1})$ we have $x \in \alpha \odot \alpha_1$
      \end{enumerate}
      this gives $\alpha \subseteq \alpha \odot \alpha_1$. Or to summarize we
      have $\alpha = \alpha \odot \alpha_1$. Now we have
      \begin{eqnarray*}
        \alpha & = & \alpha \odot \alpha_1\\
        & = & \alpha \cdot \alpha_1\\
        & \equallim_{\tmop{commutativity}} & \alpha_1 \cdot \alpha
      \end{eqnarray*}
      \item Here we have
      \begin{eqnarray*}
        \alpha \cdot \alpha_1 & = & - ((- \alpha) \cdot \alpha_1)\\
        & \equallim_{2. a} & - (- \alpha)\\
        & = & \alpha\\
        & \equallim_{\tmop{commutativity}} & \alpha_1 \cdot \alpha
      \end{eqnarray*}
      \item Here we have
      \begin{eqnarray*}
        \alpha \cdot \alpha_1 & = & 0\\
        & = & \alpha_1 \cdot \alpha
      \end{eqnarray*}
    \end{enumerate}
    \item As $\alpha \in \mathbbm{R}_0$ we have the following cases to
    consider
    \begin{enumerate}
      \item Take then $x \in \alpha \odot \alpha^{- 1} = \alpha \odot
      \tmop{rep} (\alpha)$ then we have either
      \begin{enumerate}
        \item then as $0 < 1$ we have $x < 1 \Rightarrow x \in \alpha_1$
        
        \item then $\exists s \in \alpha \vdash 0 < s$ and $\exists t \in
        \alpha^{- 1} \vdash 0 < t$ such that $s \cdot t = x$. Now for $t$ we
        have the following cases
        \begin{enumerate}
          \item then from $0 < s$ we have $x = s \cdot t \leqslant 0 < 1
          \Rightarrow x < 1 \Rightarrow x \in \alpha_1$
          
          \item then $t^{- 1} \in \mathbbm{Q} \backslash \alpha$, $0 < t^{-
          1}$ and $t^{- 1} \neq \min (\mathbbm{Q} \backslash \alpha)$. As $s
          \in \alpha$ we have from $t^{- 1} \in \mathbbm{Q} \backslash \alpha$
          and \ref{dedekind's cut} , 3 that $s < t^{- 1} \Rightarrow x = s
          \cdot t < 1 \Rightarrow x \in \alpha_1$
        \end{enumerate}
      \end{enumerate}
      Take now $x \in \alpha_1 \Rightarrow x < 1$ then we have either
      \begin{enumerate}
        \item $\Rightarrow x \in \alpha \odot \tmop{rep} (\alpha)$
        
        \item then we have $x^{- 1} \in \mathbbm{Q} \backslash \alpha, 0 <
        x^{- 1}$ and $x^{- 1} \neq \min (\mathbbm{Q} \backslash \alpha)$. From
        $0 < x^{- 1}$ we have by \ref{0<less>s=<gtr>0<less>s^-1} that $0 < x
        \Rightarrow 0 < x < 1 \Rightarrow 0 < 1 \um x$. Since $\alpha \in
        \mathbbm{R}_+ \Rightarrow 0 \in \alpha$ and as $\max (\alpha)$ does
        not exists there exists a $s_1 \in \alpha$ such that $0 < s_1$. From
        $0 < x < 1$ we have $1^{- 1} = 1 < x^{- 1}$ and thus $0 < 1 - x < x^{-
        1} (x - 1) \Rightarrow 0 < s_1 \cdot (x - 1) \cdot x^{- 1}$. If we
        note $\varepsilon = s_1 \cdot (x \um 1) \cdot x^{- 1}$ and $0 <
        \varepsilon$. Using \ref{lemma for sum of reals} there exists a $s_2
        \in \alpha$ such that $\varepsilon \upl s_2 \nin \alpha$. Then we can
        find a $s_3 \in \alpha$ such that $\varepsilon \upl s_2 \leqslant
        \varepsilon \upl s_3 \in \mathbbm{Q} \backslash \alpha, \varepsilon
        \upl s_3 \neq \min (\mathbbm{Q} \backslash \alpha)$ [[If $\varepsilon
        \upl s_2 \neq \min (\mathbbm{Q} \backslash \alpha)$ we take $s_3 =
        s_2$, else if $\varepsilon \upl s_2 = \min (\mathbbm{Q} \backslash
        \alpha)$ then as $\max (\alpha)$ does not exists there exists a $s_3
        \in \alpha$ with $s_2 < s_3 \Rightarrow \varepsilon \upl s_2 <
        \varepsilon \upl s_3$, now $\varepsilon \upl s_3 \in \mathbbm{Q}
        \backslash \alpha$ [otherwise $\varepsilon \upl s_3 \in \alpha
        \Rightarrowlim_{\text{\ref{dedekind's cut}}} \varepsilon + s_3 <
        \varepsilon \upl s_2$ contradicting $\varepsilon \upl s_2 <
        \varepsilon \upl s_3$], we have then $\varepsilon \upl s_3 \neq \min
        (\mathbbm{Q} \backslash \alpha)$]]. Take now $s = \max (s_1, s_3
        \nosymbol)$. Then as $0 \in \alpha$ we have as $\varepsilon \upl s_2
        \in \mathbbm{Q} \backslash \alpha$ by \ref{dedekind's cut} that $0 <
        \varepsilon \upl s_2 \leqslant \varepsilon \upl s$, $\varepsilon \upl
        s \in \mathbbm{Q} \backslash \alpha$ [if $\varepsilon \upl s \in
        \alpha \Rightarrowlim_{\varepsilon \upl s_2 \in \mathbbm{Q} \backslash
        \alpha \nocomma, \text{\ref{dedekind's cut}}} \varepsilon \upl s <
        \varepsilon \upl s_2$ contradicting $\varepsilon \upl s_2 <
        \varepsilon \upl s$] and $\varepsilon \upl s \neq \min (\mathbbm{Q}
        \backslash \alpha)$ hence $0 < (\varepsilon \upl s)^{- 1} \in
        \tmop{rep} (\alpha)$. As $s_1, s_3 \in \alpha \Rightarrow s = \max
        (s_1, s_2) \Rightarrow s \in \alpha$ and from $0 < s_1 \leqslant \max
        (x_1, s_2) \Rightarrow 0 < s$. The last two results gives us $s \cdot
        (\varepsilon + s)^{- 1} \in \alpha \odot \tmop{rep} (\alpha)$. Now
        from $s_1 \leqslant s$ we have by $0 < x^{- 1}, (1 \um x)$ that
        $\varepsilon = s_1 \cdot (1 \um x) \cdot x^{- 1} \leqslant s \cdot (1
        \um x) \cdot x^{- 1} \Rightarrow 0 < \varepsilon \upl s \leqslant s
        \upl s \cdot (1 \um x) \cdot x^{- 1} = s \cdot (1 \upl (1 \um x) \cdot
        x^{- 1}) \Rightarrow s^{- 1} \cdot (1 \upl (1 \um x) \cdot x^{- 1}) =
        (s \cdot (1 \upl (1 \um x) \cdot x^{- 1}))^{- 1} \leqslant
        (\varepsilon \upl s)^{- 1} \Rightarrow (1 \upl (1 \um x) \cdot x^{-
        1})^{- 1} \leqslant s \cdot (\varepsilon \upl s)^{- 1}$. Now $1 \upl
        (1 \um x) \cdot x^{- 1} = 1 \upl x^{- 1} \um x \cdot x^{- 1} = x^{- 1}
        \Rightarrow x \leqslant s \cdot (\varepsilon \upl s)^{- 1}$, as $s
        \cdot (\varepsilon \upl s)^{- 1} \in \alpha \odot \alpha^{- 1}$ we
        have by \ref{property to determine membership of a cut} that $x \in
        \alpha \odot \tmop{rep} (\alpha)$ \ 
      \end{enumerate}
      So we have proved that $\alpha \odot \tmop{rep} (\alpha) = \alpha_1$ and
      thus
      \begin{eqnarray*}
        \alpha \cdot \alpha^{- 1} & = & \alpha \odot \alpha^{- 1}\\
        & = & \alpha \odot \tmop{rep} (\alpha)\\
        & = & \alpha_1\\
        & \equallim_{\tmop{commutativity}} & \alpha^{- 1} \cdot \alpha
      \end{eqnarray*}
      \item We have then
      \begin{eqnarray*}
        \alpha \cdot \alpha^{- 1} & = & \alpha \cdot (\um (\tmop{rep} (\um
        \alpha)))\\
        & = & - (\alpha \cdot (- \alpha)^{- 1})\\
        & = & - (- ((\um \alpha) \cdot (- \alpha)^{- 1}))\\
        & = & ((- \alpha) \cdot (- \alpha)^{- 1})\\
        & \equallim_{3. b} & a_1\\
        & \equallim_{\tmop{commutativity}} & \alpha^{- 1} \cdot \alpha
      \end{eqnarray*}
    \end{enumerate}
    \item We have the following cases to consider
    \begin{enumerate}
      \item $\alpha \in \mathbbm{R}_+, \beta \in \mathbbm{R}_+, \gamma \in
      \mathbbm{R}_+$ We have then
      \begin{eqnarray*}
        x \in \alpha \cdot (\beta \upl \gamma) & \Rightarrow & x \in \alpha
        \odot (\beta \upl \gamma)
      \end{eqnarray*}
      we have then two possible cases
      \begin{enumerate}
        \item then $x \in \alpha \odot \beta$ also $0 \in \alpha \odot \gamma
        \Rightarrow x = x \upl 0 \in (\alpha \odot \beta) \upl (\alpha \odot
        \gamma)$
        
        \item then $x = s \cdot t$ where $s \in \alpha \wedge 0 < s$ and $t
        \in \beta \upl \gamma \wedge 0 < t \Rightarrow \exists u \in \beta
        \wedge v \in \gamma$ with $t = u \upl v$. Using \ref{the rational
        numbers form a field} we have that $x = s \cdot t = s \cdot (u \upl v)
        = s \cdot u + s \cdot v$. We have now the following possibilities
        \begin{enumerate}
          \item $u \leqslant 0 \wedge v \leqslant 0$ this case does not apply
          because it gives rise to the contradiction $0 < t = u \upl v
          \leqslant 0 \Rightarrow 0 < 0$.
          
          \item $u \leqslant 0 \wedge 0 < v \Rightarrowlim_{0 < s} s \cdot u
          \leqslant 0 \Rightarrow s \cdot u \in \alpha \odot \beta \wedge s
          \cdot t \in \alpha \odot \gamma \Rightarrow x = s \cdot u \upl s
          \cdot v \in (\alpha \odot \beta) \upl (\alpha \odot \gamma)
          \Rightarrow x \in (\alpha \odot \beta) \upl (\alpha \odot \gamma)$
          
          \item $0 < u \wedge v \leqslant 0 \Rightarrow s \cdot v \leqslant 0
          \Rightarrow s \cdot u \in \alpha \odot \beta \wedge s \cdot t \in
          \alpha \odot \gamma \Rightarrow x = s \cdot u \upl s \cdot v \in
          (\alpha \odot \beta) \upl (\alpha \odot \gamma) \Rightarrow x \in
          (\alpha \odot \beta) \upl (\alpha \odot \gamma)$
          
          \item $0 < u \wedge 0 < v \Rightarrow s \cdot u \in \alpha \odot
          \beta \wedge s \cdot t \in \alpha \odot \gamma \Rightarrow x = s
          \cdot u \upl s \cdot v \in (\alpha \odot \beta) \upl (\alpha \odot
          \gamma) \Rightarrow x \in (\alpha \odot \beta) \upl (\alpha \odot
          \gamma)$
        \end{enumerate}
        So we have $\alpha \odot (\beta \upl \gamma) \subseteq \alpha \odot
        \beta + \alpha \odot \gamma$
      \end{enumerate}
      Consider now
      \begin{eqnarray*}
        x \in \alpha \odot \beta + \alpha \odot \gamma & \tmop{then} x = r
        \upl t & \tmop{with} r \in \alpha \odot \beta \wedge t \in \alpha
        \odot \gamma
      \end{eqnarray*}
      we have then the following possibilities
      \begin{enumerate}
        \item $\Rightarrow x \in \alpha \odot (\beta \upl \gamma)$
        
        \item we have now the following possible sub-cases
        \begin{enumerate}
          \item then we would have the contradiction $x \leqslant 0$ so this
          case does not applies.
          
          \item then $\exists u \in \alpha, \exists v \in \gamma$ such that $t
          = u \cdot v \wedge 0 < u \wedge 0 < v \Rightarrow t = u \cdot v = u
          \cdot (0 \upl v) \in \alpha \odot (\beta + \gamma)$ since $r
          \leqslant 0 \Rightarrow x = r \upl t \leqslant 0 \upl t \Rightarrow
          x \leqslant t$ we have by $t \in \alpha \odot (\beta \upl \gamma)$
          and \ref{property to determine membership of a cut} that $x \in
          \alpha \odot (\beta + \gamma)$
          
          \item then $\exists u \in \alpha, \exists v \in \gamma$ such that $r
          = u \cdot v \Rightarrow r = u \cdot v \upl u \cdot 0 = u \cdot (v
          \upl 0) \in \alpha \odot (\beta + \gamma)$ since $t \leqslant 0
          \Rightarrow x = r \upl t \leqslant r \Rightarrow x \leqslant r$ and
          from \ref{property to determine membership of a cut} and $r \in
          \alpha \odot (\beta \upl \gamma)$ we would have $x \in \alpha \odot
          (\beta \upl \gamma)$
          
          \item then there exists $u \in \alpha, v \in \beta, u' \in \alpha,
          v' \in \gamma$ with $r = u \cdot v \wedge t = u' \cdot v' \wedge 0 <
          u \wedge 0 < v \wedge 0 < u' \wedge 0 < v'$. We have now to consider
          the following possibilities
          \begin{enumerate}
            \item then $x = u \cdot v \upl u' \cdot v' = u \cdot v \upl u
            \cdot v' = u \cdot (v \upl v') \Rightarrowlim_{0 < u \wedge 0 < v
            \upl v'} x = u \cdot (v \upl v') \in \alpha \odot (\beta \upl
            \gamma) \Rightarrow x \in \alpha \odot (\beta \upl \gamma)$
            
            \item then as $0 < u' \wedge 0 < v \upl v'  \text{we have } u'
            \cdot (v \upl v') \in \alpha \odot (\beta \upl \gamma)$ and from
            $u < u' \Rightarrow u \cdot v < u' \cdot v' \Rightarrow x = u
            \cdot v \upl u' \cdot v' \leqslant u' \cdot v \noplus \upl u'
            \cdot v' = u' \cdot (v \upl v') \in \alpha \odot (\beta \upl
            \gamma) \Rightarrowlim_{\text{\ref{property to determine
            membership of a cut}}} x \in \alpha \odot (\beta \upl \gamma)$
            
            \item then as $0 < u \wedge 0 < v \upl v'$ we have $u \cdot (v
            \upl v') \in \alpha \odot (\beta \upl \gamma)$. From $u' < u$ we
            have $u' \cdot v' < u \cdot .v' \Rightarrow x = u \cdot v \upl u'
            \cdot v' < u \cdot v + u \cdot v' = u \cdot (v \upl v') \in \alpha
            \odot (\beta \upl \gamma) \Rightarrowlim_{\text{\ref{property to
            determine membership of a cut}}} x \in \alpha \odot (\beta +
            \gamma)$
          \end{enumerate}
        \end{enumerate}
      \end{enumerate}
      So we have $\alpha \odot \beta + \alpha \odot \gamma \subseteq \alpha
      \odot (\beta + \gamma) \Rightarrow \alpha \odot (\beta + \gamma) =
      \alpha \odot \beta \upl \alpha \odot \gamma$. So finally we have
      \begin{eqnarray*}
        \alpha \cdot (\beta \upl \gamma) & = & \alpha \odot (\beta \upl
        \gamma)\\
        & = & \alpha \odot \beta \upl \alpha \odot \gamma\\
        & = & \alpha \cdot \beta \upl \alpha \cdot \gamma
      \end{eqnarray*}
      \item $\alpha \in \mathbbm{R}_+, \beta \in \mathbbm{R}_+, \gamma \in
      \mathbbm{R}_-$ We have now two sub-cases
      \begin{enumerate}
        \item First as $\gamma \in \mathbbm{R}_-$ we have $\um \gamma \in
        \mathbbm{R}_+$ and thus we have using (4.a) that
        \begin{eqnarray*}
          \alpha \cdot (\beta \upl (\um \gamma)) & = & \alpha \cdot \beta \upl
          \alpha \cdot (- \gamma)\\
          & = & \alpha \cdot \beta \upl (- (\alpha \cdot \gamma))
        \end{eqnarray*}
        now note that a $\beta, \um \gamma \in \mathbbm{R}_+$ we have $0 \in
        \beta, 0 \in - \gamma \Rightarrow 0 = 0 \upl 0 \in \beta \upl (\um
        \gamma) \Rightarrow \beta \upl (- \gamma) \in \mathbbm{R}_+$, next as
        we have that
        \begin{eqnarray*}
          \beta \upl \beta & = & (\beta \upl (\um \gamma)) \upl (\beta \upl
          \gamma)
        \end{eqnarray*}
        we have then by multiplication both sides with $\alpha$
        \begin{eqnarray*}
          \alpha \cdot (\beta \upl \beta) & = & \alpha \cdot ((\beta \upl (\um
          \gamma)) \upl (\beta \upl \gamma))\\
          & \Rightarrowlim_{\beta \in \mathbbm{R}_+, \beta \upl \gamma \in
          \mathbbm{R}_+, \beta \upl \gamma \in \mathbbm{R}_+ \tmop{and} 4. a}
          & \\
          \alpha \cdot \beta \upl \alpha \cdot \beta & = & \alpha \cdot (\beta
          \upl (- \gamma)) \upl \alpha \cdot (\beta + \gamma)\\
          & \equallim_{\tmop{previous} \tmop{remark}} & \alpha \cdot \beta
          \upl (\um (\alpha \cdot \gamma)) \upl \alpha \cdot (\beta \upl
          \gamma)\\
          & \Rightarrow & \\
          \alpha \cdot \beta \upl \alpha \cdot \gamma & = & \alpha \cdot
          (\beta \upl \gamma)
        \end{eqnarray*}
        \item Then we have $\um (\beta \upl \gamma) = (\um \gamma) \upl (-
        \beta) \in \mathbbm{R}_+, - \gamma \in \mathbbm{R}_+, \um \beta \in
        R_-$ and thus
        \begin{eqnarray*}
          \alpha \cdot (\beta \upl \gamma) & = & - (\alpha \cdot (\um (\beta
          \upl \gamma)))\\
          & = & \um (\alpha \cdot ((\um \beta) \upl (\um \gamma)))\\
          & = & \um (\alpha \cdot ((\um \gamma) \upl (\um \beta)))\\
          & \equallim_{4. a.i} & - (\alpha \cdot (\um \gamma) \upl \alpha
          \cdot (- \beta))\\
          & = & - (- (\alpha \cdot \matheuler) + (- (\alpha \cdot \beta)))\\
          & = & \alpha \cdot \gamma + \alpha \cdot \beta\\
          & = & \alpha \cdot \beta + \alpha \cdot \gamma
        \end{eqnarray*}
      \end{enumerate}
      \item $\alpha \in \mathbbm{R}_+, \beta \in \mathbbm{R}_+, \gamma = 0$
      Here we have
      \begin{eqnarray*}
        \alpha \cdot (\beta + \gamma) & = & a \cdot (\beta \upl 0)\\
        & = & \alpha \cdot \beta\\
        & = & \alpha \cdot \beta \upl 0\\
        & = & \alpha \cdot \beta \upl \alpha \cdot 0\\
        & = & \alpha \cdot \beta \upl \alpha \cdot \gamma
      \end{eqnarray*}
      \item $\alpha \in \mathbbm{R}_+, \beta \in \mathbbm{R}_-, \gamma \in
      \mathbbm{R}_+$ Here we have
      \begin{eqnarray*}
        \alpha \cdot (\beta \upl \gamma) & = & \alpha \cdot (\gamma \upl
        \beta)\\
        & \equallim_{4. b} & \alpha \cdot \gamma \upl \alpha \cdot \beta\\
        & = & \alpha \cdot \beta \upl \alpha \cdot \gamma
      \end{eqnarray*}
      \item $\alpha \in \mathbbm{R}_+, \beta \in \mathbbm{R}_-, \gamma \in
      \mathbbm{R}_-$ then we have $\um \beta, \um \gamma \in \mathbbm{R}_+$
      and then we have
      \begin{eqnarray*}
        \alpha \cdot (\beta \upl \gamma) & = & \um (\alpha \cdot (\um (\beta
        \upl \gamma)))\\
        & = & \um (\alpha \cdot ((\um \beta) \upl (\um \gamma)))\\
        & \equallim_{4. a} & - (\alpha \cdot (- \beta) \upl \alpha \cdot (-
        \gamma))\\
        & = & - (- (\alpha \cdot \beta) + (\um (\alpha \cdot \gamma)))\\
        & = & \alpha \cdot \beta \upl \alpha \cdot \gamma
      \end{eqnarray*}
      \item $\alpha \in \mathbbm{R}_+, \beta \in \mathbbm{R}_-, \gamma = 0$ We
      have then
      \begin{eqnarray*}
        \alpha \cdot (\beta \upl \gamma) & = & \alpha \cdot \beta\\
        & = & \alpha \cdot \beta + 0\\
        & = & \alpha \cdot \beta \upl \alpha \cdot \gamma
      \end{eqnarray*}
      
      
      \item $\alpha \in \mathbbm{R}_+, \beta = 0, \gamma \in \mathbbm{R}_+$
      then we have
      \begin{eqnarray*}
        \alpha \cdot (\beta \upl \gamma) & = & \alpha \cdot \gamma\\
        & = & 0 \upl \alpha \cdot \gamma\\
        & = & \alpha \cdot \beta + \alpha \cdot \gamma
      \end{eqnarray*}
      \item $\alpha \in \mathbbm{R}_+, \beta = 0, \gamma \in \mathbbm{R}_-$
      then we have
      \begin{eqnarray*}
        \alpha \cdot (\beta \upl \gamma) & = & \alpha \cdot \gamma\\
        & = & 0 \upl \alpha \cdot \gamma\\
        & = & \alpha \cdot \beta + \alpha \cdot \gamma
      \end{eqnarray*}
      \item $\alpha \in \mathbbm{R}_+, \beta = 0, \gamma = 0$ \ then we have
      \begin{eqnarray*}
        \alpha \cdot (\beta \upl \gamma) & = & \alpha \cdot \gamma\\
        & = & 0 \upl \alpha \cdot \gamma\\
        & = & \alpha \cdot \beta + \alpha \cdot \gamma
      \end{eqnarray*}
      \item $\alpha \in \mathbbm{R}_-, \beta \in \mathbbm{R}_+, \gamma \in
      \mathbbm{R}_+$ then we have $- \alpha \in \mathbbm{R}_+$ and thus
      \begin{eqnarray*}
        \alpha \cdot (\beta \upl \gamma) & = & - ((- \alpha) \cdot (\beta \upl
        \gamma))\\
        & \equallim_{4. a} & - ((- \alpha) \cdot \beta + (\um \alpha) \cdot
        \gamma)\\
        & = & - (- (\alpha \cdot \beta) \upl (\um (\alpha \cdot \gamma)))\\
        & = & \alpha \cdot \beta \upl \alpha \cdot \gamma
      \end{eqnarray*}
      \item $\alpha \in \mathbbm{R}_-, \beta \in \mathbbm{R}_+, \gamma \in
      \mathbbm{R}_-$then we have $- \alpha \in \mathbbm{R}_+$ and thus
      \begin{eqnarray*}
        \alpha \cdot (\beta \upl \gamma) & = & - ((- \alpha) \cdot (\beta \upl
        \gamma))\\
        & \equallim_{4. b} & - ((- \alpha) \cdot \beta + (\um \alpha) \cdot
        \gamma)\\
        & = & - (- (\alpha \cdot \beta) \upl (\um (\alpha \cdot \gamma)))\\
        & = & \alpha \cdot \beta \upl \alpha \cdot \gamma
      \end{eqnarray*}
      \item $\alpha \in \mathbbm{R}_-, \beta \in \mathbbm{R}_+, \gamma = 0$
      then we have $- \alpha \in \mathbbm{R}_+$ and thus
      \begin{eqnarray*}
        \alpha \cdot (\beta \upl \gamma) & = & - ((- \alpha) \cdot (\beta \upl
        \gamma))\\
        & \equallim_{4. c} & - ((- \alpha) \cdot \beta + (\um \alpha) \cdot
        \gamma)\\
        & = & - (- (\alpha \cdot \beta) \upl (\um (\alpha \cdot \gamma)))\\
        & = & \alpha \cdot \beta \upl \alpha \cdot \gamma
      \end{eqnarray*}
      \item $\alpha \in \mathbbm{R}_-, \beta \in \mathbbm{R}_-, \gamma \in
      \mathbbm{R}_+$ then we have $- \alpha \in \mathbbm{R}_+$ and thus
      \begin{eqnarray*}
        \alpha \cdot (\beta \upl \gamma) & = & - ((- \alpha) \cdot (\beta \upl
        \gamma))\\
        & \equallim_{4. d} & - ((- \alpha) \cdot \beta + (\um \alpha) \cdot
        \gamma)\\
        & = & - (- (\alpha \cdot \beta) \upl (\um (\alpha \cdot \gamma)))\\
        & = & \alpha \cdot \beta \upl \alpha \cdot \gamma
      \end{eqnarray*}
      \item $\alpha \in \mathbbm{R}_-, \beta \in \mathbbm{R}_-, \gamma \in
      \mathbbm{R}_-$ then we have $- \alpha \in \mathbbm{R}_+$ and thus
      \begin{eqnarray*}
        \alpha \cdot (\beta \upl \gamma) & = & - ((- \alpha) \cdot (\beta \upl
        \gamma))\\
        & \equallim_{4. e} & - ((- \alpha) \cdot \beta + (\um \alpha) \cdot
        \gamma)\\
        & = & - (- (\alpha \cdot \beta) \upl (\um (\alpha \cdot \gamma)))\\
        & = & \alpha \cdot \beta \upl \alpha \cdot \gamma
      \end{eqnarray*}
      \item $\alpha \in \mathbbm{R}_-, \beta \in \mathbbm{R}_-, \gamma = 0$
      then we have $- \alpha \in \mathbbm{R}_+$ and thus
      \begin{eqnarray*}
        \alpha \cdot (\beta \upl \gamma) & = & - ((- \alpha) \cdot (\beta \upl
        \gamma))\\
        & \equallim_{4. f} & - ((- \alpha) \cdot \beta + (\um \alpha) \cdot
        \gamma)\\
        & = & - (- (\alpha \cdot \beta) \upl (\um (\alpha \cdot \gamma)))\\
        & = & \alpha \cdot \beta \upl \alpha \cdot \gamma
      \end{eqnarray*}
      \item $\alpha \in \mathbbm{R}_-, \beta = 0, \gamma \in \mathbbm{R}_+$
      then we have $- \alpha \in \mathbbm{R}_+$ and thus
      \begin{eqnarray*}
        \alpha \cdot (\beta \upl \gamma) & = & - ((- \alpha) \cdot (\beta \upl
        \gamma))\\
        & \equallim_{4. g} & - ((- \alpha) \cdot \beta + (\um \alpha) \cdot
        \gamma)\\
        & = & - (- (\alpha \cdot \beta) \upl (\um (\alpha \cdot \gamma)))\\
        & = & \alpha \cdot \beta \upl \alpha \cdot \gamma
      \end{eqnarray*}
      \item $\alpha \in \mathbbm{R}_-, \beta = 0, \gamma \in \mathbbm{R}_-$
      then we have $- \alpha \in \mathbbm{R}_+$ and thus
      \begin{eqnarray*}
        \alpha \cdot (\beta \upl \gamma) & = & - ((- \alpha) \cdot (\beta \upl
        \gamma))\\
        & \equallim_{4. h} & - ((- \alpha) \cdot \beta + (\um \alpha) \cdot
        \gamma)\\
        & = & - (- (\alpha \cdot \beta) \upl (\um (\alpha \cdot \gamma)))\\
        & = & \alpha \cdot \beta \upl \alpha \cdot \gamma
      \end{eqnarray*}
      \item $\alpha \in \mathbbm{R}_-, \beta = 0, \gamma = 0$ then we have $-
      \alpha \in \mathbbm{R}_+$ and thus
      \begin{eqnarray*}
        \alpha \cdot (\beta \upl \gamma) & = & - ((- \alpha) \cdot (\beta \upl
        \gamma))\\
        & \equallim_{4. i} & - ((- \alpha) \cdot \beta + (\um \alpha) \cdot
        \gamma)\\
        & = & - (- (\alpha \cdot \beta) \upl (\um (\alpha \cdot \gamma)))\\
        & = & \alpha \cdot \beta \upl \alpha \cdot \gamma
      \end{eqnarray*}
      \item $\alpha \in 0, \beta \in \mathbbm{R}_+, \gamma \in \mathbbm{R}_+$
      then we have
      \begin{eqnarray*}
        \alpha \cdot (\beta \upl \gamma) & = & 0\\
        & = & 0 \upl 0\\
        & = & \alpha \cdot \beta \upl \alpha \cdot \gamma
      \end{eqnarray*}
      \item $\alpha \in 0, \beta \in \mathbbm{R}_+, \gamma \in \mathbbm{R}_-$
      then we have
      \begin{eqnarray*}
        \alpha \cdot (\beta \upl \gamma) & = & 0\\
        & = & 0 \upl 0\\
        & = & \alpha \cdot \beta \upl \alpha \cdot \gamma
      \end{eqnarray*}
      \item $\alpha \in 0, \beta \in \mathbbm{R}_+, \gamma = 0$ then we have
      \begin{eqnarray*}
        \alpha \cdot (\beta \upl \gamma) & = & 0\\
        & = & 0 \upl 0\\
        & = & \alpha \cdot \beta \upl \alpha \cdot \gamma
      \end{eqnarray*}
      \item $\alpha \in 0, \beta \in \mathbbm{R}_-, \gamma \in \mathbbm{R}_+$
      then we have
      \begin{eqnarray*}
        \alpha \cdot (\beta \upl \gamma) & = & 0\\
        & = & 0 \upl 0\\
        & = & \alpha \cdot \beta \upl \alpha \cdot \gamma
      \end{eqnarray*}
      \item $\alpha \in 0, \beta \in \mathbbm{R}_-, \gamma \in \mathbbm{R}_-$
      then we have
      \begin{eqnarray*}
        \alpha \cdot (\beta \upl \gamma) & = & 0\\
        & = & 0 \upl 0\\
        & = & \alpha \cdot \beta \upl \alpha \cdot \gamma
      \end{eqnarray*}
      \item $\alpha \in 0, \beta \in \mathbbm{R}_-, \gamma = 0$ then we have
      \begin{eqnarray*}
        \alpha \cdot (\beta \upl \gamma) & = & 0\\
        & = & 0 \upl 0\\
        & = & \alpha \cdot \beta \upl \alpha \cdot \gamma
      \end{eqnarray*}
      \item $\alpha \in 0, \beta = 0, \gamma \in \mathbbm{R}_+$ then we have
      \begin{eqnarray*}
        \alpha \cdot (\beta \upl \gamma) & = & 0\\
        & = & 0 \upl 0\\
        & = & \alpha \cdot \beta \upl \alpha \cdot \gamma
      \end{eqnarray*}
      \item $\alpha \in 0, \beta = 0, \gamma \in \mathbbm{R}_-$ then we have
      \begin{eqnarray*}
        \alpha \cdot (\beta \upl \gamma) & = & 0\\
        & = & 0 \upl 0\\
        & = & \alpha \cdot \beta \upl \alpha \cdot \gamma
      \end{eqnarray*}
      \item $\alpha \in 0, \beta = 0, \gamma = 0$ then we have
      \begin{eqnarray*}
        \alpha \cdot (\beta \upl \gamma) & = & 0\\
        & = & 0 \upl 0\\
        & = & \alpha \cdot \beta \upl \alpha \cdot \gamma
      \end{eqnarray*}
    \end{enumerate}
    \item
    \begin{enumerate}
      \item $\alpha \in \mathbbm{R}_+, \beta \in \mathbbm{R}_+, \gamma \in
      \mathbbm{R}_+$ First take $x \in \alpha \cdot (\beta \cdot \gamma)$ then
      we have the following cases to consider
      \begin{enumerate}
        \item then $x \in \{ r \in \mathbbm{Q} | r \leqslant 0 \nobracket \}
        \Rightarrow x \in (\alpha \cdot \beta) \cdot \gamma$
        
        \item then there exists a $s \in \alpha, t \in (\beta \cdot \gamma)$
        with $x = s \cdot t$ and $0 < s, t$, as $0 < t \in \beta \cdot \gamma$
        we must have that there exists a $u \in \beta, v \in \gamma$ such that
        $t = u \cdot v$ and $0 < u, v$. So we have $x = s \cdot t = s \cdot (u
        \cdot v) \equallim_{\text{\ref{the rational numbers form a field}}} (s
        \cdot u) \cdot v$. But $s \cdot u \in \alpha \cdot \beta \Rightarrow x
        = (s \cdot u) \cdot v \in (\alpha \cdot \beta) \cdot \gamma
        \Rightarrow x \in (\alpha \cdot \beta) \cdot \gamma$
      \end{enumerate}
      it follows that $\alpha \cdot (\beta \cdot \gamma) \subseteq (\alpha
      \cdot \beta) \cdot \gamma$. Now if $x \in (\alpha \cdot \beta) \cdot
      \gamma \equallim_{\tmop{commutativity}} \gamma \cdot (\alpha \cdot
      \beta) \equallim_{\tmop{commutativity}} \gamma \cdot (\beta \cdot
      \alpha) \subseteq_{\tmop{we} \tmop{have} \tmop{just} \tmop{proved}
      \tmop{this}} (\gamma \cdot \beta) \cdot \alpha
      \equallim_{\tmop{commutativity} \tmop{twice}} \alpha \cdot (\beta \cdot
      \gamma) \Rightarrow x \in \alpha \cdot (\beta \cdot \gamma)$ proving
      that $(\alpha \cdot \beta) \cdot \gamma \subseteq \alpha \cdot (\beta
      \cdot \gamma)$ and thus finally $\alpha \cdot (\beta \cdot \gamma) =
      (\alpha \cdot \beta) \cdot \gamma$
      
      \item $\alpha \in \mathbbm{R}_+, \beta \in \mathbbm{R}_+, \gamma \in
      \mathbbm{R}_-$ in this case we have $- \gamma \in \mathbbm{R}_+$
      \begin{eqnarray*}
        \alpha \cdot (\beta \cdot \gamma) & = & \alpha \cdot (\um (\beta \cdot
        (- \gamma)))\\
        & = & - (\alpha \cdot (\beta \cdot (- \gamma)))\\
        & \equallim_{5. a} & - ((\alpha \cdot \beta) \cdot (- \gamma))\\
        & = & (\alpha \cdot \beta) \cdot \gamma
      \end{eqnarray*}
      \item $\alpha \in \mathbbm{R}_+, \beta \in \mathbbm{R}_+, \gamma = 0$ in
      this case we have
      \begin{eqnarray*}
        \alpha \cdot (\beta \cdot \gamma) & = & \alpha \cdot (\beta \cdot 0)\\
        & = & \alpha \cdot 0\\
        & = & 0\\
        & = & (\alpha \cdot \beta) \cdot 0\\
        & = & (\alpha \cdot \beta) \cdot \gamma
      \end{eqnarray*}
      \item $\alpha \in \mathbbm{R}_+, \beta \in \mathbbm{R}_-, \gamma \in
      \mathbbm{R}_+$ we have then $\um \beta \in \mathbbm{R}_+$ and
      \begin{eqnarray*}
        \alpha \cdot (\beta \cdot \gamma) & = & \alpha \cdot (- ((- \beta)
        \cdot \gamma))\\
        & = & - (\alpha \cdot ((- \beta) \cdot \gamma))\\
        & \equallim_{5. a} & - ((\alpha \cdot (- \beta)) \cdot \gamma)\\
        & = & - ((\um (\alpha \cdot \beta)) \cdot \gamma)\\
        & = & (\alpha \cdot \beta) \cdot \gamma
      \end{eqnarray*}
      \item $\alpha \in \mathbbm{R}_+, \beta \in \mathbbm{R}_-, \gamma \in
      \mathbbm{R}_-$ we have then $- \beta, - \gamma \in \mathbbm{R}_+$
      \begin{eqnarray*}
        \alpha \cdot (\beta \cdot \gamma) & = & \alpha \cdot (- ((- \beta)
        \cdot \gamma))\\
        & = & \alpha \cdot ((- \beta) \cdot (- \gamma))\\
        & = & (\alpha \cdot (- \beta)) \cdot (- \gamma)\\
        & = & - ((\alpha \cdot (- \beta)) \cdot \gamma)\\
        & = & - ((\um (\alpha \cdot \beta)) \cdot \gamma)\\
        & = & (\alpha \cdot \beta) \cdot \gamma
      \end{eqnarray*}
      \item $\alpha \in \mathbbm{R}_+, \beta \in \mathbbm{R}_-, \gamma = 0$
      here we have
      \begin{eqnarray*}
        \alpha \cdot (\beta \cdot \gamma) & = & \alpha \cdot (\beta \cdot 0)\\
        & = & \alpha \cdot 0\\
        & = & 0\\
        & = & (\alpha \cdot \beta) \cdot 0\\
        & = & (\alpha \cdot \beta) \cdot \gamma
      \end{eqnarray*}
      \item $\alpha \in \mathbbm{R}_+, \beta = 0, \gamma \in \mathbbm{R}_+$ we
      have now
      \begin{eqnarray*}
        \alpha \cdot (\beta \cdot \gamma) & = & \alpha \cdot (0 \cdot
        \gamma)\\
        & = & \alpha \cdot 0\\
        & = & 0\\
        & = & 0 \cdot \gamma\\
        & = & (\alpha \cdot 0) \cdot \gamma\\
        & = & (\alpha \cdot \beta) \cdot \gamma
      \end{eqnarray*}
      \item $\alpha \in \mathbbm{R}_+, \beta = 0, \gamma \in \mathbbm{R}_-$ we
      have now
      \begin{eqnarray*}
        \alpha \cdot (\beta \cdot \gamma) & = & \alpha \cdot (0 \cdot
        \gamma)\\
        & = & \alpha \cdot 0\\
        & = & 0\\
        & = & 0 \cdot \gamma\\
        & = & (\alpha \cdot 0) \cdot \gamma\\
        & = & (\alpha \cdot \beta) \cdot \gamma
      \end{eqnarray*}
      \item $\alpha \in \mathbbm{R}_+, \beta = 0, \gamma = 0$ we have now
      \begin{eqnarray*}
        \alpha \cdot (\beta \cdot \gamma) & = & \alpha \cdot (0 \cdot
        \gamma)\\
        & = & \alpha \cdot 0\\
        & = & 0\\
        & = & 0 \cdot \gamma\\
        & = & (\alpha \cdot 0) \cdot \gamma\\
        & = & (\alpha \cdot \beta) \cdot \gamma
      \end{eqnarray*}
      \item $\alpha \in \mathbbm{R}_-, \beta \in \mathbbm{R}_+, \gamma \in
      \mathbbm{R}_+$ here $\um \alpha \in \mathbbm{R}_+$ and
      \begin{eqnarray*}
        \alpha \cdot (\beta \cdot \gamma) & = & - ((- \alpha) \cdot (\beta
        \cdot \gamma))\\
        & \equallim_{5. a} & - (((- \alpha) \cdot \beta) \cdot \gamma)\\
        & = & - ((- (\alpha \cdot \beta)) \cdot \gamma)\\
        & = & (\alpha \cdot \beta) \cdot \gamma
      \end{eqnarray*}
      \item $\alpha \in \mathbbm{R}_-, \beta \in \mathbbm{R}_+, \gamma \in
      \mathbbm{R}_-$ here $\um \alpha \in \mathbbm{R}_+$ and
      \begin{eqnarray*}
        \alpha \cdot (\beta \cdot \gamma) & = & - ((- \alpha) \cdot (\beta
        \cdot \gamma))\\
        & \equallim_{5. b} & - (((- \alpha) \cdot \beta) \cdot \gamma)\\
        & = & - ((- (\alpha \cdot \beta)) \cdot \gamma)\\
        & = & (\alpha \cdot \beta) \cdot \gamma
      \end{eqnarray*}
      \item $\alpha \in \mathbbm{R}_-, \beta \in \mathbbm{R}_+, \gamma = 0$
      here $\um \alpha \in \mathbbm{R}_+$ and
      \begin{eqnarray*}
        \alpha \cdot (\beta \cdot \gamma) & = & - ((- \alpha) \cdot (\beta
        \cdot \gamma))\\
        & \equallim_{5. c} & - (((- \alpha) \cdot \beta) \cdot \gamma)\\
        & = & - ((- (\alpha \cdot \beta)) \cdot \gamma)\\
        & = & (\alpha \cdot \beta) \cdot \gamma
      \end{eqnarray*}
      \item $\alpha \in \mathbbm{R}_-, \beta \in \mathbbm{R}_-, \gamma \in
      \mathbbm{R}_+$ here $\um \alpha \in \mathbbm{R}_+$ and
      \begin{eqnarray*}
        \alpha \cdot (\beta \cdot \gamma) & = & - ((- \alpha) \cdot (\beta
        \cdot \gamma))\\
        & \equallim_{5. d} & - (((- \alpha) \cdot \beta) \cdot \gamma)\\
        & = & - ((- (\alpha \cdot \beta)) \cdot \gamma)\\
        & = & (\alpha \cdot \beta) \cdot \gamma
      \end{eqnarray*}
      \item $\alpha \in \mathbbm{R}_-, \beta \in \mathbbm{R}_-, \gamma \in
      \mathbbm{R}_-$ here $\um \alpha \in \mathbbm{R}_+$ and
      \begin{eqnarray*}
        \alpha \cdot (\beta \cdot \gamma) & = & - ((- \alpha) \cdot (\beta
        \cdot \gamma))\\
        & \equallim_{5. e} & - (((- \alpha) \cdot \beta) \cdot \gamma)\\
        & = & - ((- (\alpha \cdot \beta)) \cdot \gamma)\\
        & = & (\alpha \cdot \beta) \cdot \gamma
      \end{eqnarray*}
      \item $\alpha \in \mathbbm{R}_-, \beta \in \mathbbm{R}_-, \gamma = 0$
      here $\um \alpha \in \mathbbm{R}_+$ and
      \begin{eqnarray*}
        \alpha \cdot (\beta \cdot \gamma) & = & - ((- \alpha) \cdot (\beta
        \cdot \gamma))\\
        & \equallim_{5. f} & - (((- \alpha) \cdot \beta) \cdot \gamma)\\
        & = & - ((- (\alpha \cdot \beta)) \cdot \gamma)\\
        & = & (\alpha \cdot \beta) \cdot \gamma
      \end{eqnarray*}
      \item $\alpha \in \mathbbm{R}_-, \beta = 0, \gamma \in \mathbbm{R}_+$
      here $\um \alpha \in \mathbbm{R}_+$ and
      \begin{eqnarray*}
        \alpha \cdot (\beta \cdot \gamma) & = & - ((- \alpha) \cdot (\beta
        \cdot \gamma))\\
        & \equallim_{5. g} & - (((- \alpha) \cdot \beta) \cdot \gamma)\\
        & = & - ((- (\alpha \cdot \beta)) \cdot \gamma)\\
        & = & (\alpha \cdot \beta) \cdot \gamma
      \end{eqnarray*}
      \item $\alpha \in \mathbbm{R}_-, \beta = 0, \gamma \in \mathbbm{R}_-$
      here $\um \alpha \in \mathbbm{R}_+$ and
      \begin{eqnarray*}
        \alpha \cdot (\beta \cdot \gamma) & = & - ((- \alpha) \cdot (\beta
        \cdot \gamma))\\
        & \equallim_{5. h} & - (((- \alpha) \cdot \beta) \cdot \gamma)\\
        & = & - ((- (\alpha \cdot \beta)) \cdot \gamma)\\
        & = & (\alpha \cdot \beta) \cdot \gamma
      \end{eqnarray*}
      \item $\alpha \in \mathbbm{R}_-, \beta = 0, \gamma = 0$ here $\um \alpha
      \in \mathbbm{R}_+$ and
      \begin{eqnarray*}
        \alpha \cdot (\beta \cdot \gamma) & = & - ((- \alpha) \cdot (\beta
        \cdot \gamma))\\
        & \equallim_{5. i} & - (((- \alpha) \cdot \beta) \cdot \gamma)\\
        & = & - ((- (\alpha \cdot \beta)) \cdot \gamma)\\
        & = & (\alpha \cdot \beta) \cdot \gamma
      \end{eqnarray*}
      \item $\alpha \in 0, \beta \in \mathbbm{R}_+, \gamma \in \mathbbm{R}_+$
      in this case we have
      \begin{eqnarray*}
        \alpha \cdot (\beta \cdot \gamma) & = & 0 \cdot (\beta \cdot \gamma)\\
        & = & 0\\
        & = & 0 \cdot \gamma\\
        & = & (0 \cdot \beta) \cdot \gamma\\
        & = & (\alpha \cdot \beta) \cdot \gamma
      \end{eqnarray*}
      \item $\alpha \in 0, \beta \in \mathbbm{R}_+, \gamma \in \mathbbm{R}_-$
      in this case we have
      \begin{eqnarray*}
        \alpha \cdot (\beta \cdot \gamma) & = & 0 \cdot (\beta \cdot \gamma)\\
        & = & 0\\
        & = & 0 \cdot \gamma\\
        & = & (0 \cdot \beta) \cdot \gamma\\
        & = & (\alpha \cdot \beta) \cdot \gamma
      \end{eqnarray*}
      \item $\alpha \in 0, \beta \in \mathbbm{R}_+, \gamma = 0$ in this case
      we have
      \begin{eqnarray*}
        \alpha \cdot (\beta \cdot \gamma) & = & 0 \cdot (\beta \cdot \gamma)\\
        & = & 0\\
        & = & 0 \cdot \gamma\\
        & = & (0 \cdot \beta) \cdot \gamma\\
        & = & (\alpha \cdot \beta) \cdot \gamma
      \end{eqnarray*}
      \item $\alpha \in 0, \beta \in \mathbbm{R}_-, \gamma \in \mathbbm{R}_+$
      in this case we have
      \begin{eqnarray*}
        \alpha \cdot (\beta \cdot \gamma) & = & 0 \cdot (\beta \cdot \gamma)\\
        & = & 0\\
        & = & 0 \cdot \gamma\\
        & = & (0 \cdot \beta) \cdot \gamma\\
        & = & (\alpha \cdot \beta) \cdot \gamma
      \end{eqnarray*}
      \item $\alpha \in 0, \beta \in \mathbbm{R}_-, \gamma \in \mathbbm{R}_-$
      in this case we have
      \begin{eqnarray*}
        \alpha \cdot (\beta \cdot \gamma) & = & 0 \cdot (\beta \cdot \gamma)\\
        & = & 0\\
        & = & 0 \cdot \gamma\\
        & = & (0 \cdot \beta) \cdot \gamma\\
        & = & (\alpha \cdot \beta) \cdot \gamma
      \end{eqnarray*}
      \item $\alpha \in 0, \beta \in \mathbbm{R}_-, \gamma = 0$ in this case
      we have
      \begin{eqnarray*}
        \alpha \cdot (\beta \cdot \gamma) & = & 0 \cdot (\beta \cdot \gamma)\\
        & = & 0\\
        & = & 0 \cdot \gamma\\
        & = & (0 \cdot \beta) \cdot \gamma\\
        & = & (\alpha \cdot \beta) \cdot \gamma
      \end{eqnarray*}
      \item $\alpha \in 0, \beta = 0, \gamma \in \mathbbm{R}_+$ in this case
      we have
      \begin{eqnarray*}
        \alpha \cdot (\beta \cdot \gamma) & = & 0 \cdot (\beta \cdot \gamma)\\
        & = & 0\\
        & = & 0 \cdot \gamma\\
        & = & (0 \cdot \beta) \cdot \gamma\\
        & = & (\alpha \cdot \beta) \cdot \gamma
      \end{eqnarray*}
      \item $\alpha \in 0, \beta = 0, \gamma \in \mathbbm{R}_-$ in this case
      we have
      \begin{eqnarray*}
        \alpha \cdot (\beta \cdot \gamma) & = & 0 \cdot (\beta \cdot \gamma)\\
        & = & 0\\
        & = & 0 \cdot \gamma\\
        & = & (0 \cdot \beta) \cdot \gamma\\
        & = & (\alpha \cdot \beta) \cdot \gamma
      \end{eqnarray*}
      \item $\alpha \in 0, \beta = 0, \gamma = 0$ in this case we have
      \begin{eqnarray*}
        \alpha \cdot (\beta \cdot \gamma) & = & 0 \cdot (\beta \cdot \gamma)\\
        & = & 0\\
        & = & 0 \cdot \gamma\\
        & = & (0 \cdot \beta) \cdot \gamma\\
        & = & (\alpha \cdot \beta) \cdot \gamma
      \end{eqnarray*}
    \end{enumerate}
  \end{enumerate}
\end{proof}

\begin{notation}
  If $x, y \in \mathbbm{R}, x \neq 0$ then we note
  \begin{enumerate}
    \item $x^{- 1} \equallim_{\tmop{notation}} \frac{1}{x}$
    
    \item $y \cdot x^{- 1} = \frac{y}{x}$
  \end{enumerate}
\end{notation}

\begin{theorem}
  \label{sum and product of rational cuts}$\forall r, s \in \mathbbm{Q}$ we
  have
  \begin{enumerate}
    \item $\alpha_r \upl \alpha_s = \alpha_{r \upl s}$
    
    \item $\alpha_r \cdot \alpha_s = \alpha_{r \cdot s}$
    
    \item if $\alpha_r \neq 0$ then $(\alpha_r)^{- 1} = \alpha_{r^{- 1}}$
  \end{enumerate}
\end{theorem}

\begin{proof}[$\alpha_r \in \mathbbm{R}_+, \alpha_s \in \mathbbm{R}_+$][$x
\leqslant 0$][$0 < x$][$x \leqslant 0$][$0 < x$][$\alpha_r \in \mathbbm{R}_+,
\alpha_s \in \mathbbm{R}_-$][$\alpha_r \in \mathbbm{R}_+, \alpha_s =
0$][$\alpha_r \in \mathbbm{R}_-, \alpha_s \in \mathbbm{R}_+$][$\alpha_r \in
\mathbbm{R}_-, \alpha_s \in \mathbbm{R}_-$][$\alpha_r \in \mathbbm{R}_-,
\alpha_s = 0$][$\alpha_r = 0, \alpha_s \in \mathbbm{R}_+$][$\alpha_r = 0,
\alpha_s \in \mathbbm{R}_-$][$\alpha_r = 0, \alpha_s = 0$]
  
  \begin{enumerate}
    \item If $x \in \alpha_r \upl \alpha_s$ then there exists $u \in \alpha_r,
    v \in \alpha_s$ with $x = u \upl v$. From $u \in \alpha_r, v \in \alpha_s$
    we have $u < r, v < s \Rightarrow u \upl v < r \upl v, r \upl v < r \upl s
    \Rightarrow u \upl v < r \upl s \Rightarrow x = \upsilon \upl v \in
    \alpha_{r \upl s} \Rightarrow \alpha_r \upl \alpha_s \subseteq \alpha_{r
    \upl s}$. If $x \in \alpha_{r \upl s} \Rightarrow x < r \upl s \Rightarrow
    x \um r < s$. Using the density theorem of $\mathbbm{Q}$ (see
    \ref{rational numbers are dense}) there exists a $z \in \mathbbm{Q}$ with
    $x \um r < z < s$. Then $z \in \alpha_s$ and $\varepsilon = z \um (x \um
    r) > 0 \Rightarrow \um \varepsilon < 0 \Rightarrow r \upl (- \varepsilon)
    < r \Rightarrow r \um \varepsilon \in \alpha_r$. As $\alpha_r \upl
    \alpha_s \ni r \um \varepsilon + z = r - (z \um (x \um r)) \upl z = r \um
    z \upl x \um r + z = x \Rightarrow x \in \alpha_r \upl \alpha_s
    \Rightarrow \alpha_{r \upl s} \subseteq \alpha_r \upl \alpha_s$. So we
    conclude that $\alpha_r \upl \alpha_s = \alpha_{r \upl s}$.
    
    \item We have the following possible cases for $\alpha_r, \alpha_s$ (see
    \ref{disjoint union of reals})
    \begin{enumerate}
      \item First as $0 \in \alpha_r, 0 \in \alpha_s \Rightarrow 0 < r \wedge
      0 < s$. Take $x \in \alpha_r \odot \alpha_s$ then we have the following
      possibilities:
      \begin{enumerate}
        \item as $0 < r, 0 < s$ we have $0 < r \cdot s \Rightarrow x \leqslant
        0 < r \cdot s \Rightarrow x \in \alpha_{r \cdot s}$
        
        \item in this case we have $x = u \cdot v$ where $u \in \alpha_r, v
        \in \alpha_s, 0 < u, 0 < v$. As $0 < u < r, 0 < v < s \Rightarrow u
        \cdot v < r \cdot v, r \cdot v < r \cdot s \Rightarrow u \cdot v < r
        \cdot s \Rightarrow x < r \cdot s \Rightarrow x \in \alpha_{r \cdot
        s}$
      \end{enumerate}
      we conclude thus that $\alpha_r \odot \alpha_s \subseteq \alpha_{r \cdot
      s}$. Take now $x \in \alpha_{r \cdot s} \Rightarrow x < r \cdot s$ we
      have then the following cases for $x$
      \begin{enumerate}
        \item then trivially $x \in \alpha_r \odot \alpha_s$
        
        \item then $0 < x$. As $0 < r$ we have by the density theorem (see
        \ref{rational numbers are dense}) that there exists a $\varepsilon_1
        \in \mathbbm{Q}$ such that $0 < \varepsilon_1 < r$. From $x < r \cdot
        s$ we have $0 < r \cdot s \um x \Rightarrowlim_{0 < s \Rightarrow 0 <
        s^{- 1}} 0 < (r \cdot s \um x) \cdot s^{- 1}$ and by the density
        theorem again we have that there exists a $\varepsilon_2 \in
        \mathbbm{Q}$ such that $0 < \varepsilon_2 < (r \cdot s \um x) \cdot
        s^{- 1}$. Take now $\varepsilon = \min (\varepsilon_1, \varepsilon_2)$
        then $0 < \varepsilon \leqslant \varepsilon_1 < r$ and $0 <
        \varepsilon \leqslant \varepsilon_2 < (r \cdot s \um x) \cdot s^{- 1}
        \Rightarrow - ((r \cdot s \um x) \cdot s^{- 1}) < - \varepsilon
        \Rightarrow r \um ((r \cdot s \um x) \cdot s^{- 1}) < r \um
        \varepsilon \Rightarrow x \cdot s^{- 1} < r \um \varepsilon
        \Rightarrowlim_{0 < x, 0 < s^{- 1} \Rightarrow 0 \in x \cdot s^{- 1}}
        0 < x \cdot s^{- 1} < r \um \varepsilon
        \Rightarrowlim_{\text{\ref{some properties of rational numbers}}} 0 <
        (r \um \varepsilon)^{- 1} < x^{- 1} \cdot s \Rightarrowlim_{0 < x} 0 <
        x \cdot (r \um \varepsilon)^{- 1} < s \Rightarrow 0 < x \cdot (r \um
        \varepsilon)^{- 1} \in \alpha_s$. As $\varepsilon < 0 \Rightarrow 0 <
        - \varepsilon$ and $\varepsilon < r$ we have $0 < r \um \varepsilon =
        r \upl (- \varepsilon) < r \Rightarrow 0 < r \um \varepsilon \in
        \alpha_r$ and thus $(r \um \varepsilon) (x \cdot (r \um
        \varepsilon)^{- 1}) \in \alpha_r \odot \alpha_s \Rightarrow x \in
        \alpha_r \odot \alpha_s$. We conclude thus that $\alpha_{r \cdot s}
        \subseteq \alpha_r \odot \alpha_s$.
      \end{enumerate}
      So we finally reach the conclusion that $\alpha_{r \cdot s} = \alpha_r
      \odot \alpha_s$. As $\alpha_r, \alpha_s \in \mathbbm{R}_+$ we have
      $\alpha_r \cdot \alpha_s = \alpha_r \odot \alpha_s = \alpha_{r \cdot s}
      \Rightarrow \alpha_r \cdot \alpha_s = \alpha_{r \cdot s}$
      
      \item Then $\alpha_{- s} \equallim_{\text{\ref{negative of rational
      cut}}} - \alpha_s \in \mathbbm{R}_+$ and $\alpha_r \cdot \alpha_s =
      \alpha_r \cdot (\um (- \alpha_s)) = - (\alpha_r \cdot (- \alpha_s)) = -
      (\alpha_r \cdot \alpha_{\um s}) \equallim_{(a)} \um \alpha_{r \cdot (-
      s)} = - \alpha_{- (r \cdot s)} \equallim_{\text{\ref{negative of
      rational cut}}} - (- (\alpha_{r \cdot s})) = \alpha_{r \cdot s}$
      
      \item $\Rightarrow \alpha_s = \alpha_0
      \Rightarrowlim_{\text{\ref{rational cuts}}} s = 0$ . Then \ $\alpha_r
      \cdot \alpha_s = \alpha_r \cdot 0 = 0 = \alpha_0 = \alpha_{r \cdot 0} =
      \alpha_{r \cdot s}$
      
      \item Then $\alpha_r \cdot \alpha_s = \alpha_s \cdot \alpha_r
      \equallim_{(b)} \alpha_{s \cdot r} = \alpha_{r \cdot s}$
      
      \item Then $\alpha_{- r} = \um \alpha_r \in \mathbbm{R}_+, \alpha_{- s}
      = - \alpha_s \in \mathbbm{R}_+$ and $\alpha_r \cdot \alpha_s = (- (-
      \alpha_r)) \cdot (- (- \alpha_s)) = (- \alpha_r) \cdot (- \alpha_s) =
      \alpha_{- r} \cdot \alpha_s \equallim_{(a)} \alpha_{(- r) \cdot (- s)} =
      \alpha_{r \cdot s}$
      
      \item $\Rightarrow \alpha_s = \alpha_0
      \Rightarrowlim_{\text{\ref{rational cuts}}} s = 0$ . Then \ $\alpha_r
      \cdot \alpha_s = \alpha_r \cdot 0 = 0 = \alpha_0 = \alpha_{r \cdot 0} =
      \alpha_{r \cdot s}$
      
      \item $\Rightarrow \alpha_r = \alpha_0 \Rightarrow r = 0$. Then
      $\alpha_r \cdot \alpha_s = 0 \cdot \alpha_s = 0 = \alpha_0 = \alpha_{0
      \cdot s} = \alpha_{r \cdot s}$
      
      \item $\Rightarrow \alpha_r = \alpha_0 \Rightarrow r = 0$. Then
      $\alpha_r \cdot \alpha_s = 0 \cdot \alpha_s = 0 = \alpha_0 = \alpha_{0
      \cdot s} = \alpha_{r \cdot s}$
      
      \item $\Rightarrow \alpha_r = \alpha_0 \Rightarrow r = 0$. Then
      $\alpha_r \cdot \alpha_s = 0 \cdot \alpha_s = 0 = \alpha_0 = \alpha_{0
      \cdot s} = \alpha_{r \cdot s}$
    \end{enumerate}
    \item This follows from \ref{inverse of element in Q in R}.
  \end{enumerate}
\end{proof}

\begin{theorem}
  \label{rationals in reals form a subfield}$\langle
  \mathbbm{Q}_{\mathbbm{R}}, \upl, \cdot \rangle$ forms a sub-field of
  $\langle \mathbbm{R}, \upl, \cdot \rangle$, further the function
  $i_{\mathbbm{Q}} : \mathbbm{Q} \rightarrow \mathbbm{Q}_{\mathbbm{R}}$
  defined by $r \rightarrow \alpha_r$ is a field isomorphism. So we can
  consider $\mathbbm{Q}_{\mathbbm{R}}$ as the set of rational numbers embedded
  in the set of reals.
\end{theorem}

\begin{proof}[injectivity][surjectivity]
  First we prove that $\langle \mathbbm{Q}_{\mathbbm{R}}, +, \cdot \rangle$ is
  a sub field
  \begin{enumerate}
    \item If $x, y \in \mathbbm{Q}_{\mathbbm{R}} \Rightarrow \exists r, s \in
    \mathbbm{Q} \vdash x = \alpha_r$ and $y = \alpha_s$ and thus $x \upl y =
    \alpha_r \upl \alpha_s \equallim_{\text{previous theorem}} \alpha_{r \upl
    s} \in \mathbbm{Q}_{\mathbbm{R}}$
    
    \item If $x, y \in \mathbbm{Q}_{\mathbbm{R}} \Rightarrow \exists r, s \in
    \mathbbm{Q} \vdash x = \alpha_r$ and $y = \alpha_s$ and thus $x \cdot y =
    \alpha_r \cdot \alpha_s \equallim_{\text{previous theorem}} \alpha_{r
    \cdot s} \in \mathbbm{Q}_{\mathbbm{R}}$
    
    \item If $x \in \mathbbm{Q}_{\mathbbm{R}} \backslash \{ 0 \}$ then
    $\exists r \in \mathbbm{Q}$ with $x = \alpha_r \neq 0 = \alpha_0
    \Rightarrow r \in \mathbbm{Q} \backslash \{ 0 \} \Rightarrow r^{- 1}$
    exists and if we take $x' = \alpha_{r^{- 1}}$ then we have $x' \cdot x
    \equallim_{\tmop{commutativity}} x \cdot x' = \alpha_r \cdot \alpha_{r^{-
    1}} = \alpha_{r \cdot r^{- 1}} = \alpha_1 = 1$, so $x^{- 1} = x' =
    \alpha_{r^{- 1}}^{} \in \mathbbm{Q}_{\mathbbm{R}}$
    
    \item $0 = \alpha_0 \in \mathbbm{Q}_{\mathbbm{R}}$
    
    \item $1 = \alpha_1 \in \mathbbm{Q}_{\mathbbm{R}}$
  \end{enumerate}
  Next we prove that $i_{\mathbbm{Q}}$ is a bijection
  \begin{enumerate}
    \item If $r, s \in \mathbbm{Q}$ is such that $i_{\mathbbm{Q}} (r) =
    i_{\mathbbm{Q}} (s) = \alpha_r = \alpha_s
    \Rightarrowlim_{\text{\ref{rational cuts}}} r = s$
    
    \item If $x \in \mathbbm{Q}_{\mathbbm{R}} \Rightarrow \exists r \in
    \mathbbm{Q} \vdash x = \alpha_r = i_{\mathbbm{Q}} (r) .$
  \end{enumerate}
  Finally we prove the homeomorphism properties
  \begin{enumerate}
    \item If $r, s \in \mathbbm{Q}_{\mathbbm{R}} \Rightarrow i_{\mathbbm{Q}}
    (r) \upl i_{\mathbbm{Q}} (s) = \alpha_r \upl \alpha_s
    \equallim_{\tmop{previous} \tmop{theorem}} \alpha_{r \upl s} =
    i_{\mathbbm{Q}} (r \upl s)$
    
    \item If $r, s \in \mathbbm{Q}_{\mathbbm{R}} \Rightarrow i_Q (r \cdot s) =
    \alpha_{r \cdot s} \equallim_{\tmop{previous} \tmop{theorem}} \alpha_r
    \cdot \alpha_s = i_{\mathbbm{Q}} (r) \cdot i_{\mathbbm{Q}} (s)$
    
    \item $i_{\mathbbm{Q}} (1) = \alpha_1 = 1$
  \end{enumerate}
\end{proof}

\subsection{Power in $\mathbbm{R}$}

\begin{definition}
  As $\langle \mathbbm{R}, \cdot \rangle$ is a abelian semi-group we have by
  \ref{iteration over a group} that given a $a \in \mathbbm{R}$ and $n \in
  \mathbbm{N}$ that there exists a $a^n$ such that
  \begin{eqnarray*}
    a^0 & = & 1\\
    a^{n \upl 1} & = & a^n \cdot a \equallim_{\tmop{abelian}} a \cdot a^n
  \end{eqnarray*}
\end{definition}

\begin{theorem}
  If $n, n' \in \mathbbm{N}$ and $a \in \mathbbm{R}$ then $a^{n' \upl n} =
  a^{n'} \cdot a^n$
\end{theorem}

\begin{proof}
  We prove this by induction on $n$. So let $X = \{ n \in \mathbbm{N}|a^{n'
  \upl n} = a^{n'} \cdot a^n \}$ then we have
  \begin{enumerate}
    \item If $n = 0$ then $a^{n' \upl n} = a^{n' \upl 0} = a^{n'} = a^{n'}
    \cdot 1 = a^{n'} \cdot a^0 \Rightarrow 0 \in X$
    
    \item If $n \in X$ then $a^{n' \upl (n \upl 1)} = a^{(n' \upl n) \upl 1} =
    a^{(n' \upl n)} \cdot a \equallim_{n \in X} (a^{n'} \cdot a^n) \cdot a =
    a^{n'} \cdot (a^n \cdot a) = a^{n'} \cdot a^{n \upl 1}$ and thus $n \upl 1
    \in X$
  \end{enumerate}
  Using mathematical induction (see \ref{mathematical induction}) we have $X
  =\mathbbm{N}$ proving the theorem
\end{proof}

\begin{theorem}
  \label{power in the whole numbers embedded in the reals}In $\mathbbm{R}$ we
  have
  \begin{eqnarray*}
    0^n & = & 0 (\tmop{if} n \neq 0)\\
    1^n & = & 1\\
    (- 1)^n & = & - 1 \tmop{or} 1\\
    (- 1)^{2 \cdot n} & = & 1\\
    (- 1)^{2 \cdot n \upl 1} & = & - 1
  \end{eqnarray*}
\end{theorem}

\begin{proof}
  
  \begin{enumerate}
    \item If $n \neq 0 \Rightarrow \exists m \in \mathbbm{N} \vdash n = m \upl
    1$ then $0^n = 0^{(m \upl 1)} = 0^m \cdot 0 = 0$
    
    \item $1^n = 1$ is proved by induction on $n$, let $X = \{ n \in
    \mathbbm{N}|1^n = 1 \}$ then
    \begin{enumerate}
      \item $1^0 = 1 \Rightarrow 0 \in X$
      
      \item If $n \in X \Rightarrow 1^{n \upl 1} = 1^n \cdot 1 \equallim_{n
      \in X} 1 \cdot 1 = 1 \Rightarrow n \upl 1 \in X$
    \end{enumerate}
    so $X =\mathbbm{N}$
    
    \item $(- 1)^n = \pm 1$ is proved by induction on $n$, let $X = \{ n \in
    \mathbbm{N}| (- 1)^n = - 1 \tmop{or} 1 \}$ then
    \begin{enumerate}
      \item $(- 1)^0 = 1 \Rightarrow 0 \in X$
      
      \item If $n \in X$ then $(- 1)^{n \upl 1} = (- 1)^n \cdot (- 1)
      \equallim_{n \in X} (- 1) \cdot (- 1) \vee 1 \cdot (- 1) = 1 \vee - 1
      \Rightarrow n \upl 1 \in X$
    \end{enumerate}
    so $X =\mathbbm{N}$
    
    \item $(- 1)^{2 \cdot n} = (- 1)^{(1 \upl 1) \cdot n} = (- 1)^{n \upl n} =
    (- 1)^n \cdot (- 1)^n \equallim_{(3)} (- 1) \cdot (- 1) \tmop{or} 1 \cdot
    1 = 1$
    
    \item $(- 1)^{2 \cdot n \upl 1} = (- 1)^{2 \cdot n} \cdot (- 1)
    \equallim_{(4)} 1 \cdot (- 1) = - 1$
  \end{enumerate}
\end{proof}

\section{Order relation on $\mathbbm{R}$}

\begin{theorem}
  \label{0<less>a,b=<less>0<less>a+b,a.b a,b reals}If $\alpha, \beta \in
  \mathbbm{R}_+$ then $\alpha \upl \beta \in \mathbbm{R}_+$ and $\alpha \cdot
  \beta \in \mathbbm{R}_+$
\end{theorem}

\begin{proof}
  If $\alpha, \beta \in \mathbbm{R}$ then $0 \in \alpha \wedge 0 \in \beta$
  then $0 = 0 \upl 0 \in \alpha \upl \beta \Rightarrow \alpha \upl \beta \in
  \mathbbm{R}_+$. Also $0 \in \{ r \in \mathbbm{Q} | r \leqslant 0 \nobracket
  \} \subseteq \alpha \odot \beta \equallim_{a, b \in \mathbbm{R}_+} \alpha
  \cdot \beta \Rightarrow \alpha \cdot \beta \in \mathbbm{R}_+$
\end{proof}

We define now relations $< \subseteq \mathbbm{R} \times \mathbbm{R}$ and
$\leqslant \subseteq \mathbbm{R} \times \mathbbm{R}$ as follows

\begin{definition}
  \label{strict order in the reals}{\index{$<$}}$< = \{ (a, \beta) \in
  \mathbbm{R} \times \mathbbm{R} | \beta \upl (\um \alpha) \in \mathbbm{R}_+
  \nobracket \}$, so $\alpha < \beta \Leftrightarrow (a, \beta) \in <
  \Leftrightarrow \beta \upl (\um \alpha) = \beta \um \alpha \in
  \mathbbm{R}_+$.
\end{definition}

\begin{definition}
  \label{order in the reals}{\index{$\leqslant$}}$\leqslant = \{ (\alpha,
  \beta) \in \mathbbm{R} \times \mathbbm{R} | \alpha = \beta \vee \beta \upl
  (\um \alpha) \in \mathbbm{R}_+ \nobracket \}$, so $\alpha \leqslant b
  \Leftrightarrow (\alpha, \beta) \in \leqslant \Leftrightarrow \alpha = \beta
  \vee \beta \upl (\um \alpha) \in \mathbbm{R}_+ \Leftrightarrow \alpha =
  \beta \vee \alpha < \beta$
\end{definition}

\begin{theorem}
  \label{equivalence of order and subset in the reals}$\forall \alpha, \beta
  \in \mathbbm{R}$ we have
  \begin{enumerate}
    \item $\alpha < \beta$ iff $\alpha \subset \beta$
    
    \item $\alpha \leqslant \beta$ iff $\alpha \subseteq \beta$
  \end{enumerate}
\end{theorem}

\begin{proof}[$\alpha < \beta \Rightarrow \alpha \subset \beta$][$\alpha
\subset \beta \Rightarrow \alpha < \beta$][$\alpha \leqslant \beta \Rightarrow
\alpha \subseteq \beta$][$\alpha < \beta$][{\alpha}={\beta}][$\alpha \subseteq
\beta \Rightarrow \alpha \leqslant \beta$][$\alpha = \beta$][$\alpha \subset
\beta$]
  Be care full in this prove $<$ can mean a relation in $\mathbbm{R}$ or in
  $\mathbbm{Q}$, context tells us which is which.
  \begin{enumerate}
    \item 
    \begin{enumerate}
      \item If $\alpha < \beta \Rightarrow \beta \upl (\um \alpha) \in
      \mathbbm{R}_+ \Rightarrow 0 \in \beta \upl (\um \alpha)$ . If $r \in
      \alpha \Rightarrowlim_{\ref{negative cut}} - r \nin - \alpha \Rightarrow
      - r \in \mathbbm{Q} \backslash (- \alpha)$, now as $0 \in \beta \upl
      (\um \alpha)$ there exists a $s \in \beta, t \in - \alpha$ such that $0
      = s \upl t \Rightarrow s = - t$. By \ref{dedekind's cut}, $t \in -
      \alpha$ and $- r \in \mathbbm{Q} \backslash (- \alpha)$ we have $t <
      (\um r) \Rightarrowlim_{\text{\ref{q<less>=r=<gtr>-r<less>=-q for
      rational numbers}}} r < (- t) = s \in \beta
      \Rightarrowlim_{\text{\ref{property to determine membership of a cut}}}
      r \in \beta$ and thus we have $\alpha \subseteq \beta$. If now $\alpha =
      \beta \Rightarrow \beta \upl (- \alpha) = 0 = \alpha_0 \Rightarrow 0 \in
      \beta \upl (- \alpha) = \alpha_0 \Rightarrow 0 < 0$ a contradiction, so
      we must have $\alpha \neq \beta \Rightarrow \alpha \subset \beta$.
      
      \item If $\alpha \subset \beta \Rightarrow \exists r \in \beta$ with $r
      \nin \alpha \Rightarrow r \in \mathbbm{Q} \backslash \alpha$ as $\max
      (\beta)$ does not exists there exists a $r' \in \beta$with $r < r'$. If
      $r' \in \alpha$ then using \ref{dedekind's cut} we have $r' < r$
      contradicting $r < r'$ and thus we must have $r' \nin \alpha \Rightarrow
      r' \in \mathbbm{Q} \backslash \alpha$. Hence $r, r' \in \mathbbm{Q}
      \backslash \alpha$ and $r < r' \Rightarrow r' \neq \min (\mathbbm{Q}
      \backslash \alpha) \Rightarrow - r' \in - \alpha$. We have then as $r'
      \in \beta$, $- r' \in - \alpha$ and $0 = r' \upl (- r') \in \beta \upl
      (- \alpha) \Rightarrow \beta \upl (- \alpha) \in \mathbbm{R}_+
      \Rightarrow \alpha < \beta$
    \end{enumerate}
    \item
    \begin{enumerate}
      \item As $a \leqslant b$ we have the following two cases
      \begin{enumerate}
        \item Using 1.a we have then $\alpha \subset \beta \Rightarrow \alpha
        \subseteq \beta$
        
        \item $\Rightarrow \alpha \subseteq \beta$
      \end{enumerate}
      \item As $\alpha \subseteq \beta$ we have the following two exclusive
      cases to consider
      \begin{enumerate}
        \item then $\alpha = \beta \Rightarrow \alpha \leqslant \beta$
        
        \item then using 1.b we have $\alpha < \beta \Rightarrow \alpha
        \leqslant \beta$
      \end{enumerate}
    \end{enumerate}
  \end{enumerate}
\end{proof}

\begin{theorem}
  \label{the set of reals is fully-ordered}$\langle \mathbbm{R}, \leqslant
  \rangle$ forms a partially ordered set that is fully-ordered
\end{theorem}

\begin{proof}[Reflexivity][Anti-symmetry][Transitivity][Fully-ordered][$\alpha
\upl (- \beta) = 0$][$\alpha \upl (\um \beta) \in \mathbbm{R}_+$][$\alpha \upl
(- \beta) \in \mathbbm{R}_-$]
  
  \begin{enumerate}
    \item If $\alpha \in \mathbbm{R} \Rightarrow \alpha \subseteq \alpha
    \Rightarrowlim_{\text{\ref{equivalence of order and subset in the reals}}}
    \alpha \leqslant \alpha_{\text{}}$
    
    \item If $\alpha, \beta \in \mathbbm{R}$ with $\alpha \leqslant \beta$ and
    $\beta \leqslant \alpha$ then by \ref{equivalence of order and subset in
    the reals} we have $\alpha \subseteq \beta$ and $\beta \subseteq \alpha$
    so $\alpha = \beta$
    
    \item If $\alpha, \beta, \gamma \in \mathbbm{R}$ with $\alpha \leqslant
    \beta$ and $\beta \leqslant \gamma$ then by \ref{equivalence of order and
    subset in the reals} we have $\alpha \subseteq \beta$ and $\beta \subseteq
    \gamma \Rightarrow \alpha \subseteq \gamma$ and by \ref{equivalence of
    order and subset in the reals} again we have $\alpha \leqslant \gamma$
    
    \item If $\alpha, \beta \in \mathbbm{R}$ then by \ref{disjoint union of
    reals} we have for $\alpha \upl (- \beta)$ either
    \begin{enumerate}
      \item From this we have $\alpha = \beta \Rightarrow \alpha \subseteq
      \beta \Rightarrowlim_{\text{\ref{equivalence of order and subset in the
      reals}}} \alpha \leqslant \beta$
      
      \item From this we have by definition $\beta \leqslant \alpha$
      
      \item In this case we have $\um (\alpha \upl (- \beta)) \in
      \mathbbm{R}_+ \Rightarrow (- \alpha) \upl \beta \in \mathbbm{R}_+
      \Rightarrow \beta \upl (- \alpha) \in \mathbbm{R}_+$ and thus $\alpha
      \leqslant \beta$
    \end{enumerate}
  \end{enumerate}
\end{proof}

\begin{theorem}[$x^{- 1} = 0$][$x^{- 1} < 0$][$0 < x^{- 1}$][$x = 0$][$x <
0$][$0 < x$][$x = 0$][$0 < x$]
  \label{properties of positive, negative real numbers}We have the following
  for the set of reals
  \begin{enumerate}
    \item $\alpha \in \mathbbm{R}_+$ iff $0 < \alpha$
    
    \item $\alpha \in \mathbbm{R}_-$ iff $\alpha < 0$
    
    \item If $\alpha, \beta \in \mathbbm{R}$ then we have the following
    exclusive possibilities
    \begin{enumerate}
      \item $\alpha < \beta$
      
      \item $\beta < \alpha$
      
      \item $\alpha = \beta$
    \end{enumerate}
    \item If $\alpha, \beta \in \mathbbm{R}$ with $\alpha < \beta \Rightarrow
    - \beta < \um \alpha$
    
    \item If $\alpha, \beta, \gamma \in \mathbbm{R}$ with $\alpha < \beta
    \Rightarrow \alpha \upl \gamma < \beta \upl \gamma$
    
    \item If $\alpha, \beta \in \mathbbm{R}$ and $\alpha < \beta$ with $\gamma
    \in \mathbbm{R}_+$ then $\alpha \cdot \gamma < \beta \cdot \gamma$
    
    \item If $\alpha, \beta \in \mathbbm{R}$ and $\alpha < \beta$ with $\gamma
    \in \mathbbm{R}_-$ then $\beta \cdot \gamma < \alpha \cdot \gamma$
    
    \item If $0 < x \Rightarrow 0 < x^{- 1}$
    
    \item If $0 < x < y \Rightarrow y^{- 1} < x^{- 1}$
    
    \item If $x \in \mathbbm{R} \Rightarrow 0 \leqslant x^2$ and if $x \neq 0$
    then $0 < x^2$
    
    \item If $x, y \in \mathbbm{R}_+ \bigcup \{ 0 \}$ are such that $x < y$
    then $x^2 < y^2$ (so $^2 : \{ x \in \mathbbm{R}|0 \leqslant x \}
    \rightarrow \{ x \in \mathbbm{R}|0 \leqslant x \}$ is a strictly
    increasing function)
    
    \item If $\alpha \in \mathbbm{R}_+$ and $n \in \mathbbm{N}$ then $\alpha^n
    \in \mathbbm{R}_+$
    
    \item If $\alpha \in \mathbbm{R}_+$ so that $0 < \alpha < 1$ then if $n
    \in \mathbbm{N}_0$ we have $0 < \alpha^n < \alpha$
  \end{enumerate}
  \begin{proof}
    
    \begin{enumerate}
      \item If $\alpha \in \mathbbm{R}_+$ then $\alpha \upl (\um 0) = \alpha
      \in \mathbbm{R}_+ \Rightarrow 0 < \alpha$. On the other hand if $0 <
      \alpha$ then $\mathbbm{R}_+ \ni \alpha \upl (- 0) = \alpha \Rightarrow
      \alpha \in \mathbbm{R}_+$.
      
      \item If $\alpha \in \mathbbm{R}_-$ then $\um \alpha \in \mathbbm{R}_+
      \Rightarrow 0 \upl (- \alpha) \in \mathbbm{R}_+ \Rightarrow \alpha < 0$.
      If $\alpha < 0 \Rightarrow 0 \upl (- \alpha) \in \mathbbm{R}_+
      \Rightarrow - \alpha \in \mathbbm{R}_+ \Rightarrow \alpha \in
      \mathbbm{R}_-$
      
      \item If $\alpha \comma \beta \in \mathbbm{R}$ then we have by
      \ref{disjoint union of reals} the following exclusive possible cases for
      $\alpha \upl (- \beta)$
      \begin{enumerate}
        \item $\alpha \upl (- \beta) \in \mathbbm{R}_+ \Leftrightarrow \beta <
        \alpha$
        
        \item $\alpha \upl (- \beta) \in \mathbbm{R}_- \Leftrightarrow -
        (\alpha \upl (- \beta)) \in \mathbbm{R}_+ \Leftrightarrow \beta \upl
        (- \alpha) \in \mathbbm{R}_+ \Leftrightarrow \alpha < \beta$
        
        \item $\alpha \upl (- \beta) = 0 \Leftrightarrow \alpha = \beta$
      \end{enumerate}
      \item If $\alpha, \beta \in \mathbbm{R}$ with $\alpha < \beta$ then
      $\beta \upl (- \alpha) \in \mathbbm{R}_+ \Rightarrow (- \alpha) \upl (-
      (- \beta)) \in \mathbbm{R}_+ \Rightarrow - \beta < - \alpha$
      
      \item If $\alpha, \beta, \gamma \in \mathbbm{R}$ with $\alpha < \beta$
      then $\beta \upl (- \alpha) \in \mathbbm{R}_+ \Rightarrow \beta \upl
      \gamma + (- \gamma) \upl (- \alpha) \in \mathbbm{R}_+ \Rightarrow (\beta
      \upl \gamma) \upl (\um (\alpha \upl \gamma)) \in \mathbbm{R}_+
      \Rightarrow \alpha \upl \gamma < \beta \upl \gamma$
      
      \item If $\alpha, \beta \in \mathbbm{R}$ with $\alpha < \beta$ and
      $\gamma \in \mathbbm{R}_+$ then $\beta \upl (- \alpha) \in
      \mathbbm{R}_+$ then using \ref{0<less>a,b=<less>0<less>a+b,a.b a,b
      reals} we have $(\beta \upl (- \alpha)) \cdot \gamma \in \mathbbm{R}_+
      \Rightarrow \beta \cdot \gamma \upl (- \alpha) \cdot \gamma \in
      \mathbbm{R}_+ \Rightarrow (\beta \cdot \gamma) \upl (\um (\alpha \cdot
      \gamma)) \in \mathbbm{R}_+ \Rightarrow \alpha \cdot \gamma < \beta \cdot
      \gamma$
      
      \item If $\alpha, \beta \in \mathbbm{R}$ with $\alpha < \beta$ and
      $\gamma \in \mathbbm{R}_-$ then $\beta \upl (- \alpha) \in
      \mathbbm{R}_+$ and $- \gamma \in \mathbbm{R}_+$ then using
      \ref{0<less>a,b=<less>0<less>a+b,a.b a,b reals} we have $\mathbbm{R}_+
      \ni (\beta \upl (\um \alpha)) \cdot (\um \gamma) = \um ((\beta \upl (-
      \alpha)) \cdot \gamma) = - (\beta \cdot \gamma \upl (- (\alpha \cdot
      \gamma))) = (\alpha \cdot \gamma) \upl (- (\beta \cdot \gamma))
      \Rightarrow \beta \cdot \gamma < \alpha \cdot \gamma$
      
      \item Let $0 < x$ consider then the following possible cases for $x^{-
      1}$ are
      \begin{enumerate}
        \item then $1 = x \cdot x^{- 1} = 0$ contradicting $1 \neq 0$
        
        \item then using (6) $1 = x \cdot x^{- 1} < 0 \cdot x = 0 \Rightarrow
        1 < 0$ contradicting $0 < 1$
        
        \item this is the only case left and proves thus (8)
      \end{enumerate}
      \item If $0 < x < y \Rightarrowlim_{(8)} 0 < x^{- 1}, y^{- 1}$ and using
      (6) we have $1 = x \cdot x^{- 1} < y \cdot x^{- 1} \Rightarrow 1 < y
      \cdot x^{- 1} \Rightarrow y^{- 1} = 1 \cdot y^{- 1} < (y \cdot x^{- 1})
      \cdot y^{- 1} = x^{- 1} \Rightarrow y^{- 1} < x^{- 1}$
      
      \item If $x \in \mathbbm{R}$ then we have
      \begin{enumerate}
        \item $\Rightarrow x^2 = x \cdot x = 0 \cdot 0 = 0 \Rightarrow 0
        \leqslant x^2$
        
        \item $\Rightarrow - 0 < - x \Rightarrow 0 < - x \Rightarrow 0 < (- x)
        \cdot (- x) = (- 1)^2 (x \cdot x) = x^2 \Rightarrow 0 < x^2$
        
        \item $\Rightarrow 0 \cdot x < x \cdot x \Rightarrow 0 < x^2$
      \end{enumerate}
      \item If $x, y \in \mathbbm{R}_+ \bigcup \{ 0 \}$ is such that $x < y$
      then we have to consider the following cases for $x$
      \begin{enumerate}
        \item then $x^2 = 0 \cdot 0 < y \Rightarrow x^2 = 0 \equallim_{(6)} 0
        \cdot y < y \cdot y = y^2 \Rightarrow x^2 < y^2$
        
        \item then from $0 < x < y$ we have $x, y \in R_+$ and using (6) we
        get $x \cdot x < y \cdot x$ and $x \cdot y < y \cdot y$ giving $x
        \cdot x < y \cdot y \Rightarrow x^2 < y^2$
      \end{enumerate}
      \item If $\alpha \in \mathbbm{R}_+$ then $0 < \alpha$ we proceed now by
      induction on $n$ to prove that $\forall n \in \mathbbm{N}$ we have $0 <
      \alpha^n$. So let $X = \{ n \in \mathbbm{N}| \tmop{if} 0 < \alpha
      \tmop{then} 0 < \alpha^n \}$ then we have:
      \begin{enumerate}
        \item $\alpha^0 = 1 > 0 \Rightarrow 0 \in X$
        
        \item If $n \in X$ then if $0 < \alpha$ we have $0 < \alpha^n$. Now
        $\alpha^{n + 1} = \alpha^n \cdot \alpha \Rightarrowlim_{0 < \alpha^n,
        \alpha \in \mathbbm{R}_+ \tmop{and} (6)} 0 < \alpha^n \cdot \alpha =
        \alpha^{n \upl 1} \Rightarrow n \upl 1 \in X$
      \end{enumerate}
      Using mathematical induction (see \ref{mathematical induction}) we have
      $X =\mathbbm{N}$ proving our assertion.
      
      \item Let $\alpha \in \mathbbm{R}_+$ with $0 < \alpha < 1$ then if $n
      \in \mathbbm{N}_0$ we have by (12) already $0 < \alpha^n$. Now to prove
      $\alpha^n < \alpha$ take $S = \{ n <\mathbbm{N}_0 | \tmop{if} \alpha < 1
      \tmop{then} \alpha^n < \alpha \}$ then we have
      \begin{enumerate}
        \item if $n = 1$ then $\alpha^1 = \alpha < 1$ so that $1 \in S$
        
        \item if $n \in S$ then $\alpha^{n + 1} = \alpha^n \cdot \alpha <
        \alpha$ [as $n \in S \Rightarrow \alpha^n < 1 \Rightarrowlim_{0 <
        \alpha \tmop{and} (6)} \alpha^n \cdot \alpha < 1 \cdot \alpha =
        \alpha$]
      \end{enumerate}
      
    \end{enumerate}
  \end{proof}
\end{theorem}

\begin{theorem}
  \label{zero sum in R}If $\alpha, \beta \in \mathbbm{R}$ and $0 \leqslant
  \alpha, \beta$ then from $\alpha + \beta = 0$ we have $\alpha = \beta = 0$
\end{theorem}

\begin{proof}[$0 < \alpha$][$0 < \beta$][$0 = \alpha, \beta$]
  We have then to consider the following cases
  \begin{enumerate}
    \item $\Rightarrow \beta = 0 + \beta < \alpha + \beta = 0 \Rightarrow
    \beta < 0$ a contradiction
    
    \item $\Rightarrow \alpha = 0 + \alpha < \alpha + \beta = 0 \Rightarrow
    \alpha < 0$ a contradiction
    
    \item this is what we have to prove
  \end{enumerate}
\end{proof}

\begin{lemma}
  \label{order relation on rational cuts}If $r, s \in \mathbbm{Q}$ with $r <
  s$ \ then $\alpha_r < \alpha_s$ (and thus we have from $r \leqslant s
  \Rightarrow r = s$ or $r < s \Rightarrow \alpha_r = \alpha_s$ or $\alpha_r <
  \alpha_s \Rightarrow \alpha_r \leqslant \alpha_s$)
\end{lemma}

\begin{proof}
  If $r, s \in \mathbbm{Q}$ with $r < s$ then if $u \in \alpha_r \Rightarrow u
  < r \Rightarrowlim_{r < s} u < s \Rightarrow u \in \alpha_s \Rightarrow
  \alpha_r \subseteq \alpha_s$. Also $r < s \Rightarrow r \in \alpha_s$ and $r
  \nin \alpha_r \Rightarrow \alpha_r \neq \alpha_s \Rightarrow \alpha_r <
  \alpha_s$
\end{proof}

\begin{theorem}
  The field isomorphism $i_{\mathbbm{Q}} : \mathbbm{Q} \rightarrow
  \mathbbm{Q}_{\mathbbm{R}}$ defined by $q \rightarrow \alpha_q$ is order
  preserving.
\end{theorem}

\begin{proof}
  This is trivial by using the previous lemma. If $r, s \in \mathbbm{Q}$ with
  $r \leqslant s \Rightarrow \alpha_r \leqslant \alpha_s \Rightarrow
  i_{\mathbbm{Q}} (r) \leqslant i_{\mathbbm{Q}} (s)$
\end{proof}

\begin{theorem}[$\mathbbm{R} \tmop{is} \tmop{conditional} \tmop{complete}$]
  \label{the reals are conditional complete}{\index{$\mathbbm{R} \tmop{is}
  \tmop{conditional} \tmop{complete}$}}$\langle \mathbbm{R}, \leqslant
  \rangle$ is conditional complete (see \ref{conditonal complete classes}).
  Using the definition of conditional completeness this means that $\forall S
  \subseteq \mathbbm{R} \vdash S \neq \emptyset$ for which there exists a $b
  \in \mathbbm{R}$ such that $\forall s \in S$ we have $s \leqslant b$
  (existence of a upper bound) we have the existence of $\sup (S)$ (a lowest
  upper bound). In other words: any nonempty set in $\mathbbm{R}$ with a upper
  bound has a lowest upper bound.
\end{theorem}

\begin{proof}[$\gamma \neq \emptyset$][$\gamma \neq \mathbbm{Q}$][$\forall r
\in \gamma \wedge \forall s \in \mathbbm{Q} \backslash \gamma$ we have $r <
s$][$\max (\gamma) \tmop{does} \tmop{not} \tmop{exists}$]
  Let $S \subseteq \mathbbm{R}$ with $S \neq \emptyset$ and the existence of a
  $\beta \in \mathbbm{R}$ such that $\forall \alpha \in S$ we have $\alpha
  \leqslant \beta$. Define then $\gamma = \{ r \in \mathbbm{Q} | \exists
  \alpha \in S \nobracket \vdash r \in \alpha \}$ \ (in another notation
  $\gamma = \bigcup_{\alpha \in S} \alpha$). We prove now that $\gamma$ is a
  Dedekind's cut and thus that $\gamma \in \mathbbm{R}$
  \begin{enumerate}
    \item As $S \neq \emptyset$ there exists a $\alpha \in S \subseteq
    \mathbbm{R}$ and by \ref{dedekind's cut} , 1 there exists a $r \in \alpha
    \subseteq \mathbbm{Q} \Rightarrow r \in \gamma \Rightarrow \gamma \neq
    \emptyset$
    
    \item If $r \in \gamma \Rightarrow \exists \alpha \in S$ such that $r \in
    \alpha$ and as $\alpha \leqslant \beta \Rightarrow \alpha \subseteq \beta
    \Rightarrow r \in \beta$ and thus we have $\gamma \subseteq \beta$ and as
    by \ref{dedekind's cut} , 2 we have $\beta \neq \mathbbm{Q}$ we have also
    $\gamma \neq \mathbbm{Q}$
    
    \item So let $r \in \gamma$ and $s \in \mathbbm{Q} \backslash \gamma$.
    From $r \in \gamma$ we have that $\exists \alpha \in S \vdash r \in
    \alpha$. From $s \in \mathbbm{Q} \backslash \gamma$ we have that $\forall
    \tau \in S$ we have that $s \nin \tau$ so in particular we have $s \nin
    \alpha \Rightarrow s \in \mathbbm{Q} \backslash \alpha$. Using
    \ref{dedekind's cut} , 3 for $\alpha$ we have thus $r < s$.
    
    \item We prove this by contradiction, so assume $m = \max (\gamma)$
    exists. Then as $m \in \gamma$ there exists a $\alpha \in S$ such that $m
    \in \alpha$ and as $\max (\alpha) $ does not exists there exist a $m' \in
    \alpha$ with $m < m'$, and from $m' \in \alpha \in S$ we derive $m' \in
    \gamma \Rightarrow m' < m < m' \Rightarrow m' < m'$ a contradiction. So
    $\max (\gamma)$ does not exists.
  \end{enumerate}
  Next we prove that $\gamma$ is a upper bound for $S$. If $\alpha \in S$ then
  $\forall r \in \alpha$ we have $r \in \{ s \in \mathbbm{Q} | \exists \alpha
  \in S \nobracket \vdash s \in \alpha \} = \gamma \Rightarrow \alpha
  \subseteq \gamma \Rightarrow \alpha \leqslant \gamma$. Finally we prove that
  it is the least upper bound. So if $\gamma' \in \upsilon (S) = \{ u \in
  \mathbbm{R} | u \tmop{is} a \tmop{upper} \tmop{bound} \tmop{of} S \nobracket
  \}$ then if $r \in \gamma \Rightarrow \exists \alpha \in S \vdash r \in
  \alpha$, \ as $\alpha \leqslant \gamma' \Rightarrow \alpha \subseteq
  \gamma'$ we have $r \in \gamma'$ and thus we have $\gamma \subseteq \gamma'
  \Rightarrow \gamma \leqslant \gamma'$ and thus is a least element of
  $\upsilon (S)$ and thus we have $\gamma = \sup (S)$ 
\end{proof}

\begin{theorem}
  \label{sup(a.S)}If $S \subseteq \mathbbm{R}$ has a supremum then if $\alpha
  \in \mathbbm{R}$ with $\alpha \geqslant 0$ then $\alpha \cdot S = \{ a \cdot
  s|s \in S \}$ has a supremum where $\sup (\{ \alpha \cdot s|s \in S \}) =
  \alpha \cdot \sup (S)$
\end{theorem}

\begin{proof}
  If $\alpha = 0$ then $\alpha \cdot S = \{ 0 \}$ with $\sup (\alpha \cdot S)
  = \sup (\{ 0 \}) = 0$ so we only have to prove the case that $\alpha > 0$.
  By the definition of $\sup (S)$ we have that $\forall s \in S$ that $s
  \leqslant \sup (S)$ so if $y \in \alpha \cdot S$ then $y = \alpha \cdot s
  \leqslant \alpha \cdot \sup (S)$ [as $\alpha > 0$] proving that $\alpha
  \cdot \sup (S)$ is a upper bound for $\alpha \cdot S$. So by the above
  theorem $\alpha \cdot S$ has a $\sup (\alpha \cdot S)$ with by definition
  $\sup (\alpha \cdot S) \leqslant \alpha \cdot \sup (S)$. If now $\sup
  (\alpha \cdot S) < \alpha \cdot \sup (S)$ then $\frac{1}{\alpha} \cdot \sup
  (\alpha \cdot S) < \sup (S)$ and there exists a $s \in S$ with
  $\frac{1}{\alpha} \cdot \sup (\alpha \cdot S) < s \leqslant \sup (S)$ giving
  $\sup (\alpha \cdot S) < \alpha \cdot s \leqslant \alpha \cdot \sup (S)$
  which as $\alpha \cdot s \in \alpha \cdot S$ contradicts the fact that $\sup
  (\alpha \cdot S)$ is the supremum of $\alpha \cdot S$ so we must have $\sup
  (\alpha \cdot S) = \alpha \cdot \sup (S)$.
\end{proof}

\begin{theorem}
  \label{sup(S+T)}If $S, T \subseteq \mathbbm{R}$ have a supremum then $S + T
  = \{ s + t|s \in S \wedge t \in T \}$ has a supremum with $\sup (S + T) =
  \sup (S) + \sup (T)$
\end{theorem}

\begin{proof}
  As $S, T$ have a supremum there we have $\forall s \in S, \forall t \in T$
  that $s \leqslant \sup (S) \wedge t \leqslant \sup (T)$ so that if $r = s +
  t \in S + T$ then $r = s + t \leqslant \sup (S) + \sup (T)$ so $S + T$ is
  upper bounded above by $\sup (S) + \sup (T)$ and thus $S + T$ has a supremum
  with $\sup (S + T) \leqslant \sup (S) + \sup (T)$. Assume now that $\sup (S
  + T) < \sup (S) + \sup (T)$ and take now $\varepsilon = \sup (S) + \sup (T)
  - \sup (S + T) \Rightarrow \varepsilon > 0$. Then $\sup (S) -
  \frac{\varepsilon}{2} < \sup (S)$ and $\sup (T) - \frac{\varepsilon}{2} <
  \sup (T)$ so that $\exists s \in S$ and $t \in T$ with $\sup (S) -
  \frac{\varepsilon}{2} < s \leqslant \sup (S)$ and $\sup (T) -
  \frac{\varepsilon}{2} < t \leqslant \sup (S) \Rightarrow \sup (S) + \sup (T)
  - \varepsilon < s + t \leqslant \sup (S) + \sup (T) \Rightarrow \sup (S + T)
  < s + t$ contradicting the fact that $\sup (S + T)$ is the supremum of $S +
  T$ \ as $s + t \in S + T$ and thus we reach a contradiction of our
  assumption. So we must have $\sup (S + T) = \sup (S) + \sup (T)$.
\end{proof}

\begin{theorem}
  \label{rationals embedded in the reals are not conditional complete}$\langle
  \mathbbm{Q}_{\mathbbm{R}}, \leqslant \rangle$ is not conditional complete
\end{theorem}

\begin{proof}
  Using the fact that $\langle \mathbbm{Q}, \leqslant \rangle$ is not
  conditional complete (see \ref{the set of rationals is not conditional
  complete}) we have that there exists a nonempty set $A' \subseteq Q$ which
  is bounded above by a $u' \in \mathbbm{Q}$ such that $\sup (A')$ does not
  exists. Define now $A = i_{\mathbbm{Q}} (A')$ and $u = i_{\mathbbm{Q}} (u')$
  then we have the following:
  \begin{enumerate}
    \item As $A'$ is not empty there exists a $x \in A'$ then $i_{\mathbbm{Q}}
    (x) \in A \Rightarrow A$ is not empty.
    
    \item If $a \in A \Rightarrow \exists a' \in A'$ such that $a =
    i_{\mathbbm{Q}} (a')$, as $u$ is a upper bound of $A'$ we have that $a'
    \leqslant u' \Rightarrowlim_{i_{\mathbbm{Q}} \tmop{is} \tmop{order}
    \tmop{preserving}} a = i_{\mathbbm{Q}} (a') \leqslant i_{\mathbbm{Q}} (u')
    = u \Rightarrow A$ is bounded above by $u$.
  \end{enumerate}
  So $A$ is a non-empty set that is bounded above. Assume now that $s = \sup
  (A)$ exists. Take now $s' = i_{\mathbbm{Q}}^{- 1} (s)$ then we have:
  \begin{enumerate}
    \item If $a' \in A'$ then if $s' < a'$ we have by the order preserving of
    $i_{\mathbbm{Q}}$ and the fact that it is injective that $s =
    i_{\mathbbm{Q}} (s') < i_{\mathbbm{Q}} (a') \in A$ so $s$ is not a upper
    bound of $A$ contradicting the fact that $s$ is the lowest upper bound. So
    we must have that $a' \leqslant s'$ and as $a'$ was chosen arbitrary $s'$
    is a upper bound of $A'$
    
    \item Let $b'$ be another upper bound of $A'$ and $b = i_{\mathbbm{Q}}
    (b')$. If now $a \in A$ then $a' = i_{\mathbbm{Q}}^{- 1} (a) \in A'$ [as
    $i_{\mathbbm{Q}} (a') = a \in A$] and thus $a' \leqslant b'$ using then
    the order preserving of $i_{\mathbbm{Q}}$ we have then $a =
    i_{\mathbbm{Q}} (a') \leqslant i_{\mathbbm{Q}} (b') = b \Rightarrow b$ is
    a upper bound of $A$ and as $s$ is the least upper bound we have $s
    \leqslant b$. If now $b' < s'$ then we have using order preserving and
    injectivity of $i_{\mathbbm{Q}}$ that $b = i_{\mathbbm{Q}} (b') <
    i_{\mathbbm{Q}} (s') = s \Rightarrow b < s$ contradicting the fact that
    $s$ is the lowest upper bound of $A$. So we must have that $s' \leqslant
    b'$
  \end{enumerate}
  From the above we conclude that $s' = \sup (A')$ is a lower upper bound of
  $A'$ contradicting the fact that we have chosen $A'$ so that no upper bound
  exists. So we conclude that $A$ has no lowest upper bound and as $A$ is a
  non empty set that has a upper bound we conclude that $\langle
  \mathbbm{Q}_{\mathbbm{R}}, \leqslant \rangle$ is not conditional complete.
\end{proof}

\begin{corollary}
  \label{irrationals}{\index{irrational numbers}}$\mathbbm{Q}_{\mathbbm{R}}
  \subset \mathbbm{R}$ so there exists a $r \in \mathbbm{R}$ that is not in
  $\mathbbm{Q}$. In other words we have $\mathbbm{R} \backslash
  \mathbbm{Q}_{\mathbbm{R}} \neq \emptyset$. $\mathbbm{R} \backslash
  \mathbbm{Q}_{\mathbbm{R}}$ is called the set of irrational numbers.
\end{corollary}

\begin{proof}
  As $\langle \mathbbm{Q}_{\mathbbm{R}}, \leqslant \rangle$ is not conditional
  complete (see previous theorem) there exists a non-empty set $A$ with a
  upper bound $u$ so that $\sup (A)$ does not exists in $\mathbbm{Q}$. Because
  $\mathbbm{Q}_{\mathbbm{R}} \subseteq \mathbbm{R}$ we have $\emptyset \neq A
  \subseteq \mathbbm{R}$ and that $A$ has the upper bound $u \in
  \mathbbm{Q}_{\mathbbm{R}} \subseteq \mathbbm{R}$. As $\mathbbm{R}$ is
  conditional complete (see \ref{the reals are conditional complete}) there
  exists a lowest upper bound $s = \sup (A)$, now if $s \in
  \mathbbm{Q}_{\mathbbm{R}}$ it would be a upper bound of $A$ and if $b \in
  \mathbbm{Q}_{\mathbbm{R}}$ of $A$ is another upper bound of $A$ it is also a
  upper bound of $A$ in $\mathbbm{R}$ and thus $s \leqslant b$ so $s$ would be
  a upper bound of $A$ in $\mathbbm{Q}_{\mathbbm{R}}$ contradicting the fact
  that $\sup (A)$ does not exists in $\mathbbm{Q}_{\mathbbm{R}}$. So $s \nin
  \mathbbm{Q}_{\mathbbm{R}}$.
\end{proof}

\begin{definition}
  \label{integers embedded in the
  reals}{\index{$\mathbbm{Z}_{\mathbbm{R}}$}}$\mathbbm{Z}_{\mathbbm{R}} = \{
  \alpha_r | r \in \mathbbm{Z}_{\mathbbm{Q}} \nobracket \} \subseteq
  \mathbbm{Q}_{\mathbbm{R}} \subseteq \mathbbm{R}$. Note that $0 = \alpha_0, 1
  = \alpha_1$ are elements of $\mathbbm{Z}_{\mathbbm{R}}$.
\end{definition}

\begin{theorem}
  \label{whole numbers embedded in the reals from a sub ring}$\langle
  \mathbbm{Z}_{\mathbbm{R}}, \upl, \cdot \rangle$ forms a sub ring of $\langle
  \mathbbm{R}, \upl, \cdot \rangle$. Also $i_{\mathbbm{Q}_{\mathbbm{Z}}} :
  \mathbbm{Z} \rightarrow \mathbbm{Z}_{\mathbbm{R}}$ defined by $x \rightarrow
  \alpha_{\frac{x}{1}}$ is a ring isomorphism that is also order preserving.
  So we can consider $\mathbbm{Z}_{\mathbbm{R}}$ as the set of integer numbers
  embedded in the set of reals.
\end{theorem}

\begin{proof}[injective][surjective]
  Let $x, y \in \mathbbm{Z}_{\mathbbm{R}}$ then we have the existence of a $u,
  v \in \mathbbm{Z}_{\mathbbm{Q}} \subseteq \mathbbm{Q}$ so that $x =
  \alpha_u$ and $y = \alpha_v$. Further using the fact that
  $\mathbbm{Z}_{\mathbbm{Q}}$ is a sub-ring of $\mathbbm{Q}$ (see
  \ref{embedding of the whole numbers in the rationals}) we have that $u \upl
  v \in \mathbbm{Z}_{\mathbbm{Q}}$ and $u \cdot v \in
  \mathbbm{Z}_{\mathbbm{Q}}$ and $- u \in \mathbbm{Z}_{\mathbbm{Q}}$. This
  gives then
  \begin{enumerate}
    \item $x \upl y = \alpha_u \upl \alpha_v \equallim_{\text{\ref{sum and
    product of rational cuts}}} \alpha_{u \upl v} \in
    \mathbbm{Z}_{\mathbbm{R}}$
    
    \item $x \cdot y = \alpha_u \cdot \alpha_v \equallim_{\text{\ref{sum and
    product of rational cuts}}} \alpha_u \cdot \alpha_v = \alpha_{u \cdot v}
    \in \mathbbm{Z}_{\mathbbm{R}}$
    
    \item $- x = - (\alpha_u) \equallim_{\text{\ref{negative of rational
    cut}}} \alpha_{\um u} \in \mathbbm{Z}_{\mathbbm{R}}$ (as $- u \in
    \mathbbm{Z}_{\mathbbm{Q}}$)
    
    \item $1 \in \mathbbm{Z}_{\mathbbm{Q}}$ (see definition)
    
    \item $0 \in \mathbbm{Z}_{\mathbbm{Q}}$ (see definition)
  \end{enumerate}
  this proves that $\langle \mathbbm{Z}_{\mathbbm{R}}, \upl, \cdot \rangle$ is
  a sub-ring of $\langle \mathbbm{R}, +, \cdot \rangle$. Now to prove that
  $i_{\mathbbm{Q}_{\mathbbm{Z}}}$ is \ a order preserving ring isomorphism we
  first prove that it is a bijection:
  \begin{enumerate}
    \item If $x, y \in \mathbbm{Z}$ are such that $\alpha_{\frac{x}{1}} =
    \alpha_{\frac{y}{1}}$ then by \ref{rational cuts} we have $\frac{x}{1} =
    \frac{y}{1} \Rightarrow x \cdot 1 = y \cdot 1 \Rightarrow x = y$ proving
    injectivity.
    
    \item If $z \in \mathbbm{Z}_{\mathbbm{R}}$ then there exists a $r \in
    \mathbbm{Z}_{\mathbbm{Q}}$ such that $z = \alpha_r$. As $r \in
    \mathbbm{Z}_{\mathbbm{Q}}$ there exists a $u \in \mathbbm{Z}$ such that $r
    = \frac{u}{1}$, but then $z = \alpha_{\frac{u}{1}} =
    i_{\mathbbm{Q}_{\mathbbm{Z}}} (u)$ proving surjectivity.
  \end{enumerate}
  Finally to prove that $i_{\mathbbm{Q}_{\mathbbm{Z}}}$ is a a order
  preserving ring isomorphism, note that
  \begin{enumerate}
    \item $\forall x, y \in \mathbbm{Z}$ we have that $\frac{x \upl y}{1} =
    \frac{x}{1} \upl \frac{y}{1}$ so $i_{\mathbbm{Q}_{\mathbbm{Z}}} (x \upl y)
    = \alpha_{\frac{x \upl y}{1}} = \alpha_{\frac{x}{1} \upl \frac{y}{1}}
    \equallim_{\text{\ref{sum and product of rational cuts}}}
    \alpha_{\frac{x}{1}} \upl \alpha_{\frac{y}{1}} =
    i_{\mathbbm{Q}_{\mathbbm{Z}}} (x) \upl i_{\mathbbm{Q}_{\mathbbm{Z}}} (y)$
    
    \item $\forall x, y \in \mathbbm{Z}$ we have that $\frac{x \cdot y}{1} =
    \frac{x}{1} \cdot \frac{y}{1}$ so $i_{\mathbbm{Q}_{\mathbbm{Z}}} (x \cdot
    y) = \alpha_{\frac{x \cdot y}{1}} = \alpha_{\frac{x}{1} \cdot \frac{y}{1}}
    \equallim_{\text{\ref{sum and product of rational cuts}}}
    \alpha_{\frac{x}{1}} \cdot \alpha_{\frac{y}{1}} =
    i_{\mathbbm{Q}_{\mathbbm{Z}}} (x) \cdot i_{\mathbbm{Q}_{\mathbbm{Z}}} (y)$
    
    \item $\forall x, y \in \mathbbm{Z}$ with $x \leqslant y$ we have by
    \ref{embedding of the whole numbers in the rationals} that $\frac{x}{1} =
    i_{\mathbbm{Z}} (x) \leqslant i_{\mathbbm{Z}} (y) = \frac{y}{1}
    \Rightarrow \frac{x}{1} \leqslant \frac{y}{1}$. Using \ref{order relation
    on rational cuts} we have then $\alpha_{\frac{x}{1}} \leqslant
    \alpha_{\frac{y}{1}}$ and thus $i_{Q_{\mathbbm{Z}}} (x) \leqslant
    i_{\mathbbm{Q}_{\mathbbm{Z}}} (y)$.
  \end{enumerate}
  
\end{proof}

\begin{definition}
  \label{natural numbers embedded in
  reals}{\index{$\mathbbm{N}_{\mathbbm{R}}$}}$\mathbbm{N}_{\mathbbm{R}} = \{
  \alpha_r | r \in \mathbbm{N}_{\mathbbm{Q}} \nobracket \} \subseteq
  \mathbbm{Z}_{\mathbbm{R}} \subseteq \mathbbm{Q}_{\mathbbm{R}} \subseteq
  \mathbbm{R}$ (for definition of $\mathbbm{N}_{\mathbbm{Q}}$ see
  \ref{naturals embedded in the rationals}). Note that $0 = \alpha_0, 1 =
  \alpha_1$ are elements of $\mathbbm{Z}_{\mathbbm{R}}$
\end{definition}

\begin{theorem}
  \label{properties of natural numbers embedded in the reals}$\langle
  \mathbbm{N}_{\mathbbm{R}}, + \rangle$ forms a sub-semi-group of $\langle
  \mathbbm{R}, + \rangle$, $\langle \mathbbm{N}_{\mathbbm{R}}, \cdot \rangle$
  \ forms a sub-semi-group of $\langle \mathbbm{R}, \cdot \rangle$. Also
  $i_{\mathbbm{Q}_{\mathbbm{N}}} : \mathbbm{N} \rightarrow
  \mathbbm{N}_{\mathbbm{R}}$ defined by $n \rightarrow \alpha_r$ where $r =
  \frac{i_{\mathbbm{N}} (n)}{1}$ is a semi-group isomorphism that is also
  order preserving (for the addition $+$ and multiplication $\cdot$). So we
  can consider $\mathbbm{N}_{\mathbbm{R}}$ as the set of natural number
  embedded in the set of reals. Also $0 = i_{\mathbbm{Q}_{\mathbbm{N}}} (0)$
  and \ $\forall n \in \mathbbm{N}_{\mathbbm{R}}$ we have $0 \leqslant n$
\end{theorem}

\begin{proof}[Injective][Surjective]
  Let $x, y \in \mathbbm{N}_{\mathbbm{R}}$ then there exists $r, s \in
  \mathbbm{N}_{\mathbbm{Q}}$ such that $x = \alpha_r$ and $y = \alpha_s$. From
  $r, s \in \mathbbm{N}_{\mathbbm{Q}}$ we have the existence of $u, v \in
  \mathbbm{N}_{\mathbbm{Z}}$ such that $r = \frac{u}{1}$ and $s =
  \frac{v}{1}$. Using the above gives then
  \begin{enumerate}
    \item $x \upl y = \alpha_r \upl \alpha_s \equallim_{\text{\ref{sum and
    product of rational cuts}}} \alpha_{r \upl s}$. Also $r \upl s =
    \frac{u}{1} \upl \frac{v}{1} = \frac{u \cdot 1 + v \cdot 1}{1 \cdot 1} =
    \frac{u + v}{1}$ where $u \upl v \in \mathbbm{N}_{\mathbbm{Z}}$ (see
    \ref{properties of positive numbers}) and this gives that $r \upl s \in
    \mathbbm{N}_{\mathbbm{Q}} \Rightarrow x \upl y = \alpha_{r \upl s} \in
    \mathbbm{N}_{\mathbbm{R}}$
    
    \item $0 \in \mathbbm{N}_{\mathbbm{R}}$ (see note in the definition)
  \end{enumerate}
  we therefore conclude that $\langle \mathbbm{N}_{\mathbbm{R}}, + \rangle$ is
  indeed a sub-semi-group of $\langle \mathbbm{R}, \upl \rangle$.
  
  Next we prove that $\langle \mathbbm{N}_{\mathbbm{R}}, \cdot \rangle$ is a
  sub-semi-group of $\langle \tmop{Rr}, \cdot \rangle$
  \begin{enumerate}
    \item $x \cdot y = \alpha_r \cdot \alpha_s \equallim_{\text{\ref{sum and
    product of rational cuts}}} \alpha_{r \cdot s}$. Also $r \cdot s =
    \frac{u}{1} \cdot \frac{v}{1} = \frac{u \cdot v}{1}$ where $u \cdot v \in
    \mathbbm{N}_{\mathbbm{Z}}$ (see \ref{properties of positive numbers}) and
    thus $r \cdot s \in \mathbbm{N}_{\mathbbm{Q}} \Rightarrow x \cdot y =
    \alpha_{r \cdot s} \in \mathbbm{N}_{\mathbbm{R}}$
    
    \item $1 \in \mathbbm{N}_{\mathbbm{R}}$ (see note in the definition)
  \end{enumerate}
  Next we prove that $i_{\mathbbm{Q}_{\mathbbm{N}}}$ is a order preserving
  semi-group isomorphism. First we prove that $i_{\mathbbm{Q}_{\mathbbm{N}}}$
  is a bijection:
  \begin{enumerate}
    \item If $n, m$ are such that $i_{\mathbbm{Q}_{\mathbbm{N}}} (n) =
    i_{\mathbbm{Q}_{\mathbbm{N}}} (m)$ then if $r = \frac{i_{\mathbbm{N}}
    (n)}{1}, s = \frac{i_{\mathbbm{N}} (m)}{1}$ we have that $\alpha_r =
    \alpha_s \Rightarrowlim_{\text{\ref{rational cuts}}} r = s \Rightarrow
    i_{\mathbbm{N}} (n) \cdot 1 = i_{\mathbbm{N}} (m) \cdot 1 \Rightarrow
    i_{\mathbbm{N}} (n) = i_{\mathbbm{N}} (m) \Rightarrowlim_{i_{\mathbbm{N}}
    \tmop{is} a \tmop{bijection}} n = m_{}$
    
    \item If $x \in \mathbbm{N}_{\mathbbm{R}}$ then there exists a $r \in
    \mathbbm{N}_{\mathbbm{Q}}$ such that $x = \alpha_r$, as $r \in
    \mathbbm{N}_{\mathbbm{Q}}$ there exists a $u \in
    \mathbbm{N}_{\mathbbm{Z}}$ such that $r = \frac{u}{1}$, as $u \in
    \mathbbm{N}_{\mathbbm{Z}}$ there exists by \ref{properties of positive
    numbers} a $n \in \mathbbm{N}$ such that $u = i_{\mathbbm{N}} (n)
    \Rightarrow r = \frac{i_{\mathbbm{N}} (n)}{1} \Rightarrow x = \alpha_r =
    i_{\mathbbm{Q}_{\mathbbm{N}}} (n)$
  \end{enumerate}
  Next we prove that it is a semi-group isomorphism
  \begin{enumerate}
    \item If $n, m \in \mathbbm{N}$ then $n \upl m \in \mathbbm{N}$ and by
    \ref{properties of positive numbers} we have $i_{\mathbbm{N}} (n \upl m) =
    i_{\mathbbm{N}} (n) \upl i_{\mathbbm{N}} (m)$ and thus
    $\frac{i_{\mathbbm{N}} (n \upl m)}{1} = \frac{i_{\mathbbm{N}} (n) +
    i_{\mathbbm{N}} (m)}{1} = \frac{i_{\mathbbm{N}} (n)}{1} \upl
    \frac{i_{\mathbbm{N}} (m)}{1}$. So we have that
    $i_{\mathbbm{Q}_{\mathbbm{N}}} (n \upl m) = \alpha_{\frac{i_{\mathbbm{N}}
    (n \upl m)}{1}} = \alpha_{\frac{i_{\mathbbm{N}} (n)}{1} +
    \frac{i_{\mathbbm{N}} (m)}{1}} \equallim_{\text{\ref{sum and product of
    rational cuts}}} \alpha_{\frac{i_{\mathbbm{N}} (n)}{1}} +
    \alpha_{\frac{i_{\mathbbm{N}} (m)}{1}} = i_{\mathbbm{Q}_{\mathbbm{N}}} (n)
    + i_{\mathbbm{Q}_{\mathbbm{N}}} (m)$
    
    \item If $n, m \in \mathbbm{N}$ then $n \cdot m \in \mathbbm{N}$and by
    \ref{properties of positive numbers} we have $i_{\mathbbm{N}} (n \cdot m)
    = i_{\mathbbm{N}} (n) \cdot i_{\mathbbm{N}} (m)$ and thus
    $\frac{i_{\mathbbm{N}} (n \cdot m)}{1} = \frac{i_{\mathbbm{N}} (n) \cdot
    i_{\mathbbm{N}} (m)}{1} = \frac{i_{\mathbbm{N}} (n)}{1} \cdot
    \frac{i_{\mathbbm{N}} (m)}{1}$. So we have that
    $i_{\mathbbm{Q}_{\mathbbm{N}}} (n \cdot m) = \alpha_{\frac{i_{\mathbbm{N}}
    (n \cdot m)}{1}} = \alpha_{\frac{i_{\mathbbm{N}} (n)}{1} \cdot
    \frac{i_{\mathbbm{N}} (m)}{1}} \equallim_{\text{\ref{sum and product of
    rational cuts}}} \alpha_{\frac{i_{\mathbbm{N}} (n)}{1}} \cdot
    \alpha_{\frac{i_{\mathbbm{N}} (m)}{1}} = i_{\mathbbm{Q}_{\mathbbm{N}}} (n)
    \cdot i_{\mathbbm{Q}_{\mathbbm{N}}} (m)$
  \end{enumerate}
  Next to prove order preserving:
  \begin{enumerate}
    \item If $n, m \in \mathbbm{N} \vdash n \leqslant m$ then using \ref{i_N
    is order preserving} we have $i_{\mathbbm{N}} (n) \leqslant
    i_{\mathbbm{N}} (m)$ and then using \ref{embedding of the whole numbers in
    the rationals} we have $\frac{i_{\mathbbm{N}} (n)}{1} = i_{\mathbbm{Z}}
    (i_{\mathbbm{N}} (n)) \leqslant i_{\mathbbm{Z}} (i_{\mathbbm{N}} (m)) =
    \frac{i_{\mathbbm{N}} (m)}{1}$. So finally using \ref{order relation on
    rational cuts} we have $i_{\mathbbm{Q}_{\mathbbm{N}}} (n) =
    \alpha_{\frac{i_{\mathbbm{N}} (n)}{1}} \leqslant
    \alpha_{\frac{i_{\mathbbm{N}} (m)}{1}} = i_{\mathbbm{Q}_{\mathbbm{N}}}
    (m)$ proving that $i_{\mathbbm{Q}_{\mathbbm{N}}}$ is order preserving.
  \end{enumerate}
  Next if $0 \in \mathbbm{N}$ then $i_{\mathbbm{N}} (0) = \sim [(s (0), 1)] =
  \sim [(1, 1)] = 0 \in \mathbbm{Z} \Rightarrow \frac{i_{\mathbbm{N}} (0)}{1}
  = \frac{0}{1} = 0 \in \mathbbm{Q} \Rightarrow i_{\mathbbm{Q}_{\mathbbm{N}}}
  (0) = \alpha_0 = 0 \in \mathbbm{R}$.
  
  Also if $1 \in \mathbbm{N} \Rightarrow i_{\mathbbm{N}} (1) = \sim [(s (1),
  1)] = \sim [(2, 1)] = 1 \in \mathbbm{Z} \Rightarrow \frac{i_{\mathbbm{N}}
  (1)}{1} = \frac{1}{1} = 1 \in \mathbbm{Q} \Rightarrow
  i_{\mathbbm{Q}_{\mathbbm{N}}} (1) = \alpha_1 = 1 \in \mathbbm{R}.$
  
  Finally to prove that $\forall n \in \mathbbm{N}_{\mathbbm{R}}$ we have $0
  \leqslant n$. If $n \in \mathbbm{N}_{\mathbbm{R}}$ then by the above there
  exists a $m \in \mathbbm{N}$ such that $i_{\mathbbm{Q}_{\mathbbm{N}}} (m) =
  n$. Using \ref{every natural number is bigger or equal to zero} we have $0
  \leqslant m$ and using the order preserving properties of
  $i_{\mathbbm{Q}_{\mathbbm{N}}}$ we have $0 = i_{\mathbbm{Q}_{\mathbbm{N}}}
  (0) \leqslant i_{\mathbbm{Q}_{\mathbbm{N}}} (m) = n \Rightarrow 0 \leqslant
  n$
\end{proof}

\begin{theorem}
  \label{properties of the natural numbers embeded in the embedded integers}We
  have the following properties concerning $\mathbbm{N}_{\mathbbm{R}}$ and
  $\mathbbm{Z}_{\mathbbm{R}}$ (here $\um \mathbbm{N}_{\mathbbm{R}} = \{ n
  \vdash \um n \in \mathbbm{N}_{\mathbbm{R}} \} \subseteq
  \mathbbm{Z}_{\mathbbm{R}}$)
  \begin{enumerate}
    \item $\mathbbm{Z}_{\mathbbm{R}} =\mathbbm{N}_{\mathbbm{R}} \bigcup (\um
    \mathbbm{N}_{\mathbbm{R}})$ and $\mathbbm{N}_{\mathbbm{R}} \bigcap (\um
    \mathbbm{N}_{\mathbbm{R}}) = \{ 0 \}$
    
    \item If $n \in \mathbbm{Z}_{\mathbbm{R}}$ and $0 \leqslant n \Rightarrow
    n \in \mathbbm{N}_{\mathbbm{R}}$ (so $\mathbbm{N}_{\mathbbm{R}}$ is indeed
    the set of positive integers)
    
    \item $\langle \mathbbm{N}_{\mathbbm{R}}, + \rangle$ is a semi-group of
    $\langle \mathbbm{Z}_{\mathbbm{R}}, + \rangle$
    
    \item $\langle \mathbbm{N}_{\mathbbm{R}}, \cdot \rangle$ is a
    sub-semi-group of $\langle \mathbbm{Z}_{\mathbbm{R}}, \cdot \rangle$
  \end{enumerate}
\end{theorem}

\begin{proof}
  
  \begin{enumerate}
    \item As $\mathbbm{N}_{\mathbbm{R}}, \um \mathbbm{N}_{\mathbbm{R}}
    \subseteq \mathbbm{Z}_{\mathbbm{R}}$ we have $\mathbbm{N}_{\mathbbm{R}}
    \bigcup (\um \mathbbm{N}_{\mathbbm{R}}) \subseteq
    \mathbbm{Z}_{\mathbbm{R}}$. If $x \in \mathbbm{Z}_{\mathbbm{R}}$ then $x =
    \frac{n}{1}$ where $n \in \mathbbm{Z}$ then using \ref{whole numbers is
    union of positive and negative numbers} we have either
    \begin{enumerate}
      \item $n \in \mathbbm{N}_{\mathbbm{Z}} \Rightarrow x \in
      \mathbbm{N}_{\mathbbm{R}}$
      
      \item $n \in -\mathbbm{N}_{\mathbbm{Z}} \Rightarrow - n \in
      \mathbbm{N}_{\mathbbm{Z}} \Rightarrow - x = \frac{- n}{1} \in
      \mathbbm{N}_{\mathbbm{R}} \Rightarrow x \in -\mathbbm{N}_{\mathbbm{R}}$
    \end{enumerate}
    and thus $x \in \mathbbm{N}_{\mathbbm{Z}} \bigcup (\um
    \mathbbm{N}_{\mathbbm{Z}}) \Rightarrow \mathbbm{Z}_{\mathbbm{R}} \subseteq
    \mathbbm{N}_{\mathbbm{R}} \bigcup (-\mathbbm{N}_{\mathbbm{R}}) \Rightarrow
    \mathbbm{Z}_{\mathbbm{R}} =\mathbbm{N}_{\mathbbm{R}} \bigcup
    (-\mathbbm{N}_{\mathbbm{R}})$. Finally if $x \in \mathbbm{N}_{\mathbbm{R}}
    \bigcap (-\mathbbm{N}_{\mathbbm{R}})$ then $0 \leqslant x$ (as $x \in
    \mathbbm{N}_{\mathbbm{R}}$) also $- x \in \mathbbm{N}_{\mathbbm{R}}
    \Rightarrow - x = \frac{n}{1}$ where $n \in \mathbbm{N}_{\mathbbm{Z}}
    \Rightarrowlim_{\text{\ref{condition 1 for positive integers}}} 0
    \leqslant n \Rightarrowlim_{0 \leqslant 1 \Rightarrow 0 \leqslant n \cdot
    1} \tmop{sign} (- x) = 1 \Rightarrow 0 \leqslant - x \Rightarrow x
    \leqslant - 0 = 0 \Rightarrow 0 \leqslant x \leqslant 0 \Rightarrow x = 0
    \Rightarrow \mathbbm{N}_{\mathbbm{R}} \bigcap (\um
    \mathbbm{N}_{\mathbbm{R}}) = \{ 0 \}$
    
    \item If $n \in \mathbbm{Z}_{\mathbbm{R}}$ and $0 \leqslant n$ then
    $\exists q \in \mathbbm{Z}_{\mathbbm{Q}}$ such that $n = \alpha_q$ and
    thus $\exists z \in \mathbbm{Z}$ such that $q = \frac{z}{1}$. As $0 =
    \alpha_0 \leqslant \alpha_q$ we must have $0 \leqslant q$ [if $q < 0$ then
    by \ref{order relation on rational cuts} we have $\alpha_q < \alpha_0$
    contradicting $\alpha_0 \leqslant \alpha_q$]. As $0 \leqslant q
    \Rightarrow \tmop{sign} (q) = 1 \Rightarrow 0 \leqslant z \cdot 1 = z
    \Rightarrowlim_{\text{\ref{condition 1 for positive integers}}} z \in
    \mathbbm{N}_{\mathbbm{Z}} \Rightarrow q = \frac{z}{1} \in
    \mathbbm{N}_{\mathbbm{Q}} \Rightarrow n = \alpha_q \in
    \mathbbm{N}_{\mathbbm{R}} \Rightarrow n \in \mathbbm{N}_{\mathbbm{R}}$
    
    \item We can use the same prove as in \ref{properties of natural numbers
    embedded in the reals}
    
    \item We can use the same prove as in \ref{properties of natural numbers
    embedded in the reals}
  \end{enumerate}
  
\end{proof}

\begin{theorem}
  \label{the natural numbers embedded in the reals are well-ordered}$\langle
  \mathbbm{N}_{\mathbbm{R}} . \leqslant \rangle$ is well-ordered
\end{theorem}

\begin{proof}
  If $B \subseteq \mathbbm{N}_{\mathbbm{R}}$ is a nonempty subset of
  $\mathbbm{N}_{\mathbbm{R}}$ then $\exists b \in B
  \Rightarrowlim_{i_{\mathbbm{Q}_{\mathbbm{N}}} \tmop{is} \tmop{surjective}}
  \exists n \in \mathbbm{N} \vdash b = i_{\mathbbm{Q}_{\mathbbm{N}}} (n)
  \Rightarrow i_{\mathbbm{Q}_{\mathbbm{N}}}^{- 1} (B)$ is a nonempty subset of
  $\mathbbm{N}$. As $\langle \mathbbm{N}, \leqslant \rangle$ is well-ordered
  (see \ref{the natural numbers are well-ordered}) there exist a least element
  $m = \min (i_{\mathbbm{Q}_{\mathbbm{N}}^{}}^{- 1} (B))$ so $m \in
  i_{\mathbbm{Q}_{\mathbbm{N}}}^{- 1} (B)$ and $\forall n \in
  i_{\mathbbm{Q}_{\mathbbm{N}}}^{- 1} (B)$ we have $m \leqslant n$. Now $M =
  i_{\mathbbm{Q}_{\mathbbm{N}}} (m) \in B$ (as $m \in
  i_{\mathbbm{Q}_{\mathbbm{N}}}^{- 1} (B)$) and if $N \in B
  \Rightarrowlim_{i_{\mathbbm{Q}_{\mathbbm{N}}} \tmop{is} \tmop{surjective}}
  \exists n \in \mathbbm{N} \vdash i_{\mathbbm{Q}_{\mathbbm{N}}} (n) = N \in B
  \Rightarrow n \in i_{\mathbbm{Q}_{\mathbbm{N}}}^{- 1} (B) \Rightarrow m
  \leqslant n \Rightarrowlim_{i_{\mathbbm{Q}_{\mathbbm{N}}} \tmop{is}
  \tmop{order} \tmop{preserving}} M = i_{\mathbbm{Q}_{\mathbbm{N}}} (m)
  \leqslant i_{\mathbbm{Q}_{\mathbbm{N}}} (n) = N \Rightarrow M \leqslant N$
  or $M = \min (B)$ and $B$ has a least element.
\end{proof}

\begin{lemma}
  \label{0<less>n=<gtr>1<less>=n if n is a embedded natural number}If $n \in
  \mathbbm{N}_{\mathbbm{R}}$ and $0 < n \Rightarrow 1 \leqslant n$.
\end{lemma}

\begin{proof}
  As $n \in \mathbbm{N}_{\mathbbm{R}}$ there exists by surjectivity of
  $i_{\mathbbm{Q}_{\mathbbm{N}}}$ a $n' \in \mathbbm{N}$ such that
  $i_{\mathbbm{Q}_{\mathbbm{N}}} (n') = n$. Using \ref{every natural number is
  bigger or equal to zero} we have $0 \leqslant n'$, if $n' = 0 \Rightarrow n
  = i_{\mathbbm{Q}_{\mathbbm{N}}} (n') = i_{\mathbbm{Q}_{\mathbbm{N}}} (0) =
  0$ contradicting $0 < n$, this leaves us to conclude that $0 < n'$. Using
  \ref{n<less>m=<gtr>s(n)<less>=m} we have then $1 = s (n) \leqslant n'$ and
  using the order preserving properties of $i_{\mathbbm{Q}_{\mathbbm{N}}}$ and
  the fact that $i_{\mathbbm{Q}_{\mathbbm{N}}} (1) = 1$ (see \ref{properties
  of natural numbers embedded in the reals} that $1 =
  i_{\mathbbm{Q}_{\mathbbm{N}}} (1) \leqslant i_{\mathbbm{Q}_{\mathbbm{N}}}
  (n') = n \Rightarrow 1 \leqslant n$ 
\end{proof}

\begin{theorem}[Archimedean Property]
  \label{archimedean property of the reals}If $x, y \in \mathbbm{R}$ with $0 <
  x$ then $\exists n \in \mathbbm{N}_{\mathbbm{R}}$ such that $y < n \cdot x$
\end{theorem}

\begin{proof}[$y \leqslant 0$][0<y]
  We have the following cases to consider for $y$
  \begin{enumerate}
    \item  then take $1 \in \mathbbm{N}_{\mathbbm{R}}$ then we have $y
    \leqslant 0 < x \Rightarrow y < x = 1 \cdot x$
    
    \item  we prove the theorem here by contradiction, so assume that $\forall
    n \in \mathbbm{N}_{\mathbbm{R}}$ we have $n \cdot x \leqslant y$. Define
    $A = \{ n \cdot x | n \in \mathbbm{N}_{\mathbbm{R}} \nobracket \}$ then
    $\forall t \in A$ we have $t \leqslant y$ and thus $y$ is a upper bound of
    $A$. From conditional completeness (see \ref{the reals are conditional
    complete}) $\sup (A)$ exists. As $x > 0$ we have using \ref{properties of
    positive, negative real numbers} that $- x < 0 \Rightarrow \sup (A) \um x
    < \sup (A)$ as $\sup (A)$ is the least upper bound of $A$ $\sup (A) \um x$
    is not a upper bound of $A$ \ so there exists a $n \in
    \mathbbm{N}_{\mathbbm{R}}$ such that $\sup (A) \um x < n \cdot x
    \Rightarrow \sup (A) < n \cdot x \upl x = (n \upl 1) \cdot x$, now as $n,
    1 \in \mathbbm{N}_{\mathbbm{R}}$ we have using \ref{properties of natural
    numbers embedded in the reals}) that $n \upl 1 \in
    \mathbbm{N}_{\mathbbm{R}} \Rightarrow (n \upl 1) \cdot x \in A$ and from
    $\sup (A) < (n \upl 1) \cdot x \in A$ we have that $\sup (A)$ is not a
    upper bound of $A$ and reach thus a contradiction and the initial
    hypothesis is false. So we must have that $\exists n \in
    \mathbbm{N}_{\mathbbm{R}}$ such that $y < n \cdot x$.
  \end{enumerate}
\end{proof}

The following theorem shows some of the consequences of the Archimedean
property of $\mathbbm{R}$

\begin{theorem}[$m \um 1 \in \mathbbm{N}_{\mathbbm{R}} \backslash \{ 0 \}$][$m
\um 1 \nin \mathbbm{N}_{\mathbbm{R}} \backslash \{ 0 \}$]
  \label{consequence of the archimedean property for the reals}Let $x, y \in
  \mathbbm{R}$ then the following holds:
  \begin{enumerate}
    \item $\exists n \in \mathbbm{N}_{\mathbbm{R}} \vdash y < n$
    
    \item If $0 < x$ then $\exists n \in \mathbbm{N}_{\mathbbm{R}} \backslash
    \{ 0 \}$ such that $\frac{1}{n} < x$
    
    \item If $y \geqslant 0$ then $\exists n \in \mathbbm{N}_{\mathbbm{R}}
    \backslash \{ 0 \}$ such that $n \um 1 \leqslant y < n$
  \end{enumerate}
  \begin{proof}
    
    \begin{enumerate}
      \item As $0 < 1 \in \mathbbm{R}$ we have by the Archimedean property
      (see \ref{archimedean property of the reals}) that $\exists n \in
      \mathbbm{N}_{\mathbbm{R}}$ such that $y < 1 \cdot n = n$
      
      \item Using the Archimedean property on $1$ (see \ref{archimedean
      property of the reals}) we have $\exists n \in
      \mathbbm{N}_{\mathbbm{R}}$ such that $1 < x \cdot n$, as $n = 0$ would
      give $1 = 0$ a contradiction we have that $0 < n$ and using
      \ref{properties of positive, negative real numbers} we have then $0 <
      \frac{1}{n} \Rightarrow \frac{1}{n} < x$
      
      \item Consider $A = \{ n \in \mathbbm{N}_{\mathbbm{R}} \backslash \{ 0
      \} | y < n \nobracket \} \subseteq \mathbbm{N}_{\mathbbm{R}} \backslash
      \{ 0 \} \subseteq \mathbbm{N}_{\mathbbm{R}}$, using (1) there exists a
      $n \in \mathbbm{N}_{\mathbbm{R}}$ such that $0 \leqslant y < n
      \Rightarrow n \in A \Rightarrow A \neq \emptyset$. Using the fact
      $\mathbbm{N}_{\mathbbm{R}}$ is well-ordered (see \ref{the natural
      numbers embedded in the reals are well-ordered}) we have that $m = \min
      (A)$ exist. Then as $m \um 1 < m$ we have that $m \um 1 \nin A$ we have
      now the following possibilities
      \begin{enumerate}
        \item as $m \um 1 \nin A$ we must have $m \um 1 \leqslant y < m$ (as
        $m \in A$)
        
        \item as $m \in A \Rightarrow m \neq 0$ we have using \ref{properties
        of natural numbers embedded in the reals} that $0 < m
        \Rightarrowlim_{\text{\ref{0<less>n=<gtr>1<less>=n if n is a embedded
        natural number}}} 1 \leqslant m$ and as $1, m \in
        \mathbbm{N}_{\mathbbm{R}} \subseteq \mathbbm{Z}_{\mathbbm{R}}$ we have
        by \ref{whole numbers embedded in the reals from a sub ring} that $0 =
        1 \um 1 \leqslant m \um 1 \in \mathbbm{Z}_{\mathbbm{R}}
        \Rightarrowlim_{\text{\ref{properties of the natural numbers embeded
        in the embedded integers}}} m \um 1 \in \mathbbm{N}_{\mathbbm{R}}
        \Rightarrowlim_{m \um 1 \nin \mathbbm{N}_{\mathbbm{R}} \backslash \{ 0
        \}} m \um 1 = 0$ and as $m \in A, 0 \leqslant y$ we have $m \um 1 = 0
        \leqslant y < m \Rightarrow m - 1 \leqslant y < m$
      \end{enumerate}
    \end{enumerate}
  \end{proof}
\end{theorem}

\begin{theorem}[Density Theorem]
  \label{density theorem for the reals}{\index{density theorem}}If $x, y \in
  \mathbbm{R}$ such that $x < y$ then $\exists r \in
  \mathbbm{Q}_{\mathbbm{R}}$ and $\exists i \in \mathbbm{R} \backslash
  \mathbbm{Q}_{\mathbbm{R}}$ such that $x < r < y$ and $x < i < y$
\end{theorem}

\begin{proof}[0<x][0=x][x<0][$0 < y$][$y \leqslant 0$]
  First we prove that $\exists r \in \mathbbm{Q}$ such that $x < r < y$, we
  have then the following possible cases for $x$
  \begin{enumerate}
    \item then from $x < y$ we have $0 < y \um x \Rightarrow (y \um x)^{- 1}$
    is defined and by \ref{properties of positive, negative real numbers} we
    have $0 < (y \um x)^{- 1}$, then using the Archimedean property (see
    \ref{archimedean property of the reals}) and $0 < 1$ we have $\exists n
    \in \mathbbm{N}_{\mathbbm{R}}$ such that $0 < (y \um x)^{- 1} < n \cdot 1
    = n \Rightarrowlim_{0 < (y \um x)^{- 1}} 1 < n \cdot (y \um x) \Rightarrow
    1 < n \cdot y \um n \cdot x \Rightarrow n \cdot x \upl 1 < n \cdot y$ and
    $0 < n$. As $0 < x$ and $n \in \mathbbm{N}_{\mathbbm{R}} \Rightarrow 0
    \leqslant n \Rightarrow 0 \leqslant n \cdot x
    \Rightarrowlim_{\text{\ref{consequence of the archimedean property for the
    reals}}} \exists m \in \mathbbm{N}_{\mathbbm{R}} \backslash \{ 0 \}$ such
    that $m \um 1 \leqslant n \cdot x < m \Rightarrow m \leqslant n \cdot x
    \upl 1 < n \cdot y$. Using the fact that $n, m \in
    \mathbbm{N}_{\mathbbm{R}} \subseteq \mathbbm{Z}_{\mathbbm{R}} \subseteq
    \mathbbm{Q}_{\mathbbm{R}}$ so as $\mathbbm{Q}_{\mathbbm{R}}$ is a
    sub-field in $\mathbbm{R}$ (see \ref{rationals in reals form a subfield})
    and $0 < n$ we have $r = m \cdot n^{- 1} \in \mathbbm{Q}_{\mathbbm{R}}$,
    from $0 < n$ we have $0 < n^{- 1}$ and thus $r = m \cdot n^{- 1} < n \cdot
    y \cdot n^{- 1} = y \Rightarrow r < y$. Also from $n \cdot x < m$ we have
    $x = n^{- 1} \cdot n \cdot x < n^{- 1} \cdot m = r \Rightarrow x < r$. So
    we have $x < r < y$.
    
    \item then $0 < y$ and by \ref{consequence of the archimedean property
    for the reals} there exists a $n \in \mathbbm{N}_{\mathbbm{R}} \backslash
    \{ 0 \} \subseteq \mathbbm{Q}_{\mathbbm{R}} \Rightarrow 0 < n
    \Rightarrowlim_{\text{\ref{properties of positive, negative real
    numbers}}} 0 < n^{- 1}$ such that for $n^{- 1} \in
    \mathbbm{Q}_{\mathbbm{R}}$ (as $\mathbbm{Q}_{\mathbbm{R}}$ is a sub-field
    see \ref{rationals in reals form a subfield}) we have $0 < n^{- 1} < x$.
    So taking $r = n^{- 1}$ we have $0 = x < r < y$
    
    \item We have now the following possibilities for $y$
    \begin{enumerate}
      \item then if we take $r = 0 \in \mathbbm{Q}_{\mathbbm{R}} \Rightarrow x
      < r < y$
      
      \item then $x < y \leqslant 0 \Rightarrow 0 \leqslant - y < - x$ and we
      can use (1) or (2) to find a $r' \in \mathbbm{Q}_{\mathbbm{R}}$ such
      that $- y < r' < - x \Rightarrow x < - r' < y$ so using $r = - r' \in
      \mathbbm{Q}_{\mathbbm{R}}$ we have $x < r < y$ 
    \end{enumerate}
  \end{enumerate}
  Next we prove the existence of a $i \in \mathbbm{R} \backslash
  \mathbbm{Q}_{\mathbbm{R}}$ such that $x < i < y$. As $\mathbbm{R} \backslash
  \mathbbm{Q}_{\mathbbm{R}} \neq \emptyset$ (see \ref{irrationals}) there
  exists a $z \in \mathbbm{R} \backslash \mathbbm{Q}_{\mathbbm{R}}$ then from
  $x < y$ we have $x \upl z < y \upl z$ and by (1),(2) and (3) we have that
  there exists a $r \in \mathbbm{Q}_{\mathbbm{R}}$ such that $x \upl z < r < y
  \upl z \Rightarrow x < r \um z < y$. So if $i = r \um z$ then we have $x < i
  < y$. Now if $i \in \mathbbm{Q}_{\mathbbm{R}}$ then as
  $\mathbbm{Q}_{\mathbbm{R}}$ is a sub-field (see \ref{rationals in reals form
  a subfield}) and $r \in \mathbbm{Q}_{\mathbbm{R}}$ we have $z = \um i + r
  \in \mathbbm{Q}_{\mathbbm{R}} \in \mathbbm{R} \backslash
  \mathbbm{Q}_{\mathbbm{R}}$.
\end{proof}

\begin{definition}
  \label{norm in real space}{\index{absolute value}}If $x \in \mathbbm{R}$
  then $| x |$ is defined by $| x | = \left\{ \begin{array}{l}
    x \tmop{if} 0 \leqslant x\\
    - x \tmop{if} x < 0
  \end{array} \right.$and is called the absolute value
\end{definition}

\begin{theorem}[$0 \leqslant x, \alpha$][$0 \leqslant x, \alpha < 0$][$x < 0,
0 \leqslant \alpha$][$x < 0, \alpha < 0$][$0 \leqslant x, y$][$0 \leqslant x,
y < 0$][$0 \leqslant x + y$][$x + y < 0$][$x < 0, 0 \leqslant y$][$x < 0, y <
0$]
  \label{absolute value is a norm}The absolute value $| |$ satisfies the
  following
  \begin{enumerate}
    \item $\forall x \in \mathbbm{R}$ we have $| x | \geqslant 0$
    
    \item $\forall x, \alpha \in \mathbbm{R}$ we have $| \alpha \cdot x | = |
    \alpha | \cdot | x |$
    
    \item $\forall x, y \in \mathbbm{R}$ we have $| x + y | \leqslant | x | +
    | y |$
    
    \item $\forall x \in \mathbbm{R}$ we have $| x | = 0 \Leftrightarrow x =
    0$
  \end{enumerate}
  \begin{proof}
    
    \begin{enumerate}
      \item This is trivial based on the definition.
      
      \item Given $x, \alpha \in \mathbbm{R}$ we have the following cases:
      \begin{enumerate}
        \item then $0 \leqslant x \cdot \alpha$ and $| \alpha \cdot x | =
        \alpha \cdot x = | \alpha | \cdot | x |$
        
        \item then $\alpha \cdot x < 0$ and $| \alpha \cdot x | = - (\alpha
        \cdot x) = (- \alpha) \cdot x = | \alpha | \cdot | x |$
        
        \item then $\alpha \cdot x < 0$ and $| \alpha \cdot x | = - (\alpha
        \cdot x) = \alpha \cdot (- x) = | \alpha | \cdot | x |$
        
        \item then $0 < (- \alpha) \cdot (- x) = \alpha \cdot x$ and $| \alpha
        \cdot x | = \alpha \cdot x = (- \alpha) \cdot (- x) = | \alpha | \cdot
        | x |$
      \end{enumerate}
      \item Given $x, y \in \mathbbm{R}$ we have
      \begin{enumerate}
        \item then $0 \leqslant x + y$ and $| x + y | = x + y = | x | + | y |
        \Rightarrow | x + y | \leqslant | x | + | y |$
        
        \item then we have the following cases
        \begin{enumerate}
          \item then $| x + y | = x + y \Rightarrowlim_{y < 0 \Rightarrow y <
          0 < - y = | y |} x + y < x + | y | = | x | + | y | \Rightarrow | x +
          y | \leqslant | x | + | y |$
          
          \item then $| x + y | = - x - y \Rightarrowlim_{0 \leqslant x
          \Rightarrow - x \leqslant 0 \leqslant x} - x - y \leqslant x - y
          \equallim_{- y = | y |} x + | y | = | x | + | y | \Rightarrow | x +
          y | \leqslant | x | + | y |$
        \end{enumerate}
        \item then if we take $x' = y$ and $y' = x$ we have $| x + y | = | y +
        x | = | x' + y' | \leqslant | x' | + | y' |$ (case (b)) and thus $| x
        + y | \leqslant | x | + | y |$
        
        \item then $x + y < 0 \Rightarrow | x + y | = - (x + y) = (- x) + (-
        y) = | x | + | y | \Rightarrow | x + y | \leqslant | x | + | y |$
      \end{enumerate}
      \item If $x = 0 \Rightarrowlim_{0 \leqslant x} | x | = x = 0$, if $| x |
      = 0$ then if $x < 0$ we have $0 < - x = | x | = 0$ a contradiction so we
      must have $0 \leqslant x$ but then $x = | x | = 0 \Rightarrow x = 0$
    \end{enumerate}
  \end{proof}
\end{theorem}

We prove now some use full theorem needed for the calculations of limits

\begin{theorem}
  \label{n<less>2^n}$\forall n \in \mathbbm{N}_{\mathbbm{R}}$ we have that $n
  < 2^{\mathbbm{R}}$ [and thus by \ref{consequence of the archimedean property
  for the reals} we have also $\forall a \in \mathbbm{R}$ $\exists n \in
  \mathbbm{N}_{\mathbbm{R}}$ such that $a < n < 2^n$]
\end{theorem}

\begin{proof}[$n + 1 = 1$][$1 < n + 1$]
  This is proved by induction so let $S = \{ n \in \mathbbm{N}|n < 2^n \}$
  (here we have implicitly identified $\mathbbm{N}$ and
  $\mathbbm{N}_{\mathbbm{R}}$ to avoid of complicating the proof using the
  isomorphism between $\mathbbm{N}$ and $\mathbbm{N}_{\mathbbm{R}}$). Then we
  have that
  \begin{enumerate}
    \item If $n = 0$ then $0 < 1 = 2^0$ so that $0 \in S$
    
    \item If $n \in S$ then $n < 2^n$ then consider the following cases for $n
    + 1$ (as $0 < n + 1$)
    \begin{enumerate}
      \item then $n + 1 = 1 < 2 = 2^1 = 2^{n + 1} \Rightarrow n + 1 \in S$
      
      \item then $2 \leqslant n + 1 \Rightarrow 1 \leqslant n$ so that $n + 1
      \leqslant n + n = 2 \cdot n < 2 \cdot 2^n = 2^{n + 1}$ proving that $n +
      1 \in S$
    \end{enumerate}
  \end{enumerate}
\end{proof}

\begin{theorem}
  Let $x \in \mathbbm{R}$ with $x > 1$ then $x^n - 1 \geqslant n \cdot (x -
  1)$ $\forall n \in \mathbbm{N}$
\end{theorem}

\begin{proof}
  We prove this by induction, so let $S = \{ n \in \mathbbm{N}|x^n - 1
  \geqslant n \cdot (x - 1) \}$ then we have
  \begin{enumerate}
    \item if $n = 0$ then $x^n - 1 = x^0 - 1 = 1 - 1 = 0 \geqslant 0 = 0 \cdot
    (x - 1) = n \cdot (x - 1) \Rightarrow 0 \in S$
    
    \item If $n \in S$ then $x^{n + 1} - 1 = x \cdot x^{n + 1} - 1 = x \cdot
    (x^n - 1) + (x - 1) \geqslant x \cdot (n \cdot (x - 1)) + (x - 1) = (x -
    1) \cdot (x \cdot n + 1) >_{x > 1 \rightarrow n \cdot x > n} (x - 1) \cdot
    (n + 1) \Rightarrow n + 1 \in S$
  \end{enumerate}
\end{proof}

\begin{theorem}
  \label{x^n<gtr>n if x<gtr>1}If $N \in \mathbbm{N}_{\mathbbm{R}} \backslash
  \{ 0 \}$ and $x \in \mathbbm{R}$ with $x > 1$ then there exists a $n \in
  \mathbbm{N}_{\mathbbm{R}} \backslash \{ 0 \}$ such that $x^n > N$
\end{theorem}

\begin{proof}[$N = 1$][$N > 1$]
  Consider the following cases for $N \in \mathbbm{N}_{\mathbbm{R}} \backslash
  \{ 0 \}$
  \begin{enumerate}
    \item then $N = 1 < x = x^1$ proving the theorem
    
    \item take $\delta = \frac{N - 1}{x - 1} > 0$ then by the Archimedean
    property (\ref{consequence of the archimedean property for the reals})
    there exists a $n \in \mathbbm{N}_{\mathbbm{R}}$ such that $\delta < n$ as
    by definition of $\delta$ we have $N - 1 = \delta \cdot (x - 1) \leqslant
    n \cdot (x - 1) < x^n - 1$ (using the previous theorem) so that $N < x^n$
  \end{enumerate}
\end{proof}

\begin{theorem}
  \label{n<less>m=<gtr>x^n<less>x^m<less>x^n}If $x \in \mathbbm{R}$ with $0 <
  x < 1$ and $n, m \in \mathbbm{N}_{\mathbbm{R}}$ with $n < m$ then $x^m <
  x^n$
\end{theorem}

\begin{proof}
  We prove this by induction, so let $m \in \mathbbm{N}_{\mathbbm{R}}$ and
  take $S_m = \{ n \in \mathbbm{N}_0 |x^{m + n} < x^m \}$ then we have
  \begin{enumerate}
    \item if $n = 1$ then $x^{m + n} = x^{n + 1} = x \cdot (x^n) <_{x < 1
    \Rightarrow x \cdot x^n < x^n} x^n \Rightarrow 1 \in S_m$
    
    \item if $n \in S_m$ then $x^{m + (n + 1)} = x \cdot x^{m + n} <_{x < 1}
    x^{m + n} <_{n \in S_m} x^m$
  \end{enumerate}
  Using mathematical induction we have then that $S_M =\mathbbm{N}_0$. If now
  $n, m \in \mathbbm{N}_{\mathbbm{R}}$ with $n < m$ then $k = m - n > 0$ or $k
  \in \mathbbm{N}_0 = S_n$ so that $x^m = x^{n + k} < x^n$
\end{proof}

\begin{theorem}
  \label{0<less>x^n<less>e}If $\varepsilon \in \mathbbm{R}_+$ and $x \in
  \mathbbm{R}$ such that $0 < x < 1$ then $\exists N \in
  \mathbbm{N}_{\mathbbm{R}} \backslash \{ 0 \}$ such that $0 < x^n <
  \varepsilon$ \ if $n \geqslant N$
\end{theorem}

\begin{proof}
  As $\varepsilon > 0$ we have that $\frac{1}{\varepsilon}$ is defined and by
  the Archimedean property (see \ref{consequence of the archimedean property
  for the reals}) there exists a $n \in \mathbbm{N}_{\mathbbm{R}}$ such that
  $0 < \frac{1}{\varepsilon} < n$, as $0 < x < 1$ we have that $1 <
  \frac{1}{x}$ so that by \ref{x^n<gtr>n if x<gtr>1} there exists a $N \in
  \mathbbm{N}_{\mathbbm{R}} \backslash \{ 0 \}$ such that $n < x^N$ and thus
  we have that $0 < \frac{1}{\varepsilon} < \left( \frac{1}{x} \right)^N =
  \frac{1}{x^N} \Rightarrowlim_{x, \varepsilon > 0} 0 < x^N < \varepsilon$. If
  now we have $n \geqslant N$ then by the previous theorem we have $0 < x^n
  \leqslant x^N < \varepsilon$
\end{proof}

\section{Square root}

\begin{theorem}
  The function $^2 : \{ x \in \mathbbm{R}|0 \leqslant x \} \rightarrow \{ x
  \in \mathbbm{R}|0 \leqslant x \}$ defined by $x \rightarrow x^2$ (where $0
  \leqslant x^2$ by \ref{properties of positive, negative real numbers} so
  this is indeed a function) is a bijection.
\end{theorem}

\begin{proof}[injectivity][$x = 0$][$y = 0$][$0 < x, y$][$x < y$][$y <
x$][surjectivity][$y = 0$][$y = 1$][$0 < y \neq 1$][$y < 1$][$1 < y$][$1 <
y$][$y < 1$][$s_y^2 - y < 0$][$0 < s_y^2 - y$][$s_y^2 - y = 0$]
  
  \begin{enumerate}
    \item If $x, y \geqslant 0$ is such that $x^2 = y^2$ then either
    \begin{enumerate}
      \item then $0 = x^2 = y^2 \Rightarrow y^2 = 0
      \Rightarrowlim_{\text{\ref{properties of positive, negative real
      numbers}}} y = 0 \Rightarrow x = y$
      
      \item then $0 = y^2 = x^2 \Rightarrow x = 0 \Rightarrow x = y$
      
      \item then if we assume that $x \neq y$ then we have the following cases
      \begin{enumerate}
        \item then $x^2 < y \cdot x \Rightarrowlim_{x < y \Rightarrow y \cdot
        x < y^2} x^2 < y^2$ contradicting $x^2 = y^2$
        
        \item then similar we have $y^2 < x^2$ contradicting $x^2 = y^2$
      \end{enumerate}
      In all the cases we have a contradiction so we must have $x = y$.
    \end{enumerate}
    \item Given $y \in \{ x \in \mathbbm{R}|0 \leqslant x \}$ then $0
    \leqslant y$ and we have the following cases :
    \begin{enumerate}
      \item then $y = 0 = 0 \cdot 0 = 0^2 \Rightarrow 0^2 = y$
      
      \item then $1^2 = 1 \cdot 1 = 1 = y \Rightarrow 1^2 = y$
      
      \item take then $S_y = \{ t \in \mathbbm{R}|0 \leqslant t \wedge t^2
      \leqslant y \}$. Then as $0^2 = 0 < y$ we have $0 \in S_y$ and thus $0
      \nin S_y$. Now as $y \neq 1$ we have the following possibilities for $y$
      \begin{enumerate}
        \item Take now $t \in S_y$ and suppose $1 < t$. Then from $1 < t$ and
        $0 < 1 < t \Rightarrow 0 < t$ and \ref{properties of positive,
        negative real numbers} we have $t \leqslant t^2$. From $y < 1$ and $1
        < t$ we have then $y < t$ and using $t \leqslant t^2$ we have then $y
        < t^2$ which contradicts $t \in S_y \Rightarrow t^2 \leqslant y$. So
        we must have $t \leqslant 1$ and thus we have that $1$ is a upper
        bound of $S_t$.
        
        \item Take now $t \in S_y$ and suppose $y < t$. Then from $1 < y$ and
        $y < t$ we have $1 < t$. Using $1 < t \Rightarrow 0 < t$ and
        \ref{properties of positive, negative real numbers} on $1 < t$ we have
        $1 \cdot t < t \cdot t \Rightarrow t < t^2 \Rightarrowlim_{y < t} y <
        t^2$ contradicting $t \in S_y \Rightarrow t^2 \leqslant y$. So we must
        have $t \leqslant y$ and thus we have that $y$ is a upper bound of
        $S_y$.
      \end{enumerate}
      Using (i) and (ii) above we see that $S_y$ has a upper bound and is not
      empty. From the conditional completeness on the reals (see \ref{the
      reals are conditional complete}) we have that there exists a lowest
      upper bound $s_y = \sup (S_y)$. For $S_y$ we consider now again the two
      possible cases for $y \neq 1$
      \begin{enumerate}
        \item then $1^2 = 1 < y \Rightarrow 1 \in S_y$ and thus $S_y$ contains
        a $t$ with $0 < t$
        
        \item then from $0 < y$ and \ref{properties of positive, negative real
        numbers} we have $y \cdot y < 1 \cdot y \Rightarrow y^2 < y
        \Rightarrow y \in S_y$ and thus $S_y$ contains a $t$ with $0 < t$.
      \end{enumerate}
      From the above (i) and (ii) we have ,as $\forall t \in S_y$ that $t
      \leqslant s_y = \sup (S_y)$, that
      \begin{equation}
        \label{eq 8.10} 0 < s_y .
      \end{equation}
      Now from $0 < s_y$ there exists by \ref{consequence of the archimedean
      property for the reals} a $\varepsilon_0 \in \mathbbm{R}$ such that $0 <
      \varepsilon_0 < s_y$. Now for all $\varepsilon \in \mathbbm{R}$ with $0
      < \varepsilon < s_y$ we have \ $0 < s_y - \varepsilon < s_y < s_y +
      \varepsilon$, \ then using \ref{properties of positive, negative real
      numbers} we have $(s_y - \varepsilon)^2 < s + y \cdot (s_y -
      \varepsilon) \wedge (s_y - \varepsilon) \cdot s_y < s_y^2$ and $s_y^2 <
      s_y \cdot (s_y + \varepsilon) \wedge s_y \cdot (s_y + \varepsilon) <
      (s_y + \varepsilon)^2$ giving finally $(s_y - \varepsilon)^2 < s_y^2 <
      (s_y \upl \varepsilon)^2$. As $s_y$ is a upper bound of $S_y$ and $s_y <
      s_y + \varepsilon$ we must have $s_y + \varepsilon \nin S_y$ and as $0 <
      s_y < s_y \upl \varepsilon$ we must have that $y < (s_y +
      \varepsilon)^2$. Using the fact that $s_y$ is the lowest bound on $S_y
      $and the fact that $s_y - \varepsilon < s_y$ there exists a $f \in S_y$
      such that $s_y - \varepsilon < f \leqslant s_{\varepsilon}$. Using
      \ref{properties of positive, negative real numbers} we have then that $f
      \cdot (s_y - \varepsilon) < f^2 \wedge (s_y - \varepsilon)^2 < f \cdot
      (s_y - \varepsilon) \Rightarrow (s_y - \varepsilon)^2 < f^2$ and as $f
      \in S_y$ we have $f^2 \leqslant y$ giving us $(s_y - \varepsilon)^2 <
      f^2 \leqslant y < (s_y + \varepsilon)^2$. To summarize we have proved
      that if $0 < \varepsilon_0 < s_y$ that
      \begin{equation}
        \label{eq 8.1} (s_y - \varepsilon)^2 < s_y^2 < (s_y + \varepsilon)^2
      \end{equation}
      \begin{equation}
        \label{eq 8.2} (s_y - \varepsilon)^2 < y < (s_y + \varepsilon)^2
      \end{equation}
      Using \ref{properties of positive, negative real numbers} on \ref{eq
      8.2} gives
      \begin{equation}
        \label{eq 8.3} - (s_y + \varepsilon)^2 < - y < \um (s_y -
        \varepsilon)^2
      \end{equation}
      and adding \ref{eq 8.1} and \ref{eq 8.3} gives
      \begin{equation}
        \label{eq 8.4} (s_y - \varepsilon)^2 - (s_y + \varepsilon)^2 < s_y^2 -
        y < (s_y + \varepsilon)^2 - (s_y - \varepsilon)^2
      \end{equation}
      As $(s_y + \varepsilon)^2 - (s_y - \varepsilon)^2 = s_y^2 + 2 \cdot s_y
      \cdot \varepsilon + \varepsilon^2 - s_y^2 + 2 \cdot s_y \cdot
      \varepsilon - \varepsilon^2 = 4 \cdot \varepsilon \cdot s_y$ and using
      \ref{eq 8.4} we have then
      \begin{equation}
        \label{eq 8.6} - 4 \cdot \varepsilon \cdot s_y < s_y^2 - y < 4 \cdot
        \varepsilon \cdot s_y \tmop{or} - 4 \cdot \varepsilon \cdot s_y < y -
        s_y^2 < 4 \cdot \varepsilon \cdot s_y
      \end{equation}
      For $s_y^2 - y$ we have now the following possible cases to consider
      \begin{enumerate}
        \item Take then $\delta = y - s_y^2$ then $0 < \delta$. Take now
        $\varepsilon = \min \left( \frac{\delta}{4 \cdot s_y}, \varepsilon_0
        \right)$ (which is well defined as $0 < s_y$ see \ref{eq 8.10}) then
        we have $0 < \varepsilon \leqslant \varepsilon_0 < s_y$ so we have by
        \ref{eq 8.6} that $\delta = y - s^2_y < 4 \cdot \varepsilon \cdot
        s_y$. As by definition of $\varepsilon$ we have $\varepsilon \leqslant
        \frac{\delta}{4 \cdot s_y} \Rightarrow 4 \cdot \varepsilon \cdot s_y
        \leqslant \delta < 4 \cdot \varepsilon \cdot s_y$ and we reach a
        contradiction.
        
        \item Take then $\delta = s^2_y - y$ then we have (see \ref{eq 8.6})
        $0 < \delta$. Take now $\varepsilon = \min \left( \frac{\delta}{4
        \cdot s_y}, \varepsilon_0 \right)$ then we have $0 < \varepsilon
        \leqslant \varepsilon_0 < s_y$, so we have by \ref{eq 8.6} that
        $\delta = s^2_y - y < 4 \cdot \varepsilon \cdot s_y$. As by definition
        of $\varepsilon$ we have $\varepsilon \leqslant \frac{\delta}{4 \cdot
        s_y} \Rightarrow 4 \cdot \varepsilon \cdot s_y \leqslant \delta < 4
        \cdot \varepsilon \cdot s_y$ and we reach a contradiction.
        
        \item Then $s_y^2 = y$
      \end{enumerate}
      So (iii) is the only possible case [as (i) and (ii) leads to a
      contradiction] we must have $s_y^2 = y$.
    \end{enumerate}
    So in cases (a), (b) and (c) we have found a $t$ such that $t^2 = y$
    proving surjectivity.
  \end{enumerate}
\end{proof}

\begin{definition}
  \label{square root}{\index{$\sqrt{x}$}}Using the above theorem we have that
  the function $^2 : \{ x \in \mathbbm{R}|0 \leqslant x \} \rightarrow \{ x
  \in \mathbbm{R}|0 \leqslant x \}$ has a inverse mapping. This mapping is
  called the square root function and is noted by $\sqrt{}$. So $\sqrt{} : \{
  x \in \mathbbm{R}|0 \leqslant x \} \rightarrow \{ x \in \mathbbm{R} | 0
  \leqslant x \nobracket \}$ is mapping such that $\sqrt{} \circ^{} ^2 = i_{\{
  x \in \mathbbm{R}|0 \leqslant x \}} =^2 \circ \sqrt{}$ or $\forall x \in \{
  x \in \mathbbm{R}|0 \leqslant x \}$ we have $\left( \sqrt{x} \right)^2 = x$
  and $\sqrt{x^2} = x$
\end{definition}

\begin{note}
  The requirement that $0 \leqslant x$ to have $\sqrt{x^2} = x = \sqrt{x}^2$
  is really necessary as $^2 : \mathbbm{R} \rightarrow \{ x \in \mathbbm{R}|0
  \leqslant x \}$ is definitely not a injection and thus a bijection as $(-
  1)^2 = 1 = 1^2$
\end{note}

\begin{theorem}
  \label{square root is a strictly increasing function}$\sqrt{} : \{ x \in
  \mathbbm{R}|0 \leqslant x \} \rightarrow \{ x \in \mathbbm{R}|0 \leqslant x
  \}$ is a strictly increasing function
\end{theorem}

\begin{proof}
  If $x, y \in \{ x \in \mathbbm{R}|0 \leqslant x \}$ is such that $x < y$.
  Assume then that $\sqrt{y} \leqslant \sqrt{x}$ then by \ref{properties of
  positive, negative real numbers} we have $\left( \sqrt{y} \right)^2
  \leqslant \left( \sqrt{x} \right)^2 \Rightarrow y \leqslant x$ contradicting
  $x < y$ so we must have $\sqrt{x} < \sqrt{y}$
\end{proof}

\begin{theorem}
  \label{product of square roots}If $x, y \in \{ x \in \mathbbm{R}|0 \leqslant
  x \}$ then $\sqrt{x \cdot y} = \sqrt{x} \cdot \sqrt{y}$
\end{theorem}

\begin{proof}
  As $\left( \sqrt{x \cdot y} \right)^2 = x \cdot y = \left( \sqrt{x}
  \right)^2 \cdot \left( \sqrt{y} \right)^2 = \left( \sqrt{x} \cdot \sqrt{y}
  \right)^2$ we have by the fact that $^2 : \{ x \in \mathbbm{R}|0 \leqslant x
  \} \rightarrow \{ x \in \mathbbm{R}|0 \leqslant x \}$ is a bijection and
  thus injective so we have $\sqrt{x \cdot y} = \sqrt{x} \cdot \sqrt{y}$
\end{proof}

\begin{theorem}
  \label{square root of sum is lower the sum of squareroots}Given $x, y \in \{
  x \in \mathbbm{R}|0 \leqslant x \}$ then we have $\sqrt{x + y} \leqslant
  \sqrt{x} + \sqrt{y}$
\end{theorem}

\begin{proof}
  We prove this by contradiction, so assume that $ \sqrt{x} + \sqrt{y} <
  \sqrt{x + y} \Rightarrowlim_{\text{\ref{properties of positive, negative
  real numbers}}} - \sqrt{x + y} < - \left( \sqrt{x} + \sqrt{y} \right)$ and
  by multiplying by $\sqrt{x} + \sqrt{y} \geqslant 0$ we have $- \sqrt{x + y}
  \cdot \left( \sqrt{x} + \sqrt{y} \right) < - \left( \sqrt{x} + \sqrt{y}
  \right) \cdot \left( \sqrt{x} + \sqrt{y} \right) = - \left( x + y + 2 \cdot
  \sqrt{x} \cdot \sqrt{y} \right)$. Now we have by \ref{properties of
  positive, negative real numbers} that $0 \leqslant \left( \sqrt{x + y} -
  \left( \sqrt{x} + \sqrt{y} \right) \right)^2 = \left( \sqrt{x + y} \right)^2
  + \left( \sqrt{x} + \sqrt{y} \right)^2 - 2 \cdot \sqrt{x + y} \cdot \left(
  \sqrt{x} + \sqrt{y} \right) = x + y + \left( \sqrt{x} \right)^2 + \left(
  \sqrt{y} \right)^2 + 2 \cdot \sqrt{x} \cdot \sqrt{y} - 2 \cdot \sqrt{x + y}
  \cdot \left( \sqrt{x} + \sqrt{y} \right) = 2 \cdot x + 2 \cdot y + 2 \cdot
  \sqrt{x} \cdot \sqrt{y} - 2 \cdot \sqrt{x + y} \cdot \left( \sqrt{x} +
  \sqrt{y} \right) \Rightarrowlim_{\tmop{multiply} \tmop{by} 0 < \frac{1}{2}}
  0 \leqslant x + y + \sqrt{x} \cdot \sqrt{y} - \sqrt{x + y} \cdot \left(
  \sqrt{x} + \sqrt{y} \right) < x + y + \sqrt{x} \cdot \sqrt{y} - \left( x + y
  + 2 \cdot \sqrt{x}  \sqrt{y} \right) = - \sqrt{x} \cdot \sqrt{y} \leqslant 0
  \Rightarrow 0 < 0$ a contradiction. So we must have $\sqrt{x + y} \leqslant
  \sqrt{x} + \sqrt{y}$
\end{proof}



\section{Denumerability in $\mathbbm{R}$}

\begin{theorem}
  The sets $\mathbbm{N}_{\mathbbm{R}}$, $\mathbbm{Z}_{\mathbbm{R}}$ and
  $\mathbbm{Q}_{\mathbbm{R}}$ are \tmtextbf{denumerable}.
\end{theorem}

\begin{proof}
  First as $i_{\mathbbm{Q}_{\mathbbm{N}}} : \mathbbm{N} \rightarrow
  \mathbbm{N}_{\mathbbm{R}}$ is a isomorphism (see \ref{properties of natural
  numbers embedded in the reals}) we have that $\mathbbm{N}_{\mathbbm{R}}$ is
  denumerable.
  
  Second by \ref{the integer numbers are denumerable} we have that
  $\mathbbm{Z}$ is \tmtextbf{denumerable} so there exists a bijection $b :
  \mathbbm{N} \rightarrow \mathbbm{Z}$, using \ref{whole numbers embedded in
  the reals from a sub ring} we have the existence of a bijection
  $i_{\mathbbm{Q}_{\mathbbm{Z}}} : \mathbbm{Z} \rightarrow
  \mathbbm{Z}_{\mathbbm{R}}$ and thus $\mathbbm{N}$ is bijective with
  $\mathbbm{Z}_{\mathbbm{R}}$.
  
  Finally by \ref{The rational numbers are denumerable} we have that
  $\mathbbm{Q}$ is \tmtextbf{denumerable }so there exists a bijection between
  $\mathbbm{N}$ and $\mathbbm{Q}$, using \ref{rationals in reals form a
  subfield} we have that $\mathbbm{Q}$ is bijective with
  $\mathbbm{Q}_{\mathbbm{R}}$ and thus there exists a bijection between
  $\mathbbm{N}$ and $\mathbbm{Q}_{\mathbbm{R}}$.
\end{proof}

\chapter{The complex numbers}

\section{Definition and arithmetic's}

\begin{definition}
  \label{complex numbers}{\index{$\mathbbm{C}$}}The space $\langle
  \mathbbm{C}, +, \cdot \rangle$ of complex numbers is defined by: (note that
  $+, \cdot$ has different meanings in $\mathbbm{C}$ and $\mathbbm{R}$)
  \begin{enumerate}
    \item $\mathbbm{C}=\mathbbm{R} \times \mathbbm{R}$
    
    \item $\upl : \mathbbm{C} \times \mathbbm{C} \rightarrow \mathbbm{C}$ by
    $((x, y), (x', y')) \rightarrow (x, y) + (x', y')
    \equallim_{\tmop{defined}} (x + x', y + y')$
    
    \item $\cdot : \mathbbm{C} \times \mathbbm{C} \rightarrow \mathbbm{C}$ by
    $((x, y), (x', y')) \rightarrow (x, y) \cdot (x', y')
    \equallim_{\tmop{defined}} (x \cdot x' - y \cdot y', x \cdot y' + x' \cdot
    y)$
  \end{enumerate}
\end{definition}

\begin{theorem}
  \label{the complex numbers form a field}$\langle \mathbbm{C}, +, . \rangle$
  forms a field. The unit element is $(1, 0)$ noted also as $1$, the neutral
  element is $(0, 0)$ noted as $0$, $- (x, y) = (- x, - y)$ and if $(x, y)
  \neq (0, 0)$ then $(x, y)^{- 1} = \left( \frac{x^2 \cdot y}{x^3 \cdot y + x
  \cdot y^3}, \frac{- y^2 \cdot x}{x^3 \cdot y + x \cdot y^3} \right)$
\end{theorem}

\begin{proof}[associativity][neutral element][inverse
element][commutativity][distributive][neutral
element][commutative][associative][multiplicative inverse]
  First we prove that $\langle \mathbbm{C}, + \rangle$ forms a abelian group
  using the fact that $\langle \mathbbm{R}, +, \cdot \rangle$ is a field.
  \begin{enumerate}
    \item $\forall (x, y), (x', y'), (x'', y'') \in \mathbbm{C}$ we have $(x,
    y) + ((x', y') + (x'', y'')) = (x, y) + (x' + x'', y' + y'') = (x + (x' +
    x''), y + (y' + y'')) \equallim ((x + x') + x'', (y + y') + y'') = (x +
    x', y + y') + (x'', y'') = ((x, y) + (x', y')) + (x'', y'')$
    
    \item $\forall (x, y) \in \mathbbm{C}$ we have $(x, y) + (0, 0) = (x + 0,
    y + 0) = (x, y) = (x + 0, y + 0) = (x, y) + (0, 0)$
    
    \item $\forall (x, y) \in \mathbbm{C}$ we have $(x, y) + (- x, - y) = (x +
    (- x), y + (- y)) = (0, 0) = (- x + x, - y + y) = (- x, - y) + (x, y)$. So
    $(- x, - y)$ is the additive inverse of $(x, y)$.
    
    \item $\forall (x, y), (x', y') \in \mathbbm{C}$ we have $(x, y) + (x',
    y') = (x + x', y + y') = (x' + x, y' + y) = (x', y') + (x, y)$
  \end{enumerate}
  Next we prove the rest of the axioms of a field
  \begin{enumerate}
    \item $\forall (x, y), (x', y'), (x'', y'')$ we have $(x, y) \cdot ((x',
    y') + (x'', y'')) = (x, y) \cdot (x' + x'', y' + y'') = (x \cdot (x' \upl
    x'') - y \cdot (y' + y''), x \cdot (y' + y'') + y \cdot (x' + x'')) = (x
    \cdot x' + x \cdot x'' - y \cdot y' - y \cdot y'', x \cdot y' + x \cdot
    y'' + y \cdot x' + y \cdot x'') = (x \cdot x' - y \cdot y' + x \cdot x'' -
    y \cdot y'', x \cdot y' + y \cdot x' + x \cdot y'' + y \cdot x'') = (x
    \cdot x' - y \cdot y', x \cdot y' + y \cdot x') + (x \cdot x'' - y \cdot
    y'', x \cdot y'' + y \cdot x'') = (x, y) \cdot (x', y') + (x, y) \cdot
    (x'', y'')$
    
    \item $\forall (x, y) \in \mathbbm{C}$ we have $(1, 0) \cdot (x, y) = (1
    \cdot x - 0 \cdot y, 1 \cdot y + 0 \cdot x) = (x, y)$ and $(x, y) \cdot
    (1, 0) = (x \cdot 1 + y \cdot 0, x \cdot 0 + y \cdot 1) = (x, y)$
    
    \item $\forall (x, y), (x', y') \in \mathbbm{C}$ we have $(x, y) \cdot
    (x', y') = (x \cdot x' - y \cdot y', x \cdot y' + y \cdot x') = (x' \cdot
    x - y' \cdot y, y' \cdot x + x' \cdot y) = (x' \cdot x - y' \cdot y, x'
    \cdot y + y' \cdot x) = (x', y') \cdot (x, y)$
    
    \item $\forall (x, y), (x', y'), (x'', y'')$ then we have $(x, y) \cdot
    ((x', y') \cdot (x'', y'')) = (x, y) \cdot (x' \cdot x'' - y' \cdot y'',
    x' \cdot y'' + y' \cdot x'') = (x \cdot (x' \cdot x'' - y' \cdot y'') - y
    \cdot (x' \cdot y'' + y' \cdot x''), x \cdot (x' \cdot y'' + y' \cdot x'')
    + y \cdot (x' \cdot x'' - y' \cdot y'')) = (x \cdot x' \cdot x'' - x \cdot
    y' \cdot y'' - y \cdot x' \cdot y'' - y \cdot y' \cdot x'', x \cdot x'
    \cdot y'' + x \cdot y' \cdot x'' + y \cdot x' \cdot x'' - y \cdot y' \cdot
    y'') = ((x \cdot x' - y \cdot y') \cdot x'' - (x \cdot y' + y \cdot x')
    \cdot y'', (x \cdot x' - y \cdot y') \cdot y'' + (x \cdot y' + y \cdot x')
    \cdot x'') = (x \cdot x' - y \cdot y', x \cdot y' \upl y \cdot x') \cdot
    (x'', y'') = ((x, y) \cdot (x', y')) \cdot (x'', y'')$
    
    \item Given $(x, y) \in \mathbbm{C}$ with $(x, y) \neq (0, 0) \Rightarrow
    x, y \neq 0$. Now $x^3 \cdot y + x \cdot y^3 = x \cdot (x^2 \cdot y + y^3)
    = x \cdot y \cdot (x^2 + y^2)$ now using \ref{properties of positive,
    negative real numbers} we have $0 < x^2, y^2 \Rightarrow 0 < (x^2 + y^2)
    \Rightarrowlim_{x, y \neq o} x \cdot y \cdot (x^2 + y^2) \Rightarrow x^3
    \cdot y + x \cdot y^3 \neq 0$. So $\left( \frac{x^2 \cdot y}{x^3 \cdot y +
    x \cdot y^3}, \frac{- y^2 \cdot x}{x^3 \cdot y + x \cdot y^3} \right)$ is
    defined and $(x, y) \cdot \left( \frac{x^2 \cdot y}{x^3 \cdot y + x \cdot
    y^3}, \frac{- y^2 \cdot x}{x^3 \cdot y + x \cdot y^3} \right) = \left(
    \frac{x \cdot x^2 \cdot y + y \cdot y^2 \cdot x}{x^3 \cdot y + x \cdot
    y^3}, \frac{x \cdot - y^2 \cdot x + y \cdot x^2 \cdot y}{x^3 \cdot y + x
    \cdot y^3} \right) = \left( \frac{x^3 \cdot y + x \cdot y^3}{x^3 \cdot y +
    x \cdot y^3}, \frac{- x^2 \cdot y^2 + x^2 \cdot y^2}{x^3 \cdot y + x \cdot
    y^3} \right) = (1, 0) \equallim_{(3)} \left( \frac{x^2 \cdot y}{x^3 \cdot
    y + x \cdot y^3}, \frac{- y^2 \cdot x}{x^3 \cdot y + x \cdot y^3} \right)
    \cdot (x, y)$
  \end{enumerate}
\end{proof}

\begin{definition}
  \label{conjugate of a complex number}{\index{conjugate of a complex
  number}}Given $(x, y) \in \mathbbm{C}$ we define $\overline{(x, y)} = (x, -
  y)$
\end{definition}

\begin{lemma}
  \label{complex conjugate is positive}If $z, z' \in \mathbbm{C}$ then
  $\overline{z + z'} = \bar{z} + \overline{z'}$, $\overline{z \cdot z'} =
  \bar{z} \cdot \overline{z'}$ and $\overline{\bar{z}} = z$
\end{lemma}

\begin{proof}
  If $z = (x, y), z' = (x', y')$ then
  \begin{enumerate}
    \item $\overline{(x, y) + (x', y')} = \overline{(x + x', y + y')} = (x +
    x', - y - y') = (x, - y) + (x', - y') = \overline{(x, y)} + \overline{(x',
    y')}$
    
    \item $\overline{(x, y) \cdot (x', y')} = \overline{(x \cdot x' - y \cdot
    y', x \cdot y' + y \cdot x')} = (x \cdot x' - y \cdot y', - x \cdot y' - y
    \cdot x') = (x \cdot x' - (- y) \cdot (- y'), x \cdot (- y') + (- y) \cdot
    x') = (x, - y) \cdot (x', - y') = \overline{(x, y)} \cdot \overline{(x',
    y')}$
    
    \item $\overline{\bar{z}} = \overline{\overline{(x, y)}} = \overline{(x, -
    y)} = (x, y) = z$
  \end{enumerate}
\end{proof}

\begin{definition}
  If $z = (x, y) \in \mathbbm{C}$ we have $\tmop{Re} (z) = x \in \mathbbm{R}$
  and $\tmop{Im} (z) = y \in \mathbbm{R}$, $i \in \mathbbm{Z}$ is defined by
  $i = (0, 1)$ and the unit in $\mathbbm{C}$ noted as $1 = (1, 0)$
\end{definition}

\begin{theorem}
  $i^2 = - 1$
\end{theorem}

\begin{proof}
  $i^2 = (0, 1) \cdot (0, 1) = (0 \cdot 0 - 1 \cdot 1, 0 \cdot 1 + 1 \cdot 0)
  = (- 1, 0) = - (1, 0) = - 1$ (the unit in $\mathbbm{C}$)
\end{proof}

\begin{definition}
  Given $(x, y) \in \mathbbm{C}$ we define $| (x, y) | = \sqrt{x^2 + y^2} \in
  \mathbbm{R}$ (which is defined as $x^2 + y^2 \geqslant 0$) and this is
  called the complex norm.
\end{definition}

\begin{theorem}
  \label{complex norm is a norm}$\forall z, z' \in \mathbbm{C}$ we have
  \begin{enumerate}
    \item $0 \leqslant | z |$
    
    \item $| z \cdot z' | = | z | \cdot | z' |$
    
    \item $| \bar{z} | = | z |$
    
    \item $\tmop{Re} (z) \leqslant | z |$
    
    \item $| z + z' | \leqslant | z | + | z' |$
    
    \item $| z | = 0 \Leftrightarrow z = 0$
  \end{enumerate}
\end{theorem}

\begin{proof}
  
  \begin{enumerate}
    \item This is trivial as $\sqrt{} : \{ x \in \mathbbm{R}|0 \leqslant x \}
    \rightarrow \{ x \in \mathbbm{R}|0 \leqslant x \}$
    
    \item If $z = (x, y) \in \mathbbm{C}, z' = (x', y') \in \mathbbm{C}$ then
    $z \cdot z' = (x \cdot x' - y \cdot y', x \cdot y' + y \cdot x')$ and $| z
    \cdot z' | = \sqrt{(x \cdot x' - y \cdot y')^2 + (x \cdot y' + y \cdot
    x')^2} = \sqrt{x^2 \cdot x^{\prime 2} + y^2 \cdot y^{\prime 2} - 2 \cdot x
    \cdot x' \cdot y \cdot y' + x^2 \cdot y^{\prime 2} + y^2 \cdot x^{\prime
    2} + 2 \cdot x \cdot x' \cdot y \cdot y'} = \sqrt{x^2 \cdot x^{\prime 2} +
    y^2 \cdot y^{\prime 2} + x^2 \cdot y^{\prime 2} + y^2 \cdot x^{\prime 2}}
    = \sqrt{x^2 \cdot (x^{\prime 2} + y^{\prime 2}) + y^2 \cdot (x^{\prime 2}
    + y^{\prime 2})} = \sqrt{(x^2 + y^2) \cdot (x^{\prime 2} + y^{\prime 2})}
    \equallim_{0 \leqslant (x^2 + y^2), (x^{\prime 2} + y^{\prime 2})}
    \sqrt{x^2 + y^2} \cdot \sqrt{x^{\prime 2} + y^{\prime 2}} = | z | \cdot |
    z' |$
    
    \item If $z = (x, y)$ then $| \bar{z} | = | (x, - y) | = \sqrt{x^2 + (-
    y)^2} = \sqrt{x^2 + y^2} = | z |$
    
    \item If $z = (x, y)$ then $\tmop{Re} (z) = x$, now as $0 \leqslant y^2$
    we have that $| x |^2 = x^2 \leqslant x^2 + y^2 \Rightarrowlim_{\sqrt{}
    \tmop{is} \tmop{increasing} \text{(see \ref{square root is a strictly
    increasing function})}} \sqrt{| x |^2} \leqslant \sqrt{x^2 + y^2} = | z |
    \Rightarrowlim_{0 < |x} | x | = \sqrt{| x^2 |} \leqslant | z | \Rightarrow
    x \leqslant | x | \leqslant | z |$ and thus $\tmop{Re} (x) \leqslant | z
    |$
    
    \item If $z = (x, y), z' = (x', y')$ then $| z + z' | = | (x + x', y + y')
    | = \sqrt{(x + x')^2 + (y + y')^2}$. Now $(x + x')^2 + (y + y')^2 = x^2 +
    x^{\prime 2} + 2 \cdot x \cdot x' + y^2 + y^{\prime 2} + 2 \cdot y \cdot
    y' = (x^2 + y^2) + (x^{\prime 2} + y^{\prime 2}) + 2 \cdot (x \cdot x' + y
    \cdot y') = | (x, y) |^2 + | (x', y') |^2 = 2 \cdot \tmop{Re} (x \cdot x'
    + y \cdot y', - x \cdot y' + y \cdot x') = | z |^2 + | z' |^2 + 2 \cdot
    \tmop{Re} ((x, y) \cdot (x', - y')) = | z |^2 + | z' |^2 + 2 \cdot
    \tmop{Re} (z \cdot \overline{z'})$ so we have
    \begin{equation}
      \label{9.1} (x + x')^2 + (y + y')^2 = | z |^2 + | z' |^2 + 2 \cdot
      \tmop{Re} (z \cdot \overline{z'})
    \end{equation}
    Using (4) we have
    \begin{equation}
      \label{9.2} 2 \cdot \tmop{Re} (z \cdot \overline{z'}) \leqslant 2 \cdot
      | z \cdot \overline{z'} | \equallim_{(2)} 2 \cdot | z | \cdot |
      \overline{z'} | \equallim_{(3)} 2 \cdot | z | \cdot | z' |
    \end{equation}
    Using \ref{9.1} and \ref{9.2} we have then $(x + x')^2 + (y + y')^2
    \leqslant | z |^2 + | z' |^2 + 2 \cdot | z | \cdot | z' | = (| z | + | z'
    |)^2$ and using the fact that $\sqrt{}$ is increasing (see \ref{square
    root is a strictly increasing function}) we have then that $| z + z' | =
    \sqrt{(x + x')^2 + (y + y')^2} \leqslant \sqrt{(| z | + | z' |)^2}
    \equallim_{0 \leqslant | z | + | z' |} | z | + | z' |$
    
    \item If $z = (x, y)$ then if $z = 0 = (0, 0)$ we have $| z | = \sqrt{0^2
    + 0^2} = \sqrt{0} = 0$. On the other side if $| z | = 0$ then $\sqrt{x^2 +
    y^2} = 0 \Rightarrow \left( \sqrt{x^2 + y^2} \right) = 0^2 = 0
    \Rightarrowlim_{0 \leqslant x^2 + y^2} x^2 + y^2 = 0$. If now $x \neq 0
    \Rightarrow 0 < | x | \Rightarrow 0 < | x |^2 = x^2 \Rightarrowlim_{0
    \leqslant y^2 \tmop{and} \text{\ref{properties of positive, negative real
    numbers}}} 0 < x^2 + y^2$ a contradiction. So we must have that $x = 0$
    and thus $0 = x^2 + y^2 = 0^2 + y^2 = y^2$, if $y \neq 0 \Rightarrow 0 < |
    y | \Rightarrow 0 < | y |^2 = y^2 = 0$ a contradiction so we have $y = 0$
    and thus $z = (0, 0) = 0$
  \end{enumerate}
\end{proof}

\

\begin{definition}
  \label{reals embedded in complex
  numbers}{\index{$\mathbbm{R}_{\mathbbm{C}}$}}$\mathbbm{R}_{\mathbbm{C}} = \{
  (x, y) \in \mathbbm{C}|y = 0 \} \subseteq \mathbbm{C}$
\end{definition}

\begin{theorem}
  \label{subfield real embedded in the complex
  field}$\mathbbm{R}_{\mathbbm{C}}$ forms a sub-field of $\langle \mathbbm{R},
  +, \cdot \rangle$ where if $\alpha = (\alpha', 0), \beta = (\beta', 0) \in
  \mathbbm{R}$ then $\alpha + \beta = (\alpha' + \beta', 0) \nocomma$, $\alpha
  \cdot \beta = (\alpha' \cdot \beta', 0)$ and $\alpha^{- 1} = (\alpha^{\prime
  - 1}, 0)$
\end{theorem}

\begin{proof}
  If $x, y \in \mathbbm{R}_{\mathbbm{C}}$ then $x = (x', 0)$, $y = (y', 0)$ so
  that
  \begin{enumerate}
    \item $x + y = (x', 0) + (y', 0) = (x' + y', 0) \in
    \mathbbm{R}_{\mathbbm{C}}$
    
    \item $x \cdot y = (x' \cdot y' - 0 \cdot 0, x' \cdot 0 + 0 \cdot y') =
    (x' \cdot y', 0) \in \mathbbm{R}_{\mathbbm{C}}$
    
    \item If $x \in \mathbbm{R}_{\mathbbm{C}} \backslash \{ 0 \}$ then $x'
    \neq 0$ and take $x^{- 1} = (x^{\prime - 1}, 0) \in
    \mathbbm{R}_{\mathbbm{C}}$ then $x^{- 1} \cdot x \equallim_{(2)}
    (x^{\prime - 1} \cdot x', 0) = (1, 0) = 1$ so that $x^{\prime - 1} \in
    \mathbbm{R}_{\mathbbm{C}}$
    
    \item $0 = (0, 0) \in \mathbbm{R}_{\mathbbm{C}}$
    
    \item $1 = (1, 0) \in \mathbbm{R}_{\mathbbm{C}}$
  \end{enumerate}
\end{proof}

\begin{theorem}
  If $x \in \mathbbm{C}$ then
  \begin{enumerate}
    \item $x \cdot \bar{x} = (| x |^2, 0) \in \mathbbm{R}_{\mathbbm{C}}$
    
    \item if $x = \bar{x}$ then $x \in \mathbbm{R}_{\mathbbm{C}}$
  \end{enumerate}
\end{theorem}

\begin{proof}
  Let $x = (a, b)$ then we have
  \begin{enumerate}
    \item $x \cdot \bar{x} = (a, b) \cdot (a, - b) = (a \cdot a - b \cdot (-
    b), a \cdot (- b) + b \cdot a) = (a^2 + b^2, 0) = (| x |^2, 0)$
    
    \item If $x = \bar{x}$ then $(a, b) = (a, - b) \Rightarrow b = - b
    \Rightarrow 2 \cdot b = 0 \Rightarrow b = 0 \Rightarrow x = (a, 0) \in
    \mathbbm{R}_{\mathbbm{C}}$
  \end{enumerate}
\end{proof}

\begin{theorem}
  \label{embedding reals in the comples}Define $i_{\mathbbm{R}} : \mathbbm{R}
  \rightarrow \mathbbm{R}_{\mathbbm{C}}$ by $x \rightarrow i_{\mathbbm{R}} (x)
  = (x, 0)$ then this is a field isomorphism that preserves the norm $| |$
\end{theorem}

\begin{proof}[injectivity][surjectivity]
  First we prove that $i_{\mathbbm{R}}$ is a bijection
  \begin{enumerate}
    \item If $i_{\mathbbm{R}} (x) = i_{\mathbbm{R}} (x') \Rightarrow (x, 0) =
    (x', 0) \Rightarrow x = x'$
    
    \item If $z \in \mathbbm{R}_{\mathbbm{C}}$ then $x = (x, 0)$ where $x \in
    \mathbbm{R}$ and then $i_{\mathbbm{R}} (x) = (x, 0) = z$
  \end{enumerate}
  Next we prove that it preserves $+, \cdot$ and the identity
  \begin{enumerate}
    \item If $x, y \in \mathbbm{R}$ then $i_{\mathbbm{R}} (x + y) = (x + y, 0)
    = (x, 0) + (y, 0) = i_{\mathbbm{R}} (x) + i_{\mathbbm{R}} (y)$
    
    \item If $x, y \in \mathbbm{R}$ then $i_{\mathbbm{R}} (x \cdot y) = (x
    \cdot y, 0) = (x \cdot y - 0 \cdot 0, x \cdot 0 + 0 \cdot y) = (x, 0)
    \cdot (y, 0) = i_{\mathbbm{R}} (x) \cdot i_{\mathbbm{R}} (y)$
    
    \item $i_{\mathbbm{R}} (1) = (1, 0)$ the multiplicative unit in
    $\mathbbm{C}$
  \end{enumerate}
  Finally we prove that $i_{\mathbbm{R}}$ preserves norms.
  
  If $x \in \mathbbm{R}$ then $| (x, 0) | = \sqrt{x^2 + 0^2} = \sqrt{x^2} =
  \sqrt{| x |^2} = | x |$
  
  \ 
\end{proof}

The above theorem shows that we can consider $\langle \mathbbm{R}, +, \cdot
\rangle$ embedded in $\langle \mathbbm{C}, \upl, \cdot \rangle$ as the sub
field $\langle \mathbbm{R}_{\mathbbm{C}}, +, \cdot \rangle$

\begin{theorem}
  If $z \in \mathbbm{C}$ then we can write $z = x + i \cdot y$ where $x, y \in
  \mathbbm{R}_{\mathbbm{C}}$
\end{theorem}

\begin{proof}
  If $z \in \mathbbm{C}$ then $z = (x', y')$ and take $x = i_{\mathbbm{R}}
  (x'), y = i_{\mathbbm{R}} (y')$ so that $x = (x', 0)$ and $y = (y', 0)$ now
  $i \cdot y = (0, 1) \cdot (y', 0) = (0 \cdot y' - 1 \cdot 0, 0 \cdot 0 + 1
  \cdot y') = (0, y')$ so $x + i \cdot y = (x', 0) + (0, y') = (x', y') = z$
\end{proof}

\section{Order relation on $\mathbbm{C}$}

\begin{definition}
  \label{order relation in C}Define $\leqslant \in \mathbbm{C} \times
  \mathbbm{C}$ by $(x, y) \leqslant (x', y')$ iff either
  \begin{enumerate}
    \item $x < x'$
    
    \item $x = x'$ and $y \leqslant y'$
  \end{enumerate}
  Using \ref{lexical order} we have that $\leqslant$ is a partial order
  relation on $\mathbbm{C}$
\end{definition}

\begin{theorem}
  \label{order in real and complex}If $x, y \in \mathbbm{R}$ with $x \leqslant
  y$ then $i_{\mathbbm{R}} (x) \leqslant i_{\mathbbm{R}} (y)$ and if $x < y
  \Rightarrow i_{\mathbbm{R}} (x) < i_{\mathbbm{R}} (y)$
\end{theorem}

\begin{proof}[$x < y$][$x = y$]
  We have the following cases to consider
  \begin{enumerate}
    \item then $i_{\mathbbm{R}} (x) = (x, 0) \leqslant (y, 0) =
    i_{\mathbbm{R}} (y)$, as $(x, 0) \neq (y, 0)$ [if $(x, 0) = (y, 0)
    \Rightarrow x = y$] so that $i_{\mathbbm{R}} (x) < i_{\mathbbm{R}} (y)$
    
    \item then as $0 \leqslant 0$ we have $(x, 0) = i_{\mathbbm{R}} (x)
    \leqslant i_{\mathbbm{R}} (y) = (y, 0)$
  \end{enumerate}
\end{proof}

\begin{theorem}
  \label{zero sum in C}If $x, y \in \mathbbm{R}_{\mathbbm{C}}$ with $0
  \leqslant x, y$ and $x + y = 0$ then $x = y = 0$
\end{theorem}

\begin{proof}
  First there exists a $x', y' \in \mathbbm{R}$ such that $x = i_{\mathbbm{R}}
  (x') = (x', 0)$, $y = i_{\mathbbm{R}} (y') = (y, 0)$ and as $0 = (0, 0) = x
  + y = (x', 0) + (y', 0) = (x' + y', 0)$ we have $x' + y' = 0$. Also from $0
  \leqslant x$ we have $0 \leqslant x'$ [if $x' < 0$ then $0 = i_{\mathbbm{R}}
  (x') < i_{\mathbbm{R}} (0) = x \Rightarrow x < 0 \leqslant x \Rightarrow x <
  x$ a contradiction] and similar we have $0 \leqslant y'$. So using \ref{zero
  sum in R} we have then $x' = 0 = y'$ but this means that $x = y = (0, 0) =
  0$
\end{proof}

\begin{definition}
  \label{natural numbers embedded in the complex
  numbers}{\index{$\mathbbm{C}_{\mathbbm{N}}$}}Define
  $\mathbbm{N}_{\mathbbm{C}} \subseteq \mathbbm{Q}_{\mathbbm{C}} \subseteq
  \mathbbm{R}_{\mathbbm{C}} \subseteq \mathbbm{C}$ by
  $\mathbbm{N}_{\mathbbm{C}} = \{ (x', 0) \in \mathbbm{R}_{\mathbbm{C}} | x'
  \in \mathbbm{N}_{\mathbbm{R}} \}$, $\mathbbm{Q}_{\mathbbm{C}} = \{ (x', 0)
  \in \mathbbm{R}_{\mathbbm{C}} |x' \in \mathbbm{Q}_{\mathbbm{R}} \}$
\end{definition}

\begin{theorem}
  \label{embedding of natural numbers in the reals}$\langle
  \mathbbm{N}_{\mathbbm{C}}, + \rangle, \langle \mathbbm{N}_{\mathbbm{C}},
  \cdot \rangle$ forms a semi-groups and if we define
  $i_{\mathbbm{R}_{\mathbbm{N}}} : \mathbbm{N} \rightarrow
  \mathbbm{N}_{\mathbbm{C}}$ by $i_{\mathbbm{R}_{\mathbbm{N}}} =
  i_{\mathbbm{R}_{|\mathbbm{N}_{\mathbbm{R}}}} \circ
  i_{\mathbbm{Q}_{\mathbbm{N}}}$ (see \ref{properties of natural numbers
  embedded in the reals}) then $i_{\mathbbm{R}_{\mathbbm{N}}}$ is a a order
  preserving semi-group isomorphism (for $+$ and $\cdot$)
\end{theorem}

\begin{proof}
  First we prove that $\langle \mathbbm{N}_{\mathbbm{C}}, + \rangle$ and
  $\langle \mathbbm{N}_{\mathbbm{C}}, \cdot \rangle$ are semi-groups. If $x, y
  \in \mathbbm{N}_{\mathbbm{C}}$ then we have $x = (x', 0), y = (y', 0)$ and
  $x', y' \in \mathbbm{N}_{\mathbbm{R}}$ so that
  \begin{enumerate}
    \item $x + y = (x', 0) + (y', 0) = (x' + y', 0) \in
    \mathbbm{N}_{\mathbbm{C}}$ as $x' + y' \in \mathbbm{N}_{\mathbbm{R}}$ (for
    $\mathbbm{N}_{\mathbbm{R}}$ is a semi-group)
    
    \item $0 = (0, 0) \in \mathbbm{N}_{\mathbbm{C}}$
    
    \item $x \cdot y = (x' \cdot y' - 0 \cdot 0, x' \cdot 0 + 0 \cdot y') =
    (x' \cdot y', 0) \in \mathbbm{N}_{\mathbbm{C}}$ (as $x' \cdot y' \in
    \mathbbm{N}_{\mathbbm{C}}$ which is a semi-group)
    
    \item $1 = (1, 0) \in \mathbbm{N}_{\mathbbm{C}}$
  \end{enumerate}
  To prove that $i_{\mathbbm{R}_{\mathbbm{N}}}$ is a order preserving
  semi-group isomorphism, note that by \ref{properties of natural numbers
  embedded in the reals} $i_{\mathbbm{Q}_{\mathbbm{N}}}$ is a order preserving
  semi-group isomorphism and that by \ref{embedding reals in the comples},
  \ref{order in real and complex} $i_{\mathbbm{R}}$ is a order preserving
  isomorphism and thus $i_{\mathbbm{R}_{|\mathbbm{N}}}$ is a order preserving
  semi-group isomorphism. 
\end{proof}

\begin{definition}
  {\index{$\mathbbm{R}_{\mathbbm{C}+},
  \mathbbm{R}_{\mathbbm{C}-}$}}$\mathbbm{R}_{\mathbbm{C}+} = \{ \alpha \in
  \mathbbm{R}_{\mathbbm{C}} | \alpha > 0 \}$ and $\mathbbm{R}_{\mathbbm{C}-} =
  \{ \alpha \in \mathbbm{R}_{\mathbbm{C}} | \alpha < 0 \}$
\end{definition}

\begin{definition}
  If $\alpha \in \mathbbm{R}_{\mathbbm{C}}$ define then $\alpha^n$ by
  $\alpha^0 = 1$ and $\alpha^{n + 1} = a^n \cdot a$ (see \ref{iteration over a
  group})
\end{definition}

Based on the properties (see \ref{properties of positive, negative real
numbers}0) of the reals and \ref{subfield real embedded in the complex field}
we have the following theorem

\begin{theorem}
  \label{properties of positive, negative embedded reals}We have the following
  for the set of reals embedded in the complex
  \begin{enumerate}
    \item If $\alpha, \beta \in \mathbbm{R}_{\mathbbm{C}}$ then we have the
    following exclusive possibilities
    \begin{enumerate}
      \item $\alpha < \beta$
      
      \item $\beta < \alpha$
      
      \item $\alpha = \beta$
    \end{enumerate}
    \item If $\alpha, \beta \in \mathbbm{R}_{\mathbbm{C}}$ with $\alpha <
    \beta \Rightarrow - \beta < \um \alpha$
    
    \item If $\alpha, \beta, \gamma \in \mathbbm{R}_{\mathbbm{C}}$ with
    $\alpha < \beta \Rightarrow \alpha \upl \gamma < \beta \upl \gamma$
    
    \item If $\alpha, \beta \in \mathbbm{R}_{\mathbbm{C}}$ and $\alpha <
    \beta$ with $\gamma \in \mathbbm{R}_{\mathbbm{C}+}$ then $\alpha \cdot
    \gamma < \beta \cdot \gamma$
    
    \item If $\alpha, \beta \in \mathbbm{R}_{\mathbbm{C}}$ and $\alpha <
    \beta$ with $\gamma \in \mathbbm{R}_{\mathbbm{C}-}$ then $\beta \cdot
    \gamma < \alpha \cdot \gamma$
    
    \item If $0 < \alpha \Rightarrow 0 < \alpha^{- 1}$
    
    \item If $0 < \alpha < \beta \Rightarrow \beta^{- 1} < \alpha^{- 1}$
    
    \item If $\alpha \in \mathbbm{R}_{\mathbbm{C}} \Rightarrow 0 \leqslant
    \alpha^2$ and if $\alpha \neq 0$ then $0 < \alpha^2$
    
    \item If $\alpha, \beta \in \mathbbm{R}_{\mathbbm{C}+} \bigcup \{ 0 \}$
    are such that $\alpha < \beta$ then $\alpha^2 < \beta^2$ (so $^2 : \{ x
    \in \mathbbm{R}_{\mathbbm{C}} |0 \leqslant x \} \rightarrow \{ x \in
    \mathbbm{R}_{\mathbbm{C}} |0 \leqslant x \}$ is a strictly increasing
    function)
    
    \item If $\alpha = (\alpha', 0) \in \mathbbm{R}_{\mathbbm{C}+}$ \ and $n
    \in \mathbbm{N}$ then $\alpha^n = (\alpha^{\prime n}, 0) \Rightarrow
    \alpha^n \in \mathbbm{R}_{\mathbbm{C}+}$
    
    \item If $\alpha \in \mathbbm{R}_C$ then $| \alpha | = | \tmop{real}
    (\alpha) |$
  \end{enumerate}
\end{theorem}

\begin{proof}
  First note that if $\alpha, \beta \in \mathbbm{R}_{\mathbbm{C}}$ then there
  exists a $\alpha', \beta'$ such that $\alpha = (\alpha', 0), \beta =
  (\beta', 0)$ and $\alpha \leqslant \beta$ if and only if $\alpha' \leqslant
  \beta'$ also $\alpha = \beta$ if and only if $\alpha' = \beta'$ so that we
  have also $\alpha < \beta$ if and only if $\alpha' < \beta'$
  \begin{enumerate}
    \item If $\alpha = (\alpha', 0), \beta = (b', 0)$ then as for $\alpha',
    \beta'$ we have the following possibilities
    \begin{enumerate}
      \item $\alpha' < \beta'$
      
      \item $\beta' = \alpha'$
      
      \item $\alpha' = \beta'$
    \end{enumerate}
    we have the following possibilities for $\alpha, \beta$
    \begin{enumerate}
      \item $\alpha < \beta$
      
      \item $\beta < \alpha$
      
      \item $\alpha = \beta$
    \end{enumerate}
    \item If $(\alpha', 0) = \alpha < \beta = (\beta', 0) \Rightarrow \alpha'
    < \beta' \Rightarrow - \beta' < - \alpha' \Rightarrow - \alpha < - \beta$
    
    \item If $(\alpha', 0) = \alpha < \beta = (\beta', 0) \Rightarrow \alpha'
    < \beta'$ and $\gamma = (\gamma', 0)$ then $\alpha' + \gamma' < \beta' +
    \gamma' \Rightarrow \alpha + \gamma = (\alpha' + \gamma', 0) < (\beta' +
    \gamma', 0) = \beta + \gamma$
    
    \item If $\alpha, \beta \in \mathbbm{R}_{\mathbbm{C}}$ and $\gamma \in
    \mathbbm{R}_{\mathbbm{C}+}$ then $\alpha = (\alpha', 0), \beta = (\beta',
    0), \gamma = (\gamma', 0)$ where $\alpha, \beta \in \mathbbm{R}$ and
    $\gamma \in \mathbbm{R}_+$ then from $\alpha < \beta$ it follows that
    $\alpha' < \beta' \Rightarrowlim_{\gamma' \in \mathbbm{R}_+} \alpha' \cdot
    \gamma' < \beta' \cdot \gamma' \Rightarrow \alpha \cdot \gamma = (\alpha'
    \cdot \gamma', 0) < (\beta' \cdot \gamma', 0) = \beta \cdot \gamma$
    
    \item If $\alpha, \beta \in \mathbbm{R}_{\mathbbm{C}}$ and $\gamma \in
    \mathbbm{R}_{\mathbbm{C}-}$ then $\alpha = (\alpha', 0), \beta = (\beta',
    0), \gamma = (\gamma', 0)$ where $\alpha, \beta \in \mathbbm{R}$ and
    $\gamma \in \mathbbm{R}_-$ then from $\alpha < \beta$ it follows that
    $\alpha' < \beta' \Rightarrowlim_{\gamma' \in \mathbbm{R}_-} \beta' \cdot
    \gamma' < \alpha' \cdot \gamma' \Rightarrow \beta \cdot \gamma = (\beta'
    \cdot \gamma', 0) < (\alpha' \cdot \gamma', 0) = \alpha \cdot \gamma$
    
    \item If $0 = (0, 0) < \alpha = (\alpha', 0) \Rightarrow 0 < \alpha'
    \Rightarrow 0 < \alpha^{\prime - 1} \Rightarrow 0 = (0, 0) <
    (\alpha^{\prime - 1}, 0) = \alpha^{- 1}$
    
    \item If $0 < (\alpha', 0) = \alpha < \beta = (\beta', 0) \Rightarrow 0 <
    \alpha' < \beta' = > \beta^{\prime - 1} < \alpha^{\prime - 1} \Rightarrow
    \beta^{- 1} = (\beta^{\prime - 1}, 0) < (\alpha^{\prime - 1}, 0) <
    \alpha^{- 1}$
    
    \item If $\alpha = (\alpha', 0) \in \mathbbm{R}_{\mathbbm{C}}$ then $0
    \leqslant (<) \alpha^{\prime 2}$ ($<$ if $\alpha' \neq 0$) so we have $0
    \leqslant (<) (\alpha^{\prime 2}, 0) = \alpha^2$ ($<$ if $\alpha =
    (\alpha', 0) \neq (0, 0) = 0$) \
    
    \item If $\alpha = (\alpha', 0), \beta = (\beta', 0) \in
    \mathbbm{R}_{\mathbbm{C}+} \bigcup \{ 0 \}$ are such that $\alpha < \beta$
    then $\alpha', \beta' \in \mathbbm{R}_+ \bigcup \{ 0 \}$ so
    $\alpha^{\prime 2} < \beta^{\prime 2} \Rightarrow \alpha^2 =
    (\alpha^{\prime 2}, 0) < (\beta^{\prime 2}, 0) = \beta^2$
    
    \item We prove this by induction, so let $S = \{ n \in \mathbbm{N}|
    \tmop{if} (\alpha', 0) = \alpha \in \mathbbm{R}_{\mathbbm{C}+} \tmop{then}
    \alpha^n = (\alpha^{\prime n}, 0) \}$ then we have
    \begin{enumerate}
      \item if $n = 0$ then by definition we have $\alpha^0 = (\alpha', 0) =
      (\alpha^{\prime 0}, 0) \Rightarrow 0 \in S$
      
      \item If $n \in S$ then $\alpha^{n + 1} = \alpha^n \cdot a =
      (\alpha^{\prime n}, 0) \cdot (\alpha', 0) = (\alpha^{\prime n + 1}, 0)
      \Rightarrow n + 1 \in S$
    \end{enumerate}
    \item If $\alpha \in \mathbbm{R}_{\mathbbm{C}}$ then $\alpha = (\alpha',
    0)$ so that $| \alpha | = \sqrt{(\alpha^{\prime 2} + 0^2)} =
    \sqrt{\alpha^{\prime 2}} = \sqrt{| \alpha' |' 2} = | \alpha' | = |
    \tmop{real} (\alpha) |$
  \end{enumerate}
\end{proof}

\begin{definition}
  If $\alpha \in \mathbbm{R}_{\mathbbm{C}+}$ then $\sqrt{\alpha} = \left(
  \sqrt{\tmop{Re} (\alpha)}, 0 \right) = \left( \sqrt{\alpha'}, 0 \right) \in
  \mathbbm{R}_{\mathbbm{C}}$ where $\alpha = (\alpha', 0)$
\end{definition}

\begin{theorem}
  If $\alpha \in \mathbbm{R}_C$ then
  \begin{enumerate}
    \item If $\alpha \geqslant 0$ then $\left( \sqrt{\alpha} \right)^2 =
    \alpha$
    
    \item If $\alpha \in \mathbbm{C}$ then $\alpha^2 \in
    \mathbbm{R}_{\mathbbm{C}} +$ so that it's $\sqrt{}$ is defined and
    $\sqrt{\alpha^2} = (| \alpha |, 0)$
  \end{enumerate}
\end{theorem}

\begin{proof}
  
  \begin{enumerate}
    \item If $\alpha \in \mathbbm{R}_{\mathbbm{C}}$ and $\alpha \geqslant 0$
    then $\alpha \in \mathbbm{R}_{\mathbbm{C}+}$ so that $\alpha = (\alpha',
    0)$ with $\alpha' \geqslant 0$ so that $\left( \sqrt{(\alpha', 0)}
    \right)^2 = \left( \sqrt{\alpha'}, 0 \right)^2 = \left( \left( \sqrt{a'}
    \right)^2, 0 \right) = (\alpha', 0) = \alpha$
    
    \item If $\alpha \in \mathbbm{R}_{\mathbbm{C}} \Rightarrow \alpha =
    (\alpha', 0) \Rightarrow \alpha^2 = (\alpha^{\prime 2}, 0) = (| \alpha'
    |^2, 0) \Rightarrow \sqrt{\alpha^2} = \sqrt{(| \alpha' |^2, 0)} = \left(
    \sqrt{| \alpha' |^2}, 0 \right) = (| \alpha' |, 0)$
  \end{enumerate}
\end{proof}

\

\

\begin{notation}
  To avoid the use of excesive notation we use $\mathbbm{N}, \mathbbm{Q},
  \mathbbm{R}, \mathbbm{C}$ when we actually means $\mathbbm{N}_{\mathbbm{C}},
  \mathbbm{Q}_{\mathbbm{C}}, \mathbbm{R}_{\mathbbm{C}}, \mathbbm{C}$ so that
  $\mathbbm{N} \subseteq \mathbbm{Q} \subseteq \mathbbm{R} \subseteq
  \mathbbm{C}$ using the natural injections.
\end{notation}{\hspace*{\fill}}}}

\end{document}
